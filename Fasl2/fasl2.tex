\documentclass[a4paper,12pt]{article}
%\usepackage[top=1in,margin=5em]{geometry}
\usepackage[top=7em,bottom=1pt,right=0.8in,left=0.8in,headheight=65pt,headsep=1cm]{geometry}
\usepackage{tikz-page}
\usetikzlibrary{shadows.blur}
\usepackage{enumitem}
\usepackage{amsmath}
\usepackage{mhchem}
\usepackage{tasks}
\usepackage{multicol}
\usepackage{xepersian}
\usepackage{setspace}
\usepackage{textcomp, gensymb}


\pagestyle{plain}
\tikzset{
	secnode/.style={
		minimum height=1cm,
		inner xsep=20pt,
		rotate=90,
		anchor=north east,
		draw=white,
		fill=black,
		text=white,
		blur shadow={shadow blur steps=5,shadow blur extra rounding=1.3pt}},
	pagenode/.style={
		minimum width=5mm,
		minimum height=1cm,
		inner sep=2pt,
		anchor=south east,
		draw=white,
		fill=black,
		text=white,
		blur shadow={shadow blur steps=5,shadow blur extra rounding=1.3pt}}
}
\newcommand{\tikzpagelayout}{
	\draw[black,line width=2pt,rounded corners=20pt] ([xshift=10mm]page.northwest) |- ([xshift=-2cm,yshift=10mm]page.southeast);
	\node[secnode] at ([xshift=5mm]page.northwest) {سالم غذا‌های پی در | شکیباییان};
	\node[pagenode] at ([xshift=-1cm,yshift=5mm]page.southeast) {\thepage};

}

\newenvironment{en}
	{\begin{enumerate}\setlength\itemsep{-0.2em}}
	{\end{enumerate}}
	

\newenvironment{iit}
	{\begin{itemize}\setlength\itemsep{-0.5em}}
	{\end{itemize}}
	
\newcommand{\ff}{\rule{1cm}{0.15mm}\;}
\newcommand{\fb}{\rule{2cm}{0.15mm}\;}
\newcommand{\fs}{\rule{1cm}{0.15mm}}
\newcommand{\fsm}{{\rule{0.5cm}{0.15mm}\;}}
\newcommand{\lin}{\vspace{4pt}\hrule\vspace{4pt}}
\newcommand\gototask[1]{\addtocounter{task}{\numexpr#1-\value{task}\relax}}
\NewTasksEnvironment[label=\arabic*.,label-format=\bfseries,label-width=4ex]{answers}[\a]

\settextfont{XB Niloofar}
\setdigitfont{XB Niloofar}
\setstretch{1.5}
\setlist[itemize]{topsep=0pt}
\setlist[enumerate]{topsep=0pt}
\renewcommand{\headrulewidth}{0pt}
\setlength{\headheight}{1pt}
\setlength{\columnseprule}{1pt}
\setlength{\columnsep}{0.5cm}

\begin{document}
	%page 1
	\ff و \ff ،‌ اجزاء بنیادی جهان مادی هستند. انرژی از راه‌های گوناگون با ماده ارتباط دارد، چنانکه کاهش \ff خورشید موجب تولید \ff می‌شود. «غذا» همواره نقش محوری در رشد، تندرسی و زندگی انسان داشته است. پیشرفت دانش و فناوری، موجب افرایش تولید فرآورده‌های کشاورزی و دامی و تولید صنعتی غذا شده است. در تولید انبوه، به دلیل فساد مواد غذایی و دشواری نگهداری، حفظ کیفیت و ارزش مواد غذایی، اهمیت به‌سزایی دارد. همچنین در صنایع غذایی، حجم عظیمی «آب» مصرف می‌شود و تأمین غذای جامعه را مشکل‌تر می‌کند.
	\lin
	\vspace{8pt}
خود را بیازمایید صفحه ۵۱؛\\
الف) \ff و دردرجه دوم 
\ff 
 و  
\ff.
\\
ب) با حذف خوراکی‌های غیر ضروری (مانند چیپس، پفک، نوشابه) تاحدی امکان تأمین هزینه مصرف انواع \ff در سبد خانوار تأمین می‌شود. (!!)\\
پ)
	\begin{iit}
		\item توزیع شیر رایگان در مدارس، مهدکودک‌ها، پادگان‌ها و دانشگاه‌ها
	 
	 	\item دادن علوفه و داروی دامی با قیمت ارزان به دامدار 
	 
	 	\item فرهنگ‌سازی مصرف
	\end{iit}
	ت) فرهنگ‌سازی استفاده بیشتر از حبوبات (مصرف عدسی یا آش در وعده صبحانه یا عصرانه)، مصرف انواع حبوبات در سالاد\\
	\textbf {سرانه مصرف ماده غذایی، مقدار میانگین مصرف آن را به ازای هر فرد در یک گستره زمانی نشان می‌دهد.}
	\vspace{4pt}
	\hrule
	\begin{center}
		\textbf{غذا، چیزی فراتر از یک پاسخ به احساس گرسنگی است. مصرف غذا؛}
	\end{center}
	\begin{en}
		\item مورد نیاز برای ماهیچه‌ها، ارسال پیام‌های عصبی، جابه‌جایی یون‌ها و مولکول‌ها از دیواره هر یاخته را تأمین می‌کند.
		\item \ff اولیه برای ساخت و رشد بخش‌های مختلف بدن را فراهم می‌کند. (بخش عمده \ff ، \ff  و \fs ـی موجود در بدن از غذا تأمین می‌شود.) این فرآیند‌ها وابسته به انجام واکنش‌های شیمیایی هستند، که دمای بدن را نیز تنظیم و کنترل می‌کنند. هر کدام از این واکنش‌ها، «آهنگ» ویژه‌ای دارند.
	\end{en}

تغذیه درست، شامل وعده‌های غذایی است که مخلوط منابع از انواع ذره‌ها را در بر می‌گیرد. سوء تغذیه هنگامی رخ می‌نماید که وعده‌های غذایی با کمبود نوع خاصی از این ذرات همراه باشد. از طرفی، افزایش نامناسب برخی مولکول‌ها و یون‌ها در غذا نیز، سبب بیماری خواهد شد.
	\vspace{4pt}
	\hrule
	\begin{center}
		\textbf{«غذا، ماده و انرژی»}
	\end{center}
بدن برای انجام فعالیت‌های ارادی و غیرارادی، به ماده و انرژی نیاز دارد. یکی از راه‌های آزاد شدن انرژی سوخت‌ها (مانند بنزین و …) «سوزاندن» آن‌ها است. هر ماده غذایی نیز انرژی دارد و میزان انرژی به «جرم» آن بستگی دارد.
	\newpage

	%page 2

	\begin{center}
		\textbf{دمای یک ماده، از چه خبر می‌دهد؟\\
			دما: کمیتی که میزان \ff و \ff اجسام را نشان می‌دهد.}
	\end{center}
شکل ۱ صفحه ۵۴: وقتی به ظرف محتوی آب، گرما داده می‌شود، به تدریج \ff آن افرایش می‌یابد تا اینکه سرانجام \ff یا اگر به یخ داده شود، \ff می‌شود. در این حالت‌ها، با گرفتن گرما، \ff ذرات بیشتر شده و دما \ff می‌رود یا \ff ماده عوض می‌شود.
	\begin{center}
	جنبش نامنظم ذره‌ها: گاز $\bigcirc$ مایع $\bigcirc$ جامد / آب‌ گرم $\bigcirc$ آب سرد
	\end{center}
دمای بالاتر $\leftarrow$ میانگین \ff حرکت ذرات بیشتر $\leftarrow$ میانگین انرژی \ff ذرات بیشتر.\\
	\textbf{یعنی:}
	\underline{دمای ماده}
؛ معیاری برای توصیف \ff تندی و \ff انرژی جنبشی ذره‌های سازنده ماده است.\\
یکای رایج دما، درجه \ff (\qquad) اما یکای دما در ،SI \ff (\qquad) است.\\
ارزش دمایی ۱ درجه سانتی‌گراد برابر ۱ کلوین \ff .\\
لذا در فرآیند‌هایی که دما تغییر می‌کند،
$\mathrm{\Delta\theta\bigcirc\Delta T}$
است.
	\qquad\qquad\qquad \ff = \ff + \ff \\
	\textbf{با هم بیندیشیم صفحه ۵۵:}
	\begin{en}
	 \item 
	الف) شکل A نمونه‌ای از هوا را در \ff نشان می‌دهد.\\
	ب) شکل B، نمونه‌ای از هوا را در یک روز \ff نشان می‌دهد.\\
	پ) اگر مجموع انرژی جنبشی ذره‌های سازنده یک نمونه ماده، هم‌ارز با انرژی گرمایی آن باشد؛
	انرژی گرمایی \ff بیشتر بوده زیرا \ff آن بیشتر است.
	\item 
	الف) میانگین تندی مولکول‌ها در ظرف A$\bigcirc$ 	ظرف B\\
	ب) انرژی گرمایی ظرف A $\bigcirc$ظرف B (چون \ff \ff آن بیشتر است.)
	\end{en}
	با هم بیندیشیم ۱:  \ff یکسان، دمای \ff متفاوت $\leftarrow$ انرژی گرمایی متفاوت\\
	با هم بیندیشیم ۲:	\ff یکسان، \ff \ff متفاوت $\leftarrow$ انرژی گرمایی متفاوت\\
	\textbf{نتیجه:}
	 انرژی گرمایی یک نمونه ماده، هم به \ff و هم به \ff \ff بستگی دارد.\\
	 \textbf{تذکر:}
	  چون کار کردن «تعداد ذرات»، آسان نیست می‌توان به جای آن، \ff ماده را در نظر گرفت. چنانکه در فیزیک نیز، انرژی جنبشی از رابطه \fb به دست می‌آید.
	\begin{center}
  	\textbf{تهیه غذا آب‌پز، تجربه تفاوت «گرما» و «دما»}
  \end{center}
گرما، صورتی از \ff و یکای آن در ،SI \ff (\ff) است.
 ($\mathrm{\qquad 1 \qquad = 1 Kgm^2.s^{-2}}$)\\
 از یکای \ff (\ff) نیز برای بیان مقدار گرما در پزشکی و زیست‌شناسی و علم تغزیه استفاده می‌شود.\\
	\hrule
	\vspace{4pt}
	\textbf{تعریف ژول:}\\
	\textbf{تعریف کالری:}
	\begin{flushleft}
 	 $\mathrm{\ff cal = \ff J}$ 
 \end{flushleft}
 	\vspace{4pt}
	\hrule
	\vspace{4pt}
 انرژی گرمایی: \ff انرژی‌های جنبشی ذرات ماده / دما: \ff انرژی جنبشی ذرات ماده\\
 انرژی گرمایی و دما، از ویژگی‌های یک «نمونه ماده» 
$\frac{است}{نیست}$
 و 
 $\frac{می‌تواند}{نمی‌تواند}$
 برای توصیف آن «ماده» به کار رود.

	\newpage

	%page 3

	\begin{center}
	\textbf{«گرما»}
\end{center}
صورتی از \ff است، که از جسم با \ff بالاتر، به جسم با \fsm پایین‌تر منتقل می‌شود. داد و ستد گرما، می‌‌تواند موجب تغییر \ff مواد شود.
	\\
گرما، از ویژگی‌های یک «نمونه ماده» \ff و \ff برای توصیف آن «ماده» به کار رود.
	\\
هنگامی که به ۲ ماده، گرمای یکسان داده شود، لزوماً به یک اندازه \ff نمی‌شوند.
	\lin
هنگامی که به ۲ ماده، گرمای یکسان داده شود، لزوماً به یک اندازه \ff نمی‌شوند.
	\\
	\textbf{یعنی:}
 دادن گرمای یکسان به دو ماده، لزوما/حتما تغییر دمای یکسانی را موجب می‌شود/نمی‌شود.
مثال: اگر بخواهیم دمای آب و روغن زیتون* (با جرم برابر) به یک اندازه بالا رود، باید به آب، گرمای \ff بدهیم.
	\lin
* الگوی ساختاری «روغن‌ها» با «چربی‌ها» یکسان است اما تفاوت‌هایی در ساختار دارند ( مانند پیوند دوگانه بیشتر در ساختار زنجیر کربنی \ff ) که موجب تفاوت در \ff و \ff آن‌ها می‌شود. چنان که روغن‌ها در دمای عادی، \ff و چربی‌ها \ff هستند.
	\lin
	\textbf{با هم بیندیشیم صفحه ۵۷:}
\\
الف) چون \ff \ff موجود در نمونه آب، بسیار \ff از روغن زیتون است.
	دلیل: موادی چون آب و اتانول، به دلیل وجود \ff \ff بین مولکول‌های خود، گرمای ویژه بالایی دارند*. (جدول ۱ صفحه ۵۸).
	دمای آب و روغن زیتون، به یک اندازه زیاد \ff است. برای افزایش دمای آب به میزان ۵۰ درجه سانتی‌گراد، (نسبت به روغن زیتون) گرمای \ff جذب شده، پس انرژی گرمایی ظرف محتوی آب، \ff است و تخم مرغ، گرمای \ff دریافت می‌کند.
ب) ظرفیت گرمایی (C): \ff لازم برای افرایش \ff ماده به اندازه \ff درجه \ff ( یا ۱ \ff )
\\
	$C_{H_2O} = \frac{\quad J}{\quad K (\quad J.K^{-1})}\bigcirc C_{il.oil} = \frac{\quad J}{\quad k (\quad J.k^{-1})}$
	(یکای C: \ff . \ff )
	$Q = C \Delta\theta\rightarrow C = 
	\frac{Q}{\Delta\theta}\rightarrow$
\\
پ) 
	بستگی دارد به \ff ماده و \ff ماده (به خاطر تفاوت در نوع \ff یا نیرو‌های \ff \ff )
	 هرچه \ff ماده بیشتر باشد، برای رساندن آن به دمای مشخص، \ff بیشتری لازم است.
\\
ت) گرمای ویژه (c): ظرفیت گرمایی \ff \ff ماده 
	\begin{flushleft}
	$Q = mc\Delta\theta\rightarrow c = \frac{Q}{m\Delta\theta}\downarrow$\\
	$C_{H_2O}=\frac{\quad}{\quad}=\ff ( \quad\quad )\quad C_{ol.oil}=\frac{\quad}{\quad}=\ff (\quad\quad)$
یکای c: ( \ff . \ff . \ff )

\end{flushleft}
ث) رابطه C با :c
\\
هر کمیتی که از ویژگی‌های ماده باشد، (میتواند/نمیتواند) برای توصیف آن به کار رود.
\\
ظرفیت گرمایی؛ از ویژگی‌های نمونه ماده \ff و می‌تواند/نمی‌تواند برای توصیف آن ماده به کار رود.
\\ 
گرمای ویژه؛ از ویژگی‌های یک نمونه ماده \ff و \ff برای توصیف آن ماده به کار می‌رود.

	\newpage

	%page 4

	\textbf{خود را بیازمایید صفحه ۵۸:}
	\begin{en}
	\item 
	 \ff می‌یابد. با‌گذشت زمان، چای، همه/بخشی از انرژی گرمایی خود را به‌/از محیط می‌دهد/می‌گیرد پس \ff و \ff انرژی جنبشی ذرات آن، \ff می‌یابد. (کاهش \ff \ff و \ff نمونه)
	دلیل: گرما، از جایی که \ff تر است (دمای \ff) به جایی که \ff است (دمای \ff) حرکت می‌کند. دمای چای ( ) از دمای محیط ( ) \ff است و با \ff انرژی گرمایی، با آن «\ff \ff» می‌شود.
	\item 
	گرما را می‌توان هم‌ارز با آن مقدار انرژی گرمایی/دمایی داشت که به دلیل تفاوت در انرژی گرمایی/دما جاری می‌شود.
	\item 
ماده اصلی تشکیل‌دهنده هر دو، \ff است، پس به مقدار \ff موجود در آن‌ها توجه می‌کنیم. نان، \ff کمتری دارد، چون \ff شده است، پس \ff با محیط هم‌دما می‌شود.
	\\
	\textbf{ نتیجه:«آهنگ» تغییر دمای مواد مختلف (مبادله \ff با \ff) یکسان \ff .}
\end{en}
	\lin
نکته: هنگام مبادله گرما بین دو «ماده»؛ (اگر از هدر رفت یا اتلاف گرما چشم‌پوشی کنیم) مقدارگرمایی که ماده با دمای \ff است می‌دهد،		$|Q_A| = |Q_B|$
برابر با مقدار گرمایی است که ماده با دمای \ff می‌گیرد.
	\\
یعنی قدر مطلق \ff مبادله شده در آن دو، \ff است.
	\lin
	\textbf{تمرین ۱:}
	\\
 جسم A به جرم g ۱۰۰ و دمای 100 درجه سانتی‌گراد را در تماس با جسم B به جرم g ۲۰۰  و دمای ۲۰۰ درجه سانتی‌گراد قرار می‌دهیم تا «هم دما» شوند. A و B در چه دمایی، هم‌دما می‌شوند؟ (بر حسب درجه سانتی‌گراد) (المپیاد شیمی ۸۶)
	\begin{answers}(4)
		\a 180
		\a 160
		\a 150
		\a 145
	\end{answers}
راه اول:
	\begin{flushleft}
		$\mathrm{|Q_A|=|Q_B|\rightarrow}$
	\end{flushleft}
	\lin
راه دوم (هنگام تغییر فاز قابل استفاده نیست.)
	\begin{flushleft}
		$\mathrm{\theta_{تعادلی} = \frac{m_1 C_1 \theta _1 + m_2 C_2 \theta _2 }{m_1 C_1 + m_2 C_2} = \frac{\qquad\qquad\qquad\qquad}{\qquad\qquad\qquad\qquad} = \frac{\sum{(mc\theta)}}{\sum{mc}}}$
	\end{flushleft}
	\lin
	\textbf{تمرین ۲:}
به آلیاژی از تیتانیم و نیکل به جرم ۴.۲ گرم، مقدار ۲۱ ژول گرما دادیم و دمای آن C\degree10 افزایش یافت. به تقریب، چند درصد جرم این آلیاژ را نیکل تشکیل داده است؟ $C_{Ni}=0.45(J.g^{-1}.\degree C^{-1})$ $C_{Ti}=0.5(J.g^{-1}.\degree C^{-1})$
	\begin{answers}(4)
		\a 37/6
		\a 49/2
		\a 28/6
		\a 71/5
	\end{answers}

	\newpage
	
	%page 5
	
	\begin{center}
		\textbf{جاری شدن انرژی گرمایی}
		\\
		«بررسی کیفی و کمی انرژی مبادله شده بین سامانه و محیط»
	\end{center}
	\textbf{سامانه:}
بخشی از جهان، که ـــــ ـــــ را در آن بررسی می‌کنیم.
	\\
	\textbf{محیط:}
هرچه ـــــ سامانه وجود دارد.\\
مثال: بررسی مبادله گرما بین یک لیوان آب و محیط:
\\
( معمولاً سامانه با مرز‌های مشخصی از محیط جدا می‌شود. )
\\
\textbf{فرآیند جاری شدن انرژی:}
\\
\\
\\
\textbf{تمرین:}
مبادلات انرژی را هنگام مصرف بستنی با دمای ۰ درجه سانتی گراد تا هضم آن را بررسی کنید.
\\
\\
\\
\begin{multicols}{2}
	\begin{center}
		\textbf{فرآیند گرماده}
	\end{center}
	\vfill\null
	در شرایط هم‌دما ($ \Delta\theta = 0$)\\
	جاری شدن انرژی از \ff به \ff واکنش یا فرآیند، برای انجام شدن، گرما می \ff .
	\begin{center}
		سطح انرژی طرف دوم $\bigcirc$ سطح انرژی طرف اول\\
		Q $\bigcirc$ ۰
	\end{center}
	نماد Q در طرف \ff نوشته می‌شود:\\
	1-
	واکنش گرماده:
	$\ce{H2 + Cl2 -> 2HCl}$\\
	2-
	فرآیند گرماده:
	$\ce{H2O(l) -> H2O(s)}$\\
	\ff سطح انرژی سامانه
	\vfill\null
	\columnbreak
	\begin{center}
		\textbf{فرآیند گرماگیر}
	\end{center}
	\vfill\null
	در شرایط هم‌دما $( \Delta\theta = 0 )$
	\\
	جاری شدن انرژی از \ff به \ff واکنش یا فرآیند، برای انجام شدن، گرما می ‌\ff .
	\begin{center}
			سطح انرژی طرف دوم	 $\bigcirc$ سطح انرژی طرف اول
			\\
			Q$\bigcirc$۰
	\end{center}

نماد Q در طرف \ff نوشته می‌شود:
\\
1-
واکنش گرما‌گیر: 
$\ce{N2O4(g) -> 2NO2(g)}$
\\
2-
فرآیند گرماگیر:
$\ce{H2O(s) -> H2O(l)}$
\ff سطح انرژی سامانه
	\vfill\null
\end{multicols}

\newpage

%page 6 

\begin{center}
	\textbf{گرما در واکنش‌های شیمیایی }
	(گرماشیمی)
\end{center}
هر واکنش شیمیایی، ممکن است با تغییر \ff ، تولید \ff ، آزاد شدن \ff و ایجاد \ff و \ff همراه باشد،
اما:	داد و ستد \ff ، یک ویژگی بنیادی واکنش‌های شیمیایی است.
\\
ترموشیمی (گرماشیمی) به بررسی \ff و \ff گرمای واکنش‌های شیمیایی، \ff آن و تأثیری که بر \ff ماده دارد، می‌پردازد.
\lin
\textbf{بررسی شکل ۳ صفحه ۶۰:}
\\
الف) مواد غذایی، پس از گوارش، انرژی لازم برای \ff و \ff یاختهها را تأمین می‌کنند.
\\
ب) \ff سوخت‌ها، انرژی لازم برای حمل و نقل، و نیز گرمایش محیط‌های گوناگون را فراهم می‌کند.
\\
پ) زغال کک، واکنش‌دهنده‌ای رایج در استخراج آهن، و تامین‌کننده \ff لازم برای واکنش است.
\lin
منبع انرژی در بدن، \ff است. انرژی غذا، پس از انجام واکنش‌های شیمیایی گوناگون، به سلول‌ها می‌رسد. این واکنش‌ها ممکن است گرماده یا گرماگیر باشند اما فرآیند کلی اکسایش گلوکز در مجموع، گرما \ff است. البته دمای بدن تغییر محسوسی \ff
\\
دلیل: دمای واکنش‌دهنده‌ها با دمای فرآورده‌ها \ff است $(\Delta\theta		0)$
\\
\\
در‌واقع، انرژی آزاد شده در این واکنش، ناشی از تفاوت دمای مواد واکنش‌دهنده و فرآورده \ff ، بلکه تفاوت میان انرژی \ff مواد و واکنش‌دهنده و فرآورده است.
\\
انرژی پتانسیل در اینجا، به معنای انرژی ناشی از نیرو‌های \ff \ff ذرات سازنده آن است.
\\
\textbf{انرژی پتانسیل موحوددر یک نمونه ماده، انرژی \ff نام دارد.}
\\
انرژی پتانسیل در پیوند‌های مختلف، با هم \ff است، چون اتم‌های مختلفی با هم پیوند دارند. مثال:
\\
تفاوت اتم‌های دارای پیوند اشتراکی، موجب تفاوت در نیرو‌های \ff ( این نیرو‌ها، شامل «پیوند‌ها» و «نیرو‌های بین مولکولی» است. ) این نیرو‌ها، شامل «پیوند‌ها» و «نیرو‌های بین مولکولی» است.اتم‌ها ( در مولکول ) و در نتیجه؛ تفاوت در \ff پیوند‌ها است.
\\
\\
انجام واکنش شیمیایی، موجب تغییر در پیوند‌ها یا شیوه اتصال اتم‌ها با یکدیگر، و تفاوت آشکاری در انرژی \ff وابسته به آن‌ها می‌شود؛ که خود را به صورت \ff (ی مبادله‌شده) نشان می‌دهد.
\lin
\textbf{با هم بیندیشیم صفحه ۶۱	: در دو واکنش:}

\begin{en}
	
	\item  الف) واکنش‌دهنده‌ها یکسان هستند/نیستند $\leftarrow$ سطح انرژی واکنش‌دهنده‌ها یکسان \ff
	\\
 فرآورده، یکسان \ff $\leftarrow$ سطح انرژی فرآورده در دو واکنش یکسان \ff
 	\\
 	ب) در واکنش اول/دوم، سطح انرژی واکنش‌دهنده‌ها \ff $\leftarrow$ پایدارتر
 	\item  الف) چون سطح انرژی گرافیت و الماس، یکسان \ff . ( به دلیل تفاوت در نیرو‌های نگهداری )
 	\\
 	 ب) \ff پایدار‌تر است، چون فاصله کم‌تری با فرآورده دارد، گرمای سوختنی \ff دارد.
 	\\
	\textbf{	نحوه اتصال }
	اتم‌های کربن، 
	\textbf{تعداد و نوع }
	پیوند‌های اشتراکی کربن – کربن، در این دو آلوتروپ، و در نتیجه، رفتار شیمیایی آن‌ها ( مانند پایداری یا آنتالپی سوختن) متفاوت است.
	\\
	۲- پ) 
	%i hate my self
	\qquad\qquad\qquad\qquad
 	$xKj = \qquad g \times \frac{\qquad mol}{\qquad g}\times\frac{\qquad KJ}{\qquad mol}=\ff KJ$
 	
\end{en}

\newpage

%page 7

\begin{center}
	\textbf{یخچال صحرایی!}
\end{center}
دو ظرف از جنس \fsm داریم که فضای بین آن‌ها از شن خیس پر می‌شود. پارچه‌ای \fsm به عنوان درپوش، تحویه را انجام می‌دهد. آب درون ظرف درونی، به تدریج در بدنه ظرف بیرونی نفوذ می‌کند و \fsm می‌شود:
$H_2O(\quad) + Q \rightarrow H_2O(\quad)$
\\
این فرآیند، گرما \ff است و گرمای لازم را از سامانه دریافت می‌کند که باعث افت دما و خنک شدن محتویات دستگاه می‌شود.
\lin
\begin{center}
	فرآیند‌های تغییر حالت مواد
\end{center}
قث
\\
\\
\\
\\
\\
\\
\\
\\
هخثقهخ
\lin
\begin{center}
	\textbf{عوامل مؤثر بر گرمای واکنش: (یک عامل ثابت، و سه عامل متغیر)}
\end{center}
\begin{en}
	\item \ff مواد واکنش (واکنش‌دهنده‌های و فرآورده‌ها):
	مواد مختلف،‌ سطوح انرژی متفاوت دارند. گرمای واکنش، \ff سطح انرژی مواد طرف اول و دوم واکنش است. این عامل، متغیر \ff ، چون با تغییر مواد، در واقع، واکنش دیگری داریم.
	\item  \ff و \ff :
	تغییر این دو عامل، سطح \ff واکنش‌دهنده‌ها یا فرآورده‌ها را تغییر می‌دهد.
	\item \ff واکنش‌دهند‌ها:
	سطح انرژی هر ماده، به مقدار آن وابسته \ff و تغییر مقدار مواد، سطح انرژی آن را نیز تغییر می‌دهد.
\end{en}
\lin
\textbf{تمرین:}
 سوختن هر مول متان، .KJ89 انرژی آراد می‌کند. با سوختن ۱ گرم متان، چند کالُری گرما تولید می‌شود؟
\\
\\
\\
\lin
%i hate my self again
4.
 \ff \ff مواد واکنش:
در معادله «ترموشیمیایی»، باید انرژی \ff \ff در واکنش ذکر شود. حال اگر 

حالت فیزیک یکی از مواد در واکنش تغییر کند، سطح \ff آن نیز تغییر می‌کند و در نهایت، گرمای واکنش را تغییر

 می‌دهد.
\begin{flushleft}
	( در دمای \ff ) $CH_4(g) + 2O2(g) \rightarrow CO_2(g) + 2H_2O(g) + Q_1$ I)	
	\\	
		( در دمای \ff ) $CH_4(g) + 2O_2(g) \rightarrow CO_2(g) + 2H_2O(l) + Q_2$II)			
\end{flushleft}

$H_2O$ تولید شده در واکنش سوختن متان، ابتدا در دمای شعله است و حالت فیزیکی گازی دارد، اگر مقداری صبر کنیم تا سامانه با محیط، « \ff \ff » شود، $H_2O$ به حالت مایع در می‌آید.
این فرآیند (تبخیر/میعان)، خود، گرما \ff است و در رسیدن از I به II مقداری گرما \ff می‌شود. یعنی $Q_2$، از لحاظ عددی، از $Q_1$ \ff است.
\lin

\textbf{تمرین) }
 گرمای تبخیر مولی آب را برحسب $Q_1$ و $Q_2$ به دست آورید:
\begin{flushleft}
	0
	$\bigcirc$ 
	$\frac{\qquad\qquad}{\qquad\qquad}$
	=	گرمای تبخیر مولی  
\end{flushleft}
\lin
\textbf{با هم بیندیشیم ۳ صفحه ۶۲:}
\\
اولاً: میعان، گرما \ff است، پس گرمای واکنش با عدد +/- گزارش می‌شود.
\\
ثانیا: گرمای آزاد شده در میعان و نیز گرمای واکنش هردو، علامت 	دارند و مجموع آن‌ها با علامت 	باید از نظر عددی از ۴۸۴ \ff باشد ( یعنی عدد \ff )
\lin
\textbf{پرسش:}
\\
گرمای آزاد شده در کدام حالت، مقدار عددی بیشتری دارد؟ (روش: باید یک طرف کمترین و طرف دیگر بیشترین سطح انرژی را داشته باشد)
\begin{answers}(4)
	\a $2O(l) \rightarrow O_2(g)$
	\a $2O(g) \rightarrow O_2(l)$
	\a $2O(g) \rightarrow D_2(g)$
	\a $2O(l) \rightarrow O_2(l)$
\end{answers}
\lin
\begin{center}
	«آنتالپی ,(H) همان محتوای انرژی است»
\end{center}
هر نمونه ماده، دارای شمار بسیار زیادی «ذره سازنده» است. این ذره‌ها، دارای:
\\
۱- \ff نامنظم (انرژی \ff ) و ۲- \ff با یکدیگر (انرژی \ff) هستند
\\
\textbf{یک نمونه ماده، با \ff آن در \ff و \ff معین، توصیف می‌شود.}
 مانند ۲۰۰ گرم آب در دما و فشار معین یک نمونه ماده در یک ظرف، می‌تواند یک \ff به شمار آید.
\\
«انرژی کل» یک سامانه، هم ارز «محتوای \ff» یا «\ff» آن سامانه است.
یعنی: همه مواد، در دما و قشار معین، «\ff» مشخصی دارند.
\\
با انجام واکنش شیمیایی، «محتوای \ff» یا «\ff» مواد، تغییر می‌کند. (مانند نمودار ۵ صفحه ۶۴)
\\
\textbf{مهم:}
$Q_p$ 
= \ff H - \ff H = 
واکنش
$\Delta H$
$\leftarrow$
(\ff آنتالپی )
\\
$Q_p$
به معنای \ff مبادله شده در « \ff \ff » است.
\\
مقدار عددی $\Delta H$ در یک فرآیند، \ff آن را نشان می‌دهد، اما علامت + یا -، به ترتیب، \ff \ff و \ff \ff بودن آن را نشان می‌دهد.
\\
\textbf{خود را بیازمایید صفحه ۶۴ و ۶۵: 	}
\begin{en}
	\item الف) $CO_2(s)\qquad \rightarrow CO_2(g)\qquad, \Delta H \bigcirc 0$
	\\
	ب) $CH_4(g) + 2O_2(g) \rightarrow CO_2(g) + 2H_2O(g)\qquad, \Delta H \bigcirc 0$
	\\
	پ) $N_2O_4(g)\qquad \rightarrow 2NO_2(g)\qquad, \Delta H \bigcirc 0$
	\\
	ت) $N_2O(l) \rightarrow H_2O(s)\qquad, \Delta H \bigcirc 0$
	\item $3O_2(g) + \ff \leftrightarrow 2O_3(g)$
	\\
	$x(KJ) = \frac{\ff KJ}{\ff mol O_3}\times \ff mol O_3 = \ff (KJ)$
	\\
	$(\Delta H رفت = \qquad) (\Delta H برگشت = \qquad)$
\end{en}
\newpage

%page 9 

\begin{center}
	\textbf{«آنتالپی پیوند» و «میانگین آنتالپی پیوند»}
\end{center}

انجام یک واکنش شیمیایی، نشانه‌ای از تغییر در \ff \ff اتم‌ها (ذرات) به یکدیگر است، که نتیجه آن، تغییر \ff و به دنبالش تغییر \ff مواد است. یکی از خواصی که در واکنش‌های شیمیایی تغییر می‌کند، محتوای \ff مواد است. مثلاً، یک نمونه گاز هیدروژن، دارای شمار بسیار زیادی \ff دو اتمی است. با صرف \ff ، پیوند \ff بین اتم‌ها در مولکول می‌شکند و به \ff هایی تبدیل می‌شود که \ff تر و \ff \ff تر هستند. در ترموشیمی، به مقدار KJ436, آنتالپی \ff می‌گویند:
( $KJ.mol^{-1}$ )436 $\bigcirc$  = ( \ff ) H $\Delta$
\\
\\
\textbf{آنتالپی پیوند:}
انرژی لازم برای \ff ۱ \ff پیوند در مولکول \ff و تبدیل آن به اتم‌های \ff
\\
در مولکول‌هایی که «اتم مرکزی» به چند اتم یکسان با پیوند اشتراکی متصل است، (مانند $CH_4$) این پیوند‌های یکسان، آنتالپی کاملاً یکسان \ff ! در این حالت، به کار بردن اصطلاح * \ff آنتالپی پیوند، مناسب‌تر است.
\begin{flushleft}
	$CH_4(g) + 1660 KJ \rightarrow \ff (\quad) + \ff (\quad)$
	\\
	$\Delta H_{(C-H)} = \qquad \div \qquad = \qquad (KJ.mol^{-1})$
\end{flushleft}
\lin
پرسش) در چند مورد، به کار بردن میانگین آنتالپی پیوند، مناسب‌تر است؟ \ff مورد
\begin{answers}(4)
	\a $\mathrm{NH_3(g)}$
	\a ds
	\a $\mathrm{H-Br(g)}$
	\a $\mathrm{H_2O(g)}$
\end{answers}
\lin
\textbf{خود را بیازمایید صفحه ۶۶:}
\\
الف) (پیوند \ff شده $\leftarrow$ گرما \ff ) \qquad \qquad H$\Delta$ | پیوند‌ها در جدول ۲ صفحه ۶۵ مربوط به مولکول ۲ اتمی (میانگین هست/نیست.)
\\
ب) (پیوند \ff شده $\leftarrow$ گرما \ff )\qquad\qquad H $\Delta$    | پیوند‌ها در جدول ۳ صفحه ۶۶ مربوط به مولکول‌های چند اتمی ( میانگین \ff )
\\
تذکر: برای گزارش آنتالپی پیوند، همه ذرات در دو طرف واکنش به حالت \ff و همه فرآورده‌ها باید \ff باشند:(اگر قرار است همه پیوند‌ها شکسته شود.)  \qquad \qquad \qquad \qquad
$NH3(\quad) + Q \rightarrow \ff (\quad) + \ff (\quad)$
\lin
\begin{center}
	\textbf{«آنتالپی پیوند، راهی برای تعیین H $\Delta$ واکنش»}
\end{center}

۱) روش محاسباتی برای تعیین H $\Delta$واکنش:
\\
در واکنش شیمیایی، «معمولا» تعدادی پیوند \ff و تعدادی پیوند جدید \ff می‌شود.
\\
برای «شکستن» پیوند،‌ مقداری انرژی \ff می‌شود ( با علامت $\bigcirc$ گزارش می‌شود).
\\
هنگام «تشکیل» پیوند،‌ مقداری انرژی \ff می‌شود ( با علامت $\bigcirc$ گزارش می‌شود).(H $\Delta$واکنش، \ff این انرژی‌های \ff شده است.)
\\
استفاده از آنتالپی پیوند، برای تعیین H $\Delta$واکنش‌های \ff مناسب‌تر است. ( همه مواد در حالت \ff )
\\
هرچه مواد واکنش‌، مولکول‌های \ff داشته باشند، H $\Delta$محاسبه شده، با داده‌های \ff همخوانی بیشتری دارد، و هرچه مولکول‌ها پیچیده‌تر باشند، H $\Delta$ محاسبه شده با داده‌های \ff تفاوت‌های آشکار نشان می‌دهد.
\\
۲) استفاده از «آنتالپی پیوند» برای تعیین H $\Delta$ واکنش: (خود را بیازمایید ۱ صفحه ۶۷)
\\
H $\Delta$ واکنش: [مجموع آنتالپی‌های پیوند\fb]-[مجموع آنتالپی‌های پیوند\fb]

\newpage

% page 10

نکته: در جدول آنتالپی پیوند، همه اعداد علامت 	$\bigcirc$دارند و علامت $\bigcirc$	پیش از آنتالپی پیوند فرآورده‌ها، برای آن است که 	$\bigcirc$در $\bigcirc$	، $\bigcirc$	شود. ( چون در فرآورده‌ها، پیوند‌ها در حال تشکیل هستند که فرآیندی گرماده است و باید با عدد منفی نوشته شود. )



\end{document}

UNFINSH THINGS:
1- CHARTS
2- CHOICES BETWEEIN 2 OPTIONS
3- page 9 choices



