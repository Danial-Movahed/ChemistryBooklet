\documentclass[a4paper,12pt]{article}
%\usepackage[top=1in,margin=5em]{geometry}
\usepackage[top=7em,bottom=1pt,right=0.8in,left=0.8in,headheight=65pt,headsep=1cm]{geometry}
\usepackage{tikz-page}
\usetikzlibrary{shadows.blur}
\usepackage{enumitem}
\usepackage{amsmath}
\usepackage{tasks}
\usepackage{xepersian}
\usepackage{setspace}
\usepackage{textcomp, gensymb}

\pagestyle{plain}
\tikzset{
	secnode/.style={
		minimum height=1cm,
		inner xsep=20pt,
		rotate=90,
		anchor=north east,
		draw=white,
		fill=olive,
		text=white,
		blur shadow={shadow blur steps=5,shadow blur extra rounding=1.3pt}},
	pagenode/.style={
		minimum width=5mm,
		minimum height=1cm,
		inner sep=2pt,
		anchor=south east,
		draw=white,
		fill=olive,
		text=white,
		blur shadow={shadow blur steps=5,shadow blur extra rounding=1.3pt}}
}
\newcommand{\tikzpagelayout}{
	\draw[olive,line width=2pt,rounded corners=20pt] ([xshift=10mm]page.northwest) |- ([xshift=-2cm,yshift=10mm]page.southeast);
	\node[secnode] at ([xshift=5mm]page.northwest) {سالم غذا‌های پی در | شکیباییان};
	\node[pagenode] at ([xshift=-1cm,yshift=5mm]page.southeast) {\thepage};

}

\newenvironment{en}
	{\begin{enumerate}\setlength\itemsep{-0.2em}}
	{\end{enumerate}}
	

\newenvironment{iit}
	{\begin{itemize}\setlength\itemsep{-0.5em}}
	{\end{itemize}}
	
\newcommand{\ff}{\rule{1cm}{0.15mm}\;}
\newcommand{\fb}{\rule{2cm}{0.15mm}\;}
\newcommand{\fs}{\rule{1cm}{0.15mm}}
\newcommand{\fsm}{{\rule{0.5cm}{0.15mm}\;}}
\newcommand{\lin}{\vspace{4pt}\hrule\vspace{4pt}}
\newcommand\gototask[1]{\addtocounter{task}{\numexpr#1-\value{task}\relax}}
\NewTasksEnvironment[label=\arabic*.,label-format=\bfseries,label-width=4ex]{answers}[\a]

\settextfont{XB Niloofar}
\setdigitfont{XB Niloofar}
\setstretch{1.5}
\setlist[itemize]{topsep=0pt}
\setlist[enumerate]{topsep=0pt}
\renewcommand{\headrulewidth}{0pt}
\setlength{\headheight}{1pt}

\begin{document}
	%page 1

	\ff و \ff ،‌ اجزاء بنیادی جهان مادی هستند. انرژی از راه‌های گوناگون با ماده ارتباط دارد، چنانکه کاهش \ff خورشید موجب تولید \ff می‌شود. «غذا» همواره نقش محوری در رشد، تندرسی و زندگی انسان داشته است. پیشرفت دانش و فناوری، موجب افرایش تولید فرآورده‌های کشاورزی و دامی و تولید صنعتی غذا شده است. در تولید انبوه، به دلیل فساد مواد غذایی و دشواری نگهداری، حفظ کیفیت و ارزش مواد غذایی، اهمیت به‌سزایی دارد. همچنین در صنایع غذایی، حجم عظیمی «آب» مصرف می‌شود و تأمین غذای جامعه را مشکل‌تر می‌کند.
	\vspace{4pt}
		\hrule
	\vspace{12pt}

خود را بیازمایید صفحه ۵۱؛\\
الف) \ff و دردرجه دوم 
\ff 
 و  
\ff.
\\
ب) با حذف خوراکی‌های غیر ضروری (مانند چیپس، پفک، نوشابه) تاحدی امکان تأمین هزینه مصرف انواع \ff در سبد خانوار تأمین می‌شود. (!!)\\
پ)
\begin{iit}
	\item توزیع شیر رایگان در مدارس، مهدکودک‌ها، پادگان‌ها و دانشگاه‌ها
 
 	\item دادن علوفه و داروی دامی با قیمت ارزان به دامدار 
 
 	\item فرهنگ‌سازی مصرف
\end{iit}
ت) فرهنگ‌سازی استفاده بیشتر از حبوبات (مصرف عدسی یا آش در وعده صبحانه یا عصرانه)، مصرف انواع حبوبات در سالاد\\
\textbf {سرانه مصرف ماده غذایی، مقدار میانگین مصرف آن را به ازای هر فرد در یک گستره زمانی نشان می‌دهد.}
\vspace{4pt}
\hrule
\begin{center}
	\textbf{غذا، چیزی فراتر از یک پاسخ به احساس گرسنگی است. مصرف غذا؛}
\end{center}
\begin{en}
	\item مورد نیاز برای ماهیچه‌ها، ارسال پیام‌های عصبی، جابه‌جایی یون‌ها و مولکول‌ها از دیواره هر یاخته را تأمین می‌کند.
	\item \ff اولیه برای ساخت و رشد بخش‌های مختلف بدن را فراهم می‌کند. (بخش عمده \ff ، \ff  و \fs ـی موجود در بدن از غذا تأمین می‌شود.) این فرآیند‌ها وابسته به انجام واکنش‌های شیمیایی هستند، که دمای بدن را نیز تنظیم و کنترل می‌کنند. هر کدام از این واکنش‌ها، «آهنگ» ویژه‌ای دارند.
\end{en}

تغذیه درست، شامل وعده‌های غذایی است که مخلوط منابع از انواع ذره‌ها را در بر می‌گیرد. سوء تغذیه هنگامی رخ می‌نماید که وعده‌های غذایی با کمبود نوع خاصی از این ذرات همراه باشد. از طرفی، افزایش نامناسب برخی مولکول‌ها و یون‌ها در غذا نیز، سبب بیماری خواهد شد.
\vspace{4pt}
\hrule
\begin{center}
	\textbf{«غذا، ماده و انرژی»}
\end{center}
بدن برای انجام فعالیت‌های ارادی و غیرارادی، به ماده و انرژی نیاز دارد. یکی از راه‌های آزاد شدن انرژی سوخت‌ها (مانند بنزین و …) «سوزاندن» آن‌ها است. هر ماده غذایی نیز انرژی دارد و میزان انرژی به «جرم» آن بستگی دارد.
\newpage

%page 2

\begin{center}
	\textbf{دمای یک ماده، از چه خبر می‌دهد؟\\
		دما: کمیتی که میزان \ff و \ff اجسام را نشان می‌دهد.}
\end{center}
شکل ۱ صفحه ۵۴: وقتی به ظرف محتوی آب، گرما داده می‌شود، به تدریج \ff آن افرایش می‌یابد تا اینکه سرانجام \ff یا اگر به یخ داده شود، \ff می‌شود. در این حالت‌ها، با گرفتن گرما، \ff ذرات بیشتر شده و دما \ff می‌رود یا \ff ماده عوض می‌شود.
\begin{center}
جنبش نامنظم ذره‌ها: گاز $\bigcirc$ مایع $\bigcirc$ جامد / آب‌ گرم $\bigcirc$ آب سرد
\end{center}
دمای بالاتر $\leftarrow$ میانگین \ff حرکت ذرات بیشتر $\leftarrow$ میانگین انرژی \ff ذرات بیشتر.\\
\textbf{یعنی:}
\underline{دمای ماده}
؛ معیاری برای توصیف \ff تندی و \ff انرژی جنبشی ذره‌های سازنده ماده است.\\
یکای رایج دما، درجه \ff (\qquad) اما یکای دما در ،SI \ff (\qquad) است.\\
ارزش دمایی ۱ درجه سانتی‌گراد برابر ۱ کلوین \ff .\\
لذا در فرآیند‌هایی که دما تغییر می‌کند،
$\mathrm{\Delta\theta\bigcirc\Delta T}$
است.
\qquad\qquad\qquad \ff = \ff + \ff \\
\textbf{با هم بیندیشیم صفحه ۵۵:}
\begin{en}
 \item 
الف) شکل A نمونه‌ای از هوا را در \ff نشان می‌دهد.\\
ب) شکل B، نمونه‌ای از هوا را در یک روز \ff نشان می‌دهد.\\
پ) اگر مجموع انرژی جنبشی ذره‌های سازنده یک نمونه ماده، هم‌ارز با انرژی گرمایی آن باشد؛
انرژی گرمایی \ff بیشتر بوده زیرا \ff آن بیشتر است.
\item 
الف) میانگین تندی مولکول‌ها در ظرف A$\bigcirc$ 	ظرف B\\
ب) انرژی گرمایی ظرف A $\bigcirc$ظرف B (چون \ff \ff آن بیشتر است.)
\end{en}
با هم بیندیشیم ۱:  \ff یکسان، دمای \ff متفاوت $\leftarrow$ انرژی گرمایی متفاوت\\
با هم بیندیشیم ۲:	\ff یکسان، \ff \ff متفاوت $\leftarrow$ انرژی گرمایی متفاوت\\
\textbf{نتیجه:}
 انرژی گرمایی یک نمونه ماده، هم به \ff و هم به \ff \ff بستگی دارد.\\
 \textbf{تذکر:}
  چون کار کردن «تعداد ذرات»، آسان نیست می‌توان به جای آن، \ff ماده را در نظر گرفت. چنانکه در فیزیک نیز، انرژی جنبشی از رابطه \fb به دست می‌آید.
  \begin{center}
  	\textbf{تهیه غذا آب‌پز، تجربه تفاوت «گرما» و «دما»}
  \end{center}
گرما، صورتی از \ff و یکای آن در ،SI \ff (\ff) است.
 ($\mathrm{\qquad 1 \qquad = 1 Kgm^2.s^{-2}}$)\\
 از یکای \ff (\ff) نیز برای بیان مقدار گرما در پزشکی و زیست‌شناسی و علم تغزیه استفاده می‌شود.\\
 \hrule
 \vspace{4pt}
 \textbf{تعریف ژول:}\\
 \textbf{تعریف کالری:}
 \begin{flushleft}
 	 $\mathrm{\ff cal = \ff J}$ 
 \end{flushleft}
 \vspace{4pt}
\hrule
\vspace{4pt}
 انرژی گرمایی: \ff انرژی‌های جنبشی ذرات ماده / دما: \ff انرژی جنبشی ذرات ماده\\
 انرژی گرمایی و دما، از ویژگی‌های یک «نمونه ماده» 
$\frac{است}{نیست}$
 و 
 $\frac{می‌تواند}{نمی‌تواند}$
 برای توصیف آن «ماده» به کار رود.

\newpage

%page 3

\begin{center}
	\textbf{«گرما»}
\end{center}
صورتی از \ff است، که از جسم با \ff بالاتر، به جسم با \fsm پایین‌تر منتقل می‌شود. داد و ستد گرما، می‌‌تواند موجب تغییر \ff مواد شود.
\\
گرما، از ویژگی‌های یک «نمونه ماده» \ff و \ff برای توصیف آن «ماده» به کار رود.
\\
هنگامی که به ۲ ماده، گرمای یکسان داده شود، لزوماً به یک اندازه \ff نمی‌شوند.
\lin
هنگامی که به ۲ ماده، گرمای یکسان داده شود، لزوماً به یک اندازه \ff نمی‌شوند.
\\
\textbf{یعنی:}
 دادن گرمای یکسان به دو ماده، لزوما/حتما تغییر دمای یکسانی را موجب می‌شود/نمی‌شود.
مثال: اگر بخواهیم دمای آب و روغن زیتون* (با جرم برابر) به یک اندازه بالا رود، باید به آب، گرمای \ff بدهیم.
\lin
* الگوی ساختاری «روغن‌ها» با «چربی‌ها» یکسان است اما تفاوت‌هایی در ساختار دارند ( مانند پیوند دوگانه بیشتر در ساختار زنجیر کربنی \ff ) که موجب تفاوت در \ff و \ff آن‌ها می‌شود. چنان که روغن‌ها در دمای عادی، \ff و چربی‌ها \ff هستند.
\lin
\textbf{با هم بیندیشیم صفحه ۵۷:}
\\
الف) چون \ff \ff موجود در نمونه آب، بسیار \ff از روغن زیتون است.
	دلیل: موادی چون آب و اتانول، به دلیل وجود \ff \ff بین مولکول‌های خود، گرمای ویژه بالایی دارند*. (جدول ۱ صفحه ۵۸).
	دمای آب و روغن زیتون، به یک اندازه زیاد \ff است. برای افزایش دمای آب به میزان ۵۰ درجه سانتی‌گراد، (نسبت به روغن زیتون) گرمای \ff جذب شده، پس انرژی گرمایی ظرف محتوی آب، \ff است و تخم مرغ، گرمای \ff دریافت می‌کند.
ب) ظرفیت گرمایی (C): \ff لازم برای افرایش \ff ماده به اندازه \ff درجه \ff ( یا ۱ \ff )
\\
	$C_{H_2O} = \frac{\quad J}{\quad K (\quad J.K^{-1})}\bigcirc C_{il.oil} = \frac{\quad J}{\quad k (\quad J.k^{-1})}$
	(یکای C: \ff . \ff )
	$Q = C \Delta\theta\rightarrow C = 
	\frac{Q}{\Delta\theta}\rightarrow$
\\
پ) 
	بستگی دارد به \ff ماده و \ff ماده (به خاطر تفاوت در نوع \ff یا نیرو‌های \ff \ff )
	 هرچه \ff ماده بیشتر باشد، برای رساندن آن به دمای مشخص، \ff بیشتری لازم است.
\\
ت) گرمای ویژه (c): ظرفیت گرمایی \ff \ff ماده 
\begin{flushleft}
	$Q = mc\Delta\theta\rightarrow c = \frac{Q}{m\Delta\theta}\downarrow$\\
	$C_{H_2O}=\frac{\quad}{\quad}=\ff ( \quad\quad )\quad C_{ol.oil}=\frac{\quad}{\quad}=\ff (\quad\quad)$
یکای c: ( \ff . \ff . \ff )

\end{flushleft}
ث) رابطه C با :c
\\
هر کمیتی که از ویژگی‌های ماده باشد، (میتواند/نمیتواند) برای توصیف آن به کار رود.
\\
ظرفیت گرمایی؛ از ویژگی‌های نمونه ماده \ff و می‌تواند/نمی‌تواند برای توصیف آن ماده به کار رود.
\\ 
گرمای ویژه؛ از ویژگی‌های یک نمونه ماده \ff و \ff برای توصیف آن ماده به کار می‌رود.

\newpage

%page 4

\textbf{خود را بیازمایید صفحه ۵۸:}
\begin{en}
	\item 
	 \ff می‌یابد. با‌گذشت زمان، چای، همه/بخشی از انرژی گرمایی خود را به‌/از محیط می‌دهد/می‌گیرد پس \ff و \ff انرژی جنبشی ذرات آن، \ff می‌یابد. (کاهش \ff \ff و \ff نمونه)
	دلیل: گرما، از جایی که \ff تر است (دمای \ff) به جایی که \ff است (دمای \ff) حرکت می‌کند. دمای چای ( ) از دمای محیط ( ) \ff است و با \ff انرژی گرمایی، با آن «\ff \ff» می‌شود.
	\item 
	گرما را می‌توان هم‌ارز با آن مقدار انرژی گرمایی/دمایی داشت که به دلیل تفاوت در انرژی گرمایی/دما جاری می‌شود.
	\item 
ماده اصلی تشکیل‌دهنده هر دو، \ff است، پس به مقدار \ff موجود در آن‌ها توجه می‌کنیم. نان، \ff کمتری دارد، چون \ff شده است، پس \ff با محیط هم‌دما می‌شود.
	\\
	\textbf{ نتیجه:«آهنگ» تغییر دمای مواد مختلف (مبادله \ff با \ff) یکسان \ff .}
\end{en}
\lin
نکته: هنگام مبادله گرما بین دو «ماده»؛ (اگر از هدر رفت یا اتلاف گرما چشم‌پوشی کنیم) مقدارگرمایی که ماده با دمای \ff است می‌دهد،		$|Q_A| = |Q_B|$
برابر با مقدار گرمایی است که ماده با دمای \ff می‌گیرد.
\\
یعنی قدر مطلق \ff مبادله شده در آن دو، \ff است.
\lin
\textbf{تمرین ۱:}
\\
 جسم A به جرم g ۱۰۰ و دمای 100 درجه سانتی‌گراد را در تماس با جسم B به جرم g ۲۰۰  و دمای ۲۰۰ درجه سانتی‌گراد قرار می‌دهیم تا «هم دما» شوند. A و B در چه دمایی، هم‌دما می‌شوند؟ (بر حسب درجه سانتی‌گراد) (المپیاد شیمی ۸۶)
\begin{answers}(4)
	\a 180
	\a 160
	\a 150
	\a 145
\end{answers}
راه اول:
\begin{flushleft}
	$\mathrm{|Q_A|=|Q_B|\rightarrow}$
\end{flushleft}
\lin
راه دوم (هنگام تغییر فاز قابل استفاده نیست.)
\begin{flushleft}
	$\mathrm{\theta_{تعادلی} = \frac{m_1 C_1 \theta _1 + m_2 C_2 \theta _2 }{m_1 C_1 + m_2 C_2} = \frac{\qquad\qquad\qquad\qquad}{\qquad\qquad\qquad\qquad} = \frac{\sum{(mc\theta)}}{\sum{mc}}}$
\end{flushleft}
\lin
\textbf{تمرین ۲:}
به آلیاژی از تیتانیم و نیکل به جرم ۴.۲ گرم، مقدار ۲۱ ژول گرما دادیم و دمای آن C\degree10 افزایش یافت. به تقریب، چند درصد جرم این آلیاژ را نیکل تشکیل داده است؟ $C_{Ni}=0.45(J.g^{-1}.\degree C^{-1})$ $C_{Ti}=0.5(J.g^{-1}.\degree C^{-1})$
\begin{answers}(4)
	\a 37/6
	\a 49/2
	\a 28/6
	\a 71/5
\end{answers}

\newpage

%page 5

\begin{center}
	\textbf{جاری شدن انرژی گرمایی}
	\\
	«بررسی کیفی و کمی انرژی مبادله شده بین سامانه و محیط»
\end{center}
\textbf{سامانه:}
بخشی از جهان، که ـــــ ـــــ را در آن بررسی می‌کنیم.
\\
\textbf{محیط:}
هرچه ـــــ سامانه وجود دارد.\\
\end{document}