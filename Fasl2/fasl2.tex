\documentclass[a4paper,12pt]{article}
%\usepackage[top=1in,margin=5em]{geometry}
\usepackage[top=7em,bottom=1pt,right=0.8in,left=0.8in,headheight=65pt,headsep=1cm]{geometry}
\usepackage{tikz-page}
\usepackage{tikz}
\usepackage{fancybox}
\usetikzlibrary{shadows.blur}
\usepackage{enumitem}
\usepackage{zref-abspage}
\usepackage{tabularray}
\usepackage{pbox}
\usepackage[version=4]{mhchem}
\usepackage{tasks}
\usepackage{multicol}
\usepackage{chemfig}
\usepackage{perpage}
\usepackage{setspace}
 \usepackage{vwcol}  
 \usepackage{mathtools}
 \usepackage{color}
 \usepackage{lscape}
 \usepackage{rotating}
\usepackage{textcomp, gensymb}
\usepackage{xepersian}



\pagestyle{plain}
\tikzset{
	secnode/.style={
		minimum height=1cm,
		inner xsep=20pt,
		rotate=90,
		anchor=north east,
		draw=white,
		fill=black,
		text=white,
		blur shadow={shadow blur steps=5,shadow blur extra rounding=1.3pt}},
	pagenode/.style={
		minimum width=5mm,
		minimum height=1cm,
		inner sep=2pt,
		anchor=south east,
		draw=white,
		fill=black,
		text=white,
		blur shadow={shadow blur steps=5,shadow blur extra rounding=1.3pt}}
}
\newcommand{\tikzpagelayout}{
	\draw[black,line width=2pt,rounded corners=20pt] ([xshift=10mm]page.northwest) |- ([xshift=-2cm,yshift=10mm]page.southeast);
	\node[secnode] at ([xshift=5mm]page.northwest) {سالم غذا‌های پی در | شکیباییان};
	\node[pagenode] at ([xshift=-1cm,yshift=5mm]page.southeast) {\thepage};
}

\definecolor{Silver}{rgb}{0.788,0.788,0.788}
\definecolor{Black}{rgb}{0,0,0}
\definecolor{Gallery}{rgb}{0.929,0.929,0.929}
\definecolor{Alto}{rgb}{0.858,0.858,0.858}

\newenvironment{en}
	{\begin{enumerate}\setlength\itemsep{-0.2em}}
	{\end{enumerate}}
	
\newenvironment{iit}
	{\begin{itemize}\setlength\itemsep{-0.5em}}
	{\end{itemize}}

\newcommand*\circled[1]{\tikz[baseline=(char.base)]{
		\node[shape=circle,draw,inner sep=2pt] (char) {#1};}}

\settextfont{XB Niloofar}
\setdigitfont{XB Niloofar}
\setstretch{1.5}
\setlist[itemize]{topsep=0pt}
\setlist[enumerate]{topsep=0pt}
\renewcommand{\headrulewidth}{0pt}
\setlength{\headheight}{14.5pt}
\setlength{\columnseprule}{1pt}
\setlength{\columnsep}{0.5cm}
\MakePerPage{footnote}

\newcommand{\fb}{\rule{2cm}{0.15mm}\;}
\newcommand{\fs}{\rule{1cm}{0.15mm}}
\newcommand{\fsm}{{\rule{0.5cm}{0.15mm}\;}}
\newcommand{\lin}{\vspace{4pt}\hrule\vspace{4pt}}
\newcommand\gototask[1]{\addtocounter{task}{\numexpr#1-\value{task}\relax}}
\NewTasksEnvironment[label=\arabic*.,label-format=\bfseries,label-width=4ex]{answers}[\a]
\def\extra{\rule{1ex}{0ex}}
\makeatletter
\newcommand\censor{\@ifstar{\@cenmath}{\@centext}}
\newcommand\@cenmath[1]{%
	\protect\rule[-.3ex]{\widthofpbox{\extra$#1$}}{0.1ex}}
\newcommand\@centext[1]{%
	\protect\rule[-.3ex]{\widthofpbox{\extra#1}}{0.1ex}}
\makeatother

\begin{document}
%page 1
\begin{center}
	\textbf{به نام‌ِ خدا}
\end{center}
\censor{ماده} و \censor{انرژی} ،‌ اجزاء بنیادی جهان مادی هستند. انرژی از راه‌های گوناگون با ماده ارتباط دارد، چنانکه کاهش \censor{جرم} خورشید موجب تولید \censor{انرژی} می‌شود. «غذا» همواره نقش محوری در رشد، تندرسی و زندگی انسان داشته است. پیشرفت دانش و فناوری، موجب افرایش تولید فرآورده‌های کشاورزی و دامی و تولید صنعتی غذا شده است. در تولید انبوه، به دلیل فساد مواد غذایی و دشواری نگهداری، حفظ کیفیت و ارزش مواد غذایی، اهمیت به‌سزایی دارد. همچنین در صنایع غذایی، حجم عظیمی «آب» مصرف می‌شود و تأمین غذای جامعه را مشکل‌تر می‌کند.
\lin
\vspace{8pt}
خود را بیازمایید صفحه ۵۳؛\\
الف) \censor{شکر} و دردرجه دوم
\censor{نان}
و
\censor{برنج}.
\\
ب) با حذف خوراکی‌های غیر ضروری (مانند چیپس، پفک، نوشابه) تاحدی امکان تأمین هزینه مصرف انواع \censor{گوشت} در سبد خانوار تأمین می‌شود. (!!)\\
پ)
\begin{iit}
	\item توزیع شیر رایگان در مدارس، مهدکودک‌ها، پادگان‌ها و دانشگاه‌ها

	\item دادن علوفه و داروی دامی با قیمت ارزان به دامدار

	\item فرهنگ‌سازی مصرف
\end{iit}
ت) فرهنگ‌سازی استفاده بیشتر از حبوبات (مصرف عدسی یا آش در وعده صبحانه یا عصرانه)، مصرف انواع حبوبات در سالاد\\
\textbf {سرانه مصرف ماده غذایی، مقدار میانگین مصرف آن را به ازای هر فرد در یک گستره زمانی نشان می‌دهد.}
\vspace{4pt}
\hrule
\begin{center}
	\textbf{غذا، چیزی فراتر از یک پاسخ به احساس گرسنگی است. مصرف غذا؛}
\end{center}
\begin{en}
	\item \censor{انرژی}مورد نیاز برای ماهیچه‌ها، ارسال پیام‌های عصبی، جابه‌جایی یون‌ها و مولکول‌ها از دیواره هر یاخته را تأمین می‌کند.
	\item \censor{مواد} اولیه برای ساخت و رشد بخش‌های مختلف بدن را فراهم می‌کند. (بخش عمده \censor{اتم} ، \censor{مولکول}  و \fs ـی موجود در بدن از غذا تأمین می‌شود.) این فرآیند‌ها وابسته به انجام واکنش‌های شیمیایی هستند، که دمای بدن را نیز تنظیم و کنترل می‌کنند. هر کدام از این واکنش‌ها، «آهنگ» ویژه‌ای دارند.
\end{en}

تغذیه درست، شامل وعده‌های غذایی است که مخلوط منابع از انواع ذره‌ها را در بر می‌گیرد. سوء تغذیه هنگامی رخ می‌نماید که وعده‌های غذایی با کمبود نوع خاصی از این ذرات همراه باشد. از طرفی، افزایش نامناسب برخی مولکول‌ها و یون‌ها در غذا نیز، سبب بیماری خواهد شد.
\vspace{4pt}
\hrule
\begin{center}
	\textbf{«غذا، ماده و انرژی»}
\end{center}
بدن برای انجام فعالیت‌های ارادی و غیرارادی، به ماده و انرژی نیاز دارد. یکی از راه‌های آزاد شدن انرژی سوخت‌ها (مانند بنزین و …) «سوزاندن» آن‌ها است. هر ماده غذایی نیز انرژی دارد و میزان انرژی به «جرم» آن بستگی دارد.
\newpage

%page 2

\begin{center}
	\textbf{دمای یک ماده، از چه خبر می‌دهد؟\\
		دما: کمیتی که میزان \censor{گرما} و \censor{سردی} اجسام را نشان می‌دهد.}
\end{center}
شکل ۱ صفحه ۵۶: وقتی به ظرف محتوی آب، گرما داده می‌شود، به تدریج \censor{دما} آن افرایش می‌یابد تا اینکه سرانجام \censor{تبخیر} یا اگر به یخ داده شود، \censor{ذوب} می‌شود. در این حالت‌ها، با گرفتن گرما، \censor{جمبش} ذرات بیشتر شده و دما \censor{بالا} می‌رود یا \censor{حالت} ماده عوض می‌شود.
\begin{center}
	جنبش نامنظم ذره‌ها: گاز $\bigcirc$ مایع $\bigcirc$ جامد / آب‌ گرم $\bigcirc$ آب سرد
\end{center}
\underline{دمای ماده}
؛ معیاری برای توصیف \censor{میانگین} تندی و \censor{میانگین} انرژی جنبشی ذره‌های سازنده ماده است.\\
یکای رایج دما، درجه \censor{سلسیوس} (\qquad) اما یکای دما در ،SI \censor{کلوین} (\qquad) است.\\
ارزش دمایی ۱ درجه سانتی‌گراد برابر ۱ کلوین \censor{است} .\\
لذا در فرآیند‌هایی که دما تغییر می‌کند،
$\mathrm{\Delta\theta\bigcirc\Delta T}$
است.
\qquad\qquad\qquad \censor{نمیدونم} = \censor{نمیدونم} + \censor{نمیدونم} \\
\textbf{با هم بیندیشیم صفحه ۵۷:}
\begin{en}
	\item
	الف) شکل A نمونه‌ای از هوا را در \censor{شب} نشان می‌دهد.\\
	ب) شکل B، نمونه‌ای از هوا را در یک روز \censor{تابستانی} نشان می‌دهد.\\
	پ) اگر مجموع انرژی جنبشی ذره‌های سازنده یک نمونه ماده، هم‌ارز با انرژی گرمایی آن باشد؛
	انرژی گرمایی \censor{B} بیشتر بوده زیرا \censor{دمای} آن بیشتر است.
	\item
	الف) میانگین تندی مولکول‌ها در ظرف A$\bigcirc$ 	ظرف B\\
	ب) انرژی گرمایی ظرف A $\bigcirc$ظرف B (چون \censor{تعداد} \censor{مولکول} آن بیشتر است.)
\end{en}
با هم بیندیشیم ۱:  \censor{ذرات} یکسان، دمای \censor{ذرات} متفاوت $\leftarrow$ انرژی گرمایی متفاوت\\
با هم بیندیشیم ۲:	\censor{دمای} یکسان، \censor{تعداد} \censor{ذرات} متفاوت $\leftarrow$ انرژی گرمایی متفاوت\\
\textbf{نتیجه:}
انرژی گرمایی یک نمونه ماده، هم به \censor{دما} و هم به \censor{تعداد} \censor{ذرات} بستگی دارد.\\
\textbf{تذکر:}
چون کار کردن «تعداد ذرات»، آسان نیست می‌توان به جای آن، \censor{جرم} ماده را در نظر گرفت. چنانکه در فیزیک نیز، انرژی جنبشی از رابطه \fb به دست می‌آید.
\begin{center}
	\textbf{تهیه غذا آب‌پز، تجربه تفاوت «گرما» و «دما»}
\end{center}
گرما، صورتی از \censor{انرژی} و یکای آن در ،SI \censor{ژول} (\censor{J}) است.
($\mathrm{ 1 \censor{JJ} = 1 Kgm^2.s^{-2}}$).
از یکای \censor{کالری} (\censor{cal}) نیز برای بیان مقدار گرما در پزشکی و زیست‌شناسی و علم تغزیه استفاده می‌شود.
\hrule
\vspace{4pt}
\textbf{تعریف ژول:}\\
\textbf{تعریف کالری:}
\begin{flushleft}
	$\mathrm{\censor{نمیدونم} cal = \censor{نمیدونم} J}$
\end{flushleft}
\vspace{4pt}
\hrule
\vspace{4pt}
انرژی گرمایی: \censor{مجموع} انرژی‌های جنبشی ذرات ماده / دما: \censor{میانگین} انرژی جنبشی ذرات ماده\\
انرژی گرمایی و دما، از ویژگی‌های یک «نمونه ماده»
$\dfrac{\text{است}}{\text{نیست}}$
و
$\dfrac{\text{می‌تواند}}{\text{نمی‌تواند}}$
برای توصیف آن «ماده» به کار رود.

\newpage

%page 3

\begin{center}
	\textbf{«گرما»}
\end{center}
صورتی از \censor{انرژی} است، که از جسم با \censor{دمای} بالاتر، به جسم با \censor{دمای}پایین‌تر منتقل می‌شود. داد و ستد گرما، می‌‌تواند موجب تغییر \censor{دما} مواد شود.
\\
گرما، از ویژگی‌های یک «نمونه ماده» \censor{نیست} و \censor{نمی‌توان} برای توصیف آن «ماده» به کار رود.
\\
گرما، از ویژگی‌های یک  \censor{فرآیند} است، و می‌تواند برای توصیف آن \censor{فرآیند} به کار رود.
\lin
هنگامی که به ۲ ماده، گرمای یکسان داده شود، لزوماً به یک اندازه \censor{داغ} نمی‌شوند.
\\
\textbf{یعنی:}
دادن گرمای یکسان به دو ماده، $\dfrac{\text{لزوماٌ}}{\text{حتماٌ}}$تغییر دمای یکسانی را موجب $\dfrac{\text{می‌شود}}{\text{نمی‌شود}}$.
مثال: اگر بخواهیم دمای آب و روغن زیتون* (با جرم برابر) به یک اندازه بالا رود، باید به آب، گرمای \censor{بیشتری} بدهیم.
\lin
* الگوی ساختاری «روغن‌ها» با «چربی‌ها» یکسان است اما تفاوت‌هایی در ساختار دارند ( مانند پیوند دوگانه بیشتر در ساختار زنجیر کربنی \censor{روغن} ) که موجب تفاوت در \censor{ساختار} و \censor{خواص} آن‌ها می‌شود. چنان که روغن‌ها در دمای عادی، \censor{مایع} و چربی‌ها \censor{جامد۱} هستند.
\lin
\textbf{با هم بیندیشیم صفحه ۵۹:}
\\
الف) چون \censor{انرژی} \censor{گرمایی} موجود در نمونه آب، بسیار \censor{بیشتر} از روغن زیتون است.
دلیل: موادی چون آب و اتانول، به دلیل وجود \censor{پیوند} \censor{هیدروژنی} بین مولکول‌های خود، گرمای ویژه بالایی دارند*. (جدول ۱ صفحه ۶۰).
دمای آب و روغن زیتون، به یک اندازه زیاد \censor{شده} است. برای افزایش دمای آب به میزان ۵۰ درجه سانتی‌گراد، (نسبت به روغن زیتون) گرمای \censor{بیشتری} جذب شده، پس انرژی گرمایی ظرف محتوی آب، \censor{بیشتری} است و تخم مرغ، گرمای \censor{بیشتری} دریافت می‌کند.
ب) ظرفیت گرمایی (C): \censor{گرما} لازم برای افرایش \censor{دمای} ماده به اندازه \censor{۱} درجه \censor{سلسیوس} ( یا ۱ \censor{کلوین} )
\\
$\mathrm{C_{H_2O} = \dfrac{\quad J}{\quad K (\quad J.K^{-1})}\bigcirc C_{il.oil} = \dfrac{\quad J}{\quad k (\quad J.k^{-1})}}$
(یکای C: \censor{نمی} . \censor{نمم} )
$\mathrm{Q = C \Delta\theta\rightarrow C =
	\dfrac{Q}{\Delta\theta}\rightarrow}$
\\
پ)
بستگی دارد به \censor{نوع} ماده و \censor{جرم} ماده (به خاطر تفاوت در نوع \censor{پیوند} یا نیرو‌های \censor{بین} \censor{مولکولی} )
هرچه \censor{جرم} ماده بیشتر باشد، برای رساندن آن به دمای مشخص، \censor{گرمای} بیشتری لازم است.
\\
ت) گرمای ویژه (c): ظرفیت گرمایی \censor{۱} \censor{گرم} ماده
\begin{flushleft}
	$\mathrm{Q = mc\Delta\theta\rightarrow c = \dfrac{Q}{m\Delta\theta}\downarrow}$\\
	$\mathrm{C_{H_2O}=\dfrac{\quad}{\quad}=\censor{نمیدونم} ( \quad\quad )\quad C_{ol.oil}=\dfrac{\quad}{\quad}=\censor{نمیدونم} (\quad\quad)}$
	یکای c: ( \censor{نمی} . \censor{نمم} . \censor{نمم} )

\end{flushleft}
ث) رابطه C با :c
\\
هر کمیتی که از ویژگی‌های ماده باشد،$\dfrac{\text{می‌تواند}}{\text{نمی‌تواند}}$ برای توصیف آن به کار رود.
\\
ظرفیت گرمایی؛ از ویژگی‌های نمونه ماده \censor{است} و $\dfrac{\text{می‌تواند}}{\text{نمی‌تواند}}$  برای توصیف آن ماده به کار رود.
\\
گرمای ویژه؛ از ویژگی‌های یک نمونه ماده \censor{است} و \censor{می‌تواند} برای توصیف آن ماده به کار می‌رود.
\lin
ظرفیت گرمایی، به نوع ماده بستگی \censor{دارد} و به مقدار ماده بستگی\censor{دارد}.

گرمای ویژه، به نوع ماده بستگی \censor{دارد}و به مقدار ماده بستگی\censor{دارد}.
\newpage

%page 4

\textbf{خود را بیازمایید صفحه 60:}
\begin{en}
	\item
	\censor{کاهش} می‌یابد. با‌گذشت زمان، چای، $\dfrac{\text{بخشی}}{\text{همه}}$ از انرژی گرمایی خود را $\dfrac{\text{از}}{\text{به}}$ محیط $\dfrac{\text{می‌گیرد}}{\text{می‌دهد}}$ پس \censor{میانگین} و \censor{مجموع} انرژی جنبشی ذرات آن، \censor{کاهش} می‌یابد. (کاهش \censor{انرژی} \censor{گرمایی} و \censor{دما} نمونه)
	دلیل: گرما، از جایی که \censor{گرم} تر است (دمای \censor{بالاتر}) به جایی که \censor{سرد} است (دمای \censor{پایین‌تر}) حرکت می‌کند. دمای چای ( \degree C ) از دمای محیط ( \degree C) \censor{بیشتر} است و با \censor{مبادله} انرژی گرمایی، با آن «\censor{هم} \censor{دما}» می‌شود.
	\item
	گرما را می‌توان هم‌ارز با آن مقدار $\dfrac{\text{انرژی گرمایی}}{\text{دمایی}}$ داشت که به دلیل تفاوت در $\dfrac{\text{دما}}{\text{انرژی گرمایی}}$ جاری می‌شود.
	\item
	ماده اصلی تشکیل‌دهنده هر دو، \censor{یکسان} است، پس به مقدار \censor{آب} موجود در آن‌ها توجه می‌کنیم. نان، \censor{آب} کمتری دارد، چون \censor{پخته} شده است، پس \censor{سریعتر} با محیط هم‌دما می‌شود.
	\\
	\textbf{ نتیجه:«آهنگ» تغییر دمای مواد مختلف (مبادله \censor{انرژی} با \censor{محیط}) یکسان \censor{نیست} .}
\end{en}
\lin
نکته: هنگام مبادله گرما بین دو «ماده»؛ (اگر از هدر رفت یا اتلاف گرما چشم‌پوشی کنیم) مقدارگرمایی که ماده با دمای \censor{بالا} است می‌دهد،		$|Q_A| = |Q_B|$
برابر با مقدار گرمایی است که ماده با دمای \censor{پایین} می‌گیرد.
\\
یعنی قدر مطلق \censor{گرما} مبادله شده در آن دو، \censor{برابر} است.
\lin
\textbf{تمرین ۱:}
\\
جسم A به جرم g ۱۰۰ و دمای 100 درجه سانتی‌گراد را در تماس با جسم B به جرم g ۲۰۰  و دمای ۲۰۰ درجه سانتی‌گراد قرار می‌دهیم تا «هم دما» شوند. A و B در چه دمایی، هم‌دما می‌شوند؟ (بر حسب درجه سانتی‌گراد) (المپیاد شیمی ۸۶)
\begin{answers}(4)
	\a 180
	\a 160
	\a 150
	\a 145
\end{answers}
راه اول:
\begin{flushleft}
	$\mathrm{|Q_A|=|Q_B|\rightarrow}$
\end{flushleft}
\lin
راه دوم (هنگام تغییر فاز قابل استفاده نیست.)
\begin{flushleft}
	$\mathrm{\theta_{\textnormal{تعادلی}} = \dfrac{m_1 C_1 \theta _1 + m_2 C_2 \theta _2 }{m_1 C_1 + m_2 C_2} = \dfrac{\qquad\qquad\qquad\qquad}{\qquad\qquad\qquad\qquad} = \dfrac{\sum{(mc\theta)}}{\sum{mc}}}$
\end{flushleft}
\lin
\textbf{تمرین ۲:}
به آلیاژی از تیتانیم و نیکل به جرم ۴.۲ گرم، مقدار ۲۱ ژول گرما دادیم و دمای آن C\degree10 افزایش یافت. به تقریب، چند درصد جرم این آلیاژ را نیکل تشکیل داده است؟ 
$\mathrm{C_{Ni}=0.45(J.g^{-1}.\degree C^{-1})C_{Ti}=0.5(J.g^{-1}.\degree C^{-1})}$
\begin{answers}(4)
	\a 37/6
	\a 49/2
	\a 28/6
	\a 71/5
\end{answers}
\lin



\begin{center}
	\textbf{جاری شدن انرژی گرمایی}
	\\
	«بررسی کیفی و کمی انرژی مبادله شده بین سامانه و محیط»
\end{center}
\textbf{سامانه:}
بخشی از جهان، که \censor{مبادله} \censor{انرژی} را در آن بررسی می‌کنیم.\\
\textbf{محیط:}
هرچه \censor{پیرامون} سامانه وجود دارد.

مثال: بررسی مبادله گرما بین یک لیوان آب و محیط:\\
( معمولاً سامانه با مرز‌های مشخصی از محیط جدا می‌شود. )\\
\newpage
%page 5
\textbf{فرآیند جاری شدن انرژی:}
\begin{multicols}{2}
	\begin{center}
		۱- ناهم دما: ( نمودار ۲ صفحه ۵۹ )\\
		\begin{tikzpicture}[scale=2]
			\draw[->] (0,0) -- (3,0) node[right] {};
			\draw[->] (0,0) -- (0,2) node[left,yshift=-2cm] {\rotatebox{90}{انرژی (E)}};
			\draw (0,1.5) -- (1,1.5) node[above,xshift=-0.9cm] { )۳۷( شیر};
			\draw [dotted] (1,1.5) -- (1.5,0.3);
			\draw (1.5,0.3) -- (2.7,0.3) node[above,xshift=-1.2cm] { )۳۷( فرآورده};
		\end{tikzpicture}\\
		مواد در دو طرف \censor{نمیدونم}\\
		اختلاف در انرژی \censor{نمیدونم}\\
		0 $\bigcirc$ E $\Delta$ ( 0 $\bigcirc$ Q )
	\end{center}
	
		
	\columnbreak
	\begin{center}
	۲- هم دما\\
	\begin{tikzpicture}[scale=2]
		\draw[->] (0,0) -- (3,0) node[right] {};
		\draw[->] (0,0) -- (0,2) node[left,yshift=-2cm] {\rotatebox{90}{انرژی (E)}};
		\draw (0,1.8) -- (1,1.8) node[above,xshift=-0.9cm] { )60( شیر};
		\draw [dotted] (1,1.8) -- (1.5,1);
		\draw (1.5,1) -- (2.7,1) node[above,xshift=-1.2cm] { )60( شیر};
	\end{tikzpicture}\\
	مواد در دو طرف \censor{نمیدونم}\\
	اختلاف در انرژی \censor{نمیدونم}\\
	0 $\bigcirc$ E $\Delta$ ( 0 $\bigcirc$ Q )
	\end{center}
	
\end{multicols}
\lin
\textbf{تمرین:}
مبادلات انرژی را هنگام مصرف بستنی با دمای ۰ درجه سانتی گراد تا هضم آن را بررسی کنید.
\begin{flushleft}
	\begin{tikzpicture}
		\draw[->] (0,0) -- (4,0) node[right] {};
		\draw[->] (0,0) -- (0,2) node[left,yshift=-1cm] {\rotatebox{90}{انرژی (E)}};
		\draw (0,0.7) -- (1,0.7);
		\draw [dotted] (1,0.7) -- (1.5,1.5) node[above,xshift=-9.5] {\scriptsize Q $\bigcirc$ 0};
		\draw[->] (1.25,1.2) -- (1.25,1.6) node[right] {};
		\draw (1.5,1.5) -- (2.5,1.5) ;
		\draw [dotted] (2.5,1.5) -- (3,0.3) node[above,xshift=-17.5] {\scriptsize Q $\bigcirc$ 0 };
		\draw (3,0.3) -- (4,0.3) ;
	\end{tikzpicture}
\end{flushleft}
\lin
\begin{multicols}{2}
	\begin{center}
		\textbf{فرآیند گرماده}\\
		\begin{tikzpicture}
			\draw[->] (0,0) -- (3,0) node[right] {};
			\draw[->] (0,0) -- (0,2) node[left,yshift=-1cm] {\rotatebox{90}{انرژی (E)}};
			\draw (0,1.8) -- (1,1.8) node[above,xshift=-0.45cm] { آغازی};
			\draw [dotted] (1,1.8) -- (1.5,1);
			\draw (1.5,1) -- (2.7,1) node[above,xshift=-0.6cm] { پایانی};
		\end{tikzpicture}
	\end{center}
	در شرایط هم‌دما ($ \Delta\theta = 0$)\\
	جاری شدن انرژی از \censor{سامانه} به \censor{محیط}
	\\
	 واکنش یا فرآیند، برای انجام شدن، گرما می \censor{دهد} .
	\begin{center}
		سطح انرژی طرف دوم $\bigcirc$ سطح انرژی طرف اول\\
		Q $\bigcirc$ ۰
	\end{center}
	نماد Q در طرف \censor{دوم} نوشته می‌شود:\\
	1-
	واکنش گرماده:
	$\ce{H2 + Cl2 -> 2HCl + \quad}$\\
	2-
	فرآیند گرماده:
	$\ce{H2O(l) -> H2O(s) + \quad}$\\
	\censor{کاهش} سطح انرژی سامانه
	\vfill\null
	\columnbreak
	\begin{center}
		\textbf{فرآیند گرماگیر}\\
		\begin{tikzpicture}
			\draw[->] (0,0) -- (3,0) node[right] {};
			\draw[->] (0,0) -- (0,2) node[left,yshift=-1cm] {\rotatebox{90}{انرژی (E)}};
			\draw (0,0.7) -- (1,0.7) node[above,xshift=-0.45cm] { آغازی};
			\draw [dotted] (1,0.7) -- (1.5,1.5) ;
			\draw (1.5,1.5) -- (2.5,1.5)  node[above,xshift=-0.5cm] {پایانی};
		\end{tikzpicture}
	\end{center}
	در شرایط هم‌دما $( \Delta\theta = 0 )$
	\\
	جاری شدن انرژی از \censor{محیط} به \censor{سامانه}
	\\
	 واکنش یا فرآیند، برای انجام شدن، گرما می ‌\censor{گیرد} .
	\begin{center}
		سطح انرژی طرف دوم	 $\bigcirc$ سطح انرژی طرف اول
		\\
		Q$\bigcirc$۰
	\end{center}

	نماد Q در طرف \censor{اول} نوشته می‌شود:
	\\
	۱-
	واکنش گرما‌گیر:
	$\ce{N_2O_4(g) + \censor{} -> 2NO_2(g)}$
	\\
	۲-
	فرآیند گرماگیر:
	$\ce{H2O(s) + \censor{} -> H2O(l)}$
	\\
	\censor{افزایش} سطح انرژی سامانه
	\vfill\null
\end{multicols}

\newpage

%page 6 

\begin{center}
	\textbf{گرما در واکنش‌های شیمیایی }
	(گرماشیمی)
\end{center}
هر واکنش شیمیایی، ممکن است با تغییر \censor{رنگ} ، تولید \censor{رسوب} ، آزاد شدن \censor{گاز} و ایجاد \censor{نور} و \censor{صدا} همراه باشد،
اما:	داد و ستد \censor{گرما} ، یک ویژگی بنیادی واکنش‌های شیمیایی است.
\\
ترموشیمی (گرماشیمی) به بررسی \censor{کمی} و \censor{کیفی} گرمای واکنش‌های شیمیایی، \censor{تغییرات} آن و تأثیری که بر \censor{حالت} ماده دارد، می‌پردازد.
\lin
\textbf{بررسی شکل ۳ صفحه ۶۲:}
\\
الف) مواد غذایی، پس از گوارش، انرژی لازم برای \censor{سوخت} و \censor{ساز} یاخته‌ها را تأمین می‌کنند.
\\
ب) \censor{سوختن} سوخت‌ها، انرژی لازم برای حمل و نقل، و نیز گرمایش محیط‌های گوناگون را فراهم می‌کند.
\\
پ) زغال کک، 
\textbf{واکنش‌دهنده‌ای }
رایج در استخراج آهن، و تامین‌کننده \censor{انرژی} لازم برای واکنش است.
\lin
\begin{minipage}{.7\textwidth}
	منبع انرژی در بدن، \censor{غذا} است. انرژی غذا، پس از انجام واکنش‌های شیمیایی گوناگون، به سلول‌ها می‌رسد. این واکنش‌ها ممکن است گرماده یا گرماگیر باشند اما فرآیند کلی اکسایش گلوکز در مجموع، گرما \censor{ده} است. البته دمای بدن تغییر محسوسی \censor{نمیکند}.
	\\
	دلیل: دمای واکنش‌دهنده‌ها با دمای فرآورده‌ها \censor{یکی} است .
	$\mathrm{(\Delta\theta	\bigcirc0)}$
\end{minipage}%
\begin{minipage}{.3\textwidth}
	\begin{flushleft}
		\begin{tikzpicture}[scale=1.9]+
			\draw[->] (0,0) -- (0,2) node[left,yshift=-2cm] {\rotatebox{90}{انرژی (E)}};
			\draw [very thick](0,1.5) -- (1.7,1.5) node[above,xshift=-1.7cm] { \censor{نمیدونم} , \censor{نمیدونم}};
			\draw [->] (0.5,1.4) -- (0.5,0.8);
			\draw [thick](0,0.3) -- (1.7,0.3) node[above,xshift=-1.8cm] { \censor{نمیدونم} , \censor{نمیدونم}};
		\end{tikzpicture}
	\end{flushleft}
\end{minipage}
\\
در‌واقع، انرژی آزاد شده در این واکنش، ناشی از تفاوت دمای مواد واکنش‌دهنده و فرآورده \censor{نیست} ، بلکه تفاوت میان انرژی \censor{پتانسیل} مواد و واکنش‌دهنده و فرآورده است.
\\
انرژی پتانسیل در اینجا، به معنای انرژی ناشی از نیرو‌های \censor{نگه} \censor{دارنده} ذرات سازنده آن است.
\\
\textbf{انرژی پتانسیل موجود در یک نمونه ماده، انرژی \censor{شیمیایی} نام دارد.}
\lin
\begin{minipage}{0.55\textwidth}
	انرژی پتانسیل در پیوند‌های مختلف، با هم \censor{متفاوت} است، چون اتم‌های مختلفی با هم پیوند دارند. مثال:
	\\
	تفاوت اتم‌های دارای پیوند اشتراکی، موجب تفاوت در نیرو‌های \censor{نگه دارنده} ( این نیرو‌ها، شامل «پیوند‌ها» و «نیرو‌های بین مولکولی» است.) این نیرو‌ها، شامل «پیوند‌ها» و «نیرو‌های بین مولکولی» است. اتم‌ها ( در مولکول ) و در نتیجه؛ تفاوت در \censor{استحکام} پیوند‌ها است.
\end{minipage}
\begin{minipage}{0.45\textwidth}
	\begin{tikzpicture}[scale=1.25]
		\draw[->] (0,0) -- (5,0) node[right] {};
		\draw[->] (0,0) -- (0,2) node[left,yshift=-1cm] {\rotatebox{90}{انرژی (E)}};
		\draw (0,1.7) -- (2,1.7) node[above,xshift=-1.1cm] { H-H + Cl-Cl};
		\draw [dotted] (2,1.7) -- (2.5,0.8);
		\draw (2.5,0.8) -- (4.7,0.8) node[above,xshift=-1.3cm] { H-Cl , H-Cl};
	\end{tikzpicture}
\end{minipage}
انجام واکنش شیمیایی، موجب تغییر در پیوند‌ها یا شیوه اتصال اتم‌ها با یکدیگر، و تفاوت آشکاری در انرژی \censor{پتانسیل} وابسته به آن‌ها می‌شود؛ که خود را به صورت \censor{گرما} (ی مبادله‌شده) نشان می‌دهد.
\lin
\textbf{با هم بیندیشیم صفحه ۶۴	: در دو واکنش:}
\begin{en}
		\item  الف) واکنش‌دهنده‌ها یکسان $\dfrac{\text{هستند}}{\text{نیستند}}$  $\leftarrow$ سطح انرژی واکنش‌دهنده‌ها یکسان \censor{نیست}
				\fbox{
			\begin{tikzpicture}[scale=2.5]
				\draw [thick](0,0.7) -- (0.5,0.7);
				\draw [thick](0,0.5) -- (0.5,0.5);
				\draw [dotted] (0.5,0.7) -- (1,0.3);
				\draw [dotted] (0.5,0.5) -- (1,0.3);
				\draw [thick](1,0.3) -- (1.5,0.3);
			\end{tikzpicture}
		}
	\\
	فرآورده، یکسان \censor{است} $\leftarrow$ سطح انرژی فرآورده در دو واکنش یکسان \censor{است}
	\\
	ب) در واکنش $\dfrac{\text{اول}}{\text{دوم}}$، سطح انرژی واکنش‌دهنده‌ها \censor{کمتر} $\leftarrow$ پایدارتر
\newpage
	\item  الف) چون سطح انرژی گرافیت و الماس، یکسان \censor{نسیت} . ( به دلیل تفاوت در نیرو‌های نگهداری )
	\\
	ب) \censor{گرافیت} پایدار‌تر است، چون فاصله کم‌تری با فرآورده دارد، گرمای سوختنی \censor{کمتری} دارد.
	\\
	\textbf{	نحوه اتصال }
	اتم‌های کربن،
	\textbf{تعداد و نوع }
	پیوند‌های اشتراکی کربن – کربن، در این دو آلوتروپ، و در نتیجه، رفتار شیمیایی آن‌ها ( مانند پایداری یا گرمای سوختن) متفاوت است.
	\\
	پ)
	\begin{flushleft}
				$\mathrm{xKj = \qquad g \times \dfrac{\qquad mol}{\qquad g}\times\dfrac{\qquad KJ}{\qquad mol}=\censor{نمم} KJ}$
	\end{flushleft}


\end{en}
\lin
\begin{center}
	\textbf{یخچال صحرایی!}
\end{center}
دو ظرف از جنس \censor{سفال} داریم که فضای بین آن‌ها از شن خیس پر می‌شود. پارچه‌ای \censor{نخی} به عنوان درپوش، تحویه را انجام می‌دهد. آب درون ظرف درونی، به تدریج در بدنه ظرف بیرونی نفوذ می‌کند و \censor{تبخیر} می‌شود:
\begin{flushleft}
	$\mathrm{H_2O(\quad) + Q \rightarrow H_2O(\quad)}$
\end{flushleft}
این فرآیند، گرما \censor{گیر} است و گرمای لازم را از سامانه دریافت می‌کند که باعث افت دما و خنک شدن محتویات دستگاه می‌شود.
\lin
\begin{center}
	\textbf{فرآیند‌های تغییر حالت مواد}\\
	  \censor{ذوب} : تبدیل حالت \textbf{جامد} به \textbf{مایع}\\
	\censor{انجماد}: تبدیل حالت \textbf{مایع} به \textbf{جامد}\\
	\censor{تبخیر}: تبدیل حالت \textbf{مایع} به \textbf{گاز}\\
	\censor{میعان}: تبدیل حالت \textbf{گاز} به \textbf{مایع}\\
	\censor{تصعید} (\censor{نمیدونم}): تبدیل حالت \textbf{جامد} به \textbf{گاز}\\
	\censor{چگالش} (\censor{نمیدونم}): تبدیل حالت \textbf{گاز} به \textbf{جامد}
\end{center}
 
\lin
\begin{center}
	\textbf{عوامل مؤثر بر گرمای واکنش: (یک عامل ثابت، و سه عامل متغیر)}
\end{center}
\begin{en}
	\item \censor{نوع} مواد واکنش (واکنش‌دهنده‌های و فرآورده‌ها):
	مواد مختلف،‌ سطوح انرژی متفاوت دارند. گرمای واکنش، \censor{اختلاف} سطح انرژی مواد طرف اول و دوم واکنش است. این عامل، متغیر \censor{نیست} ، چون با تغییر مواد، در واقع، واکنش دیگری داریم.
	\item  \censor{فشار} و \censor{دما} :
	تغییر این دو عامل، سطح \censor{انرژی} واکنش‌دهنده‌ها یا فرآورده‌ها را تغییر می‌دهد.
	\item \censor{مقدار} واکنش‌دهند‌ها:
	سطح انرژی هر ماده، به مقدار آن وابسته \censor{است} و تغییر مقدار مواد، سطح انرژی آن را نیز تغییر می‌دهد.
\end{en}
\lin
\textbf{تمرین:}
سوختن هر مول متان، .KJ89 انرژی آراد می‌کند. با سوختن ۱ گرم متان، چند کالُری گرما تولید می‌شود؟
\vspace{4em}
4.
\censor{حالت} \censor{فیزیکی} مواد واکنش:
در معادله «ترموشیمیایی»، باید انرژی \censor{مبادله} \censor{شده} در واکنش ذکر شود. حال اگر

حالت فیزیک یکی از مواد در واکنش تغییر کند، سطح \censor{انرژی} آن نیز تغییر می‌کند و در نهایت، گرمای واکنش را تغییر

می‌دهد.
\begin{flushleft}
	( در دمای \censor{شعله} ) $\mathrm{CH_4(g) + 2O2(g) \rightarrow CO_2(g) + 2H_2O(g) + Q_1}$I)
	\\
	( در دمای \censor{اتاق(stp)} ) $\mathrm{CH_4(g) + 2O_2(g) \rightarrow CO_2(g) + 2H_2O(l) + Q_2}$II)\\
	\vspace{1em}
		\begin{tikzpicture}[scale=1.25]
		\draw[->] (0,0) -- (5,0) node[right] {};
		\draw[->] (0,0) -- (0,2) node[left,yshift=-1cm] {\rotatebox{90}{انرژی (E)}};
		\draw (0,1.7) -- (1.2,1.7) node[above,xshift=-0.65cm,font=\scriptsize] { $\mathrm{CH_4 + 2O_2}$};
		\draw [dotted] (1.2,1.7) -- (1.7,0.8);
		\draw (1.7,0.8) -- (3.1,0.8) node[above,xshift=-0.75cm,font=\scriptsize] { $\mathrm{CO_2 + 2H_2O(g)}$};
		\draw [dotted] (3.1,0.8) -- (3.9,0.3);
		\draw (3.9,0.3) -- (5.1,0.3) node[above,xshift=-0.65cm,font=\scriptsize] { $\mathrm{CO_2 + 2H_2O(l)}$};
		\end{tikzpicture}
\end{flushleft}
\lin
$\mathrm{H_2O}$ تولید شده در واکنش سوختن متان، ابتدا در دمای شعله است و حالت فیزیکی گازی دارد، اگر مقداری صبر کنیم تا سامانه با محیط، « \censor{دما} \censor{هم} » شود، $\mathrm{H_2O}$ به حالت مایع در می‌آید.
این فرآیند 
$\left(\dfrac{\text{تبخیر}}{\text{معیان}}\right)$
، خود، گرما \censor{ده} است و در رسیدن از I به II مقداری گرما \censor{آزاد} می‌شود. یعنی $\mathrm{ Q_2}$، از لحاظ عددی، از $\mathrm{Q_1}$\censor{بزرگتر} است.
\lin
\textbf{تمرین) }
گرمای تبخیر مولی آب را برحسب $\mathrm{Q_1}$ و $\mathrm{Q_2}$ به دست آورید:
\begin{flushleft}
	0
	$\bigcirc$
	$\dfrac{\qquad\qquad}{\qquad\qquad}$
	=	گرمای تبخیر مولی
\end{flushleft}
\lin
\textbf{با هم بیندیشیم ۳ صفحه ۶۴:}
\\
اولاً: میعان، گرما \censor{ده} است، پس گرمای واکنش با عدد $\dfrac{\text{+}}{\text{-}}$ گزارش می‌شود.
\\
ثانیا: گرمای آزاد شده در میعان و نیز گرمای واکنش هردو، علامت 	دارند و مجموع آن‌ها با علامت 	باید از نظر عددی از ۴۸۴ \censor{بیشتر} باشد ( یعنی عدد \censor{۵۷۲} )
\lin
\textbf{پرسش:}
\\
گرمای آزاد شده در کدام حالت، مقدار عددی بیشتری دارد؟ (روش: باید یک طرف کمترین و طرف دیگر بیشترین سطح انرژی را داشته باشد)
\begin{answers}(4)
	\a $\mathrm{2O(l) \rightarrow O_2(g)}$
	\a $\mathrm{2O(g) \rightarrow O_2(l)}$
	\a $\mathrm{2O(g) \rightarrow D_2(g)}$
	\a $\mathrm{2O(l) \rightarrow O_2(l)}$
\end{answers}
\lin
\begin{center}
	\textbf{«آنتالپی ,(H) همان محتوای انرژی است»}
\end{center}
هر نمونه ماده، دارای شمار بسیار زیادی «ذره سازنده» است. این ذره‌ها، دارای:
\\
۱- \censor{جنبش} نامنظم (انرژی \censor{جنبشی} ) و ۲- \censor{بر هم کنش} با یکدیگر (انرژی \censor{پتانسیل}) هستند
\\
\textbf{یک نمونه ماده، با \censor{مقدار} آن در \censor{دما} و \censor{فشار} معین، توصیف می‌شود.}
مانند ۲۰۰ گرم آب در دما و فشار معین یک نمونه ماده در یک ظرف، می‌تواند یک \censor{سامانه} به شمار آید.
\\
«انرژی کل» یک سامانه، هم ارز «محتوای \censor{انرزی}» یا «\censor{انتالپی}» آن سامانه است.
یعنی: همه مواد، در دما و قشار معین، «\censor{انتالپی}» مشخصی دارند.
\\
با انجام واکنش شیمیایی، «محتوای \censor{انرزی}» یا «\censor{انتالپی}» مواد، تغییر می‌کند. (مانند نمودار ۵ صفحه ۶۴)
\newpage
\textbf{مهم:}
$\mathrm{Q_p}$
= \censor{واکنشدهنده} H - \censor{فراورده} H =
واکنش
$\mathrm{\Delta H}$
$\leftarrow$
(\censor{تغییر} آنتالپی )
\\
$\mathrm{Q_p}$
به معنای \censor{گرمای} مبادله شده در « \censor{مقدار} \censor{ثابت} » است.
\\
مقدار عددی $\mathrm{\Delta H}$ در یک فرآیند، \censor{بزرگی} آن را نشان می‌دهد، اما علامت + یا -، به ترتیب، \censor{گرما} \censor{گیر} و \censor{گرما} \censor{ده} بودن آن را نشان می‌دهد.
\\
\textbf{خود را بیازمایید صفحه ۶۶ و ۶۷: 	}
\begin{en}
	\item الف) $\mathrm{CO_2(s)\qquad \rightarrow CO_2(g)\qquad, \Delta H \bigcirc 0}$
	\\
	ب) $\mathrm{CH_4(g) + 2O_2(g) \rightarrow CO_2(g) + 2H_2O(g)\qquad, \Delta H \bigcirc 0}$
	\\
	پ) $\mathrm{N_2O_4(g)\qquad \rightarrow 2NO_2(g)\qquad, \Delta H \bigcirc 0}$
	\\
	ت) $\mathrm{N_2O(l) \rightarrow H_2O(s)\qquad, \Delta H \bigcirc 0}$
	\\
	\lin
	\begin{minipage}{0.6\textwidth}
		\centering
	\item $\mathrm{3O_2(g) + \censor{نمیدونم} \leftrightarrow 2O_3(g)}$\\
	$\mathrm{x(KJ) = \dfrac{ KJ}{ mol O_3}\times \censor{22} mol O_3 = \censor{286} (KJ)}$	\\
	$\mathrm{(\Delta H \text{رفت} = \qquad\quad) (\Delta H \text{برگشت} = \qquad\quad)}$
	\end{minipage}
	\begin{minipage}{0.4\textwidth}
		\begin{tikzpicture}
		\draw[->] (0,0) -- (3,0) node[right] {واکنش پیشرفت};
		\draw[->] (0,0) -- (0,2) node[left,yshift=-1cm] {\rotatebox{90}{آنتالپی}};
		\draw (0,0.7) -- (1,0.7) node[above,xshift=-0.45cm] {\censor{آغا}};
		\draw [dotted] (1,0.7) -- (1.5,1.5) ;
		\draw [<->] (1.5,0.7) -- (1.5,1.4) ;
		\draw (1.5,1.5) -- (2.5,1.5)  node[above,xshift=-0.5cm] {\censor{پایا}};
		\end{tikzpicture}
	\end{minipage}
		\\
	
\end{en}
\lin
\begin{center}
	\textbf{«آنتالپی پیوند» و «میانگین آنتالپی پیوند»}
\end{center}
\begin{minipage}[t]{0.7\textwidth}
	انجام یک واکنش شیمیایی، نشانه‌ای از تغییر در \censor{شیوه} \censor{اتصال} اتم‌ها (ذرات) به یکدیگر است، که نتیجه آن، تغییر \censor{ساختار} و به دنبالش تغییر \censor{خواص} مواد است. یکی از خواصی که در واکنش‌های شیمیایی تغییر می‌کند، محتوای \censor{انرژی} مواد است. مثلاً، یک نمونه گاز هیدروژن، دارای شمار بسیار زیادی \censor{مولکول} دو اتمی است. با صرف \censor{انرژی} ، پیوند \censor{اشتراکی} بین اتم‌ها در مولکول می‌شکند و به \censor{اتم} هایی تبدیل می‌شود که \censor{ناپایدار} تر و \censor{ببب} \censor{انرژی} تر هستند. در ترموشیمی، به مقدار KJ436, آنتالپی \censor{پیوند} می‌گویند:
		($\mathrm{ KJ.mol^{-1}}$ )436 $\bigcirc$  = ( \quad - \quad ) H $\Delta$
\end{minipage}%
\begin{minipage}[t]{0.3\textwidth}
	\begin{flushleft}
		\begin{tikzpicture}[scale=1.5]
			\draw[->] (0,0) -- (0,2) node[left,yshift=-1cm] {\rotatebox{90}{آنتالپی}};
			\draw (0,0.7) -- (1,0.7) node[above,xshift=-0.6cm] {$\mathrm{H-H(g)}$};
			\draw [dotted] (1,0.7) -- (1.5,1.5);
			\draw [<->] (1.5,0.7) -- (1.5,1.4) node[right,yshift=-0.6cm] {$\mathrm{Q=\censor{da}KJ}$} ;
			\draw (1.5,1.5) -- (2.5,1.5)  node[above,xshift=-0.55cm] {$\mathrm{H(g)+H(g)}$};
		\end{tikzpicture}
	\end{flushleft}
\end{minipage}
\lin
\textbf{آنتالپی پیوند:}
انرژی لازم برای \censor{شکستن} ۱ \censor{مول} پیوند در مولکول \censor{گازی} و تبدیل آن به اتم‌های \censor{گاز}
\\
در مولکول‌هایی که «اتم مرکزی» به چند اتم یکسان با پیوند اشتراکی متصل است، (مانند $\mathrm{CH_4}$)این پیوند‌های یکسان، آنتالپی کاملاً یکسان \censor{ندارند} ! در این حالت، به کار بردن اصطلاح  \censor{میانگین} آنتالپی پیوند، مناسب‌تر است.
\begin{flushleft}
$\mathrm{	CH_4(g) + 1660 KJ \rightarrow \censor{نمیدونم} (\quad) + \censor{نمیدونم} (\quad)}$
	\\
$\mathrm{\Delta H_{(C-H)} = \qquad \div \qquad = \qquad (KJ.mol^{-1})}$
\end{flushleft}
\lin
پرسش) در چند مورد، به کار بردن میانگین آنتالپی پیوند، مناسب‌تر است؟ \censor{۲} مورد
\begin{answers}(4)
	\a $\mathrm{NH_3(g)}$
	\a $\mathrm{N\equiv N(g)}$
	\a $\mathrm{H-Br(g)}$
	\a $\mathrm{H_2O(g)}$
\end{answers}
نکته: در مولکول‌های ۲ اتمی، میانگین گرفتن لازم \censor{نیست}
\newpage
\textbf{خود را بیازمایید صفحه ۶۸:}
\\
الف) (پیوند \censor{شکسته} شده $\leftarrow$ گرما \censor{گیر} ) \qquad \qquad H$\Delta$ | پیوند‌ها در جدول ۲ صفحه ۶۵ مربوط به مولکول ۲ اتمی (میانگین $\dfrac{\text{هست}}{\text{نیست}}$.)
\\
ب) (پیوند \censor{تشکیل} شده $\leftarrow$ گرما \censor{ده} )\qquad\qquad H $\Delta$    | پیوند‌ها در جدول ۳ صفحه ۶۶ مربوط به مولکول‌های چند اتمی ( میانگین \censor{هست} )
\\
تذکر: برای گزارش آنتالپی پیوند، همه ذرات در دو طرف واکنش به حالت \censor{گازی} و همه فرآورده‌ها باید \censor{اتمی} باشند: (اگر قرار است همه پیوند‌ها شکسته شود.)  \qquad \qquad  \qquad\qquad\qquad
$\mathrm{NH_3(\quad) + Q \rightarrow \censor{ٔیسب} (\quad) + \censor{سیب} (\quad)}$
\lin
\begin{center}
	\textbf{«آنتالپی پیوند، راهی برای تعیین H $\Delta$ واکنش»}
\end{center}

۱) روش محاسباتی برای تعیین H $\Delta$واکنش:
\\
در واکنش شیمیایی، «معمولا» تعدادی پیوند \censor{شکسته} و تعدادی پیوند جدید \censor{تشکیل} می‌شود.
\\
برای «شکستن» پیوند،‌ مقداری انرژی \censor{مصرف} می‌شود ( با علامت $\bigcirc$ گزارش می‌شود).
\\
هنگام «تشکیل» پیوند،‌ مقداری انرژی \censor{آزاد} می‌شود ( با علامت $\bigcirc$ گزارش می‌شود).(H $\Delta$واکنش، \censor{مجموع} این انرژی‌های \censor{مبادله} شده است.)
\\
استفاده از آنتالپی پیوند، برای تعیین H $\Delta$واکنش‌های \censor{گازی} مناسب‌تر است. ( همه مواد در حالت \censor{گازی} )
\\
هرچه مواد واکنش‌، مولکول‌های \censor{ساده} داشته باشند، H $\Delta$محاسبه شده، با داده‌های \censor{تجربی} همخوانی بیشتری دارد، و هرچه مولکول‌ها پیچیده‌تر باشند، H $\Delta$ محاسبه شده با داده‌های \censor{تجربی} تفاوت‌های آشکار نشان می‌دهد.
\\
۲) استفاده از «آنتالپی پیوند» برای تعیین H $\Delta$ واکنش: (خود را بیازمایید ۱ صفحه ۶۹)
\\
H $\Delta$ واکنش: [مجموع آنتالپی‌های پیوند \fb]-[مجموع آنتالپی‌های پیوند \fb]
\lin
نکته: در جدول آنتالپی پیوند، همه اعداد علامت 	$\bigcirc$دارند و علامت $\bigcirc$	پیش از آنتالپی پیوند فرآورده‌ها، برای آن است که 	$\bigcirc$در $\bigcirc$	، $\bigcirc$	شود. ( چون در فرآورده‌ها، پیوند‌ها در حال تشکیل هستند که فرآیندی گرماده است و باید با عدد منفی نوشته شود. )
\\
خود را بیازمایید ۲ صفحه ۷۰:\\
الف)\\
ب)\\
پ)\\
\lin
تمرین ۱ اگر برای تبدیل ۱ گرم از گاز‌های متان و اتان، به اتم‌های گازی جدا از هم، به ترتیب ۱۰۳ و ۹۴ کیلوژول انرژی مصرف شود، آنتالپی \chemfig{C-[,0.5] C} چند 
$\mathrm{\dfrac{KJ}{mol} }$است؟ ($\rm C=\text{ و ۱۲} H=\text{۱}$)
\vspace{6em}
\lin
تمرین ۲ به کمک «جدول آنتالپی پیوند»، آنتالپی سوختن کامل اتانول و بنزین را به دست آورید:
\vspace{4em}

خود را بیازمایید ۲ صفحه ۷۲:\\
الف) این دو ترکیب، فرمول مولکولی $\dfrac{\text{یکسان}}{\text{متفاوت}}$، و ساختار \censor{متفاوت} دارند.\\
نتیجه: این دو ترکیب، \censor{ایزومر} \censor{ساختاری} ( هم \censor{پار} ) هستند.\\
ب) $\dfrac{\text{بله}}{\text{خیر}}$، چون ساختار آن‌ها یکسان \censor{نیست}.\\
پ) $\dfrac{\text{بله}}{\text{خیر}}$، چون تفاوت در \censor{ساختار}، موجب تفاوت در \censor{خواص} از جمله سطح انرژی است.

\Ovalbox{
	\begin{minipage}{0.97\linewidth}
			محتوای انرژی یک ترکیب، در \underline{دما} و \underline{فشار} ثابت، علاوه بر «نوع» و «تعداد» اتم‌ها به نحوه \censor{اتصال} اتم‌ها، و «نوع» پیوند‌های شیمیایی مربوط است.
	\end{minipage}
}
\\
\begin{center}
	\textbf{آشنایی با گرو‌ه‌های عاملی}
\end{center}

گروه عاملی؛ \censor{آرایش} منظمی از \censor{اتم} ها است که به مولکول دارای آن، خواص \underline{فیزیکی} و \underline{شیمیایی} ویژه می‌بخشد.\\
در گروه‌های عاملی، \censor{شیوه} اتصال اتم‌ها با یکدیگر، یا \censor{پیوند} بین آن‌ها، اهمیت ویژه دارد.\\
گروه عاملی، در تعیین \censor{خواص} ترکیبات آلی، نقش تعیین‌کننده‌ای دارد. به عنوان مثال خواص ادویه، به طور عمده وابسته به ترکیب‌های آلی موجود در آن‌ها است که در ساختار آن‌ها، علاوه بر C و H، اتم‌های \censor{O} و گاهی \censor{N} و \censor{S} وجود دارد. تفاوت در خواص ادویه، به دلیل تفاوت در ساختار این مواد آلی است. ( گروه عاملی، قسمتی از ترکیب آلی است که با دیدن آن، می‌فهمیم این ترکیب، \censor{آلکان} نیست! )
\begin{center}
	\textbf{آشنایی با برخی خانواده‌های ترکیبات آلی}
\end{center}
\vspace{0em}
\begin{landscape}
		\begin{table}
			\centering
			\begin{tblr}{
					cells = {Gallery,c},
					row{even} = {Alto},
					cell{1}{2} = {b},
					hlines,
					vlines = {Black},
					vline{1} = {-}{Silver},
				}
				نام خانواده          & {~ ~فرمول کلی~ ~ \\
					(R' یا R با )} & {~ ~ ~فرمول کلی~ ~ ~\\
					(n برحسب)} & ~ ~ گروه عاملی~ ~~ & ~ ~ نام گروه عاملی~ ~ & {~ ~روش نام‌گذاری~ ~ ~\\
					(آیوپاک)} & {تعداد پیوند\\
					کووالانسی} & ~ ~ ~ ~ ~ ~نمونه~ ~ ~ ~ ~ ~ ~ ~~ \\
				{آلکن\\~}            &                                         &                                         &                    &                        &                                        &                               &                                      \\
				{آلکین\\~}           &                                         &                                         &                    &                        &                                        &                               &                                      \\
				{آلدهید\\~}          &                                         &                                         &                    &                        &                                        &                               &                                      \\
				{کتون\\~}            &                                         &                                         &                    &                        &                                        &                               &                                      \\
				{الکل\\~}            &                                         &                                         &                    &                        &                                        &                               &                                      \\
				{اتر\\~}             &                                         &                                         &                    &                        &                                        &                               &                                      \\
				{کربوکسیلیک اسید\\~} &                                         &                                         &                    &                        &                                        &                               &                                      \\
				{استر\\~}            &                                         &                                         &                    &                        &                                        &                               &                                      \\
				{آمین\\~}            &                                         &                                         &                    &                        &                                        &                               &                                      \\
				{آمید\\~}            &                                         &                                         &                    &                        &                                        &                               &                                      
			\end{tblr}
		\end{table}
\end{landscape}
\newpage
\begin{center}
	\textbf{آنتالپی سوختن، تکیه‌گاهی برای تامین انرژی}
\end{center}

بدن ما از عذا، مواد گوناگونی شامل \censor{کربوهیدرات} ها، \censor{چربی} ها، \censor{پروتین} ها، \censor{ویتامین} ها، \censor{آب} و مواد \censor{معدنی} دریافت می‌کند.
\lin
از این بین، کربوهیدرات‌ها، چربی‌ها و پروتئین‌ها، علاوه بر:
\circled{۱}
تامین \censor{مواد} اولیه برای سوخت و ساز،
\circled{2}
تامین \censor{انرژی} یاخته‌ها نیز هستند.\\
از این سه دسته، تنها \censor{کربوهیدرات} در بدن به \censor{گلوگز} شکسته شده و در خون حل می‌شود. \censor{گلوگز}، قند خون است، خون این ماده را به یاخته‌ها می‌رساند و در آنجا \censor{اکسایش} می‌یابد و \censor{انرژی} تولید می‌کند.

بدن، بیشتر \censor{چربی} را ذخیره می‌کند چون انرژی حاصل از اکسایش جرم برابری از آن با دو ماده دیگر، بیشتر است. ( جدول ۴ صفحه ۷۰)\\
\noindent\fbox{%
	\parbox{\textwidth}{%
		\textbf{انرژی سوختی:}
		انرژی حاصل از سوختن ۱ \censor{گرم} از ماده غذایی ( یکا: \censor{KJ} ) جواب ۵ صفحه ۷۱
	}%
}
تمرین \circled{۱}: اگر درصد چربی در ترکیب یک ماده غذایی ۲٪، و درصد پروتئین و کربوهیدرات در آن، به ترتیب \underline{۳} برابر و \underline{۲۴} برابر چربی باشد، ارزش سوختی این ماده غذایی 
$\mathrm{\dfrac{KJ}{g}}$
است؟ ( راهنمایی: جرم ماده غذایی را
\underline{۱۰۰}
گرم فرض کنید. )
\vspace{6em}
نکته: جرم کربوهیدرات و پروتئین را می‌توان جمع و یکجا محاسبه کرد (چون ارزش سوختی آن‌ها یکسان است. )
\lin
تمرین \circled{۲}: با گرمای آزاد شده از سوختن 50g از ماده غذایی تمرین \circled{۱}، چند مول آب ۸۰\degree را می‌توان به جوش آورد؟ ( فرض کنید در این فرآیند، ۲۰٪ هدر رفت انرژی وجود دارد. )
$\mathrm{C(H_2O)=4.2(J.g^{-1}.\degree C^{-1})}$
\vspace{6em}
\lin
\textbf{سوختن}
برای تهیه غذای گرم، معمولا از سوخت‌های \censor{فسیلی} استفاده می‌شود. مانند \censor{متان} که (عمده) گاز شهری را تشکیل می‌دهد، در حضور اکسیژن \underline{کافی} می‌سوزد و انرژی زیادی تولید می‌کند:
\begin{flushleft}
	$\ce{CH_4(g) + O_2(g) -> CO_2(g) + H_2O(g) + 890KJ}\;\text{(موازنه کنید)}$
\end{flushleft}
\noindent\fbox{%
	\parbox{\textwidth}{%
		آنتالپی سوختن: انرژی حاصل از سوختن \underline{۱} \censor{مول} از ماده سوختنی ( یکا: \censor{$mol^{-1}$} \censor{Kj} ) جواب ۶ صفحه ۷۱
	}%
}
\lin
خود را بیازمایید صفحه ۷۳:\\
$\mathrm{\Delta H_\text{سوختن}\text{(پروپان)}\simeq-2220(KJ.mol^{-1})\hspace{2pt}\vline\hspace{2pt}\Delta H_\text{سوختن}\text{(بوتن-۱)}\simeq-2717 (KJ.mol^{-1})}$
\begin{flushleft}
	\begin{tabular}{ c c c c c }
		پروپان && اتان && متان \\ 
		$\mathrm{x_1}$ &\ce{ ->[)-CH_2-(]}& 1560-& \ce{ ->[)-CH_2-(]}&KJ890-\\
		\hline
		اتن &&پروپن&&بوتن-۱\\
		$\mathrm{x_2}$ &\ce{ ->[)-CH_2-(]}& ۲۰۵۸-& \ce{ ->[)-CH_2-(]}&KJ۱۴۱۰-\\
	\end{tabular}
\end{flushleft}
\newpage
خود را بیازمایید \underline{۲} صفحه \underline{۷۳} و \underline{۷۴}: الف) ارزش سوختی: اتان
$\bigcirc$
اتانول 
\quad|\quad
آنتالپی سوختن: اتان
$\bigcirc$
اتانول
\\
	12C=, 1H=, 16O=
\lin\\
ب)
\vspace{2em}\\
پ) سوخت‌های سبز، علاوه بر CوH، اتم \censor{اکسیژن} نیز دارند و از پسماند سویا، نی‌شکر یا سایر دانه‌های روغنی استخراج می‌شوند. سوخت سبز برای سوختن، اکسیژن \censor{کمتری} نیاز دارد.\\
همچنین اتانول، در سوختن با \censor{جرم} برابر (نسبت به هیدروکربن)،$\mathrm{ CO_2}$ کمتری تولید می‌کند; مثال:
\begin{center}
	$\mathrm{C_2H_6+\quad O_2\rightarrow\quad CO_2+\quad H_2O\qquad\quad C_2H_6O+\quad O_2\rightarrow\quad CO_2+\quad H_2O}$
\end{center}
\lin
پرسش \circled{۱}: می‌دانیم که سوختن مواد در دما‌های بالا صورت می‌گیرد. چرا در خود را بیازمایید \underline{۱} صفحه \underline{۷۱}، سوختن مواد در دمای \degree C25 مطرح شده است؟

پاسخ: منظور از عدد \degree C25 روی پیکان در این واکنش‌ها، سوختن در دمای \degree C25 $\dfrac{\text{نیست}}{\text{است}}$، بلکه به معنای اندازه‌گیری \censor{انتالپی} واکنش در دمای \degree C25 است.\\
\lin
پرسش \circled{۲}: سوختن هیدروکربن‌ها در دماهای بالا صورت می‌گیرد، پس چگونه می‌توان آنتالپی سوختن را در دمای \degree C25 اندازه‌گیری کرد؟\\
\\
\begin{minipage}{0.6\textwidth}
پاسخ: ابتدا، واکنش‌دهنده‌ها را در دمای \degree C25 وارد سامانه می‌کنیم، پس از انجام واکنش (سوختن) اجازه می‌دهیم فرآورده‌ها \censor{سرد} شوند و به دمای \degree C25 برسند. بعنی ابتدا و انتهای واکنش، در \degree C25 بررسی می‌شود. آنتالپی واکنش نیز با توجه به \censor{ابتدا} و \censor{انتها} واکنش تعیین می‌شود، حتی اگر در \censor{مسیر} به دمایی دیگر برسیم.(آنتالپی واکنش، تابع مسیر \censor{نیست}). 
\end{minipage}
\begin{minipage}{0.4\textwidth}
	\begin{tikzpicture}[scale=1.25]
		\draw[->] (0,0) -- (3,0) node[right,font=\scriptsize] {پیشرفت واکنش};
		\draw[->] (0,0) -- (0,2) node[left,yshift=-1cm] {\rotatebox{90}{آنتالپی}};
		\draw (0,1.7) -- (1.2,1.7) node[above,xshift=-0.65cm,font=\tiny] {آنتالپی واکنش‌دهنده‌ها};
		\draw [dotted] (1.2,1.7) -- (1.7,0.8);
		\draw[<->] (1.15,1.6) -- (1.15,0.8) node[left,yshift=0.4cm] {\censor{ن}};
		\draw (1.7,0.8) -- (3.1,0.8) node[above,xshift=-0.85cm,font=\scriptsize] {آنتالپی فرآورده‌ها};
	\end{tikzpicture}
\end{minipage}
\\
\\                                                                              
چنانکه در طرح بالا نیز دیده می‌شود؛ اختلاف سطح انرژی واکنش‌دهنده‌ها با فرآورده‌ها (در دمای معین)، یعنی همان \censor{انتالپی} واکنش، مقدار مشخص \censor{است} و به مسیر پیموده شده ربط \censor{ندارد}.
\lin
\textbf{نکات مهم مربوط به جدول \underline{۶} صفحه \underline{۷۱}}\\
\circled{1}
در اثر سوختن هیدروکربن‌ها و مواد آلی اکسیژن‌دار، گرما آزاد می‌شود. سوخت‌ها، موادی پر انرژی و $\dfrac{\text{پایدار}}{\text{ناپایدار}}$ هستند و فرآورده‌های سوختن، به نسبت $\dfrac{\text{پایدار}}{\text{ناپایدار}}$ ترند و این تفاوت، به صورت گرما آزاد می‌شود.\\
\circled{2}
بین چند آلکان (یا سایر هیدروکربن‌های هم خانواده) مقدار عددی آنتالپی سوختن ترکیبی بیشتر است که ${\text{سبک‌تر}}{\text{سنگین‌تر}}$ است. (وقتی \underline{مول‌های} برابر از چند هیدروکربن هم‌خانواده بسوزند، آنکه کربن \censor{بیشتری} دارد، گرمای بیشتری آزاد می‌کند.)\\
\circled{3}
بین چند آلکان (یا سایر هیدروکربن‌های هم‌خانواده) ارزش سوختی ترکیبی بیشتر است که $\dfrac{\text{سبک‌تر}}{\text{سنگین‌تر}}$ است.(وقتی \underline{جرم‌های} برابر از چند هیدروکربنی هم‌خانواده بسوزند، آنکه کربن \censor{کمتر} دارد، گرمای بیشتری آزاد می‌کند.)\\
\circled{4}
الکل‌های سنگین‌تر، نسبت به الکل‌های سبک‌تر، مقدار عددی آنتالپی سوختن \censor{بیشتر} و ارزش سوختی \censor{بیشتر} دارند.\\\textbf{(نکته \circled{2} در مورد الکل‌ها صدق \censor{میکند} و نکته \circled{2} صدق \censor{نمی‌کند}!)}\\
\circled{5}
آنتالپی سوختن \underline{۴} خانواده جدول (هم کربن): \censor{الکان} > \censor{الکن} > \censor{الکل} > \censor{الکین}\\
\circled{6}
بین آلکان، آلکن و آلکین هم کربن، \textbf{ارزش سوختی}:  \censor{الکان} > \censor{الکن} > \censor{الکل} >\censor{شسیی}\\
\textbf{تذکر:} برخلاف انتظار، دمای شعله: اتین > اتن > اتان
\\
\\
\\
\lin
\begin{center}
	\textbf{اندازه‌گیری گرمای واکنش}
\end{center}
دو روش دارد: الف) روش مستقیم(اندازه‌گیری در آزمایشگاه، به کمک ابزار) ب) روش غیرمستقیم (به کمک محاسبه)
\lin
الف) روش مستقیم (گرماسنجی یا کالری‌متری) به روش تجربی، که ابزار آن،
\Ovalbox{\textbf{گرماسنج}}
است.
\\
گرماسنج، انواع مختلف دارد و در کتاب درسی فقط به گرماسنج لیوانی اشاره شده است. (ش \underline{۸} صفحه \underline{۷۴})
\\
\Ovalbox{\textbf{گرماسنج لیوانی:}}
گرمای واکنش را در \censor{فشار} ثابت اندازه‌گیری می‌کند. (که به آن، \censor{$\Delta H$} گفته می شود.)\\
این گرما‌سنج، برای تغیین «آنتالپی \censor{انحلال}» و نیز آنتالپی واکنش‌ها در حالت «\censor{محلول}» مناسب است.\\
در این گرماسنج، مقداری آب درون لیوان یک‌بار مصرف (\underline{۲} لیوان درون هم) قرار می‌گیرد که تا حد ممکن عایق \censor{گرما} باشد. درپوش یونالیتی روی آب قرار می‌گیرد و از درون آن، یک دماسنج و یک همزن وارد آب می‌شود تا دما را در کل محلول، تا حد ممکن \censor{یکسان} سازد. با اندازه‌گیری تغییر دما ($\Delta\Theta$) در طول فرآیند، می‌توان گرمای واکنش را از فرمول
$\mathrm{Q=mc\Delta\Theta}$
محاسبه نمود.
\lin
مسئله: در یک گرماسنج لیوانی، 200mL محلول سود ۰.۱ مولار با 200mL محلول سولفوریک اسید وارد واکنش می‌شود. اگر در پایان واکنش، مقداری اسید واکنش نداده باقی‌مانده و دما به اندازه $\mathrm{(0.7\degree C)}$ افزایش یافته باشد، آنتالپی واکنش روبه‌رو، چند KJ است؟ \quad (همه گرمای واکنش، صرف بالا بردن دمای محلول شده و چگالی همه محلول‌ها
$\mathrm{\dfrac{Kg}{L}}$
\underline{1}
است. گرمای ویژه محتویات گرماسنج،
$\mathrm{\underline{4}J.g^{-1}.\degree C^{-1}}$
است.)
\begin{flushleft}
	\ce{H_2SO_4  )aq( + 2NaOH )aq( -> Na_2SO_4 )aq( + 2H_2O )l(}
\end{flushleft}
\vspace{6em}
\lin
مسئله: حل کردن ۰.۱ مول کلسیم کلرید در گرما‌سنجی حاوی
$\mathrm{0.5Kg}$
آب، دمای گرماسنج را
$\mathrm{1.2\degree C}$
بالا می‌برد. ظرفیت گرمایی گرما‌سنج، چند
$\mathrm{KJ.\degree C^{-1}}$
است؟ و اگر در ابتدای واکنش به جای کلسیم کلرید، $\mathrm{30g}$ آمونیوم نیترات ۸۰٪ خالص را در آب حل کنیم، دمای مجموعه به تقریب چند
$\mathrm{\degree C}$
تغییر می‌کند؟ افزایش می‌یابد یا کاهش؟\\(آنتالپی انحلال
$\mathrm{CaCl_{2}}$
و
$\mathrm{NH_{4}NO_{3}}$
به ترتیب -۸۵.۲ و +۲۶ کیلوژول بر مول است.)
$\mathrm{C_{H_2O}=4.2(\dfrac{J}{g.\degree C})}$
\vspace{6em}
\lin
ب) روش غیرمستقیم: گرمای واکنش را می‌توان به کمک محاسبه، و با استفاده از استوکیومتری، آنتالپی تشکیل مواد، آنتالپی پیوند، و قانون هس محاسبه کرد، که در کتاب درسی، به دو مورد آخر پرداخته شده است.
\newpage
\begin{center}
	\textbf{جمع‌پذیری گرمای واکنش‌ها، \Ovalbox{«قانون هس»}}
\end{center}
آنتالپی بسیاری از واکنش‌ها را نمی‌توان به روش \censor{تجربی} اندازه‌گیری نمود. برخی واکنش‌ها، \underline{یک مرحله} از واکنشی «\censor{چند} مرحله»  (پیچیده) هستند، و برخی از آن‌ها، به آسانی انجام نمی‌شوند، (یا اصلا انجام نمی‌شوند!)\\
در این حالات، برای محاسبه گرمای واکنش، می‌توان از قانون هس کمک گرفت.\\
براساس «قانون هس»: \Ovalbox{اگر واکنشی شامل «چند» مرحله باشد، $\mathrm{\Delta H}$ واکنش کلی، برابر \censor{مجموع} $\mathrm{\Delta H}$ مراحل آن است.}\\
به بیان دیگر: \Ovalbox{گرمای یک واکنش معین، به راهی که برای انجام آن پیش‌گرفته، وابسته \censor{نیست}.}\\
روش کار: اگر معادله واکنشی را بتوان از «مجموع» معادله چند واکنش به دست آورد؛ $\mathrm{\Delta H}$ واکنش کلی نیز از \censor{جمع} \censor{جبری} $\mathrm{\Delta H}$ همان چند واکنش (مراحل) به دست می‌آید.
\lin
مثال: حشره‌ای با نام «سوسک بمب‌افکن»، برای دفاع از خود، مخلوطی از مواد داغ را به سمت دشمن پرتاب می‌کند، که این مواد در طرف دوم واکنش کلی دیده می‌شوند. اگر واکنش کلی در واقع شامل سه مرحله با $\mathrm{\Delta H}$‌های گفته شده باشد\RTLfootnote{اگر واکنش شیمیایی با $\mathrm{\Delta H}$ وابسته به آن معرفی شود، به آن، واکنش \censor{ترمو} \censor{شیمیایی} یا \censor{گرما} \censor{شیمیایی} می‌گویند.}، $\mathrm{\Delta H}$ واکنش کلی را به دست آورید.
\begin{flushleft}
	\begin{tabular}{l l}
		$\mathrm{(\Delta H_1=177 KJ)}$  & $\circled{1}\;\ce{C_6H_6O_2 (aq) -> C_6H_4O_2 + H_2 (g)};$         \\
		$\mathrm{(\Delta H_2=-95 KJ)}$  & $\circled{2}\;\ce{H_2O_2 (aq) -> H_2O (l) + \dfrac{1}{2}O_2 (g)};$  \\
		$\mathrm{(\Delta H = -286 KJ)}$ & $\circled{3}\;\ce{H_2 (g) + \dfrac{1}{2}O_2 (g) -> H_2O (l)};$      \\
		 & \tikz\draw [black] (0,0) -- (7cm,0pt);          \\
		$\mathrm{(\Delta H=?)}$         & $\ce{C_6H_6O_2 (aq) + H_2O_2 (aq) -> C_6H_4O_2 (aq) + 2H_2O (l)};$
		:واکنش کلی
	\end{tabular}
\end{flushleft}
\lin
\textbf{توجه:}
در اکثر موارد، برای آن که از جمع‌بندی مواد در مراحل مختلف، به واکنش کلی برسیم، لازم است که تغییراتی را در واکنش‌های مراحل، انجام دهیم. این تغییرات، شامل تغییر در ضرایب، و یا جابه‌جایی واکنش‌دهنده‌ها با فرآورده‌ها است. مثلا ضریب ماده‌ای در واکنش کلی \underline{۲} اما در مراحل \underline{۱} است یا ماده‌ای در واکنش کلی در طرف اول، اما در مراحل در طرف دوم است.\\
\Ovalbox{\textbf{قوانین پایداری:}}
\begin{en}
	\item اگر ضرایب واکنشی n برابر شود، $\Delta$H واکنش باید در \censor{n} \censor{ضرب} شود.
	\item اگر جای واکنش‌دهنده(ها) با فرآورده(ها) عوض شود، $\Delta$H واکنش باید \censor{قرینه} شود(علامت \censor{منفی} بگیرد.)\\
\end{en}
\hrule
\vspace{4pt}
تمرین ۱: با توجه به
$\mathrm{\Delta H_1}$
در واکنش اول،
$\mathrm{\Delta H_2}$
و
$\mathrm{\Delta H_3}$
را به دست آورید:
\begin{flushleft}
	\begin{tabular}{l l}
		$\mathrm{; \Delta H_1=-395KJ}$             & $\ce{S (s) + \dfrac{3}{2} O_2 (g) -> SO_3 (g)}$ \\
		$\mathrm{; \Delta H_2=\censor{-790}KJ}$ & $\ce{2S (s) + 3O_2 (g) -> 2SO_3 (g)}$          \\
		$\mathrm{; \Delta H_3=\censor{395}KJ}$ & $\ce{SO_3 (g) -> S (s) + \dfrac{3}{2}O_2 (g)}$
	\end{tabular}
\end{flushleft}
\newpage
تمرین ۲: متان، ساده‌ترین هیدروکربن و نخستین عضو خانواده \censor{الکان} است، و بخش عمده \censor{گاز} \censor{طبیعی} را تشکیل می‌دهد. متان از \censor{تجزیه} گیاهان به وسیله \underline{باکتری‌های بی‌هوازی} «در آب» تولید می‌شود. نخستین بار، از سطح \censor{مرداب} جمع‌آوری شده و به \underline{گاز مرداب} معروف است. برای تهیه این گاز، می‌توان از واکنش روبه‌رو استفاده کرد:
\begin{flushleft}
	$\mathrm{\ce{C (\text{گرافیت, S}) + 2H_2 (g) -> CH_4 (g)} (\Delta H = ?)}$
\end{flushleft}
آزمایش‌ها و یافته‌های تجربی نشان می‌دهند که تامین شرایط بهینه برای انجام واکنش بالا، بسیار دشوار و پرهزینه است. برای تعیین $\mathrm{\Delta H}$ این واکنش، می‌توان از سه واکنش ترموشیمیایی دیگر بهره گرفت: ($\mathrm{\Delta H}$ واکنش بالا را محاسبه کنید.)
\begin{flushleft}
	\begin{tabular}{l l}
		$\mathrm{(\Delta H_1 = -393.5 KJ)}$ & $\circled{1} \ce{C (\text{گرافیت، S}) + O_2 (g) -> CO_2(g)}$   \\
		$\mathrm{( \Delta H_2 = -286 KJ)}$  & $\circled{2} \ce{H_2 (g) + \dfrac{1}{2}O_2 (g) -> H_2O (l)}$    \\
		$\mathrm{( \Delta H_3 = -890 KJ)}$  & $\circled{3} \ce{CH_4 (g) + 2O_2 (g) -> 2H_2O (l) + CO_2 (g)}$
	\end{tabular}
\end{flushleft}
تذکر: ترجیحا هر یک از مواد واکنش را در هر مرحله پیدا کنید که در مراحل دیگر نباشد.
\begin{flushleft}
	$\mathrm{\Delta H=}$
\end{flushleft}
\lin
تمرین ۳: آنتالپی واکنش کلی را محاسبه کنید: (خود را بیازمایید \underline{۲} صفحه \underline{۷۴})
\begin{flushleft}
	\begin{tabular}{l l}
		$\mathrm{; \Delta H_1 = -566 KJ}$ & $\ce{2CO (g) + O_2 (g) -> 2CO_2 (g)}$           \\
		$\mathrm{; \Delta H_2 = 181 KJ}$  & $\ce{N_2 (g) + O_2 (g) -> 2NO (g)}$             \\
		                                  & \tikz\draw [black] (0,0) -- (7cm,0pt);          \\
		$\mathrm{; \Delta H = ?}$         & $\ce{2CO (g) + 2NO (g) -> 2CO_2 (g) + N_2 (g)}$
	\end{tabular}
\end{flushleft}
\vspace{30pt}
\lin
\textbf{توجه: }
واکنش بالا، توسط شیمیدانان هواکرده، و برای تبدیل گاز‌های آلاینده CO و NO (که از اگزوز خودرو‌ها به هواکرده وارد می‌شوند) طراحی شده تا به گاز‌هایی با آلایندگی کمتر و پایداری \censor{بیشتر} تبدیل شوند.
\lin
تمرین ۴: (خود را بیازمایید \underline{۱} صفحه \underline{۷۴})\\
الف)
\begin{flushleft}
	\begin{tabular}{l l}
		$\mathrm{; \Delta H_1 = -286 KJ}$ & $\ce{H_2 (g) + \dfrac{1}{2}O_2 (g) -> H_2O (l)}$ \\
		$\mathrm{; \Delta H_2 = -196 KJ}$ & $\ce{2H_2O_2 (l) -> 2H_2O (l) + O_2 (g)}$       \\
		                                  & \tikz\draw [black] (0,0) -- (7cm,0pt);          \\
		$\mathrm{; \Delta H = ?}$         & $\ce{H_2 (g) + O_2 (g) -> H_2O_2 (l)}$
	\end{tabular}
\end{flushleft}
\vspace{2em}
ب) چون واکنش مستقیم
$\mathrm{ H_2}$ با $\mathrm{O_2}$ \censor{H2O}
تولید می‌کند که
$\dfrac{\text{پایدارتر}}{\text{ناپایدارتر}}$
است.
$\mathrm{H_2O_2}$
\censor{ناپایدار}
تر است و به \censor{اکسیژن} و \censor{آب} تجزیه می‌شود.
\newpage
تمرین ۵: (خود را بیازمایید \underline{۳}) الف) چون واکنش برخورد مستقیم C با
$\mathrm{O_2}$
، همواره \censor{CO2} تولید می‌کند
($\mathrm{CO_2}$
از
$\mathrm{CO}$ $\dfrac{\text{پایدارتر}}{\text{ناپایدارتر}}$
است.)\\
ب)\\
\lin
تمرین \underline{۶} (خود را بیازمایید \underline{۴}) الف) \censor{آمونیاک} پایدارتر است (سطح انرژی \censor{پایین} دارد.) دلیل: \textbf{تعداد پیوند} \underline{۲} مول آمونیاک از \underline{۱} مول هیدرازین \censor{بیشتر} است.\\
ب)\\
\\
\lin
\begin{center}
	\textbf{غذای سالم}
\end{center}
آهنگ واکنش، نشان می‌دهد هر تغییر شیمیایی، در چه گستره‌ای از \censor{زمان} رخ می‌دهد. آهنگ واکنش، معیاری برای تعیین زمان \censor{ماندگاری} مواد است.\\
هرچه گستره زمان انجام واکنش، \underline{کوچک‌تر} باشد، آهنگ انجام آن، \censor{بیشتر} است، و واکنش، \censor{سریعتر} تر انجام می‌شود.\\
برخی روش‌های افزایش زمان ماندگاری مواد غذایی:
\begin{iit}
	\item روش‌های قدیمی:
	\censor{خشک}
	کردن، تهیه \censor{ترشی} و \censor{نمک} سود کردن (شکل \underline{۱۰} صفحه \underline{۷۵})
	\item روش‌های جدید:
	تخلیه \censor{هوای} درون بسته‌بندی، \censor{انجماد} و \censor{ظروف کدر و مات}
	\begin{en}
		\item تهیه \censor{کنسرو}
		و افزودن \censor{نگه دارنده} از روش‌های جدید نگهداری مواد غذایی است.
		\item عوامل محیطی
		مانند \censor{رطوبت}، \censor{اکسیژن}، \censor{نور} و \censor{دما} در \underline{چگونگی} و \underline{زمان} نگهداری مواد غذایی موثر است.
		\item  پودر شدن
		مواد غذایی، (مانند قاووت) \censor{سطح}\censor{تماس} آنها با اکسیژن را افزایش می‌دهد و در نتیجه، سهت فساد ماده غذایی \censor{زیاد} می‌شود.
	\end{en}
\end{iit}

\begin{iit}
	\item واکنش‌های
	تخریب مواد غذایی، در محیط \censor{آبی} انجام می‌شود. با خشک کردن، \censor{رطوبت} تا حد زیادی حذف و ماندگاری زیاد می‌شود.
	\item استفاده
	از اسید خوراکی (ترشی) یا نمک (با \censor{غلظت} مناسب) امکان رشد موجودات ذره‌بینی را کم می‌کند.
\end{iit}
\textbf{نکته:}
تهیه و تولید سریعتر یا کندتر یک فرآورده (صنعتی، دارویی یا غذایی) بر \censor{کیفیت} و زمان \censor{ماندگاری} آن موثر است.
\Ovalbox{
	آهنگ انجام واکنش، در گستره‌ای از
	\censor{زمان}،
	با نام \censor{سرعت} واکنش بیان می‌شود.}
خود را بیازمایید صفحه \underline{۷۶}:\\
الف) کاهش \censor{دما} $\leftarrow$ کاهش \censor{سرعت} واکنش‌های فساد مواد $\leftarrow$ افزایش \censor{ماندگاری}\\
ب) جلوگیری از اثر مخرب \censor{نور} (و سایر امواج \censor{الکترو}\censor{مغناطیس}) بر روغن
پ) \censor{پودر} کردن مغز دانه‌های خوراکی $\leftarrow$ \censor{افزایش} سطح تماس مواد غذایی با \censor{اکسیژن هوا} $\leftarrow$ \censor{کاهش} ماندگاری
\lin
\begin{center}
	\textbf{مقایسه کیفی سرعت واکنش‌ها (شکل \underline{۱۲}) صفحه \underline{۷۸}}
\end{center}
الف) \underline{انفجار}، یک واکنش شیمیایی «\censor{بسیار} \censor{سریع}» است.\\
در انفجار، مقدار \censor{کمی} ماده منفجرشونده (حالت \censor{جامد} یا \censor{مایع})، «حجم» \censor{زیادی} از \censor{گاز‌های} داغ تولید می‌کند.
\newpage
ب) \underline{تشکیل رسوب}، واکنشی «\censor{سریع}» است. مثال:
\begin{flushleft}
	$\ce{NaCl(aq) + AgNO_3 (aq) -> AgCl(\quad) + NaNO_3 (aq)}\qquad \text{نام: \censor{نقره کلرید}}$
\end{flushleft}
پ) \underline{زنگ‌زدن}، واکنشی «\censor{کند}» است.\\
اشیای آهنی، در هوای \censor{مرطوب} زنگ می‌زنند. زنگار تولیدشده، \censor{زرد} و \censor{شکننده} است و \censor{فرو} می‌ریزد.\\
ت) \underline{پوسیدن کاغذ}، واکنشی «\censor{بسیار}\censor{کند}» است. کاغذ از \censor{سلولز} تشکیل شده و تجزیه آن به \censor{گلوگز} در گذر زمان، باعت «\censor{زرد}» و «پوسیده شدن» کاغذ از کتاب‌های قدیمی می‌شود.
\lin
\begin{center}
	\textbf{عوامل موثر بر سرعت واکنش}\qquad
	4+1\qquad
	$\left(^{\text{\underline{۱} عامل موثر اما ثابت}}_{\text{\underline{۴} عامل موثر و متغیر}}\right)$
\end{center}
\Ovalbox{
	$^{\circled{1}}$
	افزایش \censor{دما}،
	$^{\circled{2}}$
	افزایش \censor{غلظت} واکنش‌دهنده(ها)،
	$^{\circled{3}}$
	افزایش سطح \censor{تماس}،
	$^{\circled{4}}$
	استفاده از \censor{کاتالیزگر}
}
\\
\textbf{خود را بیازمایید صفحه \underline{۸۰} و \underline{۸۱}}:

\textbf{الف)} سرعت واکنش پتاسیم با آب (شکل سمت \censor{راست}) از واکنش شدیم با آب (سمت \censor{چپ}) \censor{بیشتر} است.
دلیل: خاصیت \censor{فلزی} و \censor{واکنش} پتاسیم از سدیم بیشتر است.\RTLfootnote{
	تغییر \censor{ماهت} واکنش‌دهنده (\censor{نوع} واکنش‌دهنده)، می‌تواند واکنش سریع‌تری به راه اندازد اما نمی‌تواند عاملی برای تغییر سرعت یک واکنش مشخص باشد. (خود را بیازمایید صفحه \underline{۸۰})
}\\
یعنی:
\Ovalbox{
	\textbf{
		تغییر دادن \censor{نوع}\censor{واکنش دهنده}، می‌تواند واکنش سریع‌تری به راه بیاندازد.
	}
}

\textbf{ب)} شعله آتش، \censor{گرد} آهن موجود در کپسول چینی را داغ و \censor{سرخ} می‌کند اما پاشیدن و پخش کردن گرد آهن بر روی \censor{شعله}، سبب \censor{سوختن} آن می‌شود. (شکل سمت \censor{راست}) (تذکر: اندازه ذرات در گرد آهن از براده آهن \censor{کوچکتر} است.)\\
\circled{$\star$}
عامل موثر بر سرعت: افزایش \censor{سطح}\censor{تماس} واکنش‌دهنده‌ها

\textbf{پ)} محلول \censor{بنفش}
{\footnotesize (رنگ) }
پتاسیم پرمنگنات
($\text{(aq)}$ \censor{نمیدونم})
با یک
\censor{اسید}
آلی در دمای اتاق به \censor{کندی} واکنش می‌دهد اما با گرم شدن محلول، به \censor{سرعت} بی‌رنگ می‌شود (واکنش می‌دهد) (شکل سمت \censor{راست})\\
\circled{$\star$}
عامل موثر بر سرعت: افزایش \censor{دما}

\textbf{ت)} الیاف آهن داغ و \censor{سرخ} شده (روی شعله) در هوا $\dfrac{\text{می‌سوزد}}{\text{نمی‌سوزد}}$ اما در ارلن پر از  اکسیژن \censor{می‌سوزد} (شکل سمت \censor{راست})\\
\circled{$\star$}
عامل موثر بر سرعت: افزایش \censor{غلظت} واکنش‌دهنده

\textbf{ث)} محلول هیدروژن پراکسید (\censor{نمیدونم} $\text{(aq)}$) در دمای اتاق به \censor{کندی} تجزیه‌شده و \censor{اکسیژن} تولید می‌کند:
\begin{flushleft}
	$\ce{H_2O_2 (aq) ->[\censor{نمیدونم} (aq)] H_2O (l) + O_2 (g)}\qquad \text{(موازنه کنید)}$
\end{flushleft}
در حالی که افزودن دو قطره از محلول پتاسیم یدید
($\mathrm{(aq) \censor{KI}}$)
سرعت واکنش را به طور چشمگیری \censor{افزایش} می‌دهد (شکل سمت \censor{چپ})\\
\circled{$\star$}
عامل موثر بر سرعت: \censor{کاتالیزگر}

\textbf{ج)} بیماران دارای مشکل تنفسی، در شرایط اضطراری نیاز به تنفس از کپسول \censor{اکسیژن} دارند. دلیل: \censor{غلظت} اکسیژن در کپسول اکسیژن از \censor{هواکره} بیشتر است و با هر بار عمل دم، اکسیژن بیشتری وارد ریه می‌شود.\\
\circled{$\star$}
عامل موثر بر سرعت: افزایش \censor{غلظت} واکنش‌دهنده\\
\newpage
\textbf{ح)} برخی افراد با مصرف کلم و حبوبات، دچار نفخ می‌شوند زیرا فاقد \censor{آنزیمی} هستند که آن‌ها را کامل و سریع هضم کند.\\
دلیل: آنزیم‌ها در بدن، نقش \censor{کاتالیزگر} را دارند و «کمبود» یا «فقدان» آن‌ها، واکنش‌های هضم را \censor{کند} می‌کند.\\
\circled{$\star$}
نقش \censor{کاتالیزگر} در سرعت واکنش

\textbf{خ)} واکنش سوختن قند آغشته به \censor{خاک} \censor{باغچه} سریع‌تر از سوختن قند در حالت عادی است.\\
دلیل: در خاک باغچه، \censor{کاتالیزگر} مناسب برای این واکنش وجود دارد.\\
\circled{$\star$}
نقش \censor{کاتالیزگر} در سرعت واکنش
\lin
\begin{center}
	\textbf{پیوند با صنعت}
\end{center}
در صنایع غذایی، علاوه بر بسته‌بندی، کنسرو‌سازی، انجماد و غیره، استفاده از مواد \censor{شیمیایی} به عنوان \censor{افزودنی} سبب افزایش زمان \censor{ماندگاری} و \censor{کمیت} مواد غذایی است. «\censor{افزودنی}ها»، مواد شیمیایی مانند \censor{نگه‌دارنده}، \censor{رنگ} دهنده‌ها و \censor{حجم.} دهنده‌ها هستند که به صورت هدف‌مند به مواد غذایی افزوده می‌شوند. یکی از این افزودنی‌ها «\censor{بنزوییک} اسید» است که به طور طبیعی در \censor{تمشک} و \censor{توت‌فرنگی} وجود دارد و به عنوان \censor{نگهدارنده} به مواد غذایی افزوده می‌شود. نگه‌دارنده‌ها، \censor{سرعت} واکنش‌های شیمیایی منجر به \censor{فساد} مواد غذایی را \underline{کاهش} می‌دهند. بنزوییک اسید به علت داشتن گروه COOH جزء اسید‌های \censor{کربکسیلی} است.
\RTLfootnote{مانند $\mathrm{CH_3COOH}$ با نام \censor{اتانوییک } اسید یا \censor{استیک} اسید}
ازطرفی، بنزوییک اسید، حلقه \censor{بنزنی} دارد پس جزء ترکیبات \censor{آروماتیک} نیز هست. (اسید \censor{کربوکسیل})
\begin{center}
	\begin{tabular}{l@{\quad یا \quad}l@{\quad یا \quad}l}
		\chemname{\censor{نمیدونم} - $\mathrm{COOH}$}{$\mathrm{C_{\censor{۱}}H_{\censor{۱}}O_{\censor{۱}}}$}&\chemfig{*6([,0.5]=-=-(-[2])=-)}&\chemname{\chemfig{*6([,0.5]=-=-(-[2](-[1])(-[3]))=-)}}{بنزوییک اسید}
	\end{tabular}
\end{center}
\lin
\textbf{پیوند با ریاضی صفحه \underline{۸2} ،\underline{۸3}}:

\textbf{۱)}
 کمیت \censor{مساحت}، سطح تماس تکه زغال را با شعله در هنگام سوختن نشان می‌دهد، چون در عمق زغال، واکنش سوختن به خوبی انجام \censor{نمیدهد} (به دلیل کافی $\dfrac{\text{بودن}}{\text{نبودن}}$ \censor{اٌو دو} در دسترس)

\textbf{۲)}
 سطح آن \censor{نمیدونم} برابر و حجم آن \censor{نمیدونم} برابر می‌شود (حجم تغییر \censor{نمیکند})

\textbf{۳)}
 گرد زغال نسبت به تکه زغال، \censor{سطح}\censor{تماس} بیشتری با \censor{اکسیژن} برای سوختن دارد و سرعت سوختن گرد زغال \censor{بیشتر} است. هرچه سطح تماس بیشتر و به \censor{گرد} (\censor{پودر}) نزدیک‌تر باشد، سرعت واکنش ان با سایر مواد یا تجزیه آن، \censor{زیاد} می‌شود.\\
\begin{tikzpicture}
	\draw[-Triangle, very thick](1, 0) -- (0, 0);
\end{tikzpicture}
 برخی واکنش‌های شیمیایی مانند گوارش، تنفس، تهیه دارو‌ها و تولید فرآورده‌های صنعتی، \censor{مفید} و \censor{ضروری} هستند.\\
\begin{tikzpicture}
	\draw[-Triangle](1, 0) -- (0, 0);
\end{tikzpicture}
در چنین واکنش‌هایی باید سرعت را \censor{افزایش} داد(تا فرآورده‌های گوناگون، با صرفه اقتصادی تولید شوند.)\\
\begin{tikzpicture}
	\draw[-Triangle, very thick](1, 0) -- (0, 0);
\end{tikzpicture}
برخی دیگر از واکنش‌ها مانند «\censor{خوردگی} وسایل آهنی»، «تولید \censor{آلاینده} ها» و «\censor{زرد} و \censor{پوسیده} شدن کاغذ»، \censor{زیان} بار و \censor{ناخواسته} هستند.
\begin{tikzpicture}
	\draw[-Triangle] (1, 0) -- (0, 0);
\end{tikzpicture}
درچنین واکنش‌هایی باید به دنبال راه‌هایی برای \underline{\censor{کاهش} سرعت} یا حتی \underline{\censor{توقف} نمودن} واکنش بود.\\
برای دستیابی به چنین اهدافی، باید از \censor{سینتیک} شیمیایی کمک گرفت.\\
\Ovalbox{
	سینتیک شیمیایی، به بررسی \censor{شرایط} و \censor{چگونگی} انجام واکنش‌ها و \censor{خواص}\censor{موثر} بر سرعت واکنش‌ها می‌پردازد.
}
\begin{center}
	\textbf{سرعت تولید یا مصرف مواد شرکت‌کننده در واکنش از دیدگاه \underline{کمی}}
\end{center}
سرعت واکنش در موارد زیادی باید با دقت اندازه‌گیری شود، یعنی باید سرعت را به شکل \censor{کمی} بیان کرد. برای این کار باید \censor{پیشرفت} واکنش را به صورت «عدد» بیان کنیم.
\begin{center}
	\Ovalbox{
		پیشرفت واکنش: مصرف \censor{واکنش‌دهنده} یا تولید \censor{فراورده}
	}
\end{center}
بدیهی است که پیشرفت واکنش در گستره‌ای از \censor{زمان} انجام می‌گیرد.\\
نمونه: شکل \underline{۱۴} صفحه \underline{۸۴}: در یک واکنش شیمیایی، \censor{رنگ} خوراکی موجود در محلول، وارد واکنش‌شده و در زمان \underline{۵} دقیقه تا مرز \censor{بی گ} شدن پیش رفته است. یعنی با پیشرفت واکنش، \censor{مقار} رنگ، \censor{کهش} می‌یابد و تقریبا به \censor{صفر} می‌رسد.
برای محابسه کمی سرعت واکنش، باید بدانیم که \censor{مقدار} رنگ مصرفی چقدر بوده و در چه \censor{زمان} مصرف شده است.

\textbf{خود را بیازمایید \underline{۱} صفحه \underline{۸4}}:
 با توجه به پرسش، در اینجا باید تغییرات \censor{مدار} (\censor{مل} مصرفی) را در واحد زمان اندازه‌گیری کنیم:
\begin{flushleft}
	$\mathrm{R=\dfrac{\Delta n}{\Delta t}=\dfrac{\censor{نمیدونم} (\censor{نمیدونم})}{\censor{نمیدونم} (\censor{نمیدونم})}=\censor{نمیدونم} (\qquad\qquad)}$
\end{flushleft}
\Ovalbox{
	\begin{minipage}{0.98\linewidth}
		«\censor{مصرف}»
		یک واکنش‌دهنده یا «\censor{تولید}» یک فرآورده در گستره \censor{زمانی} قابل اندازه‌گیری را «سرعت \censor{متوسط}» (مصرف یا تولید) آن ماده می‌نامند (
		$\mathrm{\bar{R}}$
		)
	\end{minipage}
}
$\Leftarrow$
چرا سرعت «متوسط»؟ چون به صورت عادی، با گذشت زمان، سرعت مصرف یا تولید مواد \censor{کمتر} می‌شود یعنی در گستره زمانی انجام واکنش، \underline{معمولا} سرعت، \censor{یکسان} نیست.
\lin
خود را بیازمایید 
\underline{۲}:\textbf{ الف)}
 واکنش‌پذیری\quad روی $\bigcirc$ مس
\begin{center}
	\begin{vwcol}[widths={0.8,0.2}, sep=.8cm, justify=flush,rule=0pt,indent=10pt]
		\begin{minipage}{.2\linewidth}
			\Ovalbox{حذف یون تماشاچی}
		\end{minipage}\hfil
		\begin{minipage}{.8\linewidth}
			\begin{latin}
				\begin{tabular}{lllllllll}
					&$\mathrm{Zn (s)}$&+&$\mathrm{CuSO_4 (aq)}$&$\rightarrow $&$\mathrm{\qquad (aq)}$&+&$\mathrm{\qquad (s)}$&\\
					&$\mathrm{Zn (s)}$&+&$\mathrm{\qquad (aq)}$&$\rightarrow $&$\mathrm{\qquad (aq)}$&+&$\mathrm{\qquad (s)}$&\\
					$رنگ
					\rightarrow$&\censor{نقره‌ای}&&\censor{آبی}&&\censor{بی‌رنگ}&&\censor{سرخفام}&
				\end{tabular}
			\end{latin}
		\end{minipage}
	\end{vwcol}
\end{center}
\vspace{2em}
در این واکنش، \underline{«فلز \censor{Zn}» }الکترون \censor{ازدستمیدهد} و \underline{«کاتیون \censor{مس}»} الکترون \censor{گیرنده} است، پس:\\واکنش‌پذیری روی از مس \censor{بیشتر} است. (واکنش‌پذیری فلز، تقریبا معادل الکترون \censor{دهی} آن است.)

\textbf{ب)}
 با گذشت زمان، $\mathrm{Cu^{2+}}$ به \censor{Cu} تبدیل می‌شود:\\
مقدار (و غلظت)
$\mathrm{Cu^{2+}}$
 \censor{کاهش}
 می‌یابد. ( رنگ \censor{آبی} محلول، کم و کم‌تر می‌شود.) و مقدار $\mathrm{Cu}$
 \censor{بیشتر}
 می‌شود. (از \censor{محلول} خارج می‌شود و بر سطح \censor{تیغه}\censor{روی} (یا \censor{کف} ظرف) می‌نشیند)
 
\textbf{ پرسش}
) تغییرات مقدار $\mathrm{Zn}$ و $\mathrm{Zn^{2+}}$ چگونه است؟\\
 با گذشت زمان، \censor{Zn} به \censor{Zn2} تبدیل می‌شود. مقدار (و غلظت) $\mathrm{Zn^{2+}}$ \censor{افزایش} می‌یابد. (محلول نهایی \censor{بی} رنگ است.) و $\mathrm{Zn}$ \censor{zn} می‌شود (مقداری از تیغه روی \censor{خورده} می‌شود.)\\
 نکته: اگر فرض کنیم که فلز (مس) تولید شده، فقط روی تیغه (روی) بنشیند، تغییر جرم تیغه، از مقایسه جرم روی \censor{مصرف} شده با جرم مس \censor{اضافه} شده به تیغه، به دست می‌آید.
 \newpage
\textbf{ پ)}
\begin{flushleft}
	 $\mathrm{\bar{R}_{Ca^{2+}}=\dfrac{\Delta n}{\Delta t} = \dfrac{\censor{نمیدونم} (\qquad)}{\censor{نمیدونم} (\qquad)} = \censor{نمیدونم} (\qquad)}$
\end{flushleft}
 \lin
\textbf{ با هم بیندیشیم صفحه \underline{۸۵}}:
 \begin{flushleft}
 	$\ce{CaCO_3 (s) + \textbf{2}HCl (aq) -> \censor{نمیدونم} (aq) + \censor{نمیدونم} (l) + \censor{نمیدونم} (g)}$
 \end{flushleft}
 
 \textbf{الف)}
 (\censor{CO2}) \censor{نمیدونم} تولیدی، از \censor{سامانه} خارج و \censor{جرم} مخلوط باقی‌مانده \censor{کم} می‌شود.
 
 \textbf{ب)} در کتاب درسی
 
 \textbf{پ)} با گذشت زمان، مجموع جرم گاز آزاد شده، \censor{بیشتر} می‌شود (اما در مقایسه بازه‌های زمانی \underline{۱۰} ثانیه‌ای، هرچه زمان می‌گذرد، در بازه‌های بعدی، گاز \censor{کمتری} آزاد می‌شود). مثال:\\
 گاز آزاد شده در \underline{۱۰} ثانیه اول: \censor{۰.۶۶} گاز آزاد شده در \underline{۱۰} ثانیه دوم:\censor{۰.۴۴}
 
 \textbf{ت)}در ثانیه \censor{۵۰} به پایان می‌رسد چون \censor{جرم} مخلوط پس از آن تغییر \censor{نکرده} است.\\
 \Ovalbox{
	\textbf{تذکر مهم:}
	برای اندازه‌گیری سرعت،‌ باید بازه زمانی \censor{۰} تا \censor{۵۰} $\mathrm{(s)}$ را در نظر گرفت.
}

\textbf{تمرین (با هم بیندیشیم) ۲ و ۳}: در کتاب درسی 

\textbf{تمرین (با هم بیندیشیم ۴)}
: سرعت \censor{متوسط} تولید $\mathrm{CO_2}$ (\qquad) با گذشت زمان \censor{کم} می‌شود.\\
یعنی: سرعت واکنش‌های شیمیایی به تدریج \censor{کند} و \censor{کند} تر می‌شود. (واکنش در ابتدا نسبتا \censor{سریع} تر و در پایان نسبتا \censor{کند} تر انجام می‌شود.)\\
\textbf{دلیل:}
با پیشرفت واکنش، مقدار \censor{واکنش‌دهنده} ها به تدریج چه تغییری می‌کند؟ چرا؟

\textbf{با هم بیندیشیم تمرین ۵}: شیب نمودار همان
$\mathrm{\dfrac{\Delta \censor{n}}{\Delta \censor{t}}}$
است و به مرور \censor{کم} می‌شود (
$\mathrm{\dfrac{\Delta \censor{n}}{\Delta \censor{t}}}$
در واقع بیانگر \censor{سرعت} است که به تدریج \censor{کم} می‌شود.) \censor{مول} های تولیدی هر سه فرآورده در این واکنش برابر است یعنی به \censor{مولار} یکسان تولید می‌شوند.\\
$\Delta t$
نیز برای همه \censor{مواد} واکنش (از جمله فرآورده‌ها) \censor{یکسان} است. نتیجه: سرعت متوسط \censor{تولید} فرآورده‌ها \censor{یکسان} است. (چون \censor{مقدار} اولیه فرآورده‌ها صفر بوده، نمودار مقدار-زمان برای این سه ماده، یکسان است.)\lin
تمرین \circled{۱}: سرعت متوسط تولید $\mathrm{CO_2}$ را در بازه زمانی \underline{۰} تا \underline{۱۰} ثانیه به کمک جدول صفحه \underline{۸۶} به دست آورید.

تمرین \circled{۲}: سرعت متوسط تولید $\mathrm{CaCl_2}$ را در بازه زمانی \underline{۰} تا \underline{۱۰} ثانیه به کمک نمودار صفحه \underline{۸۷} به دست آورید.
\lin
\textbf{با هم بیندیشیم \underline{۵} صفحه \underline{۸۸}}:

نتیجه: 
$\mathrm{\bar{R}_{{CO_2}_{(0-10s)}}\bigcirc\bar{R}_{{CaCl_2}_{(0-10s)}}} \leftarrow$ 
نشانه: \censor{ضریب} \censor{استوکیمتری} یکسان در واکنش موازنه شده.

نکته: شیب هر دو نمودار صفحه \underline{۸۸} و \underline{۸۹}، به تدریج \censor{کم} می‌شود، چون سرعت به مرور در حال \censor{کمتر} شدن است.

تمرین \circled{۳}: عبارات زیر را با نوشتن کلمه «کاهش» یا «افزایش» کامل کنید. در واکنش شیمیایی و با گذشت زمان:\\
\begin{tabular}{r@{\qquad}r@{\qquad}r}
	مقدار واکنش‌دهنده: & شیب نمودار مصرف واکنش‌دهنده: & سرعت متوسط مصرف واکنش‌دهنده:\\
	مقدار فرآورده: & شیب نمودار تولید فرآورده: & سرعت متوسط تولید فرآورده:
\end{tabular}
\\
\\
\textbf{خود را بیازمایید \underline{۱} صفحه \underline{۸۹}}:
$\mathrm{CaCO_3  ~ \bigcirc HCl}$
$\bigcirc$
به ازای مصرف
$\bigcirc$
مول
$\mathrm{CaCO_3}$،
$\bigcirc$
مول
$\mathrm{HCl}$
مصرف می‌شود؛\\
یعنی
$\mathrm{\Delta n}$
برای \censor{HCl} باید \underline{۲} برابر \censor{CaCO} باشد ($\mathrm{\Delta t}$ برای هر دو \censor{یکسان}) $\leftarrow$
$\mathrm{\bigcirc \bar{R}_{HCl} = \bigcirc \bar{R}_{CaCO_3}}$

\Ovalbox{
نتیجه: نسبت سرعت (\censor{مصرف} یا \censor{تولید}) مواد واکنش، همان نسبت \censor{ضریب} \censor{استوکیمتری}  آن‌ها است.
}
\newpage
\textbf{خود را بیازمایید \underline{۲} صفحه \underline{۹۰}}:
\begin{flushleft}
	$\mathrm{\ce{2SO_2 (g) + O_2 (g) -> 2SO_3 (g)} \rightarrow\left(\dfrac{\bar{R}_{SO_2}}{\qquad}=\dfrac{\qquad}{\qquad}=\dfrac{\qquad}{\qquad}\right)
		\rightarrow} $
	\\
	$\mathrm{\bar{R}_{SO_2} = \qquad = \censor{3}\bar{R}_{O_2} = \quad x \qquad = \qquad \dfrac{mol}{s}\times\dfrac{\qquad}{\qquad} = \qquad (\dfrac{mol}{\quad})}$
\end{flushleft}

\textbf{خود را بیازمایید \underline{۳}}:

\textbf{ الف)}
  \censor{کم} می‌شود چون $\dfrac{\text{واکنش‌دهنده}}{\text{فرآورده}}$ است و \censor{مصرف} می‌شود (در نمودار با گذشت \censor{زمان}، \censor{مقدار} ش کم می‌شود.)
 
\textbf{ب)} علامت $\bigcirc$ دارد، چون $\mathrm{\Delta n}$ برای واکنش‌دهنده، عددی $\bigcirc$ است.\\
\Ovalbox{
برای واکنش‌دهنده‌ها
$\mathrm{\Delta n \bigcirc 0}$
و برای فرآورده‌ها
$\mathrm{\Delta n \bigcirc 0}$
است.
}

\textbf{پ)} برای آن که سرعت، همواره عددی $\bigcirc$ گزارش شود سرعت مصرف واکنش‌دهنده به صورت
$\mathrm{\left(\bar{R}=\qquad\right)}$
نوشته می‌شود، (و برای فرآورده، سرعت تولید به صورت
$\mathrm{\left(\bar{R}=\qquad\right)}$
نوشته می‌شود.)

\textbf{ت)}\\
$\mathrm{\bar{R} = -\dfrac{\qquad}{\qquad} = -\dfrac{\quad-\quad}{\quad-\quad} = -\dfrac{\quad-\quad}{\quad-\quad} = \censor{نمیدونم} \times \censor{نمیدونم} \left(\dfrac{mol}{s}\right) \times \dfrac{\censor{نمیدونم}s}{\censor{نمیدونم} min} = \censor{نمیدونم} \left(\dfrac{mol}{min}\right)}$
\lin

\begin{minipage}{0.7\textwidth}
	
تمرین \circled{۱}: نمودار تغییرات مقدار مواد در برابر زمان را برای واکنش فرضی $\mathrm{A\rightarrow B}$ رسم کنید. (فرض کنید واکنش کامل شده است.)\\
الف) زمان پایان واکنش، برای همه مواد واکنش، \censor{یکسان} است. (همه باید به \censor{t} یکسان ختم شوند.)
\end{minipage}%
\begin{minipage}{0.3\textwidth}
\begin{flushleft}
	\begin{tikzpicture}[scale=1]
		\draw[->] (0,0) -- (3,0) node[right] {t};
		\draw[->] (0,0) -- (0,3) node[left,yshift=-1.3cm] {{n}};
	\end{tikzpicture}
\end{flushleft}
\end{minipage}
ب) حتما (در اینجا) باید عرض یکسان \censor{داشته} باشند. (چون در اینجا مول مصرفی واکنش‌دهنده با مول تولیدی فرآورده \censor{یکسان} است) به دلیل \censor{ضریب} \censor{استوکیمتری} یکسان در واکنش.\\
پ) همه باید شیب \censor{کند} شونده داشته باشند (کاهش تدریجی \censor{سرعت})
\\
تمرین \circled{۲}: نمودار مقدار-زمان را برای واکنش
$\ce{N_2 (g) + H_2 (g) -> NH_3 (g)}$
(واکنش را موازنه کنید و فرض کنید که واکنش کامل شده است و مقدار اولیه واکنش‌دهنده‌ها متناسب با ضریب استوکیومتری آن است)

\begin{minipage}{.7\textwidth}
	تعریف واکنش کامل:
\end{minipage}&
\begin{minipage}{.3\textwidth}
	\begin{tikzpicture}[scale=1.25]
		\draw[->] (0,0) -- (3,0) node[right] {t};
		\draw[->] (0,0) -- (0,3) node[left,yshift=-2cm] {{n}};
	\end{tikzpicture}
\end{minipage}

\lin
نکته \circled{1}: در نمودار، نقطه شروع برای فرآورده(ها)، \underline{معمولا} \censor{۰} است. (چون در ابتدای واکنش، وجود \censor{ندارد})

نکته \circled{2}: در نمودار، نقطه شروع برای واکنش‌دهنده‌(ها)، لزوما همان ضرایب استوکیومتری \censor{نیست}.

نکته \circled{3}: اگر ضریب استوکیومتری برای دو واکنش‌دهنده، یکسان باشد، مقدارشان در شروع و در طول واکنش، لزوما با هم برابر \censor{نیست} اما اگر مقدار آنها در ابتدای واکنش یکسان باشد، نمودار آن دو، \censor{یکسان} خواهد بود.

نکته \circled{4}: اگر ضرایب استوکیومتری برای دو واکنش‌دهنده یکسان باشد، شیب نمودار مصرف آن‌ها لزوما با هم برابر \censor{است}.

نکته \circled{5}: اگر ضریب استوکیومتری برای دو فرآورده یکسان باشد، نمودار آن‌ها \censor{منطبق} است (اگر مقدار اولیه آن‌ها \censor{برابر} باشد.)

نکته \circled{6}: اگر ضریب استوکیومتری دو ماده (واکنش‌دهنده یا فرآورده) یکسان باشد، \censor{قدر} \censor{مطلق} شیب نمودار آن‌ها لزوما برابر \censor{است}.

نکته \circled{7}: ضرایب استوکیومتری مواد واکنش، لزوما مقدار واقعی آن‌ها را بیان \censor{نمیکند} و حتما \censor{نیست} تغییر مقدار (
$\Delta n$
)
آن‌ها را نشان \censor{میدهد}.\\
\\
\Ovalbox{
	\begin{minipage}{0.98\linewidth}
		\textbf{تذکر بسیار مهم:}
	\\
	مقدار نهایی فرآورده در \underline{۳} حالت یکسان \censor{است} و سرعت تولید فرآورده در \underline{۳} حالت یکسان \censor{نیست}
	\end{minipage}
}\\\\
خود را بیازمایید صفحه \underline{۸۹} و \underline{۹۰}
\begin{flushleft}
	$\mathrm{A:}$\\
$\mathrm{B:}$\\
$\mathrm{C:}$\\
\end{flushleft}
\lin
\begin{center}
	\textbf{غلظت مواد خالص}
\end{center}
می‌دانیم که غلظت مول محلول‌ها (مواد $\dfrac{\text{خالص}}{\text{ناخالص}}$) از رابطه 
$\mathrm{M = \dfrac{\qquad}{\qquad}}$
با یکای (\quad . \qquad) به دست می‌آید برای مواد خالص نیز خالص نیز غلظت مولی تعریف می‌شود (از تقسیم «چگالی» بر «جرم مولی»)
\begin{flushleft}
	$\mathrm{M = \dfrac{d (\qquad . \quad)}{\text{جرم مولی} (\qquad . \quad)}}\qquad\qquad \text{یکای M} (\qquad \qquad)$
\end{flushleft}
در مورد غلظت «گاز‌ها»، یک تفاوت مهم با «مایع» و «جامد» وجود دارد:
\\
	$
	\left.
	\begin{tabular}{r}
		\text{
			تغییر مقدار «مایع» یا «جامد» $\leftarrow$
			$\dfrac{\text{تغییر}}{\text{عدم تغییر}}$
			حجم $\leftarrow$ چگالی $\dfrac{\text{تغییر می‌کند}}{\text{ثابت است}}$ $\leftarrow$ غلظت $\dfrac{\text{ثابت است}}{\text{تغییر می‌کند}}$.
		}\\
		\text{
			تغییر مقدار «گاز» (در ظرف سربسته) $\leftarrow$ حجم، تغییر \censor{نمیم} (حجم \censor{نمیم}) $\leftarrow$ چگالی \censor{نمیم} $\leftarrow$ غلظت \censor{نمیم}
		}
	\end{tabular}
	\right\}
	$
\\
نتیجه:
\Ovalbox{
	\begin{minipage}{0.9\linewidth}
		با مصرف یا تولید مواد جامد یا مایع، غلظت آن‌ها \censor{ثابت} \censor{می‌ماند}\\
		با مصرف یا تولید مواد گازی، غلظت آن‌ها \censor{تغییر}\censor{می‌کند}
	\end{minipage}
}
\\
\\
در محاسبات سرعت، در مورد \censor{گاز} (و نیز \censor{محلول}) می‌توان؛
\\
علاوه بر تغییر \censor{مقدار} در برابر زمان (که برای همه مواد استفاده می‌شود) از تغییر \censor{غلظت} نیز سرعت را محاسبه کرد.\\
در مورد مواد مایع و جامد، فقط به کمک تغییر \censor{مقدار} در برابر زمان، می‌توان سرعت را اندازه‌گیری کرد.\\
در مورد مواد مایع و جامد، نمودار غلظت - زمان را $\dfrac{\text{می‌توان}}{\text{نمی‌توان}}$ رسم کرد ولی سرعت را نشان $\dfrac{\text{می‌دهد}}{\text{نمی‌دهد}}$.
\lin
	\begin{minipage}[t]{0.4\textwidth}
		تمرین \circled{۱}: نمودار‌های تقریبی «مقدار - زمان» و «غلظت - زمان» را برای فرآورده‌های واکنش
		$\ce{CaCO3 (s) -> CaO (\quad) + CO2 (\quad)}$
		رسم کنید.
	\end{minipage}%
	\begin{minipage}[t]{0.3\textwidth}
		\begin{center}
			
				\begin{tikzpicture}[scale=1.25]
					\draw[->] (0,0) -- (3,0) node[right] {زمان};
					\draw[->] (0,0) -- (0,3) node[left,yshift=-2cm] {{غلظت}};
				\end{tikzpicture}
			
		\end{center}
	\end{minipage}%
	\begin{minipage}[t]{0.3\textwidth}
		\begin{center}
			\begin{tikzpicture}[scale=1.25]
				\draw[->] (0,0) -- (3,0) node[right] {زمان};
				\draw[->] (0,0) -- (0,3) node[left,yshift=-2cm] {{مقدار}};
			\end{tikzpicture}
		\end{center}
	\end{minipage}

تمرین \circled{۲}: اگر در واکنشی \underline{۲} ماده A و B حضور داشته باشند و
$\mathrm{2\bar{R}_A = 3\bar{R}_B}$
باشد، معادله واکنش به چه صورت‌هایی می‌تواند نوشته شود؟ \censor{۴} حالت!

تمرین \circled{۳}: اگر در واکنشی، \underline{۲} ماده A و B حضور داشته باشند و 
$\mathrm{\dfrac{+\Delta n_{(A)}}{2\Delta t} = \dfrac{-\Delta n_{(g)}}{3\Delta t}}$\RTLfootnote{
$\dfrac{\text{لزوما}}{\text{حتما}}$
به این معنی $\dfrac{\text{نیست}}{\text{است}}$ که B، واکنش‌دهنده است، شاید \censor{نمیدونشساjdsankdsajjnسیاتیسذابذتم}!
}
باشد، معادله واکنش به چه صورت‌هایی می‌تواند نوشته شود؟ \censor{۲} حالت!
\lin
\begin{center}
	\textbf{سرعت واکنش}
\end{center}
دیدیم که سرعت مصرف یا تولید مواد واکنش، به ضریب استوکیومتری آن‌ها در واکنش موازنه شده بستگی \censor{دارد}. یعنی سرعت مواد گوناگون در واکنش، ممکن است با هم برابر باشند یا نباشند. برای درک آسان‌تر «پیشرفت واکنش» در واحد «زمان»، از کمیت دیگری به نام «\textbf{سرعت واکنش}» استفاده می‌کنند.\\
\Ovalbox{
	سرعت واکنش، از تقسیم سرعت مصرف یا تولید هر ماده بر \censor{ضریب} \censor{استوکیمتری} آن به دست می‌آید.
}\\
مثلا در مورد واکنش
$\ce{\bigcirc N2 (g) + \bigcirc H2 (g) -> \bigcirc NH3 (g)}$
می‌توان نوشت:
\begin{flushleft}
	$\mathrm{
		R_{\text{واکنش}} = \dfrac{\bar{R}_{N_2}}{\qquad} = \dfrac{\qquad}{\qquad} = \dfrac{\qquad}{\qquad}
	}$\qquad (
	با هم بیندیشیم
	\circled{۱}
	ب و پ و ت)
\end{flushleft}
\lin
نکته:  سرعت واکنش، با سرعت مصرف یا تولید ماده‌ای برابر است که ضریب استوکیومتری آن $\bigcirc$ باشد. (با هم بیندیشیم \circled{ث})
\lin
با هم بیندیشیم \circled{۱} صفحه \underline{۹۶}: الف)
\begin{flushleft}
	$
		\mathrm{
			\dfrac{{\bar{R}}_{N_2}}{\qquad} = 
			\dfrac{{\bar{R}}_{NH_3}}{\qquad}
			\rightarrow
			{\bar{R}}_{N_2} = \qquad
		}
	$\\
	\vspace{1em}
	$
		\mathrm{
			\dfrac{{\bar{R}}_{H_2}}{\qquad} = 
			\dfrac{{\bar{R}}_{N_2}}{\qquad}
			\rightarrow
			{\bar{R}}_{H_2} = \qquad
		}
	$
\end{flushleft}

ث)
\begin{flushleft}
	$
		\mathrm{
			R_{\text{کاهش}} = \bar{R} \censor{نمیدونم}
		}
	$\\
	\vspace{1em}
	$
		\mathrm{
			R_{واکنش} = -\dfrac{\Delta n (N_2)}{\Delta t} = -\dfrac{\Delta n (H_2)}{\qquad} = +\dfrac{\Delta n (NH_3)}{\qquad}
		}
$
\end{flushleft}
\lin
با هم بیندیشیم \circled{۲} صفحه \underline{۹۱}
\begin{flushleft}
	\begin{tabular}{l@{\quad}l@{\quad}l@{\quad}l@{\quad}l@{\quad}l}
	& \qquad\qquad (aq) & $\rightarrow$ & $\mathrm{H_2O (l)}$ & + & $\mathrm{C_{12}H_{22}O_{11} (aq)}$\\
	& \censor{نمیدونم} &						   &						&	   &	\text{مالتوز}
	\end{tabular}
\end{flushleft}
\ovalbox{
سمنو، که از جوانه گندم تهیه می‌شود، دارای قند \censor{گلوگز} است، که طبق واکنش بالا از قند \censor{مالتوز} تولید می‌شود.
}

الف)
\begin{flushleft}
	$\mathrm{\bar{R}_{Glucose (0-3 min)} = }$\\
	$\mathrm{\bar{R}_{Maltose (0-3 min)} = }$
\end{flushleft}
ب)
\begin{flushleft}
	$\mathrm{R_{\text{واکنش} (0-7 min)} = }$\\
	$\mathrm{R_{\text{واکنش} (7-14 min)} = }$
\end{flushleft}
پ)
\begin{tabular}{r}
	a
	مربوط به \censor{مالتوز} است چون غلظت آن به تدریج \censor{کم} می‌شود.\\
	b
	مربوط به \censor{گلوگز} است چون غلظت آن به تدریج \censor{زیاد} می‌شود.
\end{tabular}
\lin
تمرین: بدون مراجعه به نمودار، سرعت واکنش را در موارد زیر مقایسه کنید:
\begin{multicols}{3}\setlength{\columnseprule}{0pt}
	$\mathrm{R_{(0-3 min)} \bigcirc R_{(0-7 min)}}$ \vfil\null
	 $\mathrm{R_{(0-7 min)} \bigcirc R_{(7-14 min)}}$ \vfil\null
	  $\mathrm{R_{(0-7 min)} \bigcirc R_{(0-14 min)}}$
\end{multicols}
\lin
\begin{center}
\textbf{	پیوند با زندگی: «خوراکی‌های طبیعی رنگین، باز دارنده‌هایی مفید و موثر»}
\end{center}

سبزی‌ها و میوه‌ها حاوی \censor{ریز} \censor{مغذی}هایی هستند که در حفظ سلامت بافت و اندام‌ها دخالت دارند.

برخی از آن‌ها به عنوان \censor{بازدارنده} از انجام واکنش‌های ناخواسته (به دلیل حضور \censor{رادیکال}) جلوگیری می‌کنند.


\Ovalbox{
	رادیکال، گونه \censor{فعال} و \censor{ناپایدار} است که در ساختار خود، الکترون \censor{جفت} \censor{نشده} (\censor{تک} الکترون) دارد.
}


یعنی رادیکال اتم یا اتم‌هایی دارد که از قاعده \censor{هشت} \censor{تایی} پیروی نمی‌کنند (مانند $\mathrm{NO}$ یا $\mathrm{NO_2}$)\\
رادیکال‌ها، واکنش‌پذیری \censor{بالایی} دارند. در واکنش‌های متنوع و پیچیده در بدن، رادیکال‌هایی به وجود می‌آیند اکه اگر به وسیله \censor{بازدارنده‌ها} جذب نشوند با انجام واکنش‌های \censor{سریع} به بافت‌ها آسیب می‌رسانند. مصرف خوراکی‌های حاوی «بازدارنده‌ها» رادیکال‌ها را به دام می‌اندازد و سرعت واکنش‌های \censor{ناخواسته}، \censor{کاسته} می‌شود. \underline{هندوانه} و \underline{گوجه‌فرنگی} محتوی «\censor{لیکوپن}» هستند که فعالیت رادیکال‌ها را کاهش می‌دهند. لیکوپن، \censor{۱۳} پیوند دوگانه، \censor{۴۰} کربن و \censor{۵} ه۶یدروژن دارد. (شکل \underline{۱۵} صفحه \underline{۸۹})
تعداد $\mathrm{H}$ در لیکوپن: \censor{108}

\begin{center}
	\textbf{«غذا، پسماند و ردپای آن»}
\end{center}
\Ovalbox{چهره آشکار}
حدود ۳۰٪ غذایی که در جهان فراهم می‌شود به مصرف \censor{نمی‌رسد} و به \censor{زباله} تبدیل می‌شود یا از بین می‌رود.\\
\Ovalbox{چهره پنهان}
همه \censor{منابعی} که در تهیه غذا، از آغاز تا سر سفره سهم داشته‌اند.\\
تولید گاز‌های \censor{گلخانه‌ای} به ویژه \censor{CO2} که سهم تولید این گاز در \censor{ردپای} غذا، از سوختن «سوخت‌ها» \censor{بیشتر} است.

افزایش جمعیت جهان، موجب \underline{افزایش تقاضا} برای تامین \censor{غذا}، منابع \censor{آب}، \censor{انرژی}، مواد \censor{اولیه} و زمین بیشتر است. با این روند، ردپای غذا در محیط زیست \censor{بیشتر} شده و مساخت مورد نیاز برای تامین اقلام ضروری زندگی، بیشتر خواهد شد. مدیران جامعه جهانی باید با \censor{طراحی} و \censor{انتخاب} راه حل‌هایی احرایی، مناسب و هماهنگ، «\censor{بهروری}» را در مراحل \censor{تولید} و \censor{تامین} غذا افزایش دهند تا \censor{ردپا} آن کاهش یابد.

\lin

\textbf{خود را بیازمایید صفحه ۹۳}:

\begin{tabular}{rrrr}
	خرید به اندازه نیاز & $\bigcirc$ & کاهش مصرف گوشت و لبنیات & $\bigcirc$ \\
	استفاده از غذا‌های بومی فصلی & $\bigcirc$ & کاهش مصرف غذا‌های فرآوری شده & $\bigcirc$
\end{tabular}

\begin{tabular}{rr}
	\circled{1}
	کاهش مصرف انرژی &
	\circled{2}
	طراحی تولید مواد و فرآورده‌های شیمیایی سالم‌تر \\
	\circled{3}
	کاهش تولید زباله و پسماند &
	\circled{4}
	کاهش ورود مواد شیمیایی ناخواسته به محیط‌زیست
\end{tabular}













\begin{center}
	\textbf{نمودار‌های مقدار - زمان}
\end{center}
نقطه شروع نمودار (روی محور \censor{نمیدونم}) برای هر ماده 
$\equiv$
\censor{نمیدونم}\censor{نمیدونم}\censor{نمیدونم}\\
با مقایسه «شیب» نمودار مقدار-زمان برای دو ماده، نسبت \censor{نمیدونم} آن‌ها در واکنش \censor{نمیدونم} به دست می‌آید.

واکنش کامل:\\
واکنشی که در آن، $\dfrac{\text{همه}}{\text{دست کم یکی از}}$ واکنش‌دهنده‌ها، به طور کامل \censor{نمیدونم}.
\newpage

\begin{center}
	\textbf{
		نمودار‌های مقدار - زمان
	}
\end{center}
\begin{iit}
	\item
	نقطه شروع نمودار (روی محور \censor{نمیدونم}) برای هر ماده 
	$\equiv$
	\censor{نمیدونم} \censor{نمیدونم} \censor{نمیدونم}
	\item 
	با مقایسه «شیب» نمودار مقدار - زمان برای دو ماده، نسبت \censor{نمیدونم} \censor{نمیدونم} آن‌ها در واکنش \censor{نمیدونم} \censor{نمیدونم} به دست می‌آید.
\end{iit}

واکنش کامل:

واکنشی که در آن، $\frac{\text{همه}}{\text{دست کم یکی از}}$ واکنش‌دهنده‌ها، به طور کامل \censor{نمیدونم} \censor{نمیدونم}.

واکنش \censor{نمیدونم}

با واکنش‌دهنده‌های \censor{نمیدونم} و \censor{نمیدونم}

$\left.
\begin{tabular}{r}
	مقدار اولیه واکنش‌دهنده‌ها: \censor{نمیدونم}\\
	ضریب استوکیومتری: \censor{نمیدونم}
\end{tabular}
\right\}$

مثال: واکنش (
$\bigcirc A + \bigcirc B \rightarrow \censor{نمیدونم} $)
با مقدار \censor{نمیدونم} شروع شده

واکنش \censor{نمیدونم}

با واکنش‌دهنده‌های \censor{نمیدونم} \censor{نمیدونم} و \censor{نمیدونم}

$\left.
مقدار اولیه واکنش‌دهنده‌ها: \censor{نمیدونم}\\
ضریب استوکیومتری: \censor{نمیدونم}
\right\}$

مثال: واکنش (
$\bigcirc A + \bigcirc B \rightarrow \censor{نمیدونم}$)

با \censor{نمیدونم} مول A و \censor{نمیدونم} مول B شروع شده

واکنش \censor{نمیدونم}: واکنشی که در آن، $\frac{\text{همه}}{\text{$\frac{\text{دست کم یکی از}}{\text{هیچ یک از}}$}}$ واکنش‌دهنده‌ها به طور کامل \censor{نمیدونم} \censor{نمیدونم}

واکنش \censor{نمیدونم}

واکنش \censor{نمیدونم} \censor{نمیدونم}: واکنشی که سرعت آن در بازه زمانی مشخصی، \censor{نمیدونم} است (\censor{نمیدونم} نیست.)
\lin

واکنش \censor{نمیدونم}: واکنش \censor{نمیدونم} \censor{نمیدونم} که در بازه زمانی مشخصی؛ برگشت
$R_{دفن}\bigcirc R_{برگشت}$
شده است. (
$t_1$: لجظه رسیدن به \censor{نمیدونم}
)
از لحظه رسیدن به تعادل، مقدار مواد حتما \censor{نمیدونم} است اما لزوما \censor{نمیدونم} نیست.

نمودار‌های سرعت - زمان

واکنش \censor{نمیدونم}

واکنش \censor{نمیدونم}

واکنش \censor{نمیدونم} و \censor{نمیدونم}
\end{document}


