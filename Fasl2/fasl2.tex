\documentclass[a4paper,12pt]{article}
%\usepackage[top=1in,margin=5em]{geometry}
\usepackage[top=7em,bottom=1pt,right=0.8in,left=0.8in,headheight=65pt,headsep=1cm]{geometry}
\usepackage{tikz-page}
\usepackage{tikz}
\usepackage{fancybox}
\usetikzlibrary{shadows.blur}
\usepackage{enumitem}
\usepackage{zref-abspage}
\usepackage{tabularray}
\usepackage{pbox}
\usepackage[version=4]{mhchem}
\usepackage{tasks}
\usepackage{multicol}
\usepackage{chemfig}
\usepackage{perpage}
\usepackage{setspace}
 \usepackage{vwcol}  
\usepackage{textcomp, gensymb}
\usepackage{xepersian}



\pagestyle{plain}
\tikzset{
	secnode/.style={
		minimum height=1cm,
		inner xsep=20pt,
		rotate=90,
		anchor=north east,
		draw=white,
		fill=black,
		text=white,
		blur shadow={shadow blur steps=5,shadow blur extra rounding=1.3pt}},
	pagenode/.style={
		minimum width=5mm,
		minimum height=1cm,
		inner sep=2pt,
		anchor=south east,
		draw=white,
		fill=black,
		text=white,
		blur shadow={shadow blur steps=5,shadow blur extra rounding=1.3pt}}
}
\newcommand{\tikzpagelayout}{
	\draw[black,line width=2pt,rounded corners=20pt] ([xshift=10mm]page.northwest) |- ([xshift=-2cm,yshift=10mm]page.southeast);
	\node[secnode] at ([xshift=5mm]page.northwest) {سالم غذا‌های پی در | شکیباییان};
	\node[pagenode] at ([xshift=-1cm,yshift=5mm]page.southeast) {\thepage};

}

\newenvironment{en}
	{\begin{enumerate}\setlength\itemsep{-0.2em}}
	{\end{enumerate}}
	
\newenvironment{iit}
	{\begin{itemize}\setlength\itemsep{-0.5em}}
	{\end{itemize}}

\newcommand*\circled[1]{\tikz[baseline=(char.base)]{
		\node[shape=circle,draw,inner sep=2pt] (char) {#1};}}

\settextfont{XB Niloofar}
\setdigitfont{XB Niloofar}
\setstretch{1.5}
\setlist[itemize]{topsep=0pt}
\setlist[enumerate]{topsep=0pt}
\renewcommand{\headrulewidth}{0pt}
\setlength{\headheight}{14.5pt}
\setlength{\columnseprule}{1pt}
\setlength{\columnsep}{0.5cm}
\MakePerPage{footnote}

\newcommand{\fb}{\rule{2cm}{0.15mm}\;}
\newcommand{\fs}{\rule{1cm}{0.15mm}}
\newcommand{\fsm}{{\rule{0.5cm}{0.15mm}\;}}
\newcommand{\lin}{\vspace{4pt}\hrule\vspace{4pt}}
\newcommand\gototask[1]{\addtocounter{task}{\numexpr#1-\value{task}\relax}}
\NewTasksEnvironment[label=\arabic*.,label-format=\bfseries,label-width=4ex]{answers}[\a]
\def\extra{\rule{1ex}{0ex}}
\makeatletter
\newcommand\censor{\@ifstar{\@cenmath}{\@centext}}
\newcommand\@cenmath[1]{%
	\protect\rule[-.3ex]{\widthofpbox{\extra$#1$}}{0.1ex}}
\newcommand\@centext[1]{%
	\protect\rule[-.3ex]{\widthofpbox{\extra#1\extra\extra}}{0.1ex}}
\makeatother

\begin{document}
%page 1
\censor{نمیدونم} و \censor{نمیدونم} ،‌ اجزاء بنیادی جهان مادی هستند. انرژی از راه‌های گوناگون با ماده ارتباط دارد، چنانکه کاهش \censor{نمیدونم} خورشید موجب تولید \censor{نمیدونم} می‌شود. «غذا» همواره نقش محوری در رشد، تندرسی و زندگی انسان داشته است. پیشرفت دانش و فناوری، موجب افرایش تولید فرآورده‌های کشاورزی و دامی و تولید صنعتی غذا شده است. در تولید انبوه، به دلیل فساد مواد غذایی و دشواری نگهداری، حفظ کیفیت و ارزش مواد غذایی، اهمیت به‌سزایی دارد. همچنین در صنایع غذایی، حجم عظیمی «آب» مصرف می‌شود و تأمین غذای جامعه را مشکل‌تر می‌کند.
\lin
\vspace{8pt}
خود را بیازمایید صفحه ۵۱؛\\
الف) \censor{نمیدونم} و دردرجه دوم
\censor{نمیدونم}
و
\censor{نمیدونم}.
\\
ب) با حذف خوراکی‌های غیر ضروری (مانند چیپس، پفک، نوشابه) تاحدی امکان تأمین هزینه مصرف انواع \censor{نمیدونم} در سبد خانوار تأمین می‌شود. (!!)\\
پ)
\begin{iit}
	\item توزیع شیر رایگان در مدارس، مهدکودک‌ها، پادگان‌ها و دانشگاه‌ها

	\item دادن علوفه و داروی دامی با قیمت ارزان به دامدار

	\item فرهنگ‌سازی مصرف
\end{iit}
ت) فرهنگ‌سازی استفاده بیشتر از حبوبات (مصرف عدسی یا آش در وعده صبحانه یا عصرانه)، مصرف انواع حبوبات در سالاد\\
\textbf {سرانه مصرف ماده غذایی، مقدار میانگین مصرف آن را به ازای هر فرد در یک گستره زمانی نشان می‌دهد.}
\vspace{4pt}
\hrule
\begin{center}
	\textbf{غذا، چیزی فراتر از یک پاسخ به احساس گرسنگی است. مصرف غذا؛}
\end{center}
\begin{en}
	\item مورد نیاز برای ماهیچه‌ها، ارسال پیام‌های عصبی، جابه‌جایی یون‌ها و مولکول‌ها از دیواره هر یاخته را تأمین می‌کند.
	\item \censor{نمیدونم} اولیه برای ساخت و رشد بخش‌های مختلف بدن را فراهم می‌کند. (بخش عمده \censor{نمیدونم} ، \censor{نمیدونم}  و \fs ـی موجود در بدن از غذا تأمین می‌شود.) این فرآیند‌ها وابسته به انجام واکنش‌های شیمیایی هستند، که دمای بدن را نیز تنظیم و کنترل می‌کنند. هر کدام از این واکنش‌ها، «آهنگ» ویژه‌ای دارند.
\end{en}

تغذیه درست، شامل وعده‌های غذایی است که مخلوط منابع از انواع ذره‌ها را در بر می‌گیرد. سوء تغذیه هنگامی رخ می‌نماید که وعده‌های غذایی با کمبود نوع خاصی از این ذرات همراه باشد. از طرفی، افزایش نامناسب برخی مولکول‌ها و یون‌ها در غذا نیز، سبب بیماری خواهد شد.
\vspace{4pt}
\hrule
\begin{center}
	\textbf{«غذا، ماده و انرژی»}
\end{center}
بدن برای انجام فعالیت‌های ارادی و غیرارادی، به ماده و انرژی نیاز دارد. یکی از راه‌های آزاد شدن انرژی سوخت‌ها (مانند بنزین و …) «سوزاندن» آن‌ها است. هر ماده غذایی نیز انرژی دارد و میزان انرژی به «جرم» آن بستگی دارد.
\newpage

%page 2

\begin{center}
	\textbf{دمای یک ماده، از چه خبر می‌دهد؟\\
		دما: کمیتی که میزان \censor{نمیدونم} و \censor{نمیدونم} اجسام را نشان می‌دهد.}
\end{center}
شکل ۱ صفحه ۵۴: وقتی به ظرف محتوی آب، گرما داده می‌شود، به تدریج \censor{نمیدونم} آن افرایش می‌یابد تا اینکه سرانجام \censor{نمیدونم} یا اگر به یخ داده شود، \censor{نمیدونم} می‌شود. در این حالت‌ها، با گرفتن گرما، \censor{نمیدونم} ذرات بیشتر شده و دما \censor{نمیدونم} می‌رود یا \censor{نمیدونم} ماده عوض می‌شود.
\begin{center}
	جنبش نامنظم ذره‌ها: گاز $\bigcirc$ مایع $\bigcirc$ جامد / آب‌ گرم $\bigcirc$ آب سرد
\end{center}
دمای بالاتر $\leftarrow$ میانگین \censor{نمیدونم} حرکت ذرات بیشتر $\leftarrow$ میانگین انرژی \censor{نمیدونم} ذرات بیشتر.\\
\textbf{یعنی:}
\underline{دمای ماده}
؛ معیاری برای توصیف \censor{نمیدونم} تندی و \censor{نمیدونم} انرژی جنبشی ذره‌های سازنده ماده است.\\
یکای رایج دما، درجه \censor{نمیدونم} (\qquad) اما یکای دما در ،SI \censor{نمیدونم} (\qquad) است.\\
ارزش دمایی ۱ درجه سانتی‌گراد برابر ۱ کلوین \censor{نمیدونم} .\\
لذا در فرآیند‌هایی که دما تغییر می‌کند،
$\mathrm{\Delta\theta\bigcirc\Delta T}$
است.
\qquad\qquad\qquad \censor{نمیدونم} = \censor{نمیدونم} + \censor{نمیدونم} \\
\textbf{با هم بیندیشیم صفحه ۵۵:}
\begin{en}
	\item
	الف) شکل A نمونه‌ای از هوا را در \censor{نمیدونم} نشان می‌دهد.\\
	ب) شکل B، نمونه‌ای از هوا را در یک روز \censor{نمیدونم} نشان می‌دهد.\\
	پ) اگر مجموع انرژی جنبشی ذره‌های سازنده یک نمونه ماده، هم‌ارز با انرژی گرمایی آن باشد؛
	انرژی گرمایی \censor{نمیدونم} بیشتر بوده زیرا \censor{نمیدونم} آن بیشتر است.
	\item
	الف) میانگین تندی مولکول‌ها در ظرف A$\bigcirc$ 	ظرف B\\
	ب) انرژی گرمایی ظرف A $\bigcirc$ظرف B (چون \censor{نمیدونم} \censor{نمیدونم} آن بیشتر است.)
\end{en}
با هم بیندیشیم ۱:  \censor{نمیدونم} یکسان، دمای \censor{نمیدونم} متفاوت $\leftarrow$ انرژی گرمایی متفاوت\\
با هم بیندیشیم ۲:	\censor{نمیدونم} یکسان، \censor{نمیدونم} \censor{نمیدونم} متفاوت $\leftarrow$ انرژی گرمایی متفاوت\\
\textbf{نتیجه:}
انرژی گرمایی یک نمونه ماده، هم به \censor{نمیدونم} و هم به \censor{نمیدونم} \censor{نمیدونم} بستگی دارد.\\
\textbf{تذکر:}
چون کار کردن «تعداد ذرات»، آسان نیست می‌توان به جای آن، \censor{نمیدونم} ماده را در نظر گرفت. چنانکه در فیزیک نیز، انرژی جنبشی از رابطه \fb به دست می‌آید.
\begin{center}
	\textbf{تهیه غذا آب‌پز، تجربه تفاوت «گرما» و «دما»}
\end{center}
گرما، صورتی از \censor{نمیدونم} و یکای آن در ،SI \censor{نمیدونم} (\censor{نمیدونم}) است.
($\mathrm{\qquad 1 \qquad = 1 Kgm^2.s^{-2}}$).
از یکای \censor{نمیدونم} (\censor{نمیدونم}) نیز برای بیان مقدار گرما در پزشکی و زیست‌شناسی و علم تغزیه استفاده می‌شود.
\hrule
\vspace{4pt}
\textbf{تعریف ژول:}\\
\textbf{تعریف کالری:}
\begin{flushleft}
	$\mathrm{\censor{نمیدونم} cal = \censor{نمیدونم} J}$
\end{flushleft}
\vspace{4pt}
\hrule
\vspace{4pt}
انرژی گرمایی: \censor{نمیدونم} انرژی‌های جنبشی ذرات ماده / دما: \censor{نمیدونم} انرژی جنبشی ذرات ماده\\
انرژی گرمایی و دما، از ویژگی‌های یک «نمونه ماده»
$\frac{است}{نیست}$
و
$\frac{می‌تواند}{نمی‌تواند}$
برای توصیف آن «ماده» به کار رود.

\newpage

%page 3

\begin{center}
	\textbf{«گرما»}
\end{center}
صورتی از \censor{نمیدونم} است، که از جسم با \censor{نمیدونم} بالاتر، به جسم با \fsm پایین‌تر منتقل می‌شود. داد و ستد گرما، می‌‌تواند موجب تغییر \censor{نمیدونم} مواد شود.
\\
گرما، از ویژگی‌های یک «نمونه ماده» \censor{نمیدونم} و \censor{نمیدونم} برای توصیف آن «ماده» به کار رود.
\\
هنگامی که به ۲ ماده، گرمای یکسان داده شود، لزوماً به یک اندازه \censor{نمیدونم} نمی‌شوند.
\lin
هنگامی که به ۲ ماده، گرمای یکسان داده شود، لزوماً به یک اندازه \censor{نمیدونم} نمی‌شوند.
\\
\textbf{یعنی:}
دادن گرمای یکسان به دو ماده، لزوما/حتما تغییر دمای یکسانی را موجب می‌شود/نمی‌شود.
مثال: اگر بخواهیم دمای آب و روغن زیتون* (با جرم برابر) به یک اندازه بالا رود، باید به آب، گرمای \censor{نمیدونم} بدهیم.
\lin
* الگوی ساختاری «روغن‌ها» با «چربی‌ها» یکسان است اما تفاوت‌هایی در ساختار دارند ( مانند پیوند دوگانه بیشتر در ساختار زنجیر کربنی \censor{نمیدونم} ) که موجب تفاوت در \censor{نمیدونم} و \censor{نمیدونم} آن‌ها می‌شود. چنان که روغن‌ها در دمای عادی، \censor{نمیدونم} و چربی‌ها \censor{نمیدونم} هستند.
\lin
\textbf{با هم بیندیشیم صفحه ۵۷:}
\\
الف) چون \censor{نمیدونم} \censor{نمیدونم} موجود در نمونه آب، بسیار \censor{نمیدونم} از روغن زیتون است.
دلیل: موادی چون آب و اتانول، به دلیل وجود \censor{نمیدونم} \censor{نمیدونم} بین مولکول‌های خود، گرمای ویژه بالایی دارند*. (جدول ۱ صفحه ۵۸).
دمای آب و روغن زیتون، به یک اندازه زیاد \censor{نمیدونم} است. برای افزایش دمای آب به میزان ۵۰ درجه سانتی‌گراد، (نسبت به روغن زیتون) گرمای \censor{نمیدونم} جذب شده، پس انرژی گرمایی ظرف محتوی آب، \censor{نمیدونم} است و تخم مرغ، گرمای \censor{نمیدونم} دریافت می‌کند.
ب) ظرفیت گرمایی (C): \censor{نمیدونم} لازم برای افرایش \censor{نمیدونم} ماده به اندازه \censor{نمیدونم} درجه \censor{نمیدونم} ( یا ۱ \censor{نمیدونم} )
\\
$C_{H_2O} = \frac{\quad J}{\quad K (\quad J.K^{-1})}\bigcirc C_{il.oil} = \frac{\quad J}{\quad k (\quad J.k^{-1})}$
(یکای C: \censor{نمیدونم} . \censor{نمیدونم} )
$Q = C \Delta\theta\rightarrow C =
	\frac{Q}{\Delta\theta}\rightarrow$
\\
پ)
بستگی دارد به \censor{نمیدونم} ماده و \censor{نمیدونم} ماده (به خاطر تفاوت در نوع \censor{نمیدونم} یا نیرو‌های \censor{نمیدونم} \censor{نمیدونم} )
هرچه \censor{نمیدونم} ماده بیشتر باشد، برای رساندن آن به دمای مشخص، \censor{نمیدونم} بیشتری لازم است.
\\
ت) گرمای ویژه (c): ظرفیت گرمایی \censor{نمیدونم} \censor{نمیدونم} ماده
\begin{flushleft}
	$Q = mc\Delta\theta\rightarrow c = \frac{Q}{m\Delta\theta}\downarrow$\\
	$C_{H_2O}=\frac{\quad}{\quad}=\censor{نمیدونم} ( \quad\quad )\quad C_{ol.oil}=\frac{\quad}{\quad}=\censor{نمیدونم} (\quad\quad)$
	یکای c: ( \censor{نمیدونم} . \censor{نمیدونم} . \censor{نمیدونم} )

\end{flushleft}
ث) رابطه C با :c
\\
هر کمیتی که از ویژگی‌های ماده باشد، (میتواند/نمیتواند) برای توصیف آن به کار رود.
\\
ظرفیت گرمایی؛ از ویژگی‌های نمونه ماده \censor{نمیدونم} و می‌تواند/نمی‌تواند برای توصیف آن ماده به کار رود.
\\
گرمای ویژه؛ از ویژگی‌های یک نمونه ماده \censor{نمیدونم} و \censor{نمیدونم} برای توصیف آن ماده به کار می‌رود.

\newpage

%page 4

\textbf{خود را بیازمایید صفحه ۵۸:}
\begin{en}
	\item
	\censor{نمیدونم} می‌یابد. با‌گذشت زمان، چای، همه/بخشی از انرژی گرمایی خود را به‌/از محیط می‌دهد/می‌گیرد پس \censor{نمیدونم} و \censor{نمیدونم} انرژی جنبشی ذرات آن، \censor{نمیدونم} می‌یابد. (کاهش \censor{نمیدونم} \censor{نمیدونم} و \censor{نمیدونم} نمونه)
	دلیل: گرما، از جایی که \censor{نمیدونم} تر است (دمای \censor{نمیدونم}) به جایی که \censor{نمیدونم} است (دمای \censor{نمیدونم}) حرکت می‌کند. دمای چای ( ) از دمای محیط ( ) \censor{نمیدونم} است و با \censor{نمیدونم} انرژی گرمایی، با آن «\censor{نمیدونم} \censor{نمیدونم}» می‌شود.
	\item
	گرما را می‌توان هم‌ارز با آن مقدار انرژی گرمایی/دمایی داشت که به دلیل تفاوت در انرژی گرمایی/دما جاری می‌شود.
	\item
	ماده اصلی تشکیل‌دهنده هر دو، \censor{نمیدونم} است، پس به مقدار \censor{نمیدونم} موجود در آن‌ها توجه می‌کنیم. نان، \censor{نمیدونم} کمتری دارد، چون \censor{نمیدونم} شده است، پس \censor{نمیدونم} با محیط هم‌دما می‌شود.
	\\
	\textbf{ نتیجه:«آهنگ» تغییر دمای مواد مختلف (مبادله \censor{نمیدونم} با \censor{نمیدونم}) یکسان \censor{نمیدونم} .}
\end{en}
\lin
نکته: هنگام مبادله گرما بین دو «ماده»؛ (اگر از هدر رفت یا اتلاف گرما چشم‌پوشی کنیم) مقدارگرمایی که ماده با دمای \censor{نمیدونم} است می‌دهد،		$|Q_A| = |Q_B|$
برابر با مقدار گرمایی است که ماده با دمای \censor{نمیدونم} می‌گیرد.
\\
یعنی قدر مطلق \censor{نمیدونم} مبادله شده در آن دو، \censor{نمیدونم} است.
\lin
\textbf{تمرین ۱:}
\\
جسم A به جرم g ۱۰۰ و دمای 100 درجه سانتی‌گراد را در تماس با جسم B به جرم g ۲۰۰  و دمای ۲۰۰ درجه سانتی‌گراد قرار می‌دهیم تا «هم دما» شوند. A و B در چه دمایی، هم‌دما می‌شوند؟ (بر حسب درجه سانتی‌گراد) (المپیاد شیمی ۸۶)
\begin{answers}(4)
	\a 180
	\a 160
	\a 150
	\a 145
\end{answers}
راه اول:
\begin{flushleft}
	$\mathrm{|Q_A|=|Q_B|\rightarrow}$
\end{flushleft}
\lin
راه دوم (هنگام تغییر فاز قابل استفاده نیست.)
\begin{flushleft}
	$\mathrm{\theta_{\textnormal{تعادلی}} = \frac{m_1 C_1 \theta _1 + m_2 C_2 \theta _2 }{m_1 C_1 + m_2 C_2} = \frac{\qquad\qquad\qquad\qquad}{\qquad\qquad\qquad\qquad} = \frac{\sum{(mc\theta)}}{\sum{mc}}}$
\end{flushleft}
\lin
\textbf{تمرین ۲:}
به آلیاژی از تیتانیم و نیکل به جرم ۴.۲ گرم، مقدار ۲۱ ژول گرما دادیم و دمای آن C\degree10 افزایش یافت. به تقریب، چند درصد جرم این آلیاژ را نیکل تشکیل داده است؟ $C_{Ni}=0.45(J.g^{-1}.\degree C^{-1})$ $C_{Ti}=0.5(J.g^{-1}.\degree C^{-1})$
\begin{answers}(4)
	\a 37/6
	\a 49/2
	\a 28/6
	\a 71/5
\end{answers}

\newpage

%page 5

\begin{center}
	\textbf{جاری شدن انرژی گرمایی}
	\\
	«بررسی کیفی و کمی انرژی مبادله شده بین سامانه و محیط»
\end{center}
\textbf{سامانه:}
بخشی از جهان، که ـــــ ـــــ را در آن بررسی می‌کنیم.

\textbf{محیط:}
هرچه ـــــ سامانه وجود دارد.

مثال: بررسی مبادله گرما بین یک لیوان آب و محیط:\\
( معمولاً سامانه با مرز‌های مشخصی از محیط جدا می‌شود. )
\textbf{فرآیند جاری شدن انرژی:}
\\
\\
\\
\textbf{تمرین:}
مبادلات انرژی را هنگام مصرف بستنی با دمای ۰ درجه سانتی گراد تا هضم آن را بررسی کنید.
\\
\\
\\
\begin{multicols}{2}
	\begin{center}
		\textbf{فرآیند گرماده}
	\end{center}
	\vfill\null
	در شرایط هم‌دما ($ \Delta\theta = 0$)\\
	جاری شدن انرژی از \censor{نمیدونم} به \censor{نمیدونم} واکنش یا فرآیند، برای انجام شدن، گرما می \censor{نمیدونم} .
	\begin{center}
		سطح انرژی طرف دوم $\bigcirc$ سطح انرژی طرف اول\\
		Q $\bigcirc$ ۰
	\end{center}
	نماد Q در طرف \censor{نمیدونم} نوشته می‌شود:\\
	1-
	واکنش گرماده:
	$\ce{H2 + Cl2 -> 2HCl}$\\
	2-
	فرآیند گرماده:
	$\ce{H2O(l) -> H2O(s)}$\\
	\censor{نمیدونم} سطح انرژی سامانه
	\vfill\null
	\columnbreak
	\begin{center}
		\textbf{فرآیند گرماگیر}
	\end{center}
	\vfill\null
	در شرایط هم‌دما $( \Delta\theta = 0 )$
	\\
	جاری شدن انرژی از \censor{نمیدونم} به \censor{نمیدونم} واکنش یا فرآیند، برای انجام شدن، گرما می ‌\censor{نمیدونم} .
	\begin{center}
		سطح انرژی طرف دوم	 $\bigcirc$ سطح انرژی طرف اول
		\\
		Q$\bigcirc$۰
	\end{center}

	نماد Q در طرف \censor{نمیدونم} نوشته می‌شود:
	\\
	1-
	واکنش گرما‌گیر:
	$\ce{N2O4(g) -> 2NO2(g)}$
	\\
	2-
	فرآیند گرماگیر:
	$\ce{H2O(s) -> H2O(l)}$
	\censor{نمیدونم} سطح انرژی سامانه
	\vfill\null
\end{multicols}

\newpage

%page 6 

\begin{center}
	\textbf{گرما در واکنش‌های شیمیایی }
	(گرماشیمی)
\end{center}
هر واکنش شیمیایی، ممکن است با تغییر \censor{نمیدونم} ، تولید \censor{نمیدونم} ، آزاد شدن \censor{نمیدونم} و ایجاد \censor{نمیدونم} و \censor{نمیدونم} همراه باشد،
اما:	داد و ستد \censor{نمیدونم} ، یک ویژگی بنیادی واکنش‌های شیمیایی است.
\\
ترموشیمی (گرماشیمی) به بررسی \censor{نمیدونم} و \censor{نمیدونم} گرمای واکنش‌های شیمیایی، \censor{نمیدونم} آن و تأثیری که بر \censor{نمیدونم} ماده دارد، می‌پردازد.
\lin
\textbf{بررسی شکل ۳ صفحه ۶۰:}
\\
الف) مواد غذایی، پس از گوارش، انرژی لازم برای \censor{نمیدونم} و \censor{نمیدونم} یاختهها را تأمین می‌کنند.
\\
ب) \censor{نمیدونم} سوخت‌ها، انرژی لازم برای حمل و نقل، و نیز گرمایش محیط‌های گوناگون را فراهم می‌کند.
\\
پ) زغال کک، واکنش‌دهنده‌ای رایج در استخراج آهن، و تامین‌کننده \censor{نمیدونم} لازم برای واکنش است.
\lin
منبع انرژی در بدن، \censor{نمیدونم} است. انرژی غذا، پس از انجام واکنش‌های شیمیایی گوناگون، به سلول‌ها می‌رسد. این واکنش‌ها ممکن است گرماده یا گرماگیر باشند اما فرآیند کلی اکسایش گلوکز در مجموع، گرما \censor{نمیدونم} است. البته دمای بدن تغییر محسوسی \censor{نمیدونم}

دلیل: دمای واکنش‌دهنده‌ها با دمای فرآورده‌ها \censor{نمیدونم} است $(\Delta\theta		0)$

در‌واقع، انرژی آزاد شده در این واکنش، ناشی از تفاوت دمای مواد واکنش‌دهنده و فرآورده \censor{نمیدونم} ، بلکه تفاوت میان انرژی \censor{نمیدونم} مواد و واکنش‌دهنده و فرآورده است.
\\
انرژی پتانسیل در اینجا، به معنای انرژی ناشی از نیرو‌های \censor{نمیدونم} \censor{نمیدونم} ذرات سازنده آن است.
\\
\textbf{انرژی پتانسیل موحوددر یک نمونه ماده، انرژی \censor{نمیدونم} نام دارد.}
\\
انرژی پتانسیل در پیوند‌های مختلف، با هم \censor{نمیدونم} است، چون اتم‌های مختلفی با هم پیوند دارند. مثال:
\\
تفاوت اتم‌های دارای پیوند اشتراکی، موجب تفاوت در نیرو‌های \censor{نمیدونم} ( این نیرو‌ها، شامل «پیوند‌ها» و «نیرو‌های بین مولکولی» است. ) این نیرو‌ها، شامل «پیوند‌ها» و «نیرو‌های بین مولکولی» است.اتم‌ها ( در مولکول ) و در نتیجه؛ تفاوت در \censor{نمیدونم} پیوند‌ها است.
\\
\\
انجام واکنش شیمیایی، موجب تغییر در پیوند‌ها یا شیوه اتصال اتم‌ها با یکدیگر، و تفاوت آشکاری در انرژی \censor{نمیدونم} وابسته به آن‌ها می‌شود؛ که خود را به صورت \censor{نمیدونم} (ی مبادله‌شده) نشان می‌دهد.
\lin
\textbf{با هم بیندیشیم صفحه ۶۱	: در دو واکنش:}

\begin{en}

	\item  الف) واکنش‌دهنده‌ها یکسان هستند/نیستند $\leftarrow$ سطح انرژی واکنش‌دهنده‌ها یکسان \censor{نمیدونم}
	\\
	فرآورده، یکسان \censor{نمیدونم} $\leftarrow$ سطح انرژی فرآورده در دو واکنش یکسان \censor{نمیدونم}
	\\
	ب) در واکنش اول/دوم، سطح انرژی واکنش‌دهنده‌ها \censor{نمیدونم} $\leftarrow$ پایدارتر
	\item  الف) چون سطح انرژی گرافیت و الماس، یکسان \censor{نمیدونم} . ( به دلیل تفاوت در نیرو‌های نگهداری )
	\\
	ب) \censor{نمیدونم} پایدار‌تر است، چون فاصله کم‌تری با فرآورده دارد، گرمای سوختنی \censor{نمیدونم} دارد.
	\\
	\textbf{	نحوه اتصال }
	اتم‌های کربن،
	\textbf{تعداد و نوع }
	پیوند‌های اشتراکی کربن – کربن، در این دو آلوتروپ، و در نتیجه، رفتار شیمیایی آن‌ها ( مانند پایداری یا آنتالپی سوختن) متفاوت است.
	\\
	۲- پ)
	%i hate my self
	\qquad\qquad\qquad\qquad
	$xKj = \qquad g \times \frac{\qquad mol}{\qquad g}\times\frac{\qquad KJ}{\qquad mol}=\censor{نمیدونم} KJ$

\end{en}

\newpage

%page 7

\begin{center}
	\textbf{یخچال صحرایی!}
\end{center}
دو ظرف از جنس \fsm داریم که فضای بین آن‌ها از شن خیس پر می‌شود. پارچه‌ای \fsm به عنوان درپوش، تحویه را انجام می‌دهد. آب درون ظرف درونی، به تدریج در بدنه ظرف بیرونی نفوذ می‌کند و \fsm می‌شود:
$H_2O(\quad) + Q \rightarrow H_2O(\quad)$
\\
این فرآیند، گرما \censor{نمیدونم} است و گرمای لازم را از سامانه دریافت می‌کند که باعث افت دما و خنک شدن محتویات دستگاه می‌شود.
\lin
\begin{center}
	فرآیند‌های تغییر حالت مواد
\end{center}
قث
\\
\\
\\
\\
\\
\\
\\
\\
هخثقهخ
\lin
\begin{center}
	\textbf{عوامل مؤثر بر گرمای واکنش: (یک عامل ثابت، و سه عامل متغیر)}
\end{center}
\begin{en}
	\item \censor{نمیدونم} مواد واکنش (واکنش‌دهنده‌های و فرآورده‌ها):
	مواد مختلف،‌ سطوح انرژی متفاوت دارند. گرمای واکنش، \censor{نمیدونم} سطح انرژی مواد طرف اول و دوم واکنش است. این عامل، متغیر \censor{نمیدونم} ، چون با تغییر مواد، در واقع، واکنش دیگری داریم.
	\item  \censor{نمیدونم} و \censor{نمیدونم} :
	تغییر این دو عامل، سطح \censor{نمیدونم} واکنش‌دهنده‌ها یا فرآورده‌ها را تغییر می‌دهد.
	\item \censor{نمیدونم} واکنش‌دهند‌ها:
	سطح انرژی هر ماده، به مقدار آن وابسته \censor{نمیدونم} و تغییر مقدار مواد، سطح انرژی آن را نیز تغییر می‌دهد.
\end{en}
\lin
\textbf{تمرین:}
سوختن هر مول متان، .KJ89 انرژی آراد می‌کند. با سوختن ۱ گرم متان، چند کالُری گرما تولید می‌شود؟
\vspace{8em}
\lin
4.
\censor{نمیدونم} \censor{نمیدونم} مواد واکنش:
در معادله «ترموشیمیایی»، باید انرژی \censor{نمیدونم} \censor{نمیدونم} در واکنش ذکر شود. حال اگر

حالت فیزیک یکی از مواد در واکنش تغییر کند، سطح \censor{نمیدونم} آن نیز تغییر می‌کند و در نهایت، گرمای واکنش را تغییر

می‌دهد.
\begin{flushleft}
	( در دمای \censor{نمیدونم} ) $CH_4(g) + 2O2(g) \rightarrow CO_2(g) + 2H_2O(g) + Q_1$ I)
	\\
	( در دمای \censor{نمیدونم} ) $CH_4(g) + 2O_2(g) \rightarrow CO_2(g) + 2H_2O(l) + Q_2$II)
\end{flushleft}

$H_2O$ تولید شده در واکنش سوختن متان، ابتدا در دمای شعله است و حالت فیزیکی گازی دارد، اگر مقداری صبر کنیم تا سامانه با محیط، « \censor{نمیدونم} \censor{نمیدونم} » شود، $H_2O$ به حالت مایع در می‌آید.
این فرآیند (تبخیر/میعان)، خود، گرما \censor{نمیدونم} است و در رسیدن از I به II مقداری گرما \censor{نمیدونم} می‌شود. یعنی $Q_2$، از لحاظ عددی، از $Q_1$ \censor{نمیدونم} است.
\lin

\textbf{تمرین) }
گرمای تبخیر مولی آب را برحسب $Q_1$ و $Q_2$ به دست آورید:
\begin{flushleft}
	0
	$\bigcirc$
	$\frac{\qquad\qquad}{\qquad\qquad}$
	=	گرمای تبخیر مولی
\end{flushleft}
\lin
\textbf{با هم بیندیشیم ۳ صفحه ۶۲:}
\\
اولاً: میعان، گرما \censor{نمیدونم} است، پس گرمای واکنش با عدد +/- گزارش می‌شود.
\\
ثانیا: گرمای آزاد شده در میعان و نیز گرمای واکنش هردو، علامت 	دارند و مجموع آن‌ها با علامت 	باید از نظر عددی از ۴۸۴ \censor{نمیدونم} باشد ( یعنی عدد \censor{نمیدونم} )
\lin
\textbf{پرسش:}
\\
گرمای آزاد شده در کدام حالت، مقدار عددی بیشتری دارد؟ (روش: باید یک طرف کمترین و طرف دیگر بیشترین سطح انرژی را داشته باشد)
\begin{answers}(4)
	\a $2O(l) \rightarrow O_2(g)$
	\a $2O(g) \rightarrow O_2(l)$
	\a $2O(g) \rightarrow D_2(g)$
	\a $2O(l) \rightarrow O_2(l)$
\end{answers}
\lin
\begin{center}
	«آنتالپی ,(H) همان محتوای انرژی است»
\end{center}
هر نمونه ماده، دارای شمار بسیار زیادی «ذره سازنده» است. این ذره‌ها، دارای:
\\
۱- \censor{نمیدونم} نامنظم (انرژی \censor{نمیدونم} ) و ۲- \censor{نمیدونم} با یکدیگر (انرژی \censor{نمیدونم}) هستند
\\
\textbf{یک نمونه ماده، با \censor{نمیدونم} آن در \censor{نمیدونم} و \censor{نمیدونم} معین، توصیف می‌شود.}
مانند ۲۰۰ گرم آب در دما و فشار معین یک نمونه ماده در یک ظرف، می‌تواند یک \censor{نمیدونم} به شمار آید.
\\
«انرژی کل» یک سامانه، هم ارز «محتوای \censor{نمیدونم}» یا «\censor{نمیدونم}» آن سامانه است.
یعنی: همه مواد، در دما و قشار معین، «\censor{نمیدونم}» مشخصی دارند.
\\
با انجام واکنش شیمیایی، «محتوای \censor{نمیدونم}» یا «\censor{نمیدونم}» مواد، تغییر می‌کند. (مانند نمودار ۵ صفحه ۶۴)
\\
\textbf{مهم:}
$Q_p$
= \censor{نمیدونم} H - \censor{نمیدونم} H =
واکنش
$\Delta H$
$\leftarrow$
(\censor{نمیدونم} آنتالپی )
\\
$Q_p$
به معنای \censor{نمیدونم} مبادله شده در « \censor{نمیدونم} \censor{نمیدونم} » است.
\\
مقدار عددی $\Delta H$ در یک فرآیند، \censor{نمیدونم} آن را نشان می‌دهد، اما علامت + یا -، به ترتیب، \censor{نمیدونم} \censor{نمیدونم} و \censor{نمیدونم} \censor{نمیدونم} بودن آن را نشان می‌دهد.
\\
\textbf{خود را بیازمایید صفحه ۶۴ و ۶۵: 	}
\begin{en}
	\item الف) $CO_2(s)\qquad \rightarrow CO_2(g)\qquad, \Delta H \bigcirc 0$
	\\
	ب) $CH_4(g) + 2O_2(g) \rightarrow CO_2(g) + 2H_2O(g)\qquad, \Delta H \bigcirc 0$
	\\
	پ) $N_2O_4(g)\qquad \rightarrow 2NO_2(g)\qquad, \Delta H \bigcirc 0$
	\\
	ت) $N_2O(l) \rightarrow H_2O(s)\qquad, \Delta H \bigcirc 0$
	\item $3O_2(g) + \censor{نمیدونم} \leftrightarrow 2O_3(g)$
	\\
	$x(KJ) = \frac{\censor{نمیدونم} KJ}{\censor{نمیدونم} mol O_3}\times \censor{نمیدونم} mol O_3 = \censor{نمیدونم} (KJ)$
	\\
	$(\Delta H رفت = \qquad) (\Delta H برگشت = \qquad)$
\end{en}
\newpage

%page 9 

\begin{center}
	\textbf{«آنتالپی پیوند» و «میانگین آنتالپی پیوند»}
\end{center}

انجام یک واکنش شیمیایی، نشانه‌ای از تغییر در \censor{نمیدونم} \censor{نمیدونم} اتم‌ها (ذرات) به یکدیگر است، که نتیجه آن، تغییر \censor{نمیدونم} و به دنبالش تغییر \censor{نمیدونم} مواد است. یکی از خواصی که در واکنش‌های شیمیایی تغییر می‌کند، محتوای \censor{نمیدونم} مواد است. مثلاً، یک نمونه گاز هیدروژن، دارای شمار بسیار زیادی \censor{نمیدونم} دو اتمی است. با صرف \censor{نمیدونم} ، پیوند \censor{نمیدونم} بین اتم‌ها در مولکول می‌شکند و به \censor{نمیدونم} هایی تبدیل می‌شود که \censor{نمیدونم} تر و \censor{نمیدونم} \censor{نمیدونم} تر هستند. در ترموشیمی، به مقدار KJ436, آنتالپی \censor{نمیدونم} می‌گویند:
\begin{flushleft}
	( $KJ.mol^{-1}$ )436 $\bigcirc$  = ( \censor{نمیدونم} ) H $\Delta$
\end{flushleft}
\textbf{آنتالپی پیوند:}
انرژی لازم برای \censor{نمیدونم} ۱ \censor{نمیدونم} پیوند در مولکول \censor{نمیدونم} و تبدیل آن به اتم‌های \censor{نمیدونم}
\\
در مولکول‌هایی که «اتم مرکزی» به چند اتم یکسان با پیوند اشتراکی متصل است، (مانند $CH_4$) این پیوند‌های یکسان، آنتالپی کاملاً یکسان \censor{نمیدونم} ! در این حالت، به کار بردن اصطلاح * \censor{نمیدونم} آنتالپی پیوند، مناسب‌تر است.
\begin{flushleft}
	$CH_4(g) + 1660 KJ \rightarrow \censor{نمیدونم} (\quad) + \censor{نمیدونم} (\quad)$
	\\
	$\Delta H_{(C-H)} = \qquad \div \qquad = \qquad (KJ.mol^{-1})$
\end{flushleft}
\lin
پرسش) در چند مورد، به کار بردن میانگین آنتالپی پیوند، مناسب‌تر است؟ \censor{نمیدونم} مورد
\begin{answers}(4)
	\a $\mathrm{NH_3(g)}$
	\a ds
	\a $\mathrm{H-Br(g)}$
	\a $\mathrm{H_2O(g)}$
\end{answers}
\lin
\textbf{خود را بیازمایید صفحه ۶۶:}
\\
الف) (پیوند \censor{نمیدونم} شده $\leftarrow$ گرما \censor{نمیدونم} ) \qquad \qquad H$\Delta$ | پیوند‌ها در جدول ۲ صفحه ۶۵ مربوط به مولکول ۲ اتمی (میانگین هست/نیست.)
\\
ب) (پیوند \censor{نمیدونم} شده $\leftarrow$ گرما \censor{نمیدونم} )\qquad\qquad H $\Delta$    | پیوند‌ها در جدول ۳ صفحه ۶۶ مربوط به مولکول‌های چند اتمی ( میانگین \censor{نمیدونم} )
\\
تذکر: برای گزارش آنتالپی پیوند، همه ذرات در دو طرف واکنش به حالت \censor{نمیدونم} و همه فرآورده‌ها باید \censor{نمیدونم} باشند:(اگر قرار است همه پیوند‌ها شکسته شود.)  \qquad \qquad \qquad \qquad
$NH3(\quad) + Q \rightarrow \censor{نمیدونم} (\quad) + \censor{نمیدونم} (\quad)$
\lin
\begin{center}
	\textbf{«آنتالپی پیوند، راهی برای تعیین H $\Delta$ واکنش»}
\end{center}

۱) روش محاسباتی برای تعیین H $\Delta$واکنش:
\\
در واکنش شیمیایی، «معمولا» تعدادی پیوند \censor{نمیدونم} و تعدادی پیوند جدید \censor{نمیدونم} می‌شود.
\\
برای «شکستن» پیوند،‌ مقداری انرژی \censor{نمیدونم} می‌شود ( با علامت $\bigcirc$ گزارش می‌شود).
\\
هنگام «تشکیل» پیوند،‌ مقداری انرژی \censor{نمیدونم} می‌شود ( با علامت $\bigcirc$ گزارش می‌شود).(H $\Delta$واکنش، \censor{نمیدونم} این انرژی‌های \censor{نمیدونم} شده است.)
\\
استفاده از آنتالپی پیوند، برای تعیین H $\Delta$واکنش‌های \censor{نمیدونم} مناسب‌تر است. ( همه مواد در حالت \censor{نمیدونم} )
\\
هرچه مواد واکنش‌، مولکول‌های \censor{نمیدونم} داشته باشند، H $\Delta$محاسبه شده، با داده‌های \censor{نمیدونم} همخوانی بیشتری دارد، و هرچه مولکول‌ها پیچیده‌تر باشند، H $\Delta$ محاسبه شده با داده‌های \censor{نمیدونم} تفاوت‌های آشکار نشان می‌دهد.
\\
۲) استفاده از «آنتالپی پیوند» برای تعیین H $\Delta$ واکنش: (خود را بیازمایید ۱ صفحه ۶۷)
\\
H $\Delta$ واکنش: [مجموع آنتالپی‌های پیوند\fb]-[مجموع آنتالپی‌های پیوند\fb]

\newpage
% page 10

نکته: در جدول آنتالپی پیوند، همه اعداد علامت 	$\bigcirc$دارند و $\frac{\text{صسشیمیدسشت}}{\text{نتسشغلایسشتسیش}}$ علامت $\bigcirc$	پیش از آنتالپی پیوند فرآورده‌ها، برای آن است که 	$\bigcirc$در $\bigcirc$	، $\bigcirc$	شود. ( چون در فرآورده‌ها، پیوند‌ها در حال تشکیل هستند که فرآیندی گرماده است و باید با عدد منفی نوشته شود. )

خود را بیازمایید ۲ صفحه ۶۸ الف)\\
ب)\\
پ)\\
\lin
تمرین ۱ اگر برای تبدیل ۱ گرم از گاز‌های متان و اتان، به اتم‌های گازی جدا از هم، به ترتیب ۱۰۳ و ۹۴ کیلوژول انرژی مصرف شود، آنتالپی \chemfig{C-[,0.5] C} چند $\frac{KJ}{mol}$ است؟ ($\rm C=\text{ و ۱۲} H=\text{۱}$)
\vspace{10em}
\lin
تمرین ۲ به کمک «جدول آنتالپی پیوند»، آنتالپی سوختن کامل اتانول و بنزین را به دست آورید:
\vspace{8em}
\lin
خود را بیازمایید ۲ صفحه ۷۰:\\
الف) این دو ترکیب، فرمول مولکولی $\frac{\text{یکسان}}{\text{متفاوت}}$، و ساختار \censor{نمیدونم} دارند.\\
نتیجه: این دو ترکیب، \censor{نمیدونم} \censor{نمیدونم} ( هم \censor{نمیدونم} ) هستند.\\
ب) $\frac{\text{بله}}{\text{خیر}}$، چون ساختار آن‌ها یکسان \censor{نمیدونم}.\\
پ) $\frac{\text{بله}}{\text{خیر}}$، چون تفاوت در \censor{نمیدونم}، موجب تفاوت در \censor{نمیدونم} از جمله سطح انرژی است.

\Ovalbox{
	\begin{minipage}{0.97\linewidth}
			محتوای انرژی یک ترکیب، در \underline{دما} و \underline{فشار} ثابت، علاوه بر «نوع» و «تعداد» اتم‌ها به نحوه \censor{نمیدونم} اتم‌ها، و «نوع» پیوند‌های شیمیایی مربوط است.
	\end{minipage}
}
\\

\textbf{آشنایی با گرو‌ه‌های عاملی}\\
گروه عاملی؛ \censor{نمیدونم} منظمی از \censor{نمیدونم} ها است که به مولکول دارای آن، خواص \underline{فیزیکی} و \underline{شیمیایی} ویژه می‌بخشد.\\
در گروه‌های عاملی، \censor{شیوه} اتصال اتم‌ها با یکدیگر، یا \censor{پیوند} بین آن‌ها، اهمیت ویژه دارد.\\
گروه عاملی، در تعیین \censor{نمیدونم} ترکیبات آلی، نقش تعیین‌کننده‌ای دارد. به عنوان مثال خواص ادویه، به طور عمده وابسته به ترکیب‌های آلی موجود در آن‌ها است که در ساختار آن‌ها، علاوه بر C و H، اتم‌های \censor{نمیدونم} و گاهی \censor{نمیدونم} و \censor{نمیدونم} وجود دارد. تفاوت در خواص ادویه، به دلیل تفاوت در ساختار این مواد آلی است. ( گروه عاملی، قسمتی از ترکیب آلی است که با دیدن آن، می‌فهمیم این ترکیب، \censor{نمیدونم} نیست! )
\newpage
\textbf{آنتالپی سوختن، تکیه‌گاهی برای تامین انرژی}\\
بدن ما از عذا، مواد گوناگونی شامل \censor{نمیدونم} \censor{نمیدونم} ها، \censor{نمیدونم} ها، \censor{نمیدونم} ها، \censor{نمیدونم} و مواد \censor{نمیدونم} دریافت می‌کند.
\lin
از این بین، کربوهیدرات‌ها، چربی‌ها و پروتئین‌ها، علاوه بر:
\circled{۱}
تامین \censor{نمیدونم} اولیه برای سوخت و ساز،
\circled{2}
تامین \censor{نمیدونم} یاخته‌ها نیز هستند.\\
از این سه دسته، تنها \censor{نمیدونم} \censor{نمیدونم} در بدن به \censor{نمیدونم} شکسته شده و در خون حل می‌شود. \censor{نمیدونم}، قند خون است، خون این ماده را به یاخته‌ها می‌رساند و در آنجا \censor{نمیدونم} می‌یابد و \censor{نمیدونم} تولید می‌کند.

بدن، بیشتر \censor{نمیدونم} را ذخیره می‌کند چون انرژی حاصل از اکسایش جرم برابری از آن با دو ماده دیگر، بیشتر است. ( جدول ۴ صفحه ۷۰)\\
\noindent\fbox{%
	\parbox{\textwidth}{%
		\textbf{انرژی سوختی:}
		انرژی حاصل از سوختن ۱ \censor{نمیدونم} از ماده غذایی ( یکا: \censor{نمیدونم} ) جواب ۵ صفحه ۷۱
	}%
}
تمرین \circled{۱}: اگر درصد چربی در ترکیب یک ماده غذایی ۲٪، و درصد پروتئین و کربوهیدرات در آن، به ترتیب \underline{۳} برابر و \underline{۲۴} برابر چربی باشد، ارزش سوختی این ماده غذایی $\frac{KJ}{g}$ است؟ ( راهنمایی: جرم ماده غذایی را
\underline{۱۰۰}
گرم فرض کنید. )
\vspace{8em}
نکته: جرم کربوهیدرات و پروتئین را می‌توان جمع و یکجا محاسبه کرد (چون ارزش سوختی آن‌ها یکسان است. )
\lin
تمرین \circled{۲}: با گرمای آزاد شده از سوختن 50g از ماده غذایی تمرین \circled{۱}، چند مول آب ۸۰\degree را می‌توان به جوش آورد؟ ( فرض کنید در این فرآیند، ۲۰٪ هدر رفت انرژی وجود دارد. )
$\mathrm{C(H_2O)=4.2(J.g^{-1}.\degree C^{-1})}$
\newpage
\textbf{سوختن}
برای تهیه غذای گرم، معمولا از سوخت‌های \censor{نمیدونم} استفاده می‌شود. مانند \censor{نمیدونم} که (عمده) گاز شهری را تشکیل می‌دهد، در حضور اکسیژن \underline{کافی} می‌سوزد و انرژی زیادی تولید می‌کند:
\begin{flushleft}
	$\ce{CH_4(g) + O_2(g) -> CO_2(g) + H_2O(g) + 890KJ}\;\text{(موازنه کنید)}$
\end{flushleft}
\noindent\fbox{%
	\parbox{\textwidth}{%
		آنتالپی سوختن: انرژی حاصل از سوختن \underline{۱} \censor{نمیدونم} از ماده سوختنی ( یکا: \censor{نمیدونم} \censor{نمیدونم} ) جواب ۶ صفحه ۷۱
	}%
}
\vspace{5em}
\lin
خود را بیازمایید صفحه ۷۱:
$\Delta H_\text{سوختن}\text{(پروپان)}\simeq-2220(KJ.mol^{-1})\hspace{2pt}\vline\hspace{2pt}\Delta H_\text{سوختن}\text{(بوتن-۱)}\simeq-2717 (KJ.mol^{-1})$
\lin
\begin{tblr}{ccc}
	متان & اتان & پروپان \\
	-۸۹۰ KJ
	اینجا تو بکش
\end{tblr}
خود را بیازمایید \underline{۲} صفحه \underline{۷۱}: الف) ارزش سوختی: اتان $\bigcirc$ اتانول\\
آنتالپی سوختن: اتان $\bigcirc$ اتانول
\begin{flushleft}
	C=12, H=1, O=16
\end{flushleft}
\lin
ب)
\vspace{2em}\\
پ) سوخت‌های سبز، علاوه بر CوH، اتم \censor{نمیدونم} نیز دارند و از پسماند سویا، نی‌شکر یا سایر دانه‌های روغنی استخراج می‌شوند. سوخت سبز برای سوختن، اکسیژن \censor{نمیدونم} نیاز دارد.
\lin
پرسش \circled{۱}: می‌دانیم که سوختن مواد در دما‌های بالا صورت می‌گیرد. چرا در خود را بیازمایید \underline{۱} صفحه \underline{۷۱}، سوختن مواد در دمای \degree C25 مطرح شده است؟\\
پاسخ: منظور از عدد \degree C25 روی پیکان در این واکنش‌ها، سوختن در دمای \degree C25 $\frac{\text{نیست}}{\text{است}}$، بلکه به معنای اندازه‌گیری \censor{نمیدونم} واکنش در دمای \degree C25 است.
\lin
\vspace{4em}
\lin
پرسش \circled{۲}: سوختن هیدروکربن‌ها در دماهای بالا صورت می‌گیرد، پس چگونه می‌توان آنتالپی سوختن را در دمای \degree C25 اندازه‌گیری کرد؟\\
پاسخ: ابتدا، واکنش‌دهنده‌ها را در دمای \degree C25 وارد سامانه می‌کنیم، پس از انجام واکنش (سوختن) اجازه می‌دهیم فرآورده‌ها \censor{نمیدونم} شوند و به دمای \degree C25 برسند. بعنی ابتدا و انتهای واکنش، در \degree C25 بررسی می‌شود. آنتالپی واکنش نیز با توجه به \censor{نمیدونم} و \censor{نمیدونم} واکنش تعیین می‌شود، حتی اگر در \censor{نمیدونم} به دمایی دیگر برسیم.\\
(آنتالپی واکنش، تابع مسیر \censor{نمیدونم}). چنانکه در طرح بالا نیز دیده می‌شود؛ اختلاف سطح انرژی واکنش‌دهنده‌ها با فرآورده‌ها (در دمای معین)، یعنی همان \censor{نمیدونم} واکنش، مقدار مشخص \censor{نمیدونم} و به مسیر پیموده شده ربط \censor{نمیدونم}.
\newpage
\textbf{نکات مهم مربوط به جدول \underline{۶} صفحه \underline{۷۱}}\\
\circled{1}
در اثر سوختن هیدروکربن‌ها و مواد آلی اکسیژن‌دار، گرما آزاد می‌شود. سوخت‌ها، موادی پر انرژی و $\frac{\text{پایدار}}{\text{ناپایدار}}$ هستند و فرآورده‌های سوختن، به نسبت $\frac{\text{پایدار}}{\text{ناپایدار}}$ ترند و این تفاوت، به صورت گرما آزاد می‌شود.\\
\circled{2}
بین چند آلکان (یا سایر هیدروکربن‌های هم خانواده) آنتالپی سوختن ترکیبی بیشتر است که $\frac{\text{سبک‌تر}}{\text{سنگین‌تر}}$ است. (وقتی \underline{مول‌های} برابر از چند هیدروکربن هم‌خانواده بسوزند، آنکه کربن \censor{نمیدونم} دارد، گرمای بیشتری آزاد می‌کند.)\\
\circled{3}
بین چند آلکان (یا سایر هیدروکربن‌های هم‌خانواده) ارزش سوختی ترکیبی بیشتر است که $\frac{\text{سبک‌تر}}{\text{سنگین‌تر}}$ است.(وقتی \underline{جرم‌های} برابر از چند هیدروکربنی هم‌خانواده بسوزند، آنکه کربن \censor{نمیدونم} دارد، گرمای بیشتری آزاد می‌کند.)\\
\circled{4}
آنتالپی سوختن \underline{۴} خانواده جدول (هم کربن): \censor{نمیدونم} > \censor{نمیدونم} > \censor{نمیدونم} > \censor{نمیدونم}\\
\circled{5}
الکل‌های سنگین‌تر، نسبت به الکل‌های سبک‌تر، آنتالپی سوختن \censor{نمیدونم} و ارزش سوختی \censor{نمیدونم} دارند.\\\textbf{(نکته \circled{2} در مورد الکل‌ها صدق \censor{نمیدونم} و نکته $\bigcirc$ نه!)}
\lin
\begin{center}
	\textbf{اندازه‌گیری گرمای واکنش}
\end{center}
دو روش دارد: الف) روش مستقیم(اندازه‌گیری در آزمایشگاه، به کمک ابزار) ب) روش غیرمستقیم (به کمک محاسبه)
\lin
الف) روش مستقیم (گرماسنجی یا کالری‌متری) به روش تجربی، که ابزار آن،
\Ovalbox{\textbf{گرماسنج}}
است.

گرماسنج، انواع مختلف دارد و در کتاب درسی فقط به گرماسنج لیوانی اشاره شده است. (ش \underline{۸} صفحه \underline{۷۲})
\Ovalbox{\textbf{گرماسنج لیوانی:}}
گرمای واکنش را در \censor{نمیدونم} ثابت اندازه‌گیری می‌کند. (که به آن، \censor{نمیدونم} گفته می شود.)\\
این گرما‌سنج، برای تغیین «آنتالپی \censor{نمیدونم}» و نیز آنتالپی واکنش‌ها در حالت «\censor{نمیدونم}» مناسب است.\\
در این گرماسنج، مقداری آب درون لیوان یک‌بار مصرف (\underline{۲} لیوان درون هم) قرار می‌گیرد که تا حد ممکن عایق \censor{نمیدونم} باشد. درپوش یونالیتی روی آب قرار می‌گیرد و از درون آن، یک دماسنج و یک همزن وارد آب می‌شود تا دما را در کل محلول، تا حد ممکن \censor{نمیدونم} سازد. با اندازه‌گیری تغییر دما ($\Delta\Theta$) در طول فرآیند، می‌توان گرمای واکنش را از فرمول
$\mathrm{Q=mc\Delta\Theta}$
محاسبه نمود.
\lin
مسئله: در یک گرماسنج لیوانی، 200mL محلول سود ۰.۱ مولار با 200mL محلول سولفوریک اسید وارد واکنش می‌شود. اگر در پایان واکنش، مقداری اسید واکنش نداده باقی‌مانده و دما به اندازه $\mathrm{(0.7\degree C)}$ افزایش یافته باشد، آنتالپی واکنش روبه‌رو، چند KJ است؟ (همه گرمای واکنش، صرف بالا بردن دمای محلول شده و چگالی همه محلول‌ها
$\mathrm{\frac{Kg}{L}}$
\underline{1}
است. گرمای ویژه محتویات گرماسنج،
$\mathrm{\underline{4}J.g^{-1}.\degree C^{-1}}$
است.)
\begin{flushleft}
	\ce{H_2SO_4  )aq( + 2NaOH )aq( -> Na_2SO_4 )aq( + 2H_2O )l(}
\end{flushleft}
\newpage
مسئله: حل کردن ۰.۱ مول کلسیم کلرید در گرما‌سنجی حاوی
$\mathrm{0.5Kg}$
آب، دمای گرماسنج را
$\mathrm{1.2\degree C}$
بالا می‌برد. ظرفیت گرمایی گرما‌سنج، چند
$\mathrm{KJ.\degree C^{-1}}$
است؟ و اگر در ابتدای واکنش به جای کلسیم کلرید، $\mathrm{30g}$ آمونیوم نیترات ۸۰٪ خالص را در آب حل کنیم، دمای مجموعه به تقریب چند
$\mathrm{\degree C}$
تغییر می‌کند؟ افزایش می‌یابد یا کاهش؟\\(آنتالپی انحلال
$\mathrm{CaCl_{2}}$
و
$\mathrm{NH_{4}NO_{3}}$
به ترتیب -۸۵.۲ و +۲۶ کیلوژول بر مول است.)
$\mathrm{C_{H_2O}=4.2(\frac{J}{g.\degree C})}$
\vspace{8em}
\lin
ب) روش غیرمستقیم: گرمای واکنش را می‌توان به کمک محاسبه، و با استفاده از استوکیومتری، آنتالپی تشکیل مواد، آنتالپی پیوند، و قانون هس محاسبه کرد، که در کتاب درسی، به دو مورد آخر پرداخته شده است.
\lin
\begin{center}
	\underline{\textbf{جمع‌پذیری گرمای واکنش‌ها، \Ovalbox{«قانون هس»}}}
\end{center}
آنتالپی بسیاری از واکنش‌ها را نمی‌توان به روش \censor{نمیدونم} اندازه‌گیری نمود. برخی واکنش‌ها، \underline{یک مرحله} از واکنشی «\censor{نمیدونم} مرحله»  (پیچیده) هستند، و برخی از آن‌ها، به آسانی انجام نمی‌شوند، (یا اصلا انجام نمی‌شوند!)\\
در این حالات، برای محاسبه گرمای واکنش، می‌توان از قانون هس کمک گرفت.\\
براساس «قانون هس»: \Ovalbox{اگر واکنشی شامل «چند» مرحله باشد، $\mathrm{\Delta H}$ مراحل آن است.}\\
به بیان دیگر: \Ovalbox{گرمای یک واکنش معین، به راهی که برای انجام آن پیش‌گرفته، وابسته \censor{نمیدونم}.}
روش کار: اگر معادله واکنشی را بتوان از «مجموع» معادله چند واکنش به دست آورد؛ $\mathrm{\Delta H}$ واکنش کلی نیز از \censor{نمیدونم} \censor{نمیدونم} $\mathrm{\Delta H}$ همان چند واکنش (مراحل) به دست می‌آید.
\lin
مثال: حشره‌ای با نام «سوسک بمب‌افکن»، برای دفاع از خود، مخلوطی از مواد داغ را به سمت دشمن پرتاب می‌کند، که این مواد در طرف دوم واکنش کلی دیده می‌شوند. اگر واکنش کلی در واقع شامل سه مرحله با $\mathrm{\Delta H}$‌های گفته شده باشد\RTLfootnote{اگر واکنش شیمیایی با $\mathrm{\Delta H}$ وابسته به آن معرفی شود، به آن، واکنش \censor{نمیدونم} \censor{نمیدونم} یا \censor{نمیدونم} \censor{نمیدونم} می‌گویند.}، $\mathrm{\Delta H}$ واکنش کلی را به دست آورید.
\begin{flushleft}
	\begin{tabular}{l l}
		$\mathrm{(\Delta H_1=177 KJ)}$  & $\circled{1}\;\ce{C_6H_6O_2 (aq) -> C_6H_4O_2 + H_2 (g)};$         \\
		$\mathrm{(\Delta H_2=-95 KJ)}$  & $\circled{2}\;\ce{H_2O_2 (aq) -> H_2O (l) + \frac{1}{2}O_2 (g)};$  \\
		$\mathrm{(\Delta H = -286 KJ)}$ & $\circled{3}\;\ce{H_2 (g) + \frac{1}{2}O_2 (g) -> H_2O (l)};$      \\
		$\mathrm{(\Delta H=?)}$         & $\ce{C_6H_6O_2 (aq) + H_2O_2 (aq) -> C_6H_4O_2 (aq) + 2H_2O (l)};$
		:واکنش کلی
	\end{tabular}
\end{flushleft}
\newpage
\textbf{توجه:}
در اکثر موارد، برای آن که از جمع‌بندی مواد در مراحل مختلف، به واکنش کلی برسیم، لازم است که تغییراتی را در واکنش‌های مراحل، انجام دهیم. این تغییرات، شامل تغییر در ضرایب، و یا جابه‌جایی واکنش‌دهنده‌ها با فرآورده‌ها است. مثلا ضریب ماده‌ای در واکنش کلی \underline{۲} اما در مراحل \underline{۱} است یا ماده‌ای در واکنش کلی در طرف اول، اما در مراحل در طرف دوم است.\\
\Ovalbox{\textbf{قوانین پایداری:}}
\begin{en}
	\item اگر ضرایب واکنشی n برابر شود، $\Delta$H واکنش باید در \censor{نمیدونم} \censor{نمیدونم} شود.
	\item اگر جای واکنش‌دهنده(ها) با فرآورده(ها) عوض شود، $\Delta$H واکنش باید \censor{نمیدونم} شود(علامت \censor{نمیدونم} بگیرد.)\\
\end{en}
\hrule
\vspace{4pt}
تمرین ۱: با توجه به
$\mathrm{\Delta H_1}$
در واکنش اول،
$\mathrm{\Delta H_2}$
و
$\mathrm{\Delta H_3}$
را به دست آورید:
\begin{flushleft}
	\begin{tabular}{l l}
		$\mathrm{; \Delta H_1=-395KJ}$             & $\ce{S (s) + \frac{3}{2} O_2 (g) -> SO_3 (g)}$ \\
		$\mathrm{; \Delta H_2=\censor{نمیدونم}KJ}$ & $\ce{2S (s) + 3O_2 (g) -> 2SO_3 (g)}$          \\
		$\mathrm{; \Delta H_3=\censor{نمیدونم}KJ}$ & $\ce{SO_3 (g) -> S (s) + \frac{3}{2}O_2 (g)}$
	\end{tabular}
\end{flushleft}
\vspace{2em}
\lin
\vspace{5pt}
تمرین ۲: متان، ساده‌ترین هیدروکربن و نخستین عضو خانواده \censor{نمیدونم} است، و بخش عمده \censor{نمیدونم} \censor{نمیدونم} را تشکیل می‌دهد. متان از \censor{نمیدونم} گیاهان به وسیله \underline{باکتری‌های بی‌هوازی} «در آب» تولید می‌شود. نخستین بار، از سطح \censor{نمیدونم} جمع‌آوری شده و به \underline{گاز مرداب} معروف است. برای تهیه این گاز، می‌توان از واکنش روبه‌رو استفاده کرد:
\begin{flushleft}
	$\mathrm{\ce{C (\text{گرافیت, S}) + 2H_2 (g) -> CH_4 (g)} (\Delta H = ?)}$
\end{flushleft}
آزمایش‌ها و یافته‌های تجربی نشان می‌دهند که تامین شرایط بهینه برای انجام واکنش بالا، بسیار دشوار و پرهزینه است. برای تعیین $\mathrm{\Delta H}$ این واکنش، می‌توان از سه واکنش ترموشیمیایی دیگر بهره گرفت: ($\mathrm{\Delta H}$ واکنش بالا را محاسبه کنید.)
\begin{flushleft}
	\begin{tabular}{l l}
		$\mathrm{(\Delta H_1 = -393.5 KJ)}$ & $\circled{1} \ce{C (\text{گرافیت، S}) + O_2 (g) -> CO_2(g)}$   \\
		$\mathrm{( \Delta H_2 = -286 KJ)}$  & $\circled{2} \ce{H_2 (g) + \frac{1}{2}O_2 (g) -> H_2O (l)}$    \\
		$\mathrm{( \Delta H_3 = -890 KJ)}$  & $\circled{3} \ce{CH_4 (g) + 2O_2 (g) -> 2H_2O (l) + CO_2 (g)}$
	\end{tabular}
\end{flushleft}
تذکر: ترجیحا هر یک از مواد واکنش را در هر مرحله پیدا کنید که در مراحل دیگر نباشد.
\begin{flushleft}
	$\mathrm{\Delta H=}$
\end{flushleft}
\lin
تمرین ۳: آنتالپی واکنش کلی را محاسبه کنید: (خود را بیازمایید \underline{۲} صفحه \underline{۷۴})
\begin{flushleft}
	\begin{tabular}{l l}
		$\mathrm{; \Delta H_1 = -566 KJ}$ & $\ce{2CO (g) + O_2 (g) -> 2CO_2 (g)}$           \\
		$\mathrm{; \Delta H_2 = 181 KJ}$  & $\ce{N_2 (g) + O_2 (g) -> 2NO (g)}$             \\
		                                  & \tikz\draw [black] (0,0) -- (7cm,0pt);          \\
		$\mathrm{; \Delta H = ?}$         & $\ce{2CO (g) + 2NO (g) -> 2CO_2 (g) + N_2 (g)}$
	\end{tabular}
\end{flushleft}
\vspace{30pt}
\lin
\textbf{توجه: }
واکنش بالا، توسط شیمیدانان هواکرده، و برای تبدیل گاز‌های آلاینده CO و NO (که از اگزوز خودرو‌ها به هواکرده وارد می‌شوند) طراحی شده تا به گاز‌هایی با آلایندگی کمتر و پایداری \censor{نمیدونم} تبدیل شوند.
\newpage
تمرین ۴: (خود را بیازمایید \underline{۱} صفحه \underline{۷۴}) الف)\\
\begin{flushleft}
	\begin{tabular}{l l}
		$\mathrm{; \Delta H_1 = -286 KJ}$ & $\ce{H_2 (g) + \frac{1}{2}O_2 (g) -> H_2O (l)}$ \\
		$\mathrm{; \Delta H_2 = -196 KJ}$ & $\ce{2H_2O_2 (l) -> 2H_2O (l) + O_2 (g)}$       \\
		                                  & \tikz\draw [black] (0,0) -- (7cm,0pt);          \\
		$\mathrm{; \Delta H = ?}$         & $\ce{H_2 (g) + O_2 (g) -> H_2O_2 (l)}$
	\end{tabular}
\end{flushleft}
\vspace{2em}
ب) چون واکنش مستقیم
$\mathrm{ H_2}$ با $\mathrm{O_2}$ \censor{نمیدونم}
تولید می‌کند که
$\frac{\text{پایدارتر}}{\text{ناپایدارتر}}$
است.
$\mathrm{H_2O_2}$
\censor{نمیدونم}
تر است و به \censor{نمیدونم} و \censor{نمیدونم} تجزیه می‌شود.
\lin
تمرین ۵: (خود را بیازمایید \underline{۳}) الف) چون واکنش برخورد مستقیم C با
$\mathrm{O_2}$
، همواره \censor{نمیدونم} تولید می‌کند
($\mathrm{CO_2}$
از
$\mathrm{CO}$ $\frac{\text{پایدارتر}}{\text{ناپایدارتر}}$
است.)\\
ب)\\
\lin
تمرین \underline{۶} (خود را بیازمایید \underline{۴}) الف) \censor{نمیدونم} پایدارتر است (سطح انرژی \censor{نمیدونم} دارد.) دلیل: \textbf{تعداد پیوند} \underline{۲} مول آمونیاک از \underline{۱} مول هیدرازین \censor{نمیدونم} است.\\
ب)\\
\\
\lin
\begin{center}
	\textbf{غذای سالم}
\end{center}
آهنگ واکنش، نشان می‌دهد هر تغییر شیمیایی، در چه گستره‌ای از \censor{نمیدونم} رخ می‌دهد. آهنگ واکنش، معیاری برای تعیین زمان \censor{نمیدونم} مواد است.\\
هرچه گستره زمان انجام واکنش، \underline{کوچک‌تر} باشد، آهنگ انجام آن، \censor{نمیدونم} است، و واکنش، \censor{نمیدونم} تر انجام می‌شود.\\
برخی روش‌های افزایش زمان ماندگاری مواد غذایی:
\begin{iit}
	\item روش‌های قدیمی:
	\censor{نمیدونم}
	کردن، تهیه \censor{نمیدونم} و \censor{نمیدونم} سود کردن (شکل \underline{۱۰} صفحه \underline{۷۵})
	\item روش‌های جدید:
	تخلیه \censor{نمیدونم} درون بسته‌بندی، \censor{نمیدونم} و \censor{نمیدونم}
	\begin{en}
		\item تهیه \censor{نمیدونم}
		و افزودن \censor{نمیدونم} از روش‌های جدید نگهداری مواد غذایی است.
		\item عوامل محیطی
		مانند \censor{نمیدونم}، \censor{نمیدونم}، \censor{نمیدونم} و \censor{نمیدونم} در \underline{چگونگی} و \underline{زمان} نگهداری مواد غذایی موثر است.
		\item  پودر شدن
		مواد غذایی، (مانند قاووت) \censor{نمیدونم} آنها با اکسیژن را افزایش می‌دهد و در نتیجه، سهت فساد ماده غذایی \censor{نمیدونم} می‌شود.
	\end{en}
\end{iit}

\begin{iit}
	\item واکنش‌های
	تخریب مواد غذایی، در محیط \censor{نمیدونم} انجام می‌شود. با خشک کردن، \censor{نمیدونم} تا حد زیادی حذف و ماندگاری زیاد می‌شود.
	\item استفاده
	از اسید خوراکی (ترشی) یا نمک (با \censor{نمیدونم} مناسب) امکان رشد موجودات ذره‌بینی را کم می‌کند.
\end{iit}
\textbf{نکته:}
تهیه و تولید سریعتر یا کندتر یک فرآورده (صنعتی، دارویی یا غذایی) بر \censor{نمیدونم} و زمان \censor{نمیدونم} آن موثر است.
\Ovalbox{
	آهنگ انجام واکنش، در گستره‌ای از
	\censor{نمیدونم}،
	با نام \censor{نمیدونم} واکنش بیان می‌شود.}
خود را بیازمایید صفحه \underline{۷۶}:\\
الف) کاهش \censor{نمیدونم} $\leftarrow$ کاهش \censor{نمیدونم} واکنش‌های فساد مواد $\leftarrow$ افزایش \censor{نمیدونم}\\
ب) جلوگیری از اثر مخرب \censor{نمیدونم} (و سایر امواج \censor{نمیدونم}) بر روغن
پ) \censor{نمیدونم} کردن مغز دانه‌های خوراکی $\leftarrow$ \censor{نمیدونم} سطح تماس مواد غذایی با \censor{نمیدونم} $\leftarrow$ کم شدن \censor{نمیدونم}
\lin
\begin{center}
	\textbf{مقایسه کیفی سرعت واکنش‌ها (شکل \underline{۱۲}) صفحه \underline{۷۸}}
\end{center}
الف) \underline{انفجار}، یک واکنش شیمیایی «\censor{نمیدونم}» است.\\
در انفجار، مقدار \censor{نمیدونم} ماده منفجرشونده (حالت \censor{نمیدونم} یا \censor{نمیدونم})، «حجم» \censor{نمیدونم} از \censor{نمیدونم} داغ تولید می‌کند.\\
ب) \underline{تشکیل رسوب}، واکنشی «\censor{نمیدونم}» است. مثال:
\begin{flushleft}
	$\ce{NaCl(aq) + AgNO_3 (aq) -> AgCl(\quad) + NaNO_3 (aq)}\qquad \text{نام: \censor{نمیدونم}}$
\end{flushleft}
پ) \underline{زنگ‌زدن}، واکنشی «\censor{نمیدونم}» است.\\
اشیای آهنی، در خوا \censor{نمیدونم} زنگ می‌زنند. زنگار تولیدشده، \censor{نمیدونم} و \censor{نمیدونم} است و \censor{نمیدونم} می‌ریزد.\\
ت) \underline{پوسیدن کاغذ}، واکنشی «\censor{نمیدونم}» است. کاغذ از \censor{نمیدونم} تشکیل شده و تجزیه آن به \censor{نمیدونم} در گذر زمان، باعت «\censor{نمیدونم}» و «پوسیده شدن» کاغذ از کتاب‌های قدیمی می‌شود.
\lin
\begin{center}
	\textbf{عوامل موثر بر سرعت واکنش}\qquad
	4+1\qquad
	$\left(^{\text{\underline{۱} عامل موثر اما ثابت}}_{\text{\underline{۴} عامل موثر و متغیر}}\right)$
\end{center}
\Ovalbox{
	$^{\circled{1}}$
	افزایش \censor{نمیدونم}،
	$^{\circled{2}}$
	افزایش \censor{نمیدونم} واکنش‌دهنده(ها)،
	$^{\circled{3}}$
	افزایش سطح \censor{نمیدونم}،
	$^{\circled{4}}$
	استفاده از \censor{نمیدونم}
}
خود را بیازمایید صفحه \underline{۸۰} و \underline{۸۱}\\
الف) سرعت واکنش پتاسیم با آب (شکل سمت \censor{نمیدونم}) از واکنش شدیم با آب (سمت \censor{نمیدونم}) \censor{نمیدونم} است.
دلیل: خاصیت \censor{نمیدونم} و \censor{نمیدونم} پتاسیم از سدیم بیشتر است.\RTLfootnote{
	تغییر \censor{نمیدونم} واکنش‌دهنده (\censor{نمیدونم} واکنش‌دهنده)، می‌تواند واکنش سریع‌تری به راه اندازد اما نمی‌تواند عاملی برای تغییر سرعت یک واکنش مشخص باشد. (خود را بیازمایید صفحه \underline{۸۰})
}\\
یعنی:
\Ovalbox{
	\textbf{
		تغییر دادن \censor{نمیدونم}، می‌تواند واکنش سریع‌تری به راه بیاندازد.
	}
}\\
ب) شعله آتش، \censor{نمیدونم} آهن موجود در کپسول چینی را داغ و \censor{نمیدونم} می‌کند اما پاشیدن و پخش کردن گرد آهن بر روی \censor{نمیدونم}، سبب \censor{نمیدونم} آن می‌شود. (شکل سمت \censor{نمیدونم}) (تذکر: اندازه ذرات در گرد آهن از براده آهن \censor{نمیدونم} است.)\\
\circled{$\star$}
عامل موثر بر سرعت: افزایش \censor{نمیدونم} واکنش‌دهنده‌ها\\
پ) محلول \censor{نمیدونم}
{\footnotesize (رنگ) }
پتاسیم پرمنگنات
($\text{(aq)}$ \censor{نمیدونم})
با یک
\censor{نمیدونم}
آلی در دمای اتاق به \censor{نمیدونم} واکنش می‌دهد اما با گرم شدن محلول، به \censor{نمیدونم} بی‌رنگ می‌شود (واکنش می‌دهد) (شکل سمت \censor{نمیدونم})\\
\circled{$\star$}
عامل موثر بر سرعت: افزایش \censor{نمیدونم}\\
ت) الیاف آهن داغ و \censor{نمیدونم} شده (روی شعله) در هوا $\frac{\text{می‌سوزد}}{\text{نمی‌سوزد}}$ اما در ارلن پر از  اکسیژن \censor{نمیدونم} (شکل سمت \censor{نمیدونم})\\
ت) الیاف آهن داغ و \censor{نمیدونم} شده (روی شعله) در هوا $\frac{\text{می‌سوزد}}{\text{نمی‌سوزد}}$ اما در ارلن پر از اکسیژن \censor{نمیدونم} (شکل سمت \censor{نمیدونم})\\
\circled{$\star$}
عامل موثر بر سرعت: افزایش \censor{نمیدونم}\\
ث) محلول هیدروژن پراکسید (\censor{نمیدونم} $\text{(aq)}$) در دمای اتاق به \censor{نمیدونم} تجزیه‌شده و \censor{نمیدونم} تولید می‌کند:
\begin{flushleft}
	$\ce{H_2O_2 (aq) ->[\censor{نمیدونم} (aq)] H_2O (l) + O_2 (g)}\qquad \text{(موازنه کنید)}$
\end{flushleft}
در حالی که افزودن دو قطره از محلول پتاسیم یدید
($\mathrm{(aq) \censor{نمیدونم}}$)
سرعت واکنش را به طور چشمگیری \censor{نمیدونم} می‌دهد (شکل سمت \censor{نمیدونم})\\
\circled{$\star$}
عامل موثر بر سرعت: \censor{نمیدونم}\\
ج) بیماران دارای مشکل تنفسی، در شرایط اضطراری نیاز به تنفس از کپسول \censor{نمیدونم} دارند. دلیل: \censor{نمیدونم} اکسیژن در کپسول اکسیژن از \censor{نمیدونم} بیشتر است و با هر بار عمل دم، اکسیژن بیشتری وارد ریه می‌شود.\\
\circled{$\star$}
عامل موثر بر سرعت: افزایش \censor{نمیدونم} واکنش‌دهنده\\
ح) برخی افراد با مصرف کلم و حبوبات، دچار نفخ می‌شوند زیرا فاقد \censor{نمیدونم} هستند که آن‌ها را کامل و سریع هضم کند.\\
دلیل: آنزیم‌ها در بدن، نقش \censor{نمیدونم} را دارند و «کمبود» یا «فقدان» آن‌ها، واکنش‌های هضم را \censor{نمیدونم} می‌کند.\\
\circled{$\star$}
نقش \censor{نمیدونم} در سرعت واکنش\\
خ) واکنش سوختن قند آغشته به \censor{نمیدونم} \censor{نمیدونم} سریع‌تر از سوختن قند در حالت عادی است.\\
دلیل: در خاک باغچه، \censor{نمیدونم} مناسب برای این واکنش وجود دارد.\\
\circled{$\star$}
نقش \censor{نمیدونم} در سرعت واکنش
\begin{center}
	\textbf{پیوند با صنعت}
\end{center}
در صنایع غذایی، علاوه بر بسته‌بندی، کنسرو‌سازی، انجماد و غیره، استفاده از مواد \censor{نمیدونم} به عنوان \censor{نمیدونم} سبب افزایش زمان \censor{نمیدونم} و \censor{نمیدونم} مواد غذایی است. «\censor{نمیدونم}ها»، مواد شیمیایی مانند \censor{نمیدونم}، \censor{نمیدونم} دهنده‌ها و \censor{نمیدونم} دهنده‌ها هستند که به صورت هدف‌مند به مواد غذایی افزوده می‌شوند. یکی از این افزودنی‌ها «\censor{نمیدونم} اسید» است که به طور طبیعی در \censor{نمیدونم} و \censor{نمیدونم} وجود دارد و به عنوان \censor{نمیدونم} به مواد غذایی افزوده می‌شود. نگه‌دارنده‌ها، \censor{نمیدونم} واکنش‌های شیمیایی منجر به \censor{نمیدونم} مواد غذایی را \underline{کاهش} می‌دهند. بنزوییک اسید به علت داشتن گروه COOH جزء اسید‌های \censor{نمیدونم} است.
\RTLfootnote{مانند $\mathrm{CH_3COOH}$ با نام \censor{نمیدونم} اسید یا \censor{نمیدونم} اسید}
ازطرفی، بنزوییک اسید، حلقه \censor{نمیدونم} دارد پس جزء ترکیبات \censor{نمیدونم} نیز هست. (اسید \censor{نمیدونم})
\begin{center}
	\begin{tabular}{l@{\quad یا \quad}l@{\quad یا \quad}l}
		\chemname{\censor{نمیدونم} - $\mathrm{COOH}$}{$\mathrm{C_{\censor{99}}H_{\censor{99}}O_{\censor{99}}}$}&\chemfig{*6([,0.5]=-=-(-[2])=-)}&\chemname{\chemfig{*6([,0.5]=-=-(-[2](-[1])(-[3]))=-)}}{بنزوییک اسید}
	\end{tabular}
\end{center}
\newpage
پیوند با ریاضی صفحه \underline{۸2} \underline{۸3}:
۱) کمیت \censor{نمیدونم}، سطح تماس تکه زغال را با شعله در هنگام سوختن نشان می‌دهد، چون در عمق زغال، واکنش سوختن به خوبی انجام \censor{نمیدونم} (به دلیل کافی $\frac{\text{بودن}}{\text{نبودن}}$ \censor{نمیدونم} در دسترس)
۲) سطح آن \censor{نمیدونم} برابر و حجم آن \censor{نمیدونم} برابر می‌شود (حجم تغییر \censor{نمیدونم})
۳) گرد زغال نسبت به تکه زغال، \censor{نمیدونم} بیشتری با \censor{نمیدونم} برای سوختن دارد و سرعت سوختن گرد زغال \censor{نمیدونم} است. هرچه سطح تماس بیشتر و به \censor{نمیدونم} (\censor{نمیدونم}) نزدیک‌تر باشد، سرعت واکنش ان با سایر مواد یا تجزیه آن، \censor{نمیدونم} می‌شود.\\
\begin{tikzpicture}
	\draw[-Triangle, very thick](1, 0) -- (0, 0);
\end{tikzpicture}
 برخی واکنش‌های شیمیایی مانند گوارش، تنفس، تهیه دارو‌ها و تولید فرآورده‌های صنعتی، \censor{نمیدونم} و \censor{نمیدونم} هستند.\\
\begin{tikzpicture}
	\draw[-Triangle](1, 0) -- (0, 0);
\end{tikzpicture}
در چنین واکنش‌هایی باید سرعت را \censor{نمیدونم} داد(تا فرآورده‌های گوناگون، با صرفه اقتصادی تولید شوند.)\\
\begin{tikzpicture}
	\draw[-Triangle, very thick](1, 0) -- (0, 0);
\end{tikzpicture}
برخی دیگر از واکنش‌ها مانند «\censor{نمیدونم} وسایل آهنی»، «تولید \censor{نمیدونم} ها» و «\censor{نمیدونم} و \censor{نمیدونم} شدن کاغذ»، \censor{نمیدونم} بار و \censor{نمیدونم} هستند.
\begin{tikzpicture}
	\draw[-Triangle] (1, 0) -- (0, 0);
\end{tikzpicture}
درچنین واکنش‌هایی باید به دنبال راه‌هایی برای \underline{\censor{نمیدونم} سرعت} یا حتی \underline{\censor{نمیدونم} نمودن} واکنش بود.\\
برای دستیابی به چنین اهدافی، باید از \censor{نمیدونم} شیمیایی کمک گرفت.\\
\Ovalbox{
	سینتیک شیمیایی، به بررسی \censor{نمیدونم} و \censor{نمیدونم} انجام واکنش‌ها و \censor{نمیدونم} بر سرعت واکنش‌ها می‌پردازد.
}
\begin{center}
	\textbf{سرعت تولید یا مصرف مواد شرکت‌کننده در واکنش از دیدگاه \underline{کمی}}
\end{center}
سرعت واکنش در موارد زیادی باید با دقت اندازه‌گیری شود، یعنی باید سرعت را به شکل \censor{نمیدونم} بیان کرد. برای این کار باید \censor{نمیدونم} واکنش را به صورت «عدد» بیان کنیم.
\begin{center}
	\Ovalbox{
		پیشرفت واکنش: مصرف \censor{نمیدونم} یا تولید \censor{نمیدونم}
	}
\end{center}
بدیهی است که پیشرفت واکنش در گستره‌ای از \censor{نمیدونم} انجام می‌گیرد.\\
نمونه: شکل \underline{۱۴} صفحه \underline{۸۴}: در یک واکنش شیمیایی، \censor{نمیدونم} خوراکی موجود در محلول، وارد واکنش‌شده و در زمان \underline{۵} دقیقه تا مرز \censor{نمیدونم} شدن پیش رفته است. یعنی با پیشرفت واکنش، \censor{نمیدونم} رنگ، \censor{نمیدونم} می‌یابد و تقریبا به \censor{نمیدونم} می‌رسد.
برای محابسه کمی سرعت واکنش، باید بدانیم که \censor{نمیدونم} رنگ مصرفی چقدر بوده و در چه \censor{نمیدونم} مصرف شده است.
خود را بیازمایید \underline{۱} صفحه \underline{۸4}: با توجه به پرسش، در اینجا باید تغییرات \censor{نمیدونم} (\censor{نمیدونم} مصرفی) را در واحد زمان اندازه‌گیری کنیم:
\begin{flushleft}
	$\mathrm{R=\frac{\Delta n}{\Delta t}=\frac{\censor{نمیدونم} (\censor{نمیدونم})}{\censor{نمیدونم} (\censor{نمیدونم})}=\censor{نمیدونم} (\qquad\qquad)}$
\end{flushleft}
\Ovalbox{
	\begin{minipage}{0.98\linewidth}
		«\censor{نمیدونم}»
		یک واکنش‌دهنده یا «\censor{نمیدونم}» یک فرآورده در گستره \censor{نمیدونم} قابل اندازه‌گیری را «سرعت \censor{نمیدونم}» (مصرف یا تولید) آن ماده می‌نامند (
		$\mathrm{\bar{R}}$
		)
	\end{minipage}
}
$\Leftarrow$
چرا سرعت «متوسط»؟ چون به صورت عادی، با گذشت زمان، سرعت مصرف یا تولید مواد \censor{نمیدونم} می‌شود یعنی در گستره زمانی انجام واکنش، \underline{معمولا} سرعت، \censor{نمیدونم} نیست.
\lin
خود را بیازمایید \underline{۲}: الف) واکنش‌پذیری\quad روی $\bigcirc$ مس
\begin{center}
	\begin{vwcol}[widths={0.8,0.2}, sep=.8cm, justify=flush,rule=0pt,indent=10pt]
		\begin{minipage}{.2\linewidth}
			\Ovalbox{حذف یون تماشاچی}
		\end{minipage}\hfil
		\begin{minipage}{.8\linewidth}
			\begin{latin}
				\begin{tabular}{lllllllll}
					&$\mathrm{Zn (s)}$&+&$\mathrm{CuSO_4 (aq)}$&$\rightarrow $&$\mathrm{\qquad (aq)}$&+&$\mathrm{\qquad (s)}$&\\
					&$\mathrm{Zn (s)}$&+&$\mathrm{\qquad (aq)}$&$\rightarrow $&$\mathrm{\qquad (aq)}$&+&$\mathrm{\qquad (s)}$&\\
					$\rightarrow$&\censor{نمیدونم}&&\censor{نمیدونم}&&\censor{نمیدونم}&&\censor{نمیدونم}&
				\end{tabular}
			\end{latin}
		\end{minipage}
	\end{vwcol}
\end{center}
در این واکنش، \underline{«فلز \censor{نمیدونم}» }الکترون \censor{نمیدونم} و \underline{«کاتیون \censor{نمیدونم}»} الکترون \censor{نمیدونم} است، پس:\\واکنش‌پذیری روی از مس \censor{نمیدونم} است. (واکنش‌پذیری فلز، تقریبا معادل الکترون \censor{نمیدونم} آن است.)

ب) با گذشت زمان، $\mathrm{Cu^{2+}}$ به \censor{نمیدونم} تبدیل می‌شود:\\
مقدار (و غلظت)
$\mathrm{Cu^{2+}}$
 \censor{نمیدونم}
 می‌یابد. (\censor{نمیدونم} محلول، کم و کم‌تر می‌شود.) و مقدار $\mathrm{Cu}$
 \censor{نمیدونم}
 می‌شود. (از \censor{نمیدونم} خارج می‌شود و بر سطح \censor{نمیدونم} (یا \censor{نمیدونم} ظرف) می‌نشیند)
 
 پرسش) تغییرات مقدار $\mathrm{Zn}$ و $\mathrm{Zn^{2+}}$ چگونه است؟\\
 با گذشت زمان، \censor{نمیدونم} به \censor{نمیدونم} تبدیل می‌شود. مقدار (و غلظت) $\mathrm{Zn^{2+}}$ \censor{نمیدونم} می‌یابد. (محلول نهایی \censor{نمیدونم} رنگ است.) و $\mathrm{Zn}$ \censor{نمیدونم} می‌شود (مقداری از تیغه روی \censor{نمیدونم} می‌شود.)\\
 نکته: اگر فرض کنیم که فلز (مس) تولید شده، فقط روی تیغه (روی) بنشیند، تغییر جرم تیغه، از مقایسه جرم روی \censor{نمیدونم} شده با جرم مس \censor{نمیدونم} شده به تیغه، به دست می‌آید.
 
 پ)
\begin{flushleft}
	 $\mathrm{\bar{R}_{Ca^{2+}}=\frac{\Delta n}{\Delta t} = \frac{\censor{نمیدونم} (\qquad)}{\censor{نمیدونم} (\qquad)} = \censor{نمیدونم} (\qquad)}$
\end{flushleft}
 \lin
 با هم بیندیشیم صفحه \underline{۸۵}:
 \begin{flushleft}
 	$\ce{CaCO_3 (aq) + \textbf{2}HCl (aq) -> \censor{نمیدونم} (aq) + \censor{نمیدونم} (l) + \censor{نمیدونم} (g)}$
 \end{flushleft}
 الف) (\censor{نمیدونم}) \censor{نمیدونم} تولیدی، از \censor{نمیدونم} خارج و \censor{نمیدونم} مخلوط باقی‌مانده \censor{نمیدونم} می‌شود.\\
 ب) در کتاب درسی\\
 پ) با گذشت زمان، مجموع جرم گاز آزاد شده، \censor{نمیدونم} می‌شود (اما در مقایسه بازه‌های زمانی \underline{۱۰} ثانیه‌ای، هرچه زمان می‌گذرد، در بازه‌های بعدی، گاز \censor{نمیدونم} آزاد می‌شود). مثال:\\
 گاز آزاد شده در \underline{۱۰} ثانیه اول: \qquad گاز آزاد شده در \underline{۱۰} ثانیه دوم:\\
 ت)در ثانیه \censor{نمیدونم} به پایان می‌رسد چون \censor{نمیدونم} مخلوط پس از آن تغییر \censor{نمیدونم} است.\\
 \Ovalbox{
	\textbf{تذکر مهم:}
	برای اندازه‌گیری سرعت،‌ باید بازه زمانی \censor{نمیدونم} تا \censor{نمیدونم} $\mathrm{(s)}$ را در نظر گرفت.
}\\
تمرین (با هم بیندیشیم) ۲ و ۳: در کتاب درسی / تمرین (با هم بیندیشیم ۴): سرعت \censor{نمیدونم} تولید $\mathrm{CO_2}$ (\qquad) با گذشت زمان \censor{نمیدونم} می‌شود.\\
یعنی: سرعت واکنش‌های شیمیایی به تدریج \censor{نمیدونم} و \censor{نمیدونم} تر می‌شود. (واکنش در ابتدا نسبتا \censor{نمیدونم} تر و در پایان نسبتا \censor{نمیدونم} تر انجام می‌شود.)\\
\textbf{دلیل:}
با پیشرفت واکنش، مقدار \censor{نمیدونم} ها به تدریج چه تغییری می‌کند؟ چرا؟

با هم بیندیشیم تمرین ۵: شیب نمودار همان
$\mathrm{\frac{\Delta \censor{نمیدونم}}{\Delta \censor{نمیدونم}}}$
است و به مرور \censor{نمیدونم} می‌شود (
$\mathrm{\frac{\Delta \censor{نمیدونم}}{\Delta \censor{نمیدونم}}}$
در واقع بیانگر \censor{نمیدونم} است که به تدریج \censor{نمیدونم} می‌شود.) \censor{نمیدونم} های تولیدی هر سه فرآورده در این واکنش برابر است یعنی به \censor{نمیدونم} یکسان تولید می‌شوند.\\
$\Delta t$
نیز برای همه \censor{نمیدونم} واکنش (از جمله فرآورده‌ها) \censor{نمیدونم} است. نتیجه: سرعت متوسط \censor{نمیدونم} فرآورده‌ها \censor{نمیدونم} است. (چون \censor{نمیدونم} اولیه فرآورده‌ها صفر بوده، نمودار مقدار-زمان برای این سه ماده، یکسان است.)
\end{document}

% UNFINSH THINGS:
% 1- CHARTS
% 3- page 9 choices



