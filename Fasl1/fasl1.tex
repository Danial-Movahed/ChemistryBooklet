\documentclass[a4paper,12pt]{article}
%\usepackage[top=1in,margin=5em]{geometry}
\usepackage[top=7em,bottom=1pt,right=0.8in,left=0.8in,headheight=65pt,headsep=1cm]{geometry}
\usepackage{tikz-page}
\usepackage{tikz}
\usepackage{fancybox}
\usetikzlibrary{shadows.blur}
\usepackage{enumitem}
\usepackage{zref-abspage}
\usepackage{tabularray}
\usepackage{pbox}
\usepackage[version=4]{mhchem}
\usepackage{tasks}
\usepackage{multicol}
\usepackage{multirow}
\usepackage{tabularx}
\usepackage{chemfig}
\usepackage{perpage}
\usepackage{setspace}
\usepackage{vwcol}  
\usepackage{mathtools}
\usepackage{textcomp, gensymb}
\usepackage{xepersian}



\pagestyle{plain}
\tikzset{
	secnode/.style={
		minimum height=1cm,
		inner xsep=20pt,
		rotate=90,
		anchor=north east,
		draw=white,
		fill=black,
		text=white,
		blur shadow={shadow blur steps=5,shadow blur extra rounding=1.3pt}},
	pagenode/.style={
		minimum width=5mm,
		minimum height=1cm,
		inner sep=2pt,
		anchor=south east,
		draw=white,
		fill=black,
		text=white,
		blur shadow={shadow blur steps=5,shadow blur extra rounding=1.3pt}}
}
\newcommand{\tikzpagelayout}{
	\draw[black,line width=2pt,rounded corners=20pt] ([xshift=10mm]page.northwest) |- ([xshift=-2cm,yshift=10mm]page.southeast);
	\node[secnode] at ([xshift=5mm]page.northwest) {سالم غذا‌های پی در | شکیباییان};
	\node[pagenode] at ([xshift=-1cm,yshift=5mm]page.southeast) {\thepage};
	
}

\newenvironment{en}
{\begin{enumerate}\setlength\itemsep{-0.2em}}
	{\end{enumerate}}

\newenvironment{iit}
{\begin{itemize}\setlength\itemsep{-0.5em}}
	{\end{itemize}}

\newcommand*\circled[1]{\tikz[baseline=(char.base)]{
		\node[shape=circle,draw,inner sep=2pt] (char) {#1};}}

\settextfont{XB Niloofar}
\setdigitfont{XB Niloofar}
\setstretch{1.5}
\setlist[itemize]{topsep=0pt}
\setlist[enumerate]{topsep=0pt}
\renewcommand{\headrulewidth}{0pt}
\setlength{\headheight}{14.5pt}
\setlength{\columnseprule}{1pt}
\setlength{\columnsep}{0.5cm}
\MakePerPage{footnote}

\newcommand{\fb}{\rule{2cm}{0.15mm}\;}
\newcommand{\fs}{\rule{1cm}{0.15mm}}
\newcommand{\fsm}{{\rule{0.5cm}{0.15mm}\;}}
\newcommand{\lin}{\vspace{4pt}\hrule\vspace{4pt}}
\newcommand\gototask[1]{\addtocounter{task}{\numexpr#1-\value{task}\relax}}
\NewTasksEnvironment[label=\arabic*.,label-format=\bfseries,label-width=4ex]{answers}[\a]
\def\extra{\rule{1ex}{0ex}}
\makeatletter
\newcommand\censor{\@ifstar{\@cenmath}{\@centext}}
\newcommand\@cenmath[1]{%
	\protect\rule[-.3ex]{\widthofpbox{\extra$#1$}}{0.1ex}}
\newcommand\@centext[1]{%
	\protect\rule[-.3ex]{\widthofpbox{\extra#1}}{0.1ex}}
\makeatother

\begin{document}
\textbf{مواد}،
در زندگی ما، نقشی شگرف و موثر دارند. صنایع غذا، پوشاک، حمل ‌و نقل، ساختمان، ارتباطات و غیره، کم و پیش تحت تاثیر \censor{نمیدونم} هستند. رشد و گسترش تمدن بشری در گرو کشف و شناخت مواد \censor{نمیدونم} است. برای رفع نیازها، باید مواد \censor{نمیدونم} تولید شوند، یا با \censor{نمیدونم} مواد، خواص آن‌ها تغییر کند. شیمی‌دان‌ها با پی بردن به رابطه \censor{نمیدونم} مواد با \censor{نمیدونم} سازنده، دریافتند که «\censor{نمیدونم} دادن» به مواد و «\censor{نمیدونم} مواد به یکدیگر»، سبب «\censor{نمیدونم}»، و گاهی «\censor{نمیدونم}» خواص آن‌ها می‌شود. اکنون، می‌توان موادی \textbf{نو}، با ویژگی‌های \textbf{منحصر به فرد} و \textbf{دلخواه} \underline{طراحی} کرد.
\lin


خود را بیازمایید صفحه ۳: الف)
$
\left\{
\begin{tabular}{rrr}
	\text{
		مواد \censor{نمیدونم} (\censor{نمیدونم})
	} & $\leftarrow$ & \text{
		فلز
	}\\
	\text{
		مواد \censor{نمیدونم} (\censor{نمیدونم})
	} & $\leftarrow$ & \text{
		لاستیک
	}
\end{tabular}
\right.
$
$\Leftarrow$ دوچرخه

نتیجه: منشاء اجزای این فرآورده، از \censor{نمیدونم} است.

این فرآیند، شامل به دست آوردن \textbf{مواد} دلخواه از منابع مختلف، برای تولید \censor{نمیدونم} مشخص است؛ یعنی: \censor{نمیدونم} اولیه تهیه دوچرخه، به طور \censor{نمیدونم} قابل استفاده نیستند و باید شوند.

ب)\censor{نمیدونم}، کناره‌های ورق \censor{نمیدونم} برش‌خورده و کناره‌های \censor{نمیدونم} بریده‌ شده، دور ریخته \censor{نمیدونم}

\begin{tabular}{rrr}
	پ) قسمت‌های \censor{نمیدونم}، ممکن است در تماس با \underline{هوا} و \underline{رطوبت}، زنگ بزنند. & $\swarrow$ & \\
	قسمت‌های \censor{نمیدونم} و \censor{نمیدونم}، \underline{فرسوده} و \underline{کهنه} می‌شوند. & $\leftarrow$ &
	\censor{نمیدونم}
\end{tabular}

خود را بیازمایید صفحه ۳ و ۴:
الف) همه مواد \censor{نمیدونم} و \censor{نمیدونم} از کره زمین به دست می‌آیند.

\Ovalbox{
	مواد
}
$
\left.
\begin{tabular}{r}
	\underline{مستقیما}
	از کره زمین به دست می‌آیند؛ مانند فلز‌ها، نفت، الماس و طلا\\
	\underline{غیرمستقیم}
	از زمین به دست می‌آیند؛ (از مواد \censor{نمیدونم} تهیه می‌شوند) مانند لاستیک و پلاستیک
\end{tabular}
\right\}
$

ب) به سه شکل، به زمین باز می‌گردند: «\censor{نمیدونم}» و «\censor{نمیدونم}» (و برخی «\censor{نمیدونم} شده با اجزای هوا‌کره»)

پ) به تقریب، \censor{نمیدونم} کل مواد در کره زمین، \underline{ثابت} می‌ماند. هر چیزی که از زمین استخراج شده، در نهایت به صورت پسماند و زباله، به زمین باز می‌گردد.

ت) هر چه میزان بهره‌برداری از منابع، بیشتر باشد، آن کشور توسعه یافته‌تر است. ($\frac{\text{درست}}{\text{نادرست}}$)

دلیل: «\censor{نمیدونم}» ثروت ملی هستند. بهره‌برداری باید با مدیریت برداشت اصولی از \censor{نمیدونم} همراه باشد:

$\tiny ^{\circled{۱}}$
میزان بهره‌برداری مدیریت شده از منابع،
$\tiny ^{\circled{۲}}$
به داشتن \censor{نمیدونم} برداشت منابع، داشتن «\censor{نمیدونم}» های پیشرفته و
$\tiny ^{\circled{۳}}$
آموزش درست «\censor{نمیدونم} \censor{نمیدونم}» بستگی دارد.

در نظر داشتن ۳ مورد بالا، به پیشرفت پایدار می‌انجامد.

خود را بیازمایید ۳ صفحه ۴:
الف) حدود \censor{نمیدونم} میلیارد تن
ب) بیش از ۷۰ میلیارد تن برای هر سه (حدود ۱۲ میلیارد تن برای فلزها)

میزان مصرف سه منبع: \censor{نمیدونم} > \censor{نمیدونم} > \censor{نمیدونم}

شیب مصرف سه منبع: \censor{نمیدونم} > \censor{نمیدونم} > \censor{نمیدونم} \qquad (پس از سال ۲۰۰۵)
\lin

پ) زمین، منبع عظیمی از هدایای ارزشمند و ضروری برای زندگی است. سالانه، مقادیر بسیار زیادی از منابع \censor{نمیدونم}، \censor{نمیدونم} و\censor{نمیدونم}، برای مصارف گوناگون، استخراج و مورد استفاده قرار می‌گیرند. با پیشرفت «\censor{نمیدونم}» و ساخت \underline{دستگاه‌ها} و \underline{ابزار} بهتر (\censor{نمیدونم} بهتر و مدرن)، وابستگی (نیاز) به منابع، بیشتر\censor{نمیدونم}.
\lin

دانشمندان بزرگ، می‌توانند با برسی دقیق اطلاعات و یافته‌های موجود درباره \textbf{مواد} و \textbf{پدیده‌}های گوناگون، \censor{نمیدونم} ها، \censor{نمیدونم} ها و \censor{نمیدونم} بین آن‌‌ها را درک کنند. (مانند \censor{نمیدونم}، که جدول دوره ای را طراحی نمود.)

شیمی‌دان‌ها با \censor{نمیدونم} مواد و انجام \censor{نمیدونم} (استفاده از هر ۵         ) آن‌ها را دقیق برسی می‌کنند. (آزمایش: \censor{نمیدونم} کنترل شده)

هدف این برسی‌ها، یافتن اطلاعات \underline{بیشتر} و \underline{دقیق‌تر} درباره \censor{نمیدونم} های مواد است. برقراری \censor{نمیدونم} بین این داده‌ها (و اطلاعات) و نیز، ‌یافتن \censor{نمیدونم} ها و \censor{نمیدونم} ها، گامی مهم‌تر و موثر‌تر در پیشرفت علم است.

\begin{tabular}{rr}
	\Ovalbox{
		علم شیمی:
	}
	&
	\Ovalbox{
		\begin{tabular}{r}
			مطالعه \censor{نمیدونم} \censor{نمیدونم}، \censor{نمیدونم} و \censor{نمیدونم} \textbf{رفتار} عنصر‌ها و مواد \\ برای یافتن \censor{نمیدونم} ها و \censor{نمیدونم} های رفتار  \censor{نمیدونم} و \censor{نمیدونم} آن‌ها است.
		\end{tabular}
	}
\end{tabular}
\\

جدول دوره‌ای، مانند یک نقشه راه، به \underline{سازمان‌دهی}، و \underline{تجزیه و تحلیل} داده‌ها در مورد\censor{نمیدونم}، کمک می‌کند تا \censor{نمیدونم} های پنهان در رفتار عنصر‌ها، آشکار شود.

\Ovalbox{
	در جدول دوره‌ای، عنصر‌ها بر اساس بنیادی‌ترین ویژگی آن‌ها، یعنی \censor{نمیدونم} چیده شده است.
}

\Ovalbox{
	تذکر: جدول دوره‌ای جدید بر مبنای \censor{نمیدونم} اتمی و جدول دوره‌ای مندلیف بر اساس \censor{نمیدونم} اتمی مرتب شده‌اند.
}
\Ovalbox{
	جدول دوره‌ای، شامل \censor{نمیدونم} دوره، و \censor{نمیدونم} گروه است.
}

\Ovalbox{
	عنصر‌های جدول، بر اساس \censor{نمیدونم} شان در سه دسته \censor{نمیدونم}، \censor{نمیدونم} و \censor{نمیدونم} قرار می‌گیرند.
}

تعیین موقیت عنصر در جدول، ( تعیین \censor{نمیدونم} و \censor{نمیدونم} در جدول)، به پیش‌بینی \textbf{خواص} و \textbf{رفتار} عنصر، کمک زیادی می‌کند.
با برسی رفتارهای عناصر، می‌توان: \\
\circled{1} 
آن‌ها را دسته‌بندی کرد. \qquad
\circled{2}
به \censor{نمیدونم} ها و \censor{نمیدونم} های موجود در خواص، پی برد.
\lin

داوری کنید: «هر‌گاه تعداد الکترون‌های لایه ظرفیت برای اتم‌های دو عنصر، یکسان باشد، در یک گروه قرار می‌گیرند.»
پاسخ:
\vspace{3em}
\lin

$
\left.
\begin{tabular}{r}
	در عناصر \textbf{هم‌گروه}، \censor{نمیدونم} \censor{نمیدونم} \censor{نمیدونم} \censor{نمیدونم} اتم‌ها مشابه است.\\
	در عناصر \textbf{هم‌دوره}، \censor{نمیدونم} \censor{نمیدونم} \censor{نمیدونم}، یکسان است. (عدد کوانتومی \censor{نمیدونم})
\end{tabular}
\right\}
$
\lin

\begin{center}
	\textbf{الگو‌های رفتاری فلزها}
\end{center}

\begin{en}
	\item 
	 رسانایی \censor{نمیدونم} و \censor{نمیدونم}
	\item
	داشتن \censor{نمیدونم} فلزی (سطح صیقلی و درخشان)
	\item
	قابلیت تبدیل به \censor{نمیدونم} (برگه) و \censor{نمیدونم} (رشته)
	\item 
	خرد \censor{نمیدونم} در اثر ضربه ( \censor{نمیدونم} خواری)
	$\leftarrow$
	فلز‌ها در اثر ضربه، \censor{نمیدونم} می‌پذیرند.
	\item 
	استحکام و مقاومت کششی بالا
	\item 
	\censor{نمیدونم}
	 الکترون در واکنش‌های شیمیایی
\end{en}
\lin

شکل ۳ صفحه ۷:
$
\left.
\begin{tabular}{r}
	زنجیر:\\
	پل فلزی:\\
	وسایل آشپزخانه (و سیم)؛
\end{tabular}
\right\}
$





با هم بیندیشیم صفحه ۷ تا ۹: (برسی شکل الف صفحه ۷):
\circled{1}:
\qquad\qquad
\circled{2}:
\begin{tabular}{r}
	\censor{نمیدونم}
	با \censor{نمیدونم} \\
	\censor{نمیدونم}
	با \censor{نمیدونم} شبیه‌تر
\end{tabular}

نکته: در گروه‌ های جدول، \censor{نمیدونم} خواص مهم‌تر است اما \censor{نمیدونم} داریم. در دوره های جدول \censor{نمیدونم} خواص مهم‌تر است اما \censor{نمیدونم} خواص نیز داریم.

\RL{\circled{5}:}
در گروه ۱۴، از بالا به پایین، خصلت فلزی            یافته است.

\RL{\circled{6}}:
در دوره سوم، از چپ به راست، خصلت فلزی          و خصلت نافلزی            می یابد.

\Ovalbox{
	\begin{tabular}{r}
		قانون دوره ای عنصرها:\\
	خصلت فلزی عنصرها در یک دوره از چپ به راست  \censor{نمیدونم} و در هر گروه از بالا به پایین \censor{نمیدونم} می‌‌یابد.
	\end{tabular}
}

\RL{\circled{7}:}
یشترین خصلت فلزی در هر گروه، در ($\frac{\text{بالای}}{\text{پایین}}$) گروه است. (در گروه اول، عنصرِ \censor{نمیدونم})

\RL{\circled{8}:}
در هر دوره از جدول دوره‌ای ، از چپ به راست از خاصیت \censor{نمیدونم} کاسته و به خاصیت \censor{نمیدونم} افزوده می‌شود.
در گروه‌های ۱۵، ۱۶ و ۱۷، عنصرهای \censor{نمیدونم} خاصیت نافلزی بیشتری دارند زیرا از بالا به پایین، خاصیت \censor{نمیدونم} زیاد می‌شود.

بیشتر عنصرهای جدول را ($\frac{\text{فلز‌ها}}{\text{نافلز‌ها}}$) تشکیل می‌دهند که به طور عمده در «سمت \censor{نمیدونم}» و \textbf{مرکز} جدول جای دارند. \censor{نمیدونم} ها در سمت \censor{نمیدونم} و بالای جدول چیده شده‌اند. ‌شبه فلز‌ها، همانند مرزی بین فلزها و نافلز‌ها قرار دارند.

برخی رفتار‌های شبه فلز‌‌ها (به قول کتاب: خواص فیزیکی‌) به \censor{نمیدونم} شبیه‌تر\\
برخی رفتار‌های شبه فلز‌ها ( به قول کتاب: خواص شیمیایی) به \censor{نمیدونم} شبیه‌تر است.
\lin

\begin{tabular}{r}
	 {\footnotesize 
	رفتار‌ها و خواص}\\
	{\footnotesize
	شبه‌فلز‌ها}
\end{tabular}
$
\left.
\begin{tabular}{r}
	به فلز‌های شبیه‌تر: \censor{نمیدونم} \censor{نمیدونم}، \censor{نمیدونم} \censor{نمیدونم} و \censor{نمیدونم}، \censor{نمیدونم} \censor{نمیدونم} و \censor{نمیدونم}\\
	به نافلز‌ها شبیه‌تر: \censor{نمیدونم} \censor{نمیدونم} \censor{نمیدونم} \censor{نمیدونم} و \censor{نمیدونم} \censor{نمیدونم}
\end{tabular}
\right\}
$
\lin

\begin{center}
	\textbf{«نکاتی درباره فلز‌ها»}
\end{center}

\begin{en}
	\item 
	همه فلز‌ها در دمای اتاق، حالت فیزیکی \censor{نمیدونم} دارند. (به جز \censor{نمیدونم} و \censor{نمیدونم} )
	\item 
	فلز‌ها در هر ۴ دسته \censor{نمیدونم}، \censor{نمیدونم}، \censor{نمیدونم} و \censor{نمیدونم} وجود دارند. تمام عناصر دسته‌های \censor{نمیدونم} و \censor{نمیدونم} فلز هستند. عناصر دسته \censor{نمیدونم} همگی فلز هستند به جز \censor{نمیدونم} و \censor{نمیدونم} فلز‌های
	$\mathrm{Al}$، $\mathrm{Sn}$، $\mathrm{Pb}$
	در دسته \censor{نمیدونم} قرار دارند.
	\item
	اکسید‌های فلزی اغلب، در واکنش با آب، $\frac{\text{اسید}}{\text{باز}}$ تولید می‌کنند. (اکسید‌های \censor{نمیدونم} )
	
	\begin{tabular}{l l}
		\small
		\Ovalbox{
			\ce{\censor{نمیدونم} )g( + \censor{نمیدونم} )aq( -> H2O )l( + CaO )s( }
		} & 
		\small
		\Ovalbox{
			\ce{\censor{نمیدونم} )g( + \censor{نمیدونم} )aq( -> H2O )l( + Na2O )s(}
		}
	\end{tabular}
	
	تذکر: فلز‌های گروه ۱و۲‌ (به جز \censor{نمیدونم} ) نیز در آب، $\frac{\text{اسید}}{\text{باز}}$ و گاز \censor{نمیدونم} تولید می‌کنند:
	
		\begin{tabular}{l l}
		\small
		\Ovalbox{
			\ce{\censor{نمیدونم} )g( + \censor{نمیدونم} )aq( -> H2O )l( + Ca )s( }
		} & 
		\small
		\Ovalbox{
			\ce{\censor{نمیدونم} )g( + \censor{نمیدونم} )aq( -> H2O )l( + Na )s(}
		}
	\end{tabular}
	\item 
	فلز‌ها در واکنش‌های شیمیایی، به صورت \censor{نمیدونم} نوشته می‌شوند.
\end{en}

\begin{center}
	\textbf{«نکاتی در باره نافلز‌ها»}
\end{center}

\begin{en}
	\item
	در دمای اتاق، \censor{نمیدونم} حالت فیزیکی مایع دارد. (۵ عنصر) \censor{نمیدونم}، \censor{نمیدونم}، \censor{نمیدونم}، \censor{نمیدونم} و \censor{نمیدونم} \textbf{جامد} هستند. سایر نافلز‌ها شامل \censor{نمیدونم}، \censor{نمیدونم}، \censor{نمیدونم}، \censor{نمیدونم} و \censor{نمیدونم} و نیز همه عناصر گروه \censor{نمیدونم}، در دمای اتاق، حالت فیزیکی \textbf{گازی} دارند.
	\item
	نافلز‌ها عمدتا در دسته \censor{نمیدونم} جای دارند. (H و He جز دسته \censor{نمیدونم})
	\item
	اکسیدهای نافلزی، اغلب، در واکنش با آب، \censor{نمیدونم} تولید می‌کنند. (اکسید‌های \censor{نمیدونم})
	
	\begin{tabular}{l l}
		\small
		\Ovalbox{
			\ce{\censor{نمیدونم} )aq( -> H2O )l( + N2O5 )s( }
		} & 
		\small
		\Ovalbox{
			\ce{\censor{نمیدونم} )aq( -> H2O )l( + SO3 )g(}
		}
	\end{tabular}
	
	\item 
	۷ عنصر نافلزی،‌در حالت عنصری \textbf{مولکول} \censor{نمیدونم} اتمی دارند: (\censor{نمیدونم} و \censor{نمیدونم} و \censor{نمیدونم} و \censor{نمیدونم} و \censor{نمیدونم} و \censor{نمیدونم} و \censor{نمیدونم})
	\item
	معروف‌ترین الوتروپ گوگرد فرمول، \censor{نمیدونم} دارد که جامدی \censor{نمیدونم} رنگ است. (شکل بالای صفحه ۸ کتاب درسی)
	\item
	فسفر، سه الوتروپ مهم دارد: فسفر \censor{نمیدونم}، \censor{نمیدونم} و \censor{نمیدونم} (دوتای آن‌ها در شکل بالای صفحه ۸ کتاب درسی)
\end{en}
\begin{center}
	\textbf{نکاتی درباره شبه فلز‌ها}
\end{center}
از بین شبه فلز‌های جدول، در کتاب درسی فقط \censor{نمیدونم} و \censor{نمیدونم} معرفی شده‌اند. شبه فلزها:
\begin{en}
	\item 
	همانند \censor{نمیدونم} الکترون به اشتراک می‌گذارند. (در واکنش‌های شیمیایی) (الکترون نمی‌گیرند و از دست نمی‌دهند)
	\item
	همانند \censor{نمیدونم} شکننده‌اند. (در اثر ضربه \censor{نمیدونم} می‌شوند.)
	\item 
	همانند \censor{نمیدونم} رسانایی گرمایی و الکتریکی دارند. (تاحدی) $\leftarrow$ رسانایی الکتریکی: 
	$\mathrm{Ge \bigcirc Si}$
	(دلیل: افزایش خصلت \censor{نمیدونم} عناصر از بالا به پایین در هر گروه)
	\item 
	همانند \censor{نمیدونم} سطح صیقلی و درخشان دارند.
\end{en}

همه \censor{نمیدونم} عنصر جدول دوره‌ای، شناسایی و توسط آیوپاک\footnote{
آیوپاک 
\LR{(I.U.P.A.C): \textbf{I}nternational \textbf{U}nion of \textbf{P}ure \& \textbf{A}pplied \textbf{C}hemistry}
}
 تایید شده‌اند. هیچ خانه‌ای در جدول خالی نیست، و جست‌وجو برای کشف عناصر جدید، عملا به پایان رسیده است. اکنون دانشمندان به دنبال تهیه و تولید عناصر جدید به صورت «\censor{نمیدونم}» هستند.
در صورت کشف
$\footnotesize ^{(!)}$
 (تولید) این عنصر‌ها، باید آن‌ها را بر مبنای عدد «\censor{نمیدونم}»، «\censor{نمیدونم}» و غیره، در خانه‌های جدید قرار داد. برای عنصر‌های جدید (عدد اتمی بیش از \censor{نمیدونم})، در جدول دوره‌ای، جایی وجود ندارد. یکی از پیشنهاد‌ها، جایگزینی جدول فعلی با جدول ‌‌\textbf{«ژانت»} است.
 \lin
 
\begin{center}
	\textbf{
		جدول ژانت \LR{(Charles Janet)} (صفحه ۱۰ و ۱۱ کتاب درسی)
	}
\end{center}
جدول پیشنهادی ژانت، با مدل \textbf{کوانتمی}، همخوانی دارد. در هر دوره جدول ژانت، عناصری با (\censor{نمیدونم} + \censor{نمیدونم}) یکسان قرار دارند.
(در جدول فعلی، عناصر در هر دوره، \censor{نمیدونم} یکسان دارد.)
عناصر دسته s، در جدول ژانت در سمت \censor{نمیدونم} و در جدول فعلی، در سمت \censor{نمیدونم} قرار دارند.

\Ovalbox{
	نتیجه: چینش زیرلایه‌ها در جدول ژانت از \censor{نمیدونم} به \censor{نمیدونم} و در جدول فعلی، از \censor{نمیدونم} به \censor{نمیدونم} است.
}

\begin{tabular}{r}
	ترتیب پر شدن زیرلایه‌ها\\
	(در هر دوره)
\end{tabular}
$
\left.
\begin{tabular}{rrrrrrrr}
	در جدول فعلی: & \censor{نمیدونم} & ، & \censor{نمیدونم} & ، & \censor{نمیدونم} & ، & \censor{نمیدونم} \\
	در جدول ژانت: & \censor{نمیدونم} ، \censor{نمیدونم} & ، & \censor{نمیدونم} & ، & \censor{نمیدونم} & ، & \censor{نمیدونم}
\end{tabular}
\right\}
$
تمرین: مقدار 
$\mathrm{n+l}$
 را در مورد هر زیرلایه محاسبه کنید و تعیین کنید که تا پر شدن کدام زیرلایه،  ۱۱۸ عنصر کامل می‌شود؟
 
\Ovalbox{
	در جدول ژانت برای \censor{نمیدونم} عنصر، و جدول فعلی برای \censor{نمیدونم} عنصر، جایگاه تعریف شده است.
}

\Ovalbox{
	در صورت سنتز عنصر‌های ۱۱۹ و ۱۲۰، جایگاه آن‌ها در دسته \censor{نمیدونم} و ردیف \censor{نمیدونم} جدول ژانت است.
}

ااااااااااااااااااااااااااااااااااااااشیمنتشیسمنتیسمنتیشتیمکسشی
شمسیتشمنیستکمشسی

شیسمنتتیشمنستیش

شسمنیتشسمنی

\begin{center}
	\textbf{ادامه بررسی جدول دوره‌ای (تناوبی) فعلی}
\end{center}

دارای \censor{نمیدونم} عنصر، \censor{نمیدونم} دوره (تناوب، و \censor{نمیدونم} گروه، دارای ۴ دسته \censor{نمیدونم}، \censor{نمیدونم}، \censor{نمیدونم} و \censor{نمیدونم})\\
تعداد عناصر: دسته \censor{نمیدونم}، \censor{نمیدونم} عنصر، دسته \censor{نمیدونم}، \censor{نمیدونم} عنصر، دسته \censor{نمیدونم}، \censor{نمیدونم} و دسته \censor{نمیدونم}، \censor{نمیدونم} عنصر
\lin

\begin{center}
	\textbf{روند‌های تناوبی}
\end{center}

روند‌هایی هستند که در \textbf{کمیت}‌های وابسته به اتم در جدول دیده می‌شود. یعنی: \textbf{تغییرات} مشخصی که این کمیت‌ها در یک \censor{نمیدونم} (\censor{نمیدونم}) دارند، که در تناوب‌های دیگر، $\frac{\text{عینا}}{\text{کمابیش}}$ تکرار می‌شوند. روند‌های تناوبی مطرح شده در کتاب درسی:
\begin{tabular}{rrrr}
	\circled{1} 
	شعاع اتمی &
	\circled{2}
	واکنش‌پذیری: &
	آ) خاصیت فلزی &
	ب) خاصیت نافلزی
\end{tabular}

برای یافتن نحوه تغییرات روند‌های تناوبی، کافی است اثر \textbf{هسته} را بر \textbf{لایه الکترونی بیرونی} بررسی کنیم.

الف) در هر تناوب از چپ به راست، اثر هسته بر لایه الکترونی بیرونی، \censor{نمیدونم} می‌شود.\\
دلیل: تعداد لایه الکترونی در عنصر‌های یک تناوب \censor{نمیدونم} است و قدرت هسته از چپ به راست، \censor{نمیدونم} می‌یابد.

ب) در هر گروه از بالا به پایین، اثر هسته بر لایه الکترونی بیرونی، \censor{نمیدونم} می‌شود.\\
دلیل: تعداد لایه‌های الکترونی در عنصر‌های یک گروه، از بالا به پایین، \censor{نمیدونم} می‌شود اما فاصله هسته تا لایه بیرونی \censor{نمیدونم} می‌یابد.( اثر \censor{نمیدونم} از اثر \censor{نمیدونم} مهم‌تر است. ( طبق قانون کولن
$\mathrm{F=K\frac{qq'}{r^2}}$
)

تمرین: روند تغییرات را در مورد سه روند تناوبی ذکر شده در کتاب در طرح‌های روبه‌رو مشخص نمایید:

asdadsdsa

dsa
dsa
dsa

dsa

dsadsa

dsa
dsa


\begin{center}
	\textbf{شعاع اتمی}
\end{center}

مطابق مدل «کوانتومی»، اتم را مانند \censor{نمیدونم} در نظر می‌گیرند که در الکترون‌ها پیرامون هسته و در \censor{نمیدونم} الکترونی، در حال حرکت‌اند. برای هر اتم، می‌توان «شعاعی» در نظر گرفت. هر چه شعاع اتم بزرگ‌تر باشد، اندازه آن بزرگ‌تر است.

\begin{center}
	\textbf{روند تغییرات شعاع اتمی}
\end{center}
\textbf{در گروه}:
از بالا به پایین \censor{نمیدونم} می‌شود. دلیل: افزایش تعداد \censor{نمیدونم} (جدول‌های صفحه ۱۲ و ۱۳)

$
\left.
\begin{tabular}{r}
	\begin{minipage}{\linewidth}
			در هر گروه از بالا به پایین، تعداد \censor{نمیدونم} بیشتر می‌شود $\leftarrow$ که خود به تنهایی باید شعاع را \censor{نمیدونم} دهد.
	\end{minipage}\\
	\begin{minipage}{\linewidth}
			در هر گروه از بالا به پایین، قدرت \censor{نمیدونم} بیشتر می‌شود $\leftarrow$ که خود به تنهایی باید شعاع را \censor{نمیدونم} دهد.
	\end{minipage}
\end{tabular}
\right\}
$

در نهایت، در هر گروه از بالا به پایین، شعاع \censor{نمیدونم} می‌یابد؛ نتیجه: اثر «تعداد لایه» از اثر «قدرت هسته» \censor{نمیدونم}. ( دلیل: طبق قانون کولن:
$\mathrm{f=K\frac{qq'}{r^2}}$
نیروی جاذبه هسته بر الکترون‌ها، با \censor{نمیدونم} فاصله بستگی دارد اما با بار رابطه درجه \censor{نمیدونم} دارد.)
\lin

در تناوب: از چپ به راست \censor{نمیدونم} می‌شود.
دلیل: در هر دوره، تعداد \censor{نمیدونم} ثابت است اما قدرت \censor{نمیدونم} از چپ به راست بیشتر می‌شود.

پرسش: در هر دوره، با افزایش تعداد پروتون‌ها، تعداد الکترون‌ها نیز به همان اندازه افزایش می‌یابد، پس چرا اثر هسته بر لایه بیرونی، ثابت \underline{نمی‌ماند}؟

پاسخ: «نیرو»، دارای \censor{نمیدونم} است و هر الکترونی که در این \censor{نمیدونم} (جاذبه هسته)‌ قرار گیرد، جاذبه‌ای \underline{مشخص} و \underline{ثابت} دریافت \censor{نمیدونم} که افزایش الکترون‌ها بر آن مؤثر \censor{نمیدونم}. («نیرو»، مانند «انرژی» نیست و تقسیم نمی‌شود.)

نتیجه: هر هر دوره از چپ به راست، با افزایش تعداد پروتون‌ها، هر الکترون، جاذبه \censor{نمیدونم} دریافت می‌کند.
\lin

بررسی نمودار ۱ صفحه ۱۳:

نکته \circled{1}: در تناوب \censor{نمیدونم} از چپ به راست، شعاع اتمی عنصر‌ها کاهش می‌یابد.

نکته \circled{2}: بیشترین تفاوت شعاع، بین عنصر‌های گروه‌های \censor{نمیدونم} و \censor{نمیدونم} است. ( عنصر‌های \censor{نمیدونم} و \censor{نمیدونم} )

نکته \circled{3}: تفاوت شعاع عناصر (در تناوب ۳): بین نافلز‌ها $\bigcirc$ بین فلز‌ها (یعنی روند تغییرات شعاع، در $\frac{\text{اوایل}}{\text{اواخر}}$ تناوب سوم، چشمگیر‌تر است. )

\begin{center}
\textbf{
	مقایسه تغییر شعاع و واکنش پذیری عنصر‌های گروه ۱ و ۲ \censor{نمیدونم} و ۱۷
}
\end{center}
شعاع اتمی

تعداد لایه ها

نماد لایه ظرفیت

آرایش الکترونی

نماد

شعاع اتمی

تعداد لایه ها

نماد لایه ظرفیت

آرایش الکترونی

نماد
\\
\\
\\\\
با هم بیندیشیم صفحه ۱۲:
\begin{en}
	\item
	\censor{نمیدونم}
	آسان‌تر الکترون از دست می‌دهد، چون شعاع \censor{نمیدونم} دارد.
	\item
	$\frac{\text{بله}}{\text{خیر}}$،
	چون شدت واکنش \censor{نمیدونم} با گاز کلر، بیشتر است. (\censor{نمیدونم} تر به کلر الکترون می‌دهد.)
	
	\Ovalbox{
		\begin{tabular}{r}
			در واکنش لیتیم، سدیم، پتاسیم به ترتیب نور \censor{نمیدونم}، \censor{نمیدونم} و \censor{نمیدونم} ایجاد می‌شود.\\
			( انرژی نور: \censor{نمیدونم} > \censor{نمیدونم} > \censor{نمیدونم} )
			(رنگ نور ایجاد شده، با رنگ شعله این ۳ عنصر، یکسان \censor{نمیدونم})
		\end{tabular}
	}
	\item
	$\frac{\text{بله}}{\text{خیر}}$،
	هرچه شعاع اتمی فلز بزرگ‌تر باشد، \censor{نمیدونم} تر الکترون از دست می‌دهد، چون: الکترون(های) بیرونی از هسته	\censor{نمیدونم} و نیروی هسته بر آن(ها) \censor{نمیدونم} است. (در فلز‌های گروه‌های اصلی)
\end{en}

\begin{tabular}{|r|r|}
	\hline
	\begin{tabular}{r}
			واکنش فلز قلیایی (M) با گاز کلر: (واکنش‌ها موازنه شود)\\
			\ce{\censor{نمیدونم} )\censor{نمیدونم}( -> Cl2 )g( + M )s(}
	\end{tabular} &
	\begin{tabular}{r}
		واکنش فلز قلیایی خاکی ($\mathrm{X}$) با گاز کلر:\\
		\ce{\censor{نمیدونم} )\censor{نمیدونم}( -> Cl2 )g( + X )s(}
	\end{tabular}\\
	\hline
	واکنش‌پذیری: \censor{نمیدونم} > \censor{نمیدونم} > \censor{نمیدونم} &
	واکنش‌پذیری: \censor{نمیدونم} > \censor{نمیدونم} > \censor{نمیدونم}\\
	\hline
\end{tabular}
\\

\begin{tabular}{r|r}
	\begin{tabular}{r}
			واکنش‌پذیری: فلز قلیایی
		$\bigcirc$
		فلز قلیایی خاکی (هم تناوب)\\
		دلیل: تعداد لایه \censor{نمیدونم} اما هسته عنصر‌های گروه \censor{نمیدونم} قوی‌تر 
	\end{tabular}& 
	\begin{tabular}{l}
		$\mathrm{M \rightarrow M^{+}}$\\
		$\mathrm{X \rightarrow X^{2+}}$
	\end{tabular}
\end{tabular}
\lin

تمرین: واکنش‌پذیری عنصر‌های دارای اعداد اتمی ۱۱، ۱۲ و ۱۳ را مقایسه کنید:  \censor{نمیدونم}  <  \censor{نمیدونم}  <  \censor{نمیدونم}
\lin

تذکر مهم: واکنش‌پذیری عنصر‌های واسطه، در مواردی از نظام گفته شده، پیروی نمی‌کند.

{
	\Large 
	نکته مهم‌تر:
}در گروه‌های اصلی، استحکام فلز با واکنش‌پذیری آن، رابطه \censor{نمیدونم} دارد.

نتیجه؛
\begin{tabular}{rrrr}
	واکنش‌پذیری:& فلز‌های اصلی & $\bigcirc$ & فلز‌های واسطه\\
	استحکام:& فلز‌های اصلی & $\bigcirc$ & فلز‌های واسطه
\end{tabular}

\begin{center}
\textbf{روند واکنش‌پذیری نافلز‌های گروه ۱۷ (هالوژن‌ها)}
\end{center}

$
\left\{
\begin{tabular}{r}
	در گروه ۱، از بالا به پایین، «خاصیت فلزی $\equiv$ واکنش‌پذیری» \censor{نمیدونم} می‌شود.\\
	در گروه ۱۷، از بالا به پایین، «خاصیت \censor{نمیدونم} $\equiv$ واکنش‌پذیری» \censor{نمیدونم} می‌شود.
\end{tabular}
\right.
$
\begin{tabular}{r}
	به علت \censor{نمیدونم} شدن اثر هسته\\بر لایه بیرونی از بالا به پایین
\end{tabular}

خود را بیازمایید الف صفحه ۱۳:

ب) واکنش پذیری: \censor{نمیدونم} < \censor{نمیدونم} < \censor{نمیدونم}\\
دلیل: در گروه نافلزی؛ شعاع کمتر $\leftarrow$ فاصله هسته تا لایه بیرونی \censor{نمیدونم} $\leftarrow$ گرفتن الکترون، \censor{نمیدونم}

در تولید لامپ چراغ‌های جلو خودرو از \censor{نمیدونم} استفاده می‌شود.

پ)
بالای جدول
صفحه ۱۴

ت) با افزایش شعاع، خاصیت نافلزی \censor{نمیدونم} می‌شود.

پرسش مهم: کدام هالوژن، در دمای ۴۰۰ درجه سانتی‌گراد با $\mathrm{H_2}$ واکنش می‌دهد؟
\
\begin{center}
\textbf{	نکاتی درباره هالوژن‌ها}
\end{center}

\begin{en}
	\item
	هالوژن‌ها در حالت آزاد، $\frac{\text{سمی}}{\text{غیرسمی}}$ و $\frac{\text{رنگی}}{\text{غیررنگی}}$، و در حالت ترکیب، \censor{نمیدونم} و \censor{نمیدونم} هستند.
	\item
	واژه «هالوژن» به معنی \censor{نمیدونم}. این نافلز‌ها می‌توانند با اغلب فلز‌ها (به ویژه گروه  \censor{نمیدونم}) واکنش‌دهند و \censor{نمیدونم} تولید کنند. مثال:
	\ce{\censor{نمیدونم} (\censor{نمیدونم}) -> Cl2 (g) + Na (s)}
	\item 
	حالت فیزیکی هالوژن‌ها (در دمای اتاق): ($\mathrm{F_2}$: \censor{نمیدونم}) ($\mathrm{Cl_2}$: \censor{نمیدونم}) ($\mathrm{Br_2}$: \censor{نمیدونم}) ($\mathrm{I_2}$: \censor{نمیدونم})
	\item
	نقطه جوش هالوژن‌ها: \censor{نمیدونم} < \censor{نمیدونم} < \censor{نمیدونم} < \censor{نمیدونم}\\
	دلیل: در مولکول‌های $\frac{\text{قطبی}}{\text{ناقطبی}}$، با افزایش جرم و حجم مولکول، نیروی بین مولکولی \censor{نمیدونم} می‌شود.
	\item
	برای تشکیل ترکیب یونی، هالوژن‌ها با \censor{نمیدونم} یک الکترون به یون \censor{نمیدونم} تبدیل می‌شوند. (
	$\mathrm{Cl \rightarrow Cl^{\censor{نمیدونم}}}$
	)
	\item
	$\mathrm{F, Cl, Br, I}$ 
	$\frac{\text{نافلز}}{\text{فلز}}$
	هستند.
	\item
	آنیون‌های تشکیل شده توسط هالوژن‌ها، یون \censor{نمیدونم} نامیده می‌شوند. مثال:(
	$Cl^{-} \rightarrow \censor{نمیدونم}$
	)
	\item
	هالوژن‌ها در حالت آزاد (مولکول \censor{نمیدونم} اتمی) $\frac{\text{بی‌رنگ}}{\text{رنگی}}$ هستند و در حالت آنیون یا ترکیب \censor{نمیدونم} اند.
	\item
	رنگ هالوژن‌ها: ($\mathrm{F_2 (g): \censor{نمیدونم}}$) ($\mathrm{Cl_2 (g): \censor{نمیدونم}}$) ($\mathrm{Br_2 (l): \censor{نمیدونم}}$) ($\mathrm{I_2 (s): \censor{نمیدونم}}$)
	
	$^{\text{غیررسمی}}$
	(تذکر: $\mathrm{I_2}$ در حالت بخار و محلول رنگ \censor{نمیدونم} مایل به \censor{نمیدونم} دارد.)
\end{en}

\Ovalbox{
	\textbf{رابطه‌ی نمک‌ها و ترکیب‌های یونی}
}
همه \censor{نمیدونم} جزء \censor{نمیدونم} هستند اما برخی \censor{نمیدونم} \censor{نمیدونم}، \censor{نمیدونم} محسوب نمی‌شوند مانند\censor{نمیدونم} \censor{نمیدونم}.
( مانند \censor{نمیدونم} که \censor{نمیدونم} است و نمک نیست) (برسی تمرین دوره‌ای صفحه ۴۸ )

\begin{center}
	\textbf{رفتار‌های ویژه فلز‌ها}
\end{center}

رفتار‌های «کلی» فلز‌ها مشابه است اما تفاوت‌های قابل توجهی نیز دارند به طوری که: هر فلز، رفتار‌های «\censor{نمیدونم}» خود را دارد. نمونه: (شکل‌های حاشیه صفحه ۱۴)

$
\left.
\begin{tabular}{r}
	\begin{minipage}{\linewidth}
		سدیم: $\frac{\text{نرم}}{\text{سخت}}$ است. با چاقو بریده \censor{نمیدونم} و جلای نقره‌ای آن در مجاورت اکسیژن \censor{نمیدونم} به $\frac{\text{سرعت}}{\text{کندی}}$ از بین می‌رود و \censor{نمیدونم} می‌شود.
	\end{minipage}\vspace{0.5em}\\
	\begin{minipage}{\linewidth}
		آهن: محکم \censor{نمیدونم} (برای ساخت در و پنجره) و در هوای $\frac{\text{خشک}}{\text{مرطوب}}$ با \censor{نمیدونم} هوا به \censor{نمیدونم} واکنش می‌دهد و به \censor{نمیدونم} آهن تبدیل می‌شود.
	\end{minipage}\vspace{0.5em}\\
	\begin{minipage}{\linewidth}
		طلا: در گذر زمان، جلای فلزی خود را \censor{نمیدونم} و خوش رنگ و \censor{نمیدونم} می‌ماند. برخی گنبد‌ها و گلدسته‌ها با \censor{نمیدونم} نازکی از طلا \censor{نمیدونم} می‌شود.
	\end{minipage}
\end{tabular}
\right\}
$

\begin{center}
\textbf{	دنیایی رنگی با عنصر‌های دسته d}
\end{center}

رفتاری شبیه فلز‌های دسته \censor{نمیدونم} و \censor{نمیدونم} دارند: (مانند همه فلز‌ها رسانای \censor{نمیدونم} و \censor{نمیدونم} \censor{نمیدونم} هستند، \censor{نمیدونم}  خوارند و قابلیت تبدیل به \censor{نمیدونم} و \censor{نمیدونم} را دارند) اما هر یک، رفتار‌های ویژه‌ای نیز دارند. فلز‌های دسته d به فلز‌های $\frac{\text{واسته}}{\text{اصلی}}$ معروف‌اند در حالی که فلز‌های دسته s و p به فلز‌های \censor{نمیدونم} شهرت دارند.
اغلب فلز‌های واسطه در طبیعت به شکل ترکیب‌های $\frac{\text{یونی}}{\text{مولکولی}}$ (مانند \censor{نمیدونم}، \censor{نمیدونم} و غیره) یافت می‌شوند.
برای نمونه، آهن، دو اکسید طبیعی 
$\mathrm{FeO}$
(\censor{نمیدونم})
و 
$\mathrm{Fe_2O_3}$
(\censor{نمیدونم})
دارد.

\Ovalbox{
	اغلب عناصر واسطه، دو ویژگی دارند: ترکیبات \censor{نمیدونم} و ظرفیت‌های \censor{نمیدونم}.
}

رنگ سنگ‌های قیمتی فیروزه (\censor{نمیدونم})، یاقوت (\censor{نمیدونم}) و زمرد (\censor{نمیدونم}) به علت وجود ترکیبات عناصر واسطه در آن‌ها است.
\lin

\begin{center}
	«آرایش الکترونی فلز‌های واسطه»
\end{center}
زیر لایه \censor{نمیدونم} در آن‌ها در حال پر شدن است:

\begin{tabular}{ccc}
	$\mathrm{_{26}Fe:[\qquad]}\qquad\qquad\qquad\qquad$&
	$\mathrm{_{26}Fe^{2+}:}\qquad\qquad\qquad\qquad$&
	$\mathrm{_{26}Fe^{3+}:}\qquad\qquad\qquad\qquad$
\end{tabular}

نکته مهم: زیرلایه ۴s نسبت به ۳d؛ $\frac{\text{زودتر}}{\text{دیرتر}}$ پر می‌شود: چون سطح انرژی \censor{نمیدونم} دارد، و \censor{نمیدونم} خالی می‌شود:
چون \censor{نمیدونم}  \censor{نمیدونم}  \censor{نمیدونم}
\lin

تست: آرایش الکترونی
$\mathrm{[Ar]3d^5}$
متعلق به چند مورد از موارد زیر می‌تواند باشد؟
\;
$\bullet$اتم\qquad
$\bullet$کاتیون\qquad
$\bullet$آنیون

\begin{multicols}{5}
	\setlength{\columnseprule}{0pt}
	\begin{enumerate}
		\item فقط اتم
		\item فقط آنیون
		\item اتم و آنیون
		\item فقط کاتیون
		\item فقط یون
	\end{enumerate}
\end{multicols}

خود را بیازمایید ۲ صفحه ۱۶ (به همراه تمرین آرایش الکترونی چند عنصر واسطه دیگر)

آرایش الکترونی
نماد
آرایش الکترونی
نماد
آرایش الکترونی
نماد











































-





\begin{center}
\textbf{	«نکاتی درباره عناصر واسطه تناوب ۴»}
\end{center}
\begin{en}
	\item
	همه، ترکیبات \censor{نمیدونم} دارند، به جز \censor{نمیدونم} و \censor{نمیدونم}
	\item
	همه، ظرفیت‌های \censor{نمیدونم} دارند، به جز \censor{نمیدونم} (ظرفیت = \censor{نمیدونم}) و \censor{نمیدونم}  (ظرفیت =  \censor{نمیدونم} )
	\item
	مجموع ارقام عدد اتمی = شماره \censor{نمیدونم} (به جز  \censor{نمیدونم} ) مثال: (شماره  \censor{نمیدونم}  =  \censor{نمیدونم}  +  \censor{نمیدونم}
	$\mathrm{_{26}Fe} \rightarrow$
	)
	\item
	رقم «\underline{دهگان}» و «یکان» در عدد اتمی، به ترتیب برابر با شمار الکترون‌های \censor{نمیدونم} و \censor{نمیدونم} است. (به جز \censor{نمیدونم}، \censor{نمیدونم} و \censor{نمیدونم}) مثال:
	
		{\raggedright$\mathrm{_{26}Fe: [Ar]4s^{\censor{نمیدونم}} 3d^{\censor{نمیدونم}}}$\par}
	\item
	ظرفیت اصلی (کمترین ظرفیت) و بیشترین ظرفیت عناصر واسطه تناوب ۴:
	( ممکن است برخی از این عناصر، ظرفیت‌های دیگری بین این دو ظرفیت داشته باشند )
	
	\begin{tabular*}{\textwidth}{@{\extracolsep{\fill}}c|ccccc|ccc|cc}
				نماد عنصر & 
			$\mathrm{Sc}$ & $\mathrm{Ti}$ & $\mathrm{V}$ & $\mathrm{Cr}$ & $\mathrm{Mn}$ & $\mathrm{Fe}$ & $\mathrm{Co}$ & $\mathrm{Ni}$ & $\mathrm{Cu}$ & $\mathrm{Zn}$\\\hline
				ظرفیت اصلی &
			$\bigcirc$ & {} & {} & {} & {} & {} & {} & {} & $\bigcirc$ & {} \\ \hline
				بیشترین ظرفیت &
			{} & {} & {} & {} & {} & {} & {} & {} & {} & {}
	\end{tabular*}

	\item
	فقط  \censor{نمیدونم}  می‌تواند با کمترین ظرفیت (ظرفیت اصلی) و «\censor{نمیدونم}  ظرفیت» خود، به آرایش الکترونی گاز نجیب برسد.
	\item
	در این عناصر، ظرفیت اصلی (کمترین ظرفیت) برابر با  \censor{نمیدونم}  است. (به جز  \censor{نمیدونم}  و  \censor{نمیدونم})
\end{en}

خود را بیازمایید صفحه ۱۷:

الف) اسکاندیم (\censor{نمیدونم})، نخستین فلز  \censor{نمیدونم}  جدول دوره‌ای است. در وسایل خانه، مانند  \censor{نمیدونم}   \censor{نمیدونم}  و برخی  \censor{نمیدونم} وجود دارد.

\begin{center}
\textbf{	طلا (\censor{نمیدونم})}
\end{center}
طلا افزون بر ویژگی‌های مشترک با سایر فلز‌ها، ویژگی‌های منحصر به فردی نیز دارد. بسیار \censor{نمیدونم}  و  \censor{نمیدونم} \censor{نمیدونم}  است. (طلا به اندازه‌ای  \censor{نمیدونم}  و  \censor{نمیدونم}  است که می‌توان چند گرم از آن را با چکش‌کاری، به  \censor{نمیدونم}  با مساحت چند متر مربع تبدیل کرد.)
به راحتی به  \censor{نمیدونم}  و  \censor{نمیدونم}  بسیار نازک (\censor{نمیدونم} طلا) تبدیل می‌شود. رسانایی الکتریکی آن، \censor{نمیدونم}  است و در شرایط گوناگون دمایی، این رسانایی  \censor{نمیدونم}   \censor{نمیدونم}. با  \censor{نمیدونم}  های موجود در \textbf{هواکره} و  \censor{نمیدونم}، واکنش  \censor{نمیدونم}. (ساخت وسایل الکتریکی شکل صفحه ۱۷)
پرتو‌های خورشیدی، از روی ورقه طلا، \censor{نمیدونم} زیادی دارند.
طلا در طبیعت به صورت  \censor{نمیدونم}  (\censor{نمیدونم}) یافت می‌شود و مقدارش در معادن، بسیار  \censor{نمیدونم}  است. برای استخراج آن، باید حجم  \censor{نمیدونم}  از  \censor{نمیدونم} معدن استفاده شود. «استخراج طلا»، آثار  \censor{نمیدونم} \censor{نمیدونم} بر محیط زیست برجای می‌گذارد.
دانشمندان، به دنبال راه‌های جدید برای  \censor{نمیدونم}  فلز‌ها هستند که ضمن بهره‌برداری از  \censor{نمیدونم}، منجر به کاهش  \censor{نمیدونم}   \censor{نمیدونم}  محیط زیستی شود و با  \censor{نمیدونم}  هماهنگ باشد.
\begin{center}
\textbf{عنصر‌ها به چه شکلی در طبیعت یافت می‌شوند؟}
\end{center}
شکل ۹ صفحه ۱۸:  \censor{نمیدونم} \censor{نمیدونم}، \censor{نمیدونم} \censor{نمیدونم}، \censor{نمیدونم} (II) \censor{نمیدونم}  و  \censor{نمیدونم}، نمونه‌هایی از «کانی‌های» موجود در طبیعت هستند.

اغلب عناصر در طبیعت، به شکل $\frac{\text{آزاد}}{\text{ترکیب }}$ یافت می‌شوند، هرچند، برخی نافلز‌ها مانند  \censor{نمیدونم}،  \censor{نمیدونم}  و  \censor{نمیدونم} و برخی فلز‌ها مانند\censor{نمیدونم}،  \censor{نمیدونم}  و  \censor{نمیدونم} به شکل آزاد در طبیعت وجود دارند. (البته نافلز‌های مذکور، و نیز فلز  \censor{نمیدونم}  به شکل  \censor{نمیدونم}  نیز در طبیعت یافت می‌شوند.)

\Ovalbox{
در میان فلز‌ها، تنها «طلا» به شکل  \censor{نمیدونم}  ها یا  \censor{نمیدونم}  های «زرد»، لابه‌لای خاک یافت می‌شود. (حاشیه صفحه ۱۸)
}
\textbf{\Large «حالت آزاد»}
در یک عنصر یعنی، اتم‌های آن با اتمی \censor{نمیدونم}
\begin{multicols}{2}
	\setlength{\columnseprule}{0pt}
	\begin{enumerate}
		\item
		از عنصر دیگر پیوند نداده باشد.
		\item 
		دیگر پیوند نداده باشد.
	\end{enumerate}
\end{multicols}
پرسش: چند مورد، حالت آزاد هیدروژن است؟
\begin{multicols}{3}
	\setlength{\columnseprule}{0pt}
	\begin{enumerate}
		\item
		$\mathrm{H}$
		\item
		$\mathrm{H-Cl}$
		\item
		$\mathrm{H-H}$
	\end{enumerate}
\end{multicols}
\begin{center}
	روش شناسایی کاتیون‌های آهن ( واکنش‌ها، موازنه شوند. ) ( کاوش کنید ۱ صفحه ۱۹ )
\end{center}
ج) آزمایش ۱ صفحه ۱۹ ( شناسایی $\mathrm{Fe^{2+}}$ ) به کمک یون  \censor{نمیدونم}
	$\ce{\censor{a} (aq)+ \censor{a} (aq) -> \censor{a} (s) + \censor{a} (aq)}$\\
ث) رسوب  \censor{نمیدونم}  رنگ\\
چ) یون  \censor{نمیدونم} ، شناساگر یون \censor{نمیدونم} است.\\
پ) آزمایش ۲ صفحه ۱۹ ( شناسایی $\mathrm{Fe^{+3}}$ ) به کمک یون  \censor{نمیدونم}:
$\ce{\censor{نمیدونم} (aq)+ \censor{نمیدونم} (aq) -> \censor{نمیدونم} (s) + \censor{نمیدونم} (aq)}$\\
ب) رسوب \censor{نمیدونم} (\censor{نمیدونم})\\
ت) یون \censor{نمیدونم} ، شناساگر یون \censor{نمیدونم} نیز هست.\\
تذکر: روش شناسایی یک ذره، باید  \censor{نمیدونم}  \textbf{ویژه} و مشخص، ایجاد کند، به شکلی که؛\\
$\frac{\text{یون مورد نظر}}{\text{یون شناساگر}}$،
فقط با 
$\frac{\text{یون مورد نظر}}{\text{یون شناساگر}}$،
آن  \censor{نمیدونم}  را ایجاد کند.

نکته ۱: دو ترکیب یونی، در محلول  \censor{نمیدونم} (  \censor{نمیدونم}  )، فقط به شرطی واکنش می‌دهند که  \censor{نمیدونم} یا  \censor{نمیدونم} یا \censor{نمیدونم} تولید شود.

نکته ۲: در واکنش جابه‌جایی دوگانه، ظرفیت هر ذره، در دو طرف واکنش یکسان  \censor{نمیدونم}.

آزمایش ۳ صفحه ۱۹: ( واکنش‌ها موازنه شوند. ) ابتدا، میخ زنگ‌زده را در محلول $\mathrm{HCl}$ وارد می‌کنیم:
\begin{flushleft}
	$\ce{Fe_2O_3 (s) + HCl (aq) -> \censor{نمیدونم} (aq) + \censor{نمیدونم} (\censor{نمیدونم})}$ (ب
\end{flushleft}
سپس، به این سامانه، محلول آبی «سود» می‌افزاییم:
\begin{flushleft}
	$\ce{\censor{نمیدونم} (aq) + \quad NaOH (aq) -> \censor{نمیدونم} (s) + \censor{نمیدونم} (aq)}$ (پ
\end{flushleft}
ت) رسوب  \censor{نمیدونم}\\
ث) این دو واکنش نشانگر وجود یون  \censor{نمیدونم}  در زنگ آهن (\censor{نمیدونم}) است.

\begin{tabular}{rr}
	یادداشت (در حد کتاب درسی شیمی ۳): &
	\begin{tabular}{r}
		اغلب عناصر فلزی می‌توانند با 
		$\mathrm{HCl(aq)}$
		۱ مولار، واکنش دهند\\
		به جز فلز‌‌های $\mathrm{APAC}$ (\censor{نمیدونم}، \censor{نمیدونم}، \censor{نمیدونم}، \censor{نمیدونم})
	\end{tabular}
\end{tabular}

\begin{flushleft}
$
\left\{
\begin{tabular}{l}
	$\circled{I}\; \ce{Fe (s) +CuSO4 (aq) -> }$\\
	$\circled{II}\; \ce{Cu (s) +FeSO4 (aq) -> }$
\end{tabular}
\right.
$
کاوش کنید ۲ صفحه ۲۰
\end{flushleft}
$
\left.
\begin{tabular}{r}
	\text{
		در واکنش (I)، فلز سمت چپ (  \censor{a}  ) واکنش را انجام  \censor{a}  است. (  \censor{a}  می‌تواند به  \censor{a}  الکترون دهد. )
	}\\
	\text{
		در واکنش (II)، ‌فلز سمت چپ (  \censor{a}  ) واکنش را انجام  \censor{a}  است. (  \censor{a}  نمی‌تواند به  \censor{a}  الکترون دهد. )
	}
\end{tabular}
\right\}
$

نتیجه:  \censor{نمیدونم}  از  \censor{نمیدونم}  واکنش‌پذیر‌تر است.

نکته ۳: در واکنش جابه‌جایی یگانه، حتماً در واکنش، بار  \censor{نمیدونم}  ذره تغییر می‌کند.

نکته ۴: اگر واکنش «فلزی» با محلول آبی کاتیون «فلز» دیگر، خود به خود انجام‌پذیر باشد، واکنش عکس ( برگشت )، حتماً  \censor{نمیدونم}  خود به خودی است.

\begin{multicols}{3}
		\setlength{\columnseprule}{0pt}
			خوب است بدانیم: \\
		$\ce{Fe + HCl -> \censor{نمیدونم} + H2}$ \\
		$\ce{Fe + Cl2 -> \censor{نمیدونم}}$
\end{multicols}

\begin{center}
\textbf{واکنش‌پذیری}
\end{center}

واکنش‌پذیری هر فلز ( و به طور کلی هر عنصر ) تمایل آن را برای انجام  \censor{نمیدونم}   \censor{نمیدونم}  نشان می‌دهد.
اصطلاح «مس فلزی» به عنصر مس در حالت 
$\frac{\text{اتم}}{\text{کاتیون-ترکیب}}$
 اشاره دارد. عنصر می در حالت  \censor{نمیدونم}  یا  \censor{نمیدونم}  خاصیت فلزی.
هرچه عنصری واکنش‌پذیرتر باشد، تمایل آن را برای انجام واکنش ( تبدیل  \censor{نمیدونم}  به  \censor{نمیدونم}  ) بیشتر است. برای مقایسه، تعدادی فلز، از لحاظ واکنش‌پذیری در سه دسته قرار گرفته‌اند:

با هم بیندیشیم صفحه ۲۰: ( با توجه به جدول پایین صفحه ۲۰ به پرسش‌ها پاسخ دهید )

واکنش‌پذیری: ( زیاد:  \censor{نمیدونم} ، \censor{نمیدونم}) ( کم:  \censor{نمیدونم}، \censor{نمیدونم}) ( ناچیز:  \censor{نمیدونم}، \censor{نمیدونم} و \censor{نمیدونم})

الف) در \textbf{شرایط یکسان}، فلز‌ها با واکنش‌پذیری  \censor{نمیدونم}، تمایل  \censor{نمیدونم}  به تشکیل  \censor{نمیدونم}  نشان می‌دهند.

ب) در \textbf{شرایط یکسان}، سرعت واکنش‌دادن در هوای مرطوب:  \censor{نمیدونم}  <  \censor{نمیدونم}  <  \censor{نمیدونم}

پ) تأمین شرایط نگهداری فلز‌ها با واکنش‌پذیری  \censor{نمیدونم}، دشوارتر است. 
(چون با کمترین مقدار مواد، از جمله  \censor{نمیدونم}  هوا، واکنش می‌دهند و فعالیت شیمیایی آن‌ها  \censor{نمیدونم}  است.)

ت) به طور کلی، در هر واکنش شیمیایی که به طور طبیعی ( خود به خود ) انجام می‌شود؛
\begin{multicols}{2}
	\setlength{\columnseprule}{0pt}
	واکنش‌پذیری: واکنش‌دهنده‌ها $\bigcirc$ فرآورده‌ها\\
	پایداری: واکنش‌دهنده‌ها $\bigcirc$ فرآورده‌ها
\end{multicols}
* این مقایسه، در مورد واکنش پذیری عناصر در دو طرف واکنش است.

\begin{tabular}{r l}
	\setlength{\columnseprule}{0pt}
	\begin{tabular}{r}
		با هم بیندیشیم صفحه ۲۱ ت\\
		واکنش‌پذیری: \quad
		$\mathrm{Na\bigcirc Fe}$
	\end{tabular}&\qquad
	\begin{latin}
		\begin{tabular}{l l l l l l l}
			$\mathrm{FeO (s)}$ & + & $\mathrm{2 Na (s)}$ & $\ce{->[\Delta]}$ & $\mathrm{Na_2O (s)}$ & + & $\mathrm{Fe (s)}$\\
			$\mathrm{Na_2O (s)}$ & + & $\mathrm{Fe (s)}$ & \ce{->[\Delta]} & & & 
		\end{tabular}
	\end{latin}
\end{tabular}

\begin{tabular}{r l}
	\setlength{\columnseprule}{0pt}
	\begin{tabular}{r}
		با هم بیندیشیم صفحه ۲۱ ث\\
		واکنش‌پذیری: \quad
		$\mathrm{Na\bigcirc C}$
	\end{tabular}&\qquad
	\begin{latin}
		\begin{tabular}{l l l l l l l}
			$\mathrm{Na_2O}$ & + & $\mathrm{C (s)}$ & $\ce{->[\Delta]}$ & & & \\
			$\mathrm{CO_2 (g)}$ & + & $\mathrm{Na (s)}$ & \ce{->[\Delta]} & & & 
		\end{tabular}
	\end{latin}
\end{tabular}

به طور کلی:

\begin{tabular}{r l}
	\begin{tabular}{r l c r}
		واکنش‌پذیری ($\mathrm{M}$ و $\mathrm{M'}$ فلز): &
		$\mathrm{M'}$ & $\bigcirc$ & $\mathrm{M}$\\
		واکنش‌پذیری ($\mathrm{X}$ و $\mathrm{Z}$ نافلز): &
		$\mathrm{Z}$ & $\bigcirc$ & $\mathrm{X}$\\
		واکنش‌پذیری ($\mathrm{X}$، $\mathrm{Y}$ و $\mathrm{Z}$ نافلز): &
		$\mathrm{Z}$ & $\bigcirc$ & $\mathrm{Y}$
	\end{tabular}&\qquad
	\begin{latin}
		\begin{tabular}{l l l l l l l l}
			$\mathrm{MX}$ & + & $\mathrm{M'}$ & $\rightarrow$ & $\mathrm{M'X}$ & + & $\mathrm{M}$ & $\Rightarrow$\\
			$\mathrm{MX}$ & + & $\mathrm{Z}$ & $\rightarrow$ & $\mathrm{MX}$ & + & $\mathrm{X}$ & $\Rightarrow$\\
			$\mathrm{XY}$ & + & $\mathrm{Z}$ & $\rightarrow$ & $\mathrm{XZ}$ & + & $\mathrm{Y}$ & $\Rightarrow$\\
		\end{tabular}
	\end{latin}
\end{tabular}

\begin{tabular*}{\textwidth}{@{\extracolsep{\fill}}r l}
	\begin{tabular}{r}
		واکنش‌پذیری:
		$\mathrm{Zn \bigcirc Cu}$
	\end{tabular}&\qquad
	$\left\{
	\begin{latin}
		\begin{tabular}{l l l l l l l }
			$\mathrm{Zn (s)}$ & + & $\mathrm{CuSO_4 (aq)}$ & $\rightarrow$ & $\mathrm{\qquad (aq)}$ & + & 	$\mathrm{\qquad (s)}$ \\
			$\mathrm{Cu (s)}$ & + & $\mathrm{ZnSO_4 (aq)}$ & $\rightarrow$ & $\mathrm{\quad}$ & & 		$\mathrm{\quad}$ \\
		\end{tabular}
	\end{latin}
	\right.
	$
\end{tabular*}

\begin{tabular*}{\textwidth}{@{\extracolsep{\fill}}r l}
	\begin{tabular}{r}
		واکنش‌پذیری:
		$\mathrm{Al \bigcirc Cu}$
	\end{tabular}&\qquad
	$\left\{
	\begin{latin}
		\begin{tabular}{l l l l l l l }
			$\mathrm{Al (s)}$ & + & $\mathrm{CuSO_4 (aq)}$ & $\rightarrow$ & $\mathrm{\quad (aq)}$ & + & 	$\mathrm{\quad (s)}$ \\
			$\mathrm{Cu (s)}$ & + & $\mathrm{Al_2(SO4)_3 (aq)}$ & $\rightarrow$ & $\mathrm{\quad}$ & & 		$\mathrm{\quad}$ \\
		\end{tabular}
	\end{latin}
	\right.
	$
\end{tabular*}

\begin{tabular*}{\textwidth}{@{\extracolsep{\fill}}r l}
	\begin{tabular}{r}
		واکنش‌پذیری:
		$\mathrm{Fe \bigcirc Cu}$
	\end{tabular}&\qquad
	$\left\{
	\begin{latin}
		\begin{tabular}{l l l l l l l }
			$\mathrm{FeO (s)}$ & + & $\mathrm{Cu (s)}$ & $\rightarrow$ & $\mathrm{\qquad}$ &  & 	$\mathrm{\qquad}$ \\
			$\mathrm{CuO (s)}$ & + & $\mathrm{Fe (s)}$ & $\rightarrow$ & $\mathrm{\qquad}$ & & 		$\mathrm{\qquad}$ \\
		\end{tabular}
	\end{latin}
	\right.
	$
\end{tabular*}
آیا این واکنش انجام‌پذیر است؟  \censor{نمیدونم}  چون  \censor{نمیدونم}  از  \censor{نمیدونم}  واکنش‌پذیر‌تر است.
روش استخراج  \censor{نمیدونم}  فلزی از  \censor{نمیدونم}  ( \censor{نمیدونم} )  \censor{نمیدونم}  در معدن مس سرچشمه: ( تمرین دوره‌ای ۷ )
واکنش‌پذیری:



روش استخراج  \censor{نمیدونم}  فلزی از  \censor{نمیدونم}  ( \censor{نمیدونم} )  \censor{نمیدونم}  (  \censor{نمیدونم}  ) در فولاد مبارکه:
(صفحه ۲۱)
واکنش پذیری \censor{نمیدونم} : ( با هم بیندیشیم صفحه۲۱)
روش دیگری برای استخراج آهن:
آهن،  \censor{نمیدونم}  ترین عنصر کره زمین است و  \censor{نمیدونم}  مصرف سالانه را بین فلز‌ها در جهان دارد.
برای جوش دادن خطوط آهن، از واکنشی موسوم به «  \censor{نمیدونم}  » استفاده می‌شود:
) خود را بیازمایید صفحه ۲۴ (
فلزها در طبیعت، اغلب به شکل  \censor{نمیدونم}   \censor{نمیدونم}  یافت می‌شوند؛ هرچه فلزی واکنش‌پذیرتر باشد، استخراج آن  \censor{نمیدونم}  است.
هر چه تمایل فلز برای الکترون دهی بیشتر باشد تمایل کاتیون آن برای الکترون گیری کمتر است.
تمرین دوره‌ای صفحه ۴۸:
نتیجه ۱: Ne نماینده گروه  \censor{نمیدونم}  کمترین  \censor{نمیدونم}   \censor{نمیدونم}  را بین عنصر‌های دوره  \censor{نمیدونم}  دارد.
نتیجه ۲: بین عنصر گروه ۱ تا ۱۷، عنصر  \censor{نمیدونم}  ( نماینده گروه ۱۴) کمترین  \censor{نمیدونم}   \censor{نمیدونم}  را دارد.
مسئله ( خود را بیازمایید صفحه ۲۲ )
از واکنش ۴۰ گرم آهن (III) اکسید با کربن، انتظار می‌رود چند گرم آهن به دست آید ؟ C=12, O=16, H=1, Fe=56, Al=27




دنیای واقعی واکنش‌ها
۱- درصد خلوص					۲- بازده
گاهی واکنش‌های شیمیایی، مطابق آنچه انتظار می‌رود پیش نمی‌روند. ممکن است واکنش‌دهنده‌ها ناخالص باشند ( درصد خلوص)، واکنش به طور کامل انجام نشود ( به دلیل شرایط مختلف) یا همزمان، واکنش‌های ناخواسته دیگری انجام شود.( بازده )
بازده درصدی
وقتی واکنش به طور کامل در مسیر اصلی انجام نوشد مقدار فرآورده تشکیل شده در آزمایش ( مقدار  \censor{نمیدونم}  ) از آنچه در تئوری و روی کاغذ به دست آمده ( مقدار  \censor{نمیدونم}  )  \censor{نمیدونم} تر خواهد بود. ( مقدار  \censor{نمیدونم}  < مقدار  \censor{نمیدونم}  )
پیوند با ریاضی:
۲- الف ( صفحه ۲۳ ) ( ۱۰۰ \censor{نمیدونم} بازده )
۲- ب :

مسئله ۱: از تخمیر ۱.۵ تن گلوکز موجود در پسماندهای گیاهی، چند تن سوخت سبز (  \censor{نمیدونم}  ) تولید می‌شود؟(۸۰٪ = Ra)


مسئله ۲ ( تمرین دوره‌ای ۶ ): آهن (III) اکسید به عنوان  \censor{نمیدونم}  در نقاشی به کار می‌رود. ۱۰ کیلوگرم از این ماده، طبق واکنش زیر در واکنش با کار کربن مونو‌اکسید،۵۲۰۰ گرم‌ آهن تولید کرده است. بازده درصدی واکنش را به دست آورید: (خود را بیازمایید ۲ صفحه ۲۵ )


درصد خلوص
پیوند با ریاضی( ۱- الف صفحه ۲۳):
یعنی در هر  \censor{نمیدونم}  گرم از این ماده معدنی ( کانه )،  \censor{نمیدونم}  گرم  \censor{نمیدونم}  و  \censor{نمیدونم}  گرم مواد دیگر هست.
۱- ب				 \censor{نمیدونم} درصد خلوص	یا					 \censor{نمیدونم} درصد خلوص
مسئله ۳ – ۱۰ گرم آهن با خلوص ۹۵٪ را در مقدار کافی محلول هیدروکلریک اسید می‌اندازیم. حجم(g) \censor{نمیدونم} در شرایط STP، چند لیتر است؟



مهم
خود را بیازمایید ۱ صفحه ۲۴:
الف)  \censor{نمیدونم}  فعال‌تر است، چون در واکنش خود بخودی سمت  \censor{نمیدونم}  قرار دارد ( و  \censor{نمیدونم}  را از ترکیبش خارج می‌کند. )
بررسی تمرین دوره‌ای ۱، ۲ ، ۳ و ۷:


«گیاه پالایی»
یکی از روش‌های بیرون کشیدن فلز از لابه‌لای خاک، استفاده از گیاهان است. ابتدا گیاه را می‌کارند، گیاه،  \censor{نمیدونم}  را جذب می‌کند. سپس گیاه را برداشت می‌کندد،  \censor{نمیدونم}  و از  \censor{نمیدونم}  آن،  \censor{نمیدونم}  را جداسازی می‌کنند.
خود را بیازمایید ۳ صفحه ۲۵ الف:

ب:											در‌صد نیکل در خاکستر
پ:
مقرون به صرفه ( گیاه‌پالایی )
درصد فلز در سنگ معدن
درصد فلز در گیاه
فلز



Au



Cu



Ni



Zn
با مقایسه درصد «نیکل» و «روی» در سنگ معدن آن‌ها، و با توجه به حجم گیاه و آب مصرفی، و نیز سطح زیادی از زمین به که زیر کشت می‌رود، روش گیاه پالایی برای این دو فلز مقرون به صرفه  \censor{نمیدونم}  .
پیوند با صنعت: گنجینه‌های اعماق دریا
اعماق دریا، در برخی مناطق محتوی  \censor{نمیدونم}  چندین فلز واسطه (  \censor{نمیدونم}  سولفیدی ) ( شکل ۱۱ پ صفحه ۲۶ ) و در برخی مناطق دیگر، به صورت  \censor{نمیدونم}  ها و  \censor{نمیدونم}  هایی غنی از فلز‌هایی مانند  \censor{نمیدونم}  ،  \censor{نمیدونم}  ،  \censor{نمیدونم}  ،  \censor{نمیدونم}  و  \censor{نمیدونم}  است. ( شکل ۱۱ ب صفحه ۲۶ )
غلظت گونه‌های فلزی «کف اقیانوس»، نسبت به «ذخایر زیرزمینی»،  \censor{نمیدونم}  است.

جریان فلز بین «محیط زیست» و «جامعه»
استخراج فلز از سنگ معدن، در نهایت به تولید  \censor{نمیدونم}  و  \censor{نمیدونم}  گوناگون می‌انجامد. بر اساس توسعه پایدار، در تولید یک «  \censor{نمیدونم}  » یا عرضه «  \censor{نمیدونم}  »، باید همه هزینه‌ها و ملاحظه‌های  \censor{نمیدونم}  ،  \censor{نمیدونم}  و  \censor{نمیدونم}   \censor{نمیدونم}  را در نظر گرفت.
اگر مجموع هزینه‌های بهره‌برداری از یک معدن، با در نظر گرفتن این ملاحظه‌ها،  \censor{نمیدونم}  مقدار ممکن باشد، در مسیر پیشرفت پایدار حرکت می‌کنیم، رفتار‌های ما آسیب کمتری به جامعه وارد می‌کند و  \censor{نمیدونم}   \censor{نمیدونم}  زیست محیطی ما را کاهش می‌دهد.
«فرآیند استخراج فلز از طبیعت و بازگشت آن به‌ طبیعت»






با هم بیندیشیم صفحه ۲۷:
الف) یکسان  \censor{نمیدونم}  ( آهنگ مصرف		آهنگ بازگست به طبیعت )
ب) فلز‌ها، منابعی تجدید  \censor{نمیدونم}  . با تمام شدن معادن، دسترسی به آن‌ها  \censor{نمیدونم}  ، و محدود به  \censor{نمیدونم}  است.
پ) بازیافت فلز‌ها از جمله آهن؛ ردپای	 \censor{نمیدونم} را کاهش می‌دهد. ( د / ن )
سبب کاهش سرعت گر‌مای جهانی می‌شود. ( د / ن )
گونه‌های زیستی بیشتری را از بین می‌برد. ( د / ن )
به توسعه پایدار کشور کمک می‌کند. ( د / ن )
پسماند سرانه فولاد  \censor{نمیدونم}  کیلوگرم است. با انرژی ذخیره شده از بازگردانی ۷ قوطی فولادی، می‌توان یک لامپ ۶۰ واتی را حدود ۲۵ ساعت روشن نگه داشت. در استخراج ۱ کیلوگرم آهن، تقریباً  \censor{نمیدونم}  کیلوگرم سنگ معدن آهن، و  \censor{نمیدونم}  کیلوگرم از منابع معدنی دیگر مصرف می‌شود. در استخراج فلز، درصد )کمی / زیادی( از سنگ معدن به فلز تبدیل می‌شود.




ارزیابی چرخه عمر
چرخه عمر: میزان تأثیر یک فرآورده بر روی محیط زیست در طول مدت عمر آن.
ارزیابی چرخه عمر: تاثیر‌های هر فرآورده را در ۴ مرحله، بررسی می‌کند:
۱:  \censor{نمیدونم}  و  \censor{نمیدونم}  مواد خام برای تولید فراورده
۲:  \censor{نمیدونم}  \censor{نمیدونم} ۳:  \censor{نمیدونم} 		۴:  \censor{نمیدونم}
ارزیابی چرخه عمر، شامل برسی و ارزیابی میزان ( آب مصرفی)، (انرژی)(پایدار بودن فرآیند تامین مواد خام)، (میزان زباله و پسماند ایجاد شده) و سهم حمل و نقل در همه مراحل) است.
ارزیابی چرخه عمر، حاصل تلاش برای یافتن شاخص‌هایی است که کمک می‌کنند صنایع در مسیر بهره‌گیری از دانش فنی و تخصصی سازگار‌تر با
محیط زیست حرکت کنند، و رفتار و عمل‌کرد خود را در مسیر رسیدن به توسعه پایدار «اصلاح» کنند.
‌برسی چرخه عمر برای کیسه پلاستیکی و پاکت کاغذی ( صفحه ۲۹)











مرحله ۱: استخراج و تولید مواد اولیه و خام ۲: مرحله تولید 	۳: مرحله مصرف	۴ : مرحله دفع

نفت
نفت خام، یکی از سوخت‌های  \censor{نمیدونم}  است که به شکل مایعی  \censor{نمیدونم}  ،  \censor{نمیدونم}  رنگ یا  \censor{نمیدونم}  ( متمایل به  \censor{نمیدونم}  ) از زمین بیرون کشیده می‌شود. نفت خام در دنیای کنونی، دو نقش اساسی دارد: «منبع تأمین  \censor{نمیدونم}  » و «  \censor{نمیدونم}  اولیه برای تهیه مواد و کالاها»
مصرف روزانه نفت خام ( ۸۰،۰۰۰،۰۰۰ بشکه ) است که:
نیمی از آن در سوخت  \censor{نمیدونم}   \censor{نمیدونم}  ( حدود ٪  \censor{نمیدونم}  )
و نیمی دیگر در تأمین  \censor{نمیدونم}  و انرژی  \censor{نمیدونم}  ( حدود ٪  \censor{نمیدونم}  ) و تولید  \censor{نمیدونم}  و  \censor{نمیدونم}  ،  \censor{نمیدونم}  ها، مواد  \censor{نمیدونم}  و  \censor{نمیدونم}  ،  \censor{نمیدونم}  ،  \censor{نمیدونم}  ، مواد  \censor{نمیدونم}  و  \censor{نمیدونم}  ( حدود ٪  \censor{نمیدونم}  )
نفت خام، مخلوطی از هزاران ترکیب شیمیایی است که بخش عمده آن را  \censor{نمیدونم}  ‌های (شامل  \censor{نمیدونم}  و  \censor{نمیدونم}  ) گوناگون تشکیل می‌دهند. عنصر اصلی سازنده نفت خام،  \censor{نمیدونم}  است. کربن، اساس استخوان‌بندی  \censor{نمیدونم}   \censor{نمیدونم}  ها است.
کربن در خانه شماره  \censor{نمیدونم}  جدون دوره‌ای جای دارد. ( سرگروه گروه  \censor{نمیدونم}  ) و اتم آن، در لایه ظرفیت خود  \censor{نمیدونم}  الکترون دارد.
خود را بیازمایید صفحه ۳۰: الف) آرایش الکترونی فشرده:
ب) آرایش الکترون نقطه‌ای اتم کربن:  \censor{نمیدونم}  پ) انواع پیوند اشتراکی (برای رسیدن به آرایش هشتایی):  \censor{نمیدونم}  ،  \censor{نمیدونم}  و  \censor{نمیدونم}

مثال) تشکیل متان ():

\censor{نمیدونم}  \censor{نمیدونم}   \censor{نمیدونم}   \censor{نمیدونم}
\censor{نمیدونم}  \censor{نمیدونم}   \censor{نمیدونم}  و  \censor{نمیدونم}   \censor{نمیدونم}
≡-C \censor{نمیدونم}  \censor{نمیدونم}   \censor{نمیدونم}  و  \censor{نمیدونم}   \censor{نمیدونم}
=C= \censor{نمیدونم}  \censor{نمیدونم}   \censor{نمیدونم}

تمرین: آرایش الکترون نقطه‌ای اتم‌های زیر را رسم کنید:
الف) بیشترین تعداد الکترون لایه ظرفیت، مربوط به کدام گروه است؟
گروه  \censor{نمیدونم}  (  \censor{نمیدونم}  الکترون ظرفیتی )
ب) بیشترین تعداد الکترون منفرد ( تکی ) مربوط به کدام گروه است؟ گروه  \censor{نمیدونم}  (  \censor{نمیدونم}  تک الکترون)
پ) ظرفیت عناصر کدام گروه، بیشتر است؟ چرا؟ گروه  \censor{نمیدونم}  ( ظرفیت  \censor{نمیدونم}  ) ← ظرفیت اصلی گروه
مشاهده:
الف) اتم  \censor{نمیدونم}  و \censor{نمیدونم}  می‌توانند بیش از سایر فلز‌ها پیوند اشتراکی ایجاد کنند. ( با ظرفیت اصلی خود )
ب) اتم  \censor{نمیدونم}  ( و البته  \censor{نمیدونم}  ،  \censor{نمیدونم}  و  \censor{نمیدونم}  ) می‌توانند پیوند‌های دوگانه و اتم‌های  \censor{نمیدونم}  ،  \censor{نمیدونم}  و  \censor{نمیدونم}  می‌توانند پیوند سه‌گانه ایجاد کنند.
نتیجه: بیشترین و متنوع‌ترین ترکیبات، باید مربوط به گروه  \censor{نمیدونم}  باشد:
\censor{نمیدونم}  شازنده اصلی مولکول‌های زیستی و  \censor{نمیدونم}  سازنده اصلی جهان غیرزنده است.
ترکیبات کربن از سیلیسیم بسیار  \censor{نمیدونم}  است چون:
۱- پیوند‌های  \censor{نمیدونم}  تشکیل می‌دهد ( دلیل: طول پیوند  \censor{نمیدونم}  )
۲- توانایی تشکیل پیوند  \censor{نمیدونم}  و  \censor{نمیدونم}  و  \censor{نمیدونم}  را نیز دارد. ( شکل ۱۵ و ۱۶ صفحه ۳۱ )
گفتیم که نفت خام، مخلوطی از  \censor{نمیدونم}   \censor{نمیدونم}  است. هیدروکربن‌ها، دارای  \censor{نمیدونم}  و  \censor{نمیدونم}  گوناگونی هستند. البته کربن می‌تواند علاوه بر H به  \censor{نمیدونم}  و  \censor{نمیدونم}  نیز به شیوه‌های گوناگون متصل شود؛ و  \censor{نمیدونم}   \censor{نمیدونم}  ،  \censor{نمیدونم}  ،  \censor{نمیدونم}   \censor{نمیدونم}  ،  \censor{نمیدونم}  ،  \censor{نمیدونم}  و غیره را بسازد.
همچین، کربن‌ها می‌توانند به روش‌های گوناگون به هم متصل شوند و دگرشکل ( آلوتروپ ) های مختلفی مانند  \censor{نمیدونم}  ،  \censor{نمیدونم}  و غیره را ایجاد کنند.
یادآوری:تعریف و مقایسه «آلوتروپ، ایزوتوپ، ایزومر»
آلکان‌ها \censor{نمیدونم} (			)
دسته‌ای از هیدروکربن‌ها هستند که در آن‌ها، هر اتم کربن با  \censor{نمیدونم}  پیوند یگانه به اتم‌های دیگر متصل شده است ( یعنی حتماً با  \censor{نمیدونم}  اتم دیگر پیوند دارد. ).  \censor{نمیدونم}  ( C ) ساده‌ترین و نخستین عضو خانواده آلکان است. سایر اعضای خانواده، تعداد  \censor{نمیدونم}  های بیشتری دارند، که البته اتم‌های  \censor{نمیدونم}  آن‌ها نیز بیشتر می‌شود.
آلکان‌ها به دو دسته تقسیم می‌شوند:
۱- آلکن‌های  \censor{نمیدونم}   \censor{نمیدونم}  : اتم‌های  \censor{نمیدونم}  همانند یک  \censor{نمیدونم}  به دنبال هم قرار دارند. ( هر اتم کربن به  \censor{نمیدونم}  یا  \censor{نمیدونم}  اتم کربن در زنجیر کربنی متصل است. ) ( شکل ۱۸ الف)
۲-  \censor{نمیدونم}  : برخی اتم‌های کربن به شکل شاخه  \censor{نمیدونم}  (  \censor{نمیدونم}  ) به زنجیر اصلی متصل است. ( برخی اتم‌های کربن به  \censor{نمیدونم}  یا  \censor{نمیدونم}  اتم کربن در زنجیر متصل هستند.) ( شکل ۱۸ ب )
پرسش – کوچک‌ترین آلکانی که همه انواع کربن را دارد، چند اتم هیدروژن دارد؟ (حلقوی نباشد )
مدل پیوند – خط
در این روش، اتم‌های کربن با نقطه و پیوند بین آن‌ها با خط‌تیره ( پاره خط ) نشان داده می‌شوند.
اتم‌های هیدروژن، و نیز پیوند‌های C-H نشان داده  \censor{نمیدونم}  ( H متصل به اتم‌های دیگر، نشان داده  \censor{نمیدونم}  )
همچنین C-C-C با زاویه واقعی ۵/۱۰۹ نشان داده می‌شود. پیوند‌های دوگانه یا سه‌گانه نیز با دو یا سه خط نشان داده می‌شوند. سایر اتم‌ها مانند O یا N نیز نمایش داده  \censor{نمیدونم}  .
خود را بیازمایید صفحه ۳۳:
فرمول «ساختاری» یا «پیوند – خط» به همراه فرمول مولکولی را برای هر ترکیب نمایش دهید:
الف)


ب)




















پ)


ت)





















تمرین: با مدل پیوند – خط نمایش دهید:

شمار اتم‌های کربن نقش مهمی در تعیین  \censor{نمیدونم}  هیدروکربن‌ها دارد. با تغییر تعداد C،  \censor{نمیدونم}  مولکول نیز  \censor{نمیدونم}  مولکولی
تغییر می‌یابد ← تغییر نیروی  \censor{نمیدونم}  مولکولی، نقطه  \censor{نمیدونم}  و غیره


با هم بیندیشیم ۱ صفحه ۳۴: ( جمع‌بندی مهم )
بزرگ شدن اندازه مولکول: ۱.  \censor{نمیدونم}  نقطه جوش
۲.  \censor{نمیدونم}  فرار بودن ( تمایل برای تبدیل به گاز )
۳.  \censor{نمیدونم}  گران روی (  \censor{نمیدونم}  مقاوت در برابر جاری شدن )
الف) با افزایش شمار کربن ←  \censor{نمیدونم}  نقطه جوش آلکان در فشار ۱ اتمسفر ←  \censor{نمیدونم}  تعداد مولکول‌هایی که تبخیر می‌گردند (  \censor{نمیدونم}  فشار بخار )
ب) نقطه جوش:
پ) گران‌روی: 						فرار بودن:
ت) گشتاور دو قطبی آلکان‌ها صفر یا حدود  \censor{نمیدونم}  است. ( یعنی  \censor{نمیدونم}  هستند. )
ث) نیروی بین مولکولی در آلکان‌ها از نوع  \censor{نمیدونم}   \censor{نمیدونم}   \censor{نمیدونم}  است. افزایش شمار اتم‌های کربن، باعث  \censor{نمیدونم}  قدرت نیروی بین مولکولی، ( و  \censor{نمیدونم}  جرم و حجم مولکول ) و باعث  \censor{نمیدونم}  نقطه جوش می‌شود.
ج) با بزرگ‌تر شدن زنجیر کربنی، گران‌روی  \censor{نمیدونم}  می‌یابد چون مقاومت مولکول‌های بزرگ‌تر ددر برابر جاری شدن  \censor{نمیدونم}  است.
چسبندگی: 					(نیروی بین مولکولی (واندروالسی) در  \censor{نمیدونم}  قوی‌تر است. )
\censor{نمیدونم} (  \censor{نمیدونم}  )	 \censor{نمیدونم} (  \censor{نمیدونم}  )
با هم بیندیشیم ۲ صفحه ۳۵
الف) آلکان‌های  \censor{نمیدونم}  تا  \censor{نمیدونم}  کربنه در دمای ۲۲ درجه سانتی‌گراد به حالت گاز هستند.
ب) با افزایش جرم مولی آلکان، نقطه جوش  \censor{نمیدونم}  می‌یابد !!! ( این، ۴۰ بار! )
آلکان‌ها به دلیل  \censor{نمیدونم}  بودن، در آب  \censor{نمیدونم}  و می‌توان از آن‌ها برای حفاظت  \censor{نمیدونم}  استفاده کرد. قرار دادن فلز در آلکان‌های  \censor{نمیدونم}  یا  \censor{نمیدونم}  کردن سطح فلز‌ها و وسایل فلزی با آن‌ها، مانع از رسیدن  \censor{نمیدونم}  به سطح فلز می‌شود و از  \censor{نمیدونم}  فلز جلوگیری می‌کند.
آلکان‌ها، ترکیباتی سیر  \censor{نمیدونم}  هستند، ( هر اتم کربن به  \censor{نمیدونم}  اتم دیگر متصل است ). پیوند‌های آن‌ها فقط اشتراکی  \censor{نمیدونم}  است. ( دوگانه و سه‌گانه  \censor{نمیدونم}  ). آلکان‌ها تمایل زیادی برای واکنش شیمیایی  \censor{نمیدونم}  .
اگر آلکان‌ها را استنشاق کنیم، میزان سمی بودن آن‌ها  \censor{نمیدونم}  است و استنشاق آن‌ها بر شش‌ها و بدن، تأثیر چندانی ندارد ( فقط سبب کاهش  \censor{نمیدونم}  در هوای دم می‌شوند ) البته، ورود بخار  \censor{نمیدونم}  به شش‌ها از  \censor{نمیدونم}  گاز‌های تنفسی جلوگیری می‌کند و حتی ممکن است سبب مرگ شود.

خود را بیازمایید صفحه ۳۷: گشتاور دو قطبی مولکول‌های سازنده چربی‌ها، حدود  \censor{نمیدونم}  است. ( چربی‌ها،  \censor{نمیدونم}  هستند. )
الف) افرادی که با گریس کار می‌کنند، دستشان را با بنزین یا نفت ( یا مخلوطی از هیدروکربن‌ها ) می‌شویند چون شبیه،  \censor{نمیدونم}  را حل می‌کند (‌ هر دو دسته مواد،  \censor{نمیدونم}  هستند ) پس بنزین یا نفت سفید به عنوان  \censor{نمیدونم}  ، گریس را حل می‌کند.
ب) پس از شستن دست با بنزین،  \censor{نمیدونم}   \censor{نمیدونم}  پوست نیز در بنزین  \censor{نمیدونم}  و شسته می‌شود و در نتیجه پوست  \censor{نمیدونم}  ‌می‌گردد.
پ) شستن پوست یا تماس با آلکان‌های مایع در دراز مدت به ساختار پوست آسیب می‌رساند زیرا قشر  \censor{نمیدونم}  برداشته شده و پوست ( خشک / مرطوب ) و  \censor{نمیدونم}  و مستعد ابتلا به عفونت، ترک‌خوردن، اگزما یا آلرژی می‌شود.
«نامگذاری آلکان‌ها» ( پیوند با ریاضی صفحه ۳۵ )
واژه «آلکان» از دو جزء ساخته شده است. به جای لفظ «آلکـ» همواره کلمه‌ای قرار می‌گیرد که  \censor{نمیدونم}  اتم کربن را مشخص می‌کند.
اعداد یونانی ا تا ۴ به ترتیب  \censor{نمیدونم}  ،  \censor{نمیدونم}  ،  \censor{نمیدونم}  و  \censor{نمیدونم}  هستند که برای نامگذاری انتخاب نشده و به جای آن‌ها واژه‌های دیگری به کار می‌رود. اما پیشوند‌های  \censor{نمیدونم}  برای  \censor{نمیدونم}  کربن به بالا، استفاده می‌شوند.
«نامگذاری آلکان‌های شاخه‌دار»
برای نام‌گذاری آلکان‌های شاخه‌دار، باید: ۱) نام شاخه‌های جانبی ( فرعی ) را بدانیم: 			 \censor{نمیدونم}  \censor{نمیدونم}  \censor{نمیدونم} $\rightarrow$ آلکان
\censor{نمیدونم} ( \censor{نمیدونم} )									 \censor{نمیدونم} ( \censor{نمیدونم} )
۲) سپس باید زنجیر اصلی را به درستی انتخاب کنیم: زنجیری که بیشترین تعداد  \censor{نمیدونم}  را دارد. ( به شرطی که از هر کربن فقط ۱ بار عبور کنیم.) در هر مورد، دور زنجیر اصلی، کادر بکشید:



نکته ۱: اگر بتوان برای هیدروکربنی، دو زنجیر اصلی با کربن‌های برابر اما شاخه‌های فرعی متفاوت انتخاب کرد، انتخابی درست است که تعداد شاخه فرعی  \censor{نمیدونم}  دارد:
نکته ۲: گروه آلکیل ( مانند متیل یا اتیل ) در کربن ابتدایی یا پایانی زنجیر اصلی، در‌واقع، ادامه  \censor{نمیدونم}  است و شاخه فرعی محسوب  \censor{نمیدونم}  تمرین ۱: نامگذاری کنید:
۳) سپس، زنجیر اصلی انتخاب شده ار از طرفی که به  \censor{نمیدونم}   \censor{نمیدونم}  نزدیک‌تر است، شماره‌گذاری می‌کنیم. ( شماره اتصال شاخه فرقی باید  \censor{نمیدونم}  باشد. ) ( سه ترکیب قسمت ۲ را شماره گذاری نمایید.)
۴) نامگذاری:
>>> اگر تعداد شاخه یکی باشد: شماره اتصال و نام شاخه  \censor{نمیدونم}  و سپس نام  \censor{نمیدونم}   \censor{نمیدونم}  ذکر می‌شود:


با هم بیندیشیم ۱ صفحه ۳۸:
الف) اعداد، نشانگر شماره  \censor{نمیدونم}  در  \censor{نمیدونم}  اصلی است که  \censor{نمیدونم}  فرعی به آن متصل شده است و واژه بعد از آن،  \censor{نمیدونم}  شاخه فرعی را نشان می‌دهد. واژه بعدی، نام  \censor{نمیدونم}   \censor{نمیدونم}  است.
ب) شباهت این دو ترکیب، در تعداد کل  \censor{نمیدونم}  در ترکیب، و نیز تعداد کربن  \censor{نمیدونم}   \censor{نمیدونم}  و نیز، تعداد کربن و نوع  \censor{نمیدونم}   \censor{نمیدونم}  است.
تفاوت این دو ترکیب، در  \censor{نمیدونم}   \censor{نمیدونم}  اتصال شاخه فرعی است.
۳- متیل هگزان
۴- متیل هپتان
با هم بیندیشیم ۳:

زنجیر اصلی  \censor{نمیدونم}  کربنه

زنجیر اصلی  \censor{نمیدونم}  کربنه

زنجیر اصلی  \censor{نمیدونم}  کربنه
با هم بیندیشیم ۴:


انتخاب زنجیر
نام نادرست:

جهت شماره‌گذاری
انتخاب زنجیر
نام نادرست:

جهت شماره‌گذاری
انتخاب زنجیر
نام درست:
نکته مهم: متیل در کربن اول، اتیل در کربن اول و دوم، پروپیل در کربن‌های اول، دوم و سوم زنجیر، شاخه فرعی  \censor{نمیدونم}   \censor{نمیدونم}  و ادامه زنجیر محسوب  \censor{نمیدونم}
خود را بیازمایید ۱ الف صفحه ۳۹:


تمرین دوره‌ای ۵ قسمت (پ):


خود را بیازمایید ۲ صفحه ۴۰:
نکته: هالوژن‌ها نیز می‌توانند به عنوان شاخه فرعی در ترکیب‌های آلی محسوب شوند.
در نامگذاری، پسوند «ـو» به نام هالوژن افزوده می‌شود.
تذکر مهم: هالوژن‌ها ( برخلاف گروه‌های آلکیل ) در کربن اول زنجیر نیز شاخه فرعی می‌توانند باشند.



نکته: هنگامی که شاخه فرعی، فقط یک کربن اتصال در زنجیر اصلی دارد، شماره اتصال شاخه فرعی نباید ذکر شود. ( برخی کتاب‌ها می‌گویند که بهتر است گفته نشود. )
تذکر مهم: اگر تا رسیدن به وسط زنجیر بیش از یک موققیت برای شاخه فرعی وجود داست حتما شماره اتصال شاخه فرعه ذکر شود.
تمرین : ترکیبی با فرمول مولکولی \censor{نمیدونم} چند ایزومر ساختاری دارد؟

نکته: هالوژن ( می‌تواند / نمی‌تواند ) در کربن اول زنجیر نیز شاخه فرعی باشد.
نتیجه: عدد ۱ برای هالوژن‌ها ( به عنوان شاخه ) ذکر  \censor{نمیدونم}   \censor{نمیدونم}  . ( در صورت لزوم )
معرفی دو شاخه فرعی دیگر:						و
ادامه نامگذاری ( قوانین ):
>>> تعداد شاخه فرعی بیش از یک دو حالت دارد:
۱- دو یا چند شاخه فرعی اما از یک نوع			۲- دو یا چند شاخه فرعی از گونه‌های متفاوت
حالت ۱: دو یا چند شاخه فرعی اما از یک نوع
اگر تعداد شاخه فرعی، بیش از یکی باشد ( اما همه از یک نوع باشند )؛ ابتدا، «همه» شماره‌های اتصال، از  \censor{نمیدونم}  به  \censor{نمیدونم}  نوشته می‌شود ( حتی اگر  \censor{نمیدونم}  باشد. ) سپس تعداد آن شاخه ( با لفظ یونانی ) و نام آن شاخه فرعی ذکر می‌شود.






(بهتر است که کربن‌های بیشتر، در یک خط نوشته شوند که زنجیر اصلی، مستقیم باشد. )

خود را بیازمایید ۱ (ج) صفحه۴۰:

تذکر: وقتی بیش از یک شاخه فرعی داریم، شماره‌گذاری زنجیر اصلی، «باید» از طرفی انجام شود که بتوان با ارقام آن‌ها عدد  \censor{نمیدونم}  ساخت.





خود را بیازمایید ۱ ت صفحه ۳۹



حالت دوم: دو یا چند شاخه فرعی از گونه‌های متفاوت
اگر تعداد شاخه فرعی، بیش از یکی باشد اما از گونه‌های متفاوت باشند، شماره‌گذاری ( بدون توجه به انواع شاخه‌ها ) از طرفی که ارقام کوچکتر انتخاب شوند انجام می‌شود.
اما در نامگذاری: تقدم ذکر نام شاخه فرعی، بر اساس حرف اول نام آن ( در انگلیسی ) است. ← در این حالت، شماره اتصال و نام هر شاخه فرعی، جداگانه ذکر می‌شود.



یعنی: در نامگذاری، شاخه فرعی  \censor{نمیدونم}  بر  \censor{نمیدونم}  مقدم است، ( به دلیل تقدم حرف اول نام ) چه شماره اتصالش بیشتر باشد، چه کمتر و چه مساوی!
خود را بیازمایید ۱ ب صفحه ۳۹:


نکته: اگر شماره‌گذاری دو نوع شاخه فرعی، از دو طرف ارقام یکسانی بدهد، شماره‌گذاری باید از طرف آن شاخه فرعی انجام شود که شاخه مقدم در نام‌گذاری شماره  \censor{نمیدونم}  داشته باشد:
در نام‌گذاری ترکیب‌های آلی، بین عدد و عدد:  \censor{نمیدونم} ، بین عدد و کلمه:  \censor{نمیدونم}  قرار می‌گیرد و بین کلمه و کلمه:  \censor{نمیدونم}  !
نامگذاری کنید:

تمرین ۱: ایزومر‌های \censor{نمیدونم} را رسم کنید ( فرمول ساختاری و خط پیوند ) و سپس نامگذاری نمایید:




تمرین ۲: در بین ایزومر‌های \censor{نمیدونم} چند ایزومر داریم که ۴ کربن در زنجیر اصلی داشته باشند و نامگذاری کنید.


تمرین ۳: مثال‌های زیر را با مدل نقطه – خط نمایش دهید ( ابتدا زنجیر اصلی را بکشید، راحت‌تر است )
الف) ۲ – کلرو – ۳ – فلوئورو – ۴،۳ – دی متیل هپتان
ب) ۳ – ایتل – ۳،۲ – دی متیل پنتان

تمرین ۴: ترکیب زیر را نام‌گذاری کنید: (وقتی ترکیب شلوغه، نام هر شاخه را که نوشتی، در زنجیر خط بزن که تکراری ننویسی )

نکته:
تعداد پیوند‌های کربن – کربن در آلکان‌ها ( برحسب n ):
تعداد پیوند‌های کربن – هیدروژن در آلکان‌ها ( برحسب n ):
تعداد پیوند اشتراکی در آلکان‌ها ( برحسب n ):
تعداد پیوند اشتراکی در هیدروکربن‌ها (CxHy) ( برحسب x و y ):
تعداد پیوند اشتراکی در آلکن (‌ برحسب n ):
تعداد پیوند اشتراکی در آلکین ( برحسب n ):
تعداد پیوند اشتراکی در سیکلوآلکان ( برحسب n ):
تعداد پیوند C-C در آلکان ( با n کربن )،	در آلکن،		در آلکین،		در سیکلوآلکان (!)







«آلکن‌ها (		)»
این هیدروکربن‌ها در ساختار خود، یک پیوند دوگانه  \censor{نمیدونم}  –  \censor{نمیدونم}  ( 		 ) دارند. برای نامگذاری، پسوند «ـِن» را به لفظ آلک می‌افزاییم. ساده‌ترین آلکن  \censor{نمیدونم}  کربن دارد ← 		( فرمول  \censor{نمیدونم}  ) \censor{نمیدونم} یا	 \censor{نمیدونم} (‌ فرمول ساختاری کوتاه شده ) یا 		 \censor{نمیدونم} ( فرمول  \censor{نمیدونم}  ) ( نام:  \censor{نمیدونم}  )
نام قدیمی اتن، «  \censor{نمیدونم}  » بوده و در بیشتر گیاهان وجود دارد. اتن آزاد شده در گیاهانی نظیر  \censor{نمیدونم}  یا  \censor{نمیدونم}   \censor{نمیدونم}  ، موجب رسیدن سریع‌تر میوه‌های نارس می‌شود و از آن به عنوان  \censor{نمیدونم}   \censor{نمیدونم}  استفاده می‌شود.
تمرین ۱: نام، فرمول مولکولی و فرمول ساختاری و مدل خط پیوند را برای آلکنی با ۳ کربن، نشان دهید.

نکته بسیار مهم: پیوند دوگانه، باید جزء زنجیر اصلی قرار گیرد، حتی اگر مجبور باشیم، بلندترین زنجیر ممکن را انتخاب نکنیم!

تمرین ۲ : ، سه ایزومر آلکنی دارد. آن‌ها را رسم و نامگذاری کنید. ( نام:  \censor{نمیدونم}  	$\rightarrow$				) \censor{نمیدونم}
( نام:  \censor{نمیدونم}  $\rightarrow$					 ) 	( نام:  \censor{نمیدونم}  $\rightarrow$						)
نکته: در آلکن‌های چهارکربنه به بالا، باید پیش از ذکر لفظ «آلک»، شماره‌ای را ذکر کرد که جایگاه پیوند دوگانه را نشان دهد از بین دو کربنی که پیوند دوگانه دارند، باید شماره  \censor{نمیدونم}  را ذکر کرد.
تمرین ۳ : ایزومر‌های آلکنی 		را رسم و نامگذاری کنید.




تمرین ۴ – نسبت تعداد H در «سومین آلکان» به «سومین آلکن» چند است؟

تمرین ۵ – بین آلکان و آلکن هم کربن، ایزومر‌های کدام، بیشتر است؟

واکنش‌های آلکن‌ها ( سیر شدن ← فصل دوم – پلیمر شدن ← فصل سوم )
سیر شدن:
آلکن‌ها از آلکان‌ها، واکنش‌پذیری  \censor{نمیدونم}  دارند، و به خاطر وجود پیوند دوگانه، سیر  \censor{نمیدونم}  هستند. در ( C = C ) یکی از دو پیوند، از دیگر ضعیف‌تر است آسان‌تر شکسته می‌شود و دو ذره  \censor{نمیدونم}  ظرفیتی را به دو کربن، متصل می‌کند:
بررسی تمرین دوره‌ای ۸:



در واکنش سیر‌شدن، هر اتم کربن، از تمام امکان خود برای تشکیل پیوند‌های  \censor{نمیدونم}  استفاده می‌کند، ( به جای اینکه  \censor{نمیدونم}  پیوند دوگانه و  \censor{نمیدونم}  پیوند یگانه داشته باشد،  \censor{نمیدونم}  پیوند یگانه خواهد داشت. )
معمولا هر اتم کربن، ۴ پیوند اشتراکی دارد به جز:  \censor{نمیدونم}



* تذکر: واکنش آلکن‌ها با Cl-Cl نیاز به کاتالیزگر  \censor{نمیدونم}  دارد. تمرین دوره‌ای ۵ فصل ۳ ← !!
تمرین – تفاوت تعداد اتم‌های H بین واکنش‌دهنده و فرآورده در واکنش «۲و۳ – دی‌متیل – ۲ – بوتن» با برم مایع چندتا است؟ نام فرآورده چیست؟


وارد کردن آلکن در بخار برم مایع ( قرمز ) یا آب برم ( قرمز )، ترکیبی  \censor{نمیدونم}  رنگ ایجاد می‌کند که نشانگر انجام واکنش، و مهم‌ترین روش شناسایی ترکیب‌های سیر نشده از سیر شده است.
سایر هالوژن‌ها نیز می‌تواندد چنین واکنشی را انجام دهند و در مقابل ترکیب سیر‌نشده،  \censor{نمیدونم}  رنگ شوند.
تذکر: هالوژن‌ها در حالت عنصری ( آزاد )، ( رنگی / بی‌رنگ ) و در حالت ترکیب  \censor{نمیدونم}  هستند.







اسید‌های هیدرولیک نیز می‌توانند در واکنش با آلکن‌ها شرکت کنند. گاز اتن، سنگ‌بنای صنایع پتروشیمی است. با استفاده از اتن، حجم انبوهی از مواد گوناگونی تهیه می‌شود. از واکنش اتن با آب در حضور  \censor{نمیدونم}  به عنوان کاتالیز‌گر،  \censor{نمیدونم}  تولید می‌شود. که الکلی  \censor{نمیدونم}  کربنه،  \censor{نمیدونم}  رنگ، و فرّار ( نقطه جوش  \censor{نمیدونم}  تر از آب ) است. به هر نسبتی در  \censor{نمیدونم}  حل می‌شود. از مهم‌ترین  \censor{نمیدونم}  های صنعتی است و در تهیه مواد دارویی، آرایشی و بهداشتی و به عنوان «ضد عفونی کننده» به کار می‌رود.
* خود را بیازمایید ۱ صفحه ۴۲:
گوشت رنگ بخار برم را از بین برده پس چربی آن ترکیبات سیر  \censor{نمیدونم}  ( نیز ) دارد. ( که با برم واکنش می‌دهد. )
در صنعت پتروشیمی، ترکیب‌ها، مواد و وسایل گوناگون از  \censor{نمیدونم}  یا  \censor{نمیدونم}  طبیعی به دست می‌آید. ( فرآورده‌های پتروشیمیایی )
در صنایع پتروشیمی کشور‌ها، موادی نظیر  \censor{نمیدونم}  ،  \censor{نمیدونم}   \censor{نمیدونم}  و  \censor{نمیدونم}   \censor{نمیدونم}  تولید می‌شوند.
آلکین‌ها (		) ( سیر نشده‌تر از آلکن‌ها ! )
آلکین‌ها در ساختتار خود، یک پیوند سه‌گانه کربن-کربن (-C≡C-) دارند. برای نام‌گذاری، پسوند «ـین» را به لفظ آلک اضافه می‌کنیم. ساده‌ترین الکین  \censor{نمیدونم}  کربن دارد: ( گاز:  \censor{نمیدونم}  ) CH یا -C≡C-
نام قدیمی گاز اتین،  \censor{نمیدونم}  است که ( از شعله آن ) در  \censor{نمیدونم}  کاری و  \censor{نمیدونم}  کاری فلز‌ها استفاده می‌شود و به آن، جوش  \censor{نمیدونم}  نیز گفته می‌شود:							+		$\rightarrow$		+
در این روش، کلسیم  \censor{نمیدونم}  (  \censor{نمیدونم}  ) در یک مخزن نگه‌داری و با افزودن آب، به  \censor{نمیدونم}  تبدیل می‌شود.
تمرین ۱ – فرمول ساختاری و مولکولی، مدل پیوند – خط، و نام آلکین سه کربنه چیست؟ ( فرمول پیوند – خط )



تمرین ۲ – ایزومر‌های آلکنی 	 \censor{نمیدونم} را رسم و ناگذاری کنید: ( چرا کلمه آلکنی گفته شده؟ * )


تمرین ۳ – واکنش ۱ مول پروپین با ۱ مول برم مایع را بنویسید:

تمرین ۴ – واکنش ۱ مول اتین را با ۲ مول گاز کلر بنویسید:

تمرین ۵ – هر مول اتین برای سیر‌شدن کامل، به چند مول گاز هیدروژن نیاز دارد؟

تمرین ۶ – یک آلکین در اثر سیر شدن کامل با گاز هیدروژن، ۱۰٪ افزایش جرم دارد. تعداد هیدروژن آلکان هم‌کربن این آلکین چند تا است؟

تمرین ۷ – ترکیب		 		 برای سیر شدن کامل:
اولاً ) به چند مول \censor{نمیدونم} نیاز دارد؟
دوم) چند مول فرآورده تشکیل می‌شود؟
*سوم) این ترکیبا با ۱-بوتین ایزومر است یا با ۱-بوتن؟
واکنش سوختن کامل ( پارامتری بر حسب n )
آلکان، الکن و آلکین
( با n اتم کربن )
پرسش – آیا این گفته درست است؟ «کربن دارای پیوند سه‌گانه در آلکین، نمی‌تواند شاخه فرعی داشته باشد.»



هیدروکربن‌های حلقوی
خود را بیازمایید الف و ب صفحه ۴۲ :
الف) هیدروکربن‌های حلقوی سیر‌شده (  \censor{نمیدونم}  آلکان ) ← معروف‌ترین آن‌ها  \censor{نمیدونم}   \censor{نمیدونم}  است:
حلقه در سیکو هگزان سطح ( است / نیست ) .






\censor{نمیدونم}  قلمرو پیوندی اطراف هر اتم کربن
زاویه پیوندی:  \censor{نمیدونم}
همه قلمرو ها در یک صفحه :
( مدل خط – پیوندی )




فرمول مولکولی


ب) آروماتیک ← ممکن است دارای یک  \censor{نمیدونم}  ، دو  \censor{نمیدونم}  ( یا بیشتر ) باشند.
\censor{نمیدونم} ← معروف‌ترین ترکیب آروماتیک،  \censor{نمیدونم}  با  \censor{نمیدونم}  حلقه و پیوند دوگانه  \censor{نمیدونم}   \censor{نمیدونم}   \censor{نمیدونم}  است.
نفتالن نیز از ترکیبات آروماتیک (‌ دو حلقه‌ای ) است. ( و در  \censor{نمیدونم}  پیوند دوگانه دارد ) ( \censor{نمیدونم} 
$\mathrm{C_{\censor{نمیدونم}}H_{\censor{نمیدونم}}}$
)

يا \censor{نمیدونم} 						 \censor{نمیدونم} یا

نفتالن به عنوان  \censor{نمیدونم}   \censor{نمیدونم}  برای نگهداری  \censor{نمیدونم}  و  \censor{نمیدونم}  به کار می‌رود.
تمرین – هر مول بنزین، چند مول اتم هیدروژن از هر مول هگزان کم دارد؟

تست – یک آلکن، در صورت هم کربن بودن، با کدامیک هم‌پار است؟
۱) آلکین			۲) سیکلوآلکان			۳) آلکان			۴) \censor{نمیدونم} آروماتیک
تمرین – جرم مولی آلکان، آلکن، آلکین و سیکلوآلکان را بر حسب n بنویسید.
نفت، ماده‌ای که اقتصاد جهان را دگرگون ساخت
نفت خام به طور عمده مخلوطی از  \censor{نمیدونم}  و به مقدار کم برخی  \censor{نمیدونم}  ،  \censor{نمیدونم}  ،  \censor{نمیدونم}  و غیره است.
مقدار نمک و اسید در نفت خام  \censor{نمیدونم}  و در مناطق گوناگون،  \censor{نمیدونم}  است. دلیل: شرایط  \censor{نمیدونم}  و نحوه  \censor{نمیدونم}  نفت خام
← بخش عمده هیدروکربن‌های نفت خام را  \censor{نمیدونم}  تشکیل می‌دهند که به دلیل واکنش‌پذیری  \censor{نمیدونم}  به عنوان  \censor{نمیدونم}  به کار می‌روند.
← بیش از ۹۰٪ نفت خام صرف  \censor{نمیدونم}  و تأمین  \censor{نمیدونم}  می‌شود و مقدار کمی از آن در صنایع  \censor{نمیدونم}  کاربرد دارد.
با هم بیندیشیم صفحه ۴۳:
بنزین و خوراک پتروشیمی:  \censor{نمیدونم}  >  \censor{نمیدونم}  =  \censor{نمیدونم}   \censor{نمیدونم}  >  \censor{نمیدونم}   \censor{نمیدونم}
نفت سفید:  \censor{نمیدونم}  =  \censor{نمیدونم}   \censor{نمیدونم}  >  \censor{نمیدونم}   \censor{نمیدونم}  >  \censor{نمیدونم}   \censor{نمیدونم}
گازوییل:  \censor{نمیدونم}  >  \censor{نمیدونم}   \censor{نمیدونم}  >  \censor{نمیدونم}   \censor{نمیدونم}  >  \censor{نمیدونم}   \censor{نمیدونم}
نفت کوره:  \censor{نمیدونم}  <  \censor{نمیدونم}   \censor{نمیدونم}  <  \censor{نمیدونم}   \censor{نمیدونم}  <  \censor{نمیدونم}   \censor{نمیدونم}
الف) اندازه مولکول: نفت کوره \censor{نمیدونم} بنزین (  \censor{نمیدونم}  فرّارتر ← نقطه جوش  \censor{نمیدونم}  تر ⇄ جرم و اندازه مولکول  \censor{نمیدونم}  کم‌تر است )
ب) در نفت سنگین،  \censor{نمیدونم}   \censor{نمیدونم}  بیشتری هست. در نفت سبک، «  \censor{نمیدونم}  و  \censor{نمیدونم}  » ، «  \censor{نمیدونم}   \censor{نمیدونم}  » و «  \censor{نمیدونم}  » بیشتری هست.
پ) ملاک دسته‌بندی نفت خام به سبک و سنگین،  \censor{نمیدونم}   \censor{نمیدونم}  تشکیل‌دهنده آن است. ( نفت کوره ملاک است )
ت) گران‌ترین بخش نفت خام،  \censor{نمیدونم}  است و در نتیجه نفت  \censor{نمیدونم}  و نفت  \censor{نمیدونم}   \censor{نمیدونم}  ، به ترتیب، بیشتری و کمترین قیمت را دارند.
«پالایش نفت خام»
پس از جدا کردن  \censor{نمیدونم}  ،  \censor{نمیدونم}  و  \censor{نمیدونم}  ، نفت خام را پالایش می‌کنند. با استفاده از  \censor{نمیدونم}   \censor{نمیدونم}  به  \censor{نمیدونم}  ، (تقطیر  \censor{نمیدونم}  به  \censor{نمیدونم}  ، هنگامی صورت می‌گیرد که نقطه جوش اجزاء مخلوط، به هم نزدیک باشند.) هیدروکربن‌های آن، به صورت  \censor{نمیدونم} ‌ هایی با  \censor{نمیدونم}   \censor{نمیدونم}  نزدیک به هم، جدا می‌شوند.
ابتدا، نفت خام را در محفظه‌ای بزرگ  \censor{نمیدونم}  می‌دهند و آن را به  \censor{نمیدونم}  تقطیر هدایت می‌کنند. در برج تقطیر، دما از  \censor{نمیدونم}  به  \censor{نمیدونم}  کم می‌شود (  \censor{نمیدونم}  سرد‌تر است ) نفت خام داغ به قسمت  \censor{نمیدونم}  وارد می‌شود. مولکول‌های  \censor{نمیدونم}  تر و  \censor{نمیدونم}  تر، از جمله مواد  \censor{نمیدونم}   \censor{نمیدونم}  از  \censor{نمیدونم}  بیرون آمده و به سوی  \censor{نمیدونم}  برج حرکت می‌کنند. به تدریج که مولکول‌ها بالاتر می‌روند،  \censor{نمیدونم}  شده و به  \censor{نمیدونم}  تبدیل می‌شوند، و در  \censor{نمیدونم}  هایی که در فاصله‌های گوناگون برج هستند، وارد شده و از برج  \censor{نمیدونم}  می‌شوند.
پالایش نفت خام، سوخت  \censor{نمیدونم}  و مناسب در اختیار صنایع قرار می‌دهد و از سویی منجر به تولید انرژی  \censor{نمیدونم}  ارزان می‌گردد. با افزایش اهمیت و کاربرد بی‌رویه، نفت خام رو به پایان می‌رود.
زغال‌سنگ (			 \censor{نمیدونم} )
یکی دیگر از سوخت‌های  \censor{نمیدونم}  است که عمر زخایر آن به ۵۰۰ سال می‌رسد. زغال‌سنگ، می‌تواند به عنوان  \censor{نمیدونم}  ، جایگزین نفت شود، البته باعث ورود مقدار بیشتری از  \censor{نمیدونم}  به هوا نیز می‌شود و اثر  \censor{نمیدونم}  را تشدید می‌کند:
\censor{نمیدونم} بنزین:  \censor{نمیدونم}  ،  \censor{نمیدونم}  و  \censor{نمیدونم}
\censor{نمیدونم} زغال‌سنگ:  \censor{نمیدونم}  ،  \censor{نمیدونم}  ،  \censor{نمیدونم}  ،  \censor{نمیدونم}  و  \censor{نمیدونم}
گرمای آزاد شده ( به ازای ۱ گرم ): بنزین \censor{نمیدونم} زغال‌سنگ
مقدار C تولید شده: بنزین زغال‌سنگ
راه‌های بهبود کار‌آیی زغال‌سنگ:
۱)  \censor{نمیدونم}  و  \censor{نمیدونم}  زغال‌سنگ برای حذف  \censor{نمیدونم}  و ناخالصی‌های دیگر
۲) به  \censor{نمیدونم}  انداختن گاز	 \censor{نمیدونم} خارج شده از دودکش  \censor{نمیدونم}  ها به کمک
شرایط  \censor{نمیدونم}  زغال‌سنگ نیز بسیار دشوار و خطرناک است و معادل زغال‌سنگ، بار‌ها دچار  \censor{نمیدونم}   \censor{نمیدونم}  یا  \censor{نمیدونم}  شده‌اند. انفجار به دلیل  \censor{نمیدونم}  گاز  \censor{نمیدونم}  آزاد شده هنگام استخراج زغال‌سنگ است. می‌دانیم که متان گازی ( سبک/ سنگین)، بی  \censor{نمیدونم}  و بی  \censor{نمیدونم}  است و اگر مقدار آن به بیش از  \censor{نمیدونم}  درصد برسد، احتمال  \censor{نمیدونم}  وجود دارد. هرچه متان بیشتر باشد، احتمال انفجار نیز  \censor{نمیدونم}  خواهد بود.
«پیوند با صنعت»
حمل و نقل هوایی  \censor{نمیدونم}  ترین حالت حمل و نقل بوده و رو به گسترش است.
مزایا:  \censor{نمیدونم}  – عدم نیاز به  \censor{نمیدونم}  سازی و  \censor{نمیدونم}  جاده – مسافرت آسان،  \censor{نمیدونم}  رسانی خوب در مواقع  \censor{نمیدونم}
معایب:  \censor{نمیدونم}
سوخت هواپیما از پالایش  \censor{نمیدونم}   \censor{نمیدونم}  در برج تقطیر پالایشگاه‌ها تولید می‌شود و به طور عمده از نفت  \censor{نمیدونم}  تشکیل شده است. ( مخلوطی از  \censor{نمیدونم}  با  \censor{نمیدونم}  تا  \censor{نمیدونم}  کربن )
یکی از مسائل مهم در تأمین سوخت،  \censor{نمیدونم}  آن به مراکز توزیع و استفاده از آن است. که حدود ۶۶٪ از طریق خط  \censor{نمیدونم}  و تعبیه از طریق  \censor{نمیدونم}   \censor{نمیدونم}  ،  \censor{نمیدونم}  جاده‌پیما و  \censor{نمیدونم}  های نفتی انجام می‌شود.
تمرین ۱ – ۴۴.۸۱ مخلوط متان و اتن، در حضور اکسیژن کافی، به طول کامل می‌سوزند. اگر گرمای حاصل، بتواند دمای ۸.۲ کیلوگرم آب را از ۲۰ درجه سانتی‌گراد به ۱۰۰ درجه برساند، جرم اتیلن در مخلوط به تقریب، چند گرم است؟

\end{document}




