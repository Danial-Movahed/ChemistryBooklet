\documentclass[a4paper,12pt]{article}
%\usepackage[top=1in,margin=5em]{geometry}
\usepackage[top=7em,bottom=1pt,right=0.8in,left=0.8in,headheight=65pt,headsep=1cm]{geometry}
\usepackage{tikz-page}
\usepackage{tikz}
\usepackage{fancybox}
\usetikzlibrary{shadows.blur}
\usepackage{enumitem}
\usepackage{zref-abspage}
\usepackage{tabularray}
\usepackage{pbox}
\usepackage[version=4]{mhchem}
\usepackage{tasks}
\usepackage{multicol}
\usepackage{chemfig}
\usepackage{perpage}
\usepackage{setspace}
\usepackage{vwcol}  
\usepackage{mathtools}
\usepackage{textcomp, gensymb}
\usepackage{xepersian}



\pagestyle{plain}
\tikzset{
	secnode/.style={
		minimum height=1cm,
		inner xsep=20pt,
		rotate=90,
		anchor=north east,
		draw=white,
		fill=black,
		text=white,
		blur shadow={shadow blur steps=5,shadow blur extra rounding=1.3pt}},
	pagenode/.style={
		minimum width=5mm,
		minimum height=1cm,
		inner sep=2pt,
		anchor=south east,
		draw=white,
		fill=black,
		text=white,
		blur shadow={shadow blur steps=5,shadow blur extra rounding=1.3pt}}
}
\newcommand{\tikzpagelayout}{
	\draw[black,line width=2pt,rounded corners=20pt] ([xshift=10mm]page.northwest) |- ([xshift=-2cm,yshift=10mm]page.southeast);
	\node[secnode] at ([xshift=5mm]page.northwest) {سالم غذا‌های پی در | شکیباییان};
	\node[pagenode] at ([xshift=-1cm,yshift=5mm]page.southeast) {\thepage};
	
}

\newenvironment{en}
{\begin{enumerate}\setlength\itemsep{-0.2em}}
	{\end{enumerate}}

\newenvironment{iit}
{\begin{itemize}\setlength\itemsep{-0.5em}}
	{\end{itemize}}

\newcommand*\circled[1]{\tikz[baseline=(char.base)]{
		\node[shape=circle,draw,inner sep=2pt] (char) {#1};}}

\settextfont{XB Niloofar}
\setdigitfont{XB Niloofar}
\setstretch{1.5}
\setlist[itemize]{topsep=0pt}
\setlist[enumerate]{topsep=0pt}
\renewcommand{\headrulewidth}{0pt}
\setlength{\headheight}{14.5pt}
\setlength{\columnseprule}{1pt}
\setlength{\columnsep}{0.5cm}
\MakePerPage{footnote}

\newcommand{\fb}{\rule{2cm}{0.15mm}\;}
\newcommand{\fs}{\rule{1cm}{0.15mm}}
\newcommand{\fsm}{{\rule{0.5cm}{0.15mm}\;}}
\newcommand{\lin}{\vspace{4pt}\hrule\vspace{4pt}}
\newcommand\gototask[1]{\addtocounter{task}{\numexpr#1-\value{task}\relax}}
\NewTasksEnvironment[label=\arabic*.,label-format=\bfseries,label-width=4ex]{answers}[\a]
\def\extra{\rule{1ex}{0ex}}
\makeatletter
\newcommand\censor{\@ifstar{\@cenmath}{\@centext}}
\newcommand\@cenmath[1]{%
	\protect\rule[-.3ex]{\widthofpbox{\extra$#1$}}{0.1ex}}
\newcommand\@centext[1]{%
	\protect\rule[-.3ex]{\widthofpbox{\extra#1}}{0.1ex}}
\makeatother

\begin{document}
\textbf{مواد}،
در زندگی ما، نقشی شگرف و موثر دارند. صنایع غذا، پوشاک، حمل ‌و نقل، ساختمان، ارتباطات و غیره، کم و پیش تحت تاثیر «\censor{نمیدونم}» هستند. رشد و گسترش تمدن بشری در گرو کشف و شناخت مواد «\censor{نمیدونم}» است. برای رفع نیازها، باید مواد  \censor{نمیدونم} تولید شوند، یا با \censor{نمیدونم} مواد، خواص آن‌ها تغییر کند. شیمی‌دان‌ها با پی بردن به رابطه \censor{نمیدونم} مواد با \censor{نمیدونم} سازنده، دریافتند که «\censor{نمیدونم}» دادن به مواد و «\censor{نمیدونم}» مواد به یکدیگر، سبب «\censor{نمیدونم}»، و گاهی «\censor{نمیدونم}» خواص آن‌ها می‌شود. اکنون، می‌توان موادی \textbf{نو}، با ویژگی‌های \textbf{منحصر به فرد} و \textbf{دلخواه} \underline{طراحی} کرد.
\lin
خود را بیازمایید \underline{۱} صفحه \underline{۳}؛ الف)
$
\left\{
\begin{tabular}{rrr}
	\text{
		مواد \censor{نمیدونم} (\censor{نمیدونم})
	} & $\leftarrow$ & \text{
		فلز
	}\\
	\text{
		مواد \censor{نمیدونم} (\censor{نمیدونم})
	} & $\leftarrow$ & \text{
		لاستیک
	}
\end{tabular}
\right.
$
$\leftarrow$
دوچرخه

نتیجه: منشا اجزای این فرآورده، از «\censor{نمیدونم}» است.

این فرآیند، شامل به دست آوردن \textbf{\underline{مواد}} دلخواه از منابع مختلف، برای تولید \censor{نمیدونم} مشخص است\\
یعنی: \censor{نمیدونم} اولیه تهیه دوچرخه، به طور \censor{نمیدونم} قابل استفاده نیستند و باید \censor{نمیدونم} شوند.

ب) \censor{نمیدونم}، کناره‌های ورق \censor{نمیدونم} برش خورده و کناره‌های            \censor{نمیدونم} بریده شده، دور ریخته \censor{نمیدونم}

پ) قسمت‌های \censor{نمیدونم}، ممکن است در تماس با \underline{هوا} و \underline{رطوبت}، زنگ بزنند.\\
قسمت‌های \censor{نمیدونم} و \censor{نمیدونم}، \underline{فرسوده} و \underline{کهنه} می‌شوند.
 $\leftarrow$ \qquad \censor{نمیدونم}
 \lin
 خود را بیازمایید صفحه۳و۴: 
الف) همه مواد \censor{نمیدونم} و \censor{نمیدونم} از کره زمین به دست می‌آیند.

مواد به دو دسته تقسیم می‌شوند:
\begin{en}
	\item
	مستقیماً از کره زمین به دست می‌آیند؛ مانند فلز‌‌‌ها، نفت، الماس و طلا
	\item
	غیر‌مستقیم از کره زمین به دست می‌آیند؛ (از مواد \censor{نمیدونم} تهیه می‌شوند) مانند لاستیک و پلاستیک
\end{en}
ب) به سه شکل، به زمین باز می‌گردند: \censor{نمیدونم} و \censor{نمیدونم} (و برخی \censor{نمیدونم} شده با اجزای هوا‌کره)\\
پ) به تقریب، \censor{نمیدونم} کل مواد در کره زمین، \underline{ثابت} می‌ماند. هر چیزی که از زمین استخراج شده، در نهایت به صورت پسماند و زباله، به زمین باز می‌گردد.\\
ت) هر چه میزان بهره‌برداری از منابع، بیشتر باشد، آن کشور توسعه یافته‌تر است.
$\left(\frac{\text{درست}}{\text{نادرست}}\right)$\\
دلیل: «\censor{نمیدونم}»، ثروت ملی هستند. بهره‌برداری باید با مدیریت برداشت اصولی از \censor{نمیدونم} همراه باشد:
$\tiny ^{\circled{1}}$
میزان بهره‌برداری مدیریت شده از منابع،
$\tiny ^{\circled{2}}$
به داشتن \censor{نمیدونم} برداشت منابع، داشتن «\censor{نمیدونم}» های پیشرفته و
$\tiny ^{\circled{3}}$
آموزش درست «\censor{نمیدونم} \censor{نمیدونم}» بستگی دارد.\\
در نظر گرفتن \underline{۳} مورد بالا، به پیشرفت پایدار می‌انجامد.

خود را بیازمایید\underline{۳} صفحه ۴:

الف) حدود \censor{نمیدونم} میلیارد تن

ب) بیش از \underline{۷۰} میلیارد تن (برای هر سه و حدود \underline{۱۲} میلیارد تن برای فلز‌ها)

\begin{tabular}{rrrrrrr}
	میزان مصرف سه منبع: & \censor{نمیدونم} & > & \censor{نمیدونم} & > & \censor{نمیدونم} & \\
	شیب مصرف سه منبع: & \censor{نمیدونم} & > & \censor{نمیدونم} & > & \censor{نمیدونم} & (پس  از سال ۲۰۰۵)
\end{tabular}

پ) زمین، منبع عظیمی از هدایای ارزشمند و ضروری برای زندگی است. سالانه، مقادیر بسیار زیادی از منابع \censor{نمیدونم} ، \censor{نمیدونم} و \censor{نمیدونم} ، برای مصارف گوناگون، استخراج و مورد استفاده قرار می‌گیرند. با پیشرفت «\censor{نمیدونم}» و ساخت \textbf{دستگاه‌ها} و \textbf{ابزار} بهتر ( \censor{نمیدونم} بهتر و مدرن)، وابستگی (نیاز) به منابع، بیشتر\censor{نمیدونم} .
\lin
دانشمندان بزرگ، می‌توانند با برسی دقیق اطلاعات و یافته‌های موجود درباره \textbf{مواد} و \textbf{پدیده‌ها}ی گوناگون، \censor{نمیدونم}ها، \censor{نمیدونم}ها و \censor{نمیدونم} بین آن‌‌ها را درک کنند. (مانند               ، که جدول دوره ای را طراحی نمود.)
شیمی‌دان‌ها با               مواد و انجام               (استفاده از هر ۵         ) آن‌ها را دقیق برسی می‌کنند.
(آزمایش =              کنترل شده)
هدف این برسی‌ها، یافتن اطلاعات بیشتر و دقیق‌تر درباره             های مواد است. برقراری              بین این داده‌ها (و اطلاعات) و نیز، ‌یافتن           ها و          ها، گامی مهم‌تر و موثر‌تر در پیشرفت علم است.   
\end{document}




